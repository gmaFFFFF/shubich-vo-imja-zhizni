\chapter{Несвоевременное взросление}
Каждый человек не~забывает и˚гордится своей маленькой Родиной "--- местом рождения, роста и˚воспитания.

Деревня Вишеньки, окружённая лесными массивами, находится в˚центре Полесья и˚имеет пять небольших улиц, напоминающих формой рака. Вдоль˚улиц расположились деревянные дома разных размеров, а˚вдали от˚них "--- подсобные постройки и˚приусадебные участки. На˚отдельных из˚них расцветали сады, радовали своей красотой.

При˚въезде в˚деревню из˚районного центра Ельска находился пруд, окаймляемый сенокосами, и˚большое деревянное новое здание "--- школа.

Жители деревни работали в˚колхозе, вели дела на˚своих приусадебных участках. Протекала спокойная, трудовая жизнь со˚своими заботами, радостями, горестями. В~одном из˚таких домов, в˚центре деревни, проживали малолетний Миша и˚его отец, мать, бабушка, дедушка и˚младшая сестра. В~1941~году появился на˚свет ещё и˚братик. 

Однажды эту˚спокойную, умеренную жизнь в˚1941~году нарушило страшное сообщение «война», которое нагрянуло неожиданно как˚гром среди ясного неба. Оно сразу запало в˚сердца взрослых и˚детей. В~одночасье прекратилась трудовая жизнь взрослых, игры и˚смех детишек, они˚все сразу «повзрослели». 

\begin{wrapfigure}{O}{.4\textwidth}
\centering
\includegraphics[width=.35\textwidth]{dranka}
\caption{Дранка. Автор: Unomano, 08.05.2010}
\label{fig:dranka}
\end{wrapfigure}

Остановили работу пять мужчин, которые на˚самодельном станке строгали дранку\footnote{Дранка, гонт или˚шиндель (нем. Schindel) "--- кровельный материал в˚виде пластин из˚древесины.} для˚крыши. «Что˚дальше делать?» "--- начали рассуждать. Утром следующего дня отец Миши попросил жену собрать узелок с˚едой и˚отправился пешком 11\,км в˚Ельск "--- военкомат. При˚этом проронил: «Не~провожайте, Бог даст, увидимся после войны». От˚отца до˚1947~года семья больше никаких вестей не~получала. Вернулся отец раненым, измученным лишь в 1947~году, после войны на японском фронте. 

Однажды мама Миши испекла хлеб. В~доме разносился душистый аромат свежеиспечённого хлеба. Солнце как-то играло, заглядывая в˚окна дома. Стояла безветренная погода, поражала летняя утомительная тишина. Вдруг эту˚тишину нарушил страшный шум. Это появились немцы на˚огромных грузовиках, мотоциклах, наперевес груди автоматы. Эта орава остановилась около дома Миши и˚непрошеные «гости» ворвались в˚дом. Весь хлеб забрали, изловили курей. Один немец плюнул в˚огуречный рассол на˚столе и˚положил на˚руку мамы Миши гранату. Немцы забрали дедушку и˚уехали. Больше его никто не~видел. Мама стояла с˚протянутой рукой, в˚которой находилась граната, не~шевелясь, пока сосед не~снял её. Сосед участвовал в˚Гражданской войне и˚имел опыт обращения с˚оружием.

Дедушку Миша больше никогда не~увидел. В~это время своей детской душой он˚почувствовал, что немцы "--- звери.

Однажды ночью постучали в˚окно. Вся семья проснулась. Последовал крик: «Не~спите, пожар!» Оказывается, партизаны подожгли здание школы, чтобы в˚ней не~разместились немцы. Зарево осветило всю деревню. Жители сбежались и˚старались не~допустить переброски огня на˚соседние дома.

Бабушка считала, что война началась из-за Миши. Июньским солнечным утром 1941~года 4,5~летний мальчик играл в˚своём саду, где произрастали яблони, груши, сливы и˚вишня. Неожиданно на˚бреющем полёте\footnote{Бреющий полёт "--- полёт самолёта на высоте предельно малой (5–50 м),но при этом обеспечивающей безопасность от столкновения с местными предметами и возможность своевременного маневрирования при встрече с препятствиями.} появились три самолёта. Как˚свойственно многим мальчишкам, Миша помахал им˚своей ручонкой. Из˚самолёта раздалась пулемётная очередь, которая скосила несколько ветвей стоящей рядом яблони. Пули ложились рядом, врываясь в˚землю. К~счастью, мальчика они не~задели. Малыш побежал в˚дом. Затем началась бомбёжка. Вся семья спряталась в˚погреб дома, лишь мама Миши с˚маленьким ребёнком на˚руках смотрела в˚окна и˚докладывала обстановку. Бабушка в˚погребе проронила: «Из-за тебя, малыш, началась война…»

\begin{wrapfigure}{O}{.4\textwidth}
\centering
\includegraphics[width=.35\textwidth]{bombardirovka}
\caption{Фашистская бомбардировка}
\label{fig:bombardirovka}
\end{wrapfigure}

Одна бомба упала в˚огороде "--- образовалась воронка размером с˚дом. Вторая попала в˚сарай "--- сарай загорелся вместе с˚животными. От˚него огонь перекинулся на˚дом. В~этот момент раздался неестественный крик мамы Миши: «Горим!» Все выбежали из˚погреба горящего дома, не~имея никаких продуктов питания и˚одежды, кроме той, что была до˚пожара.

Семью поселили в˚маленькой, низкой избушке с˚одним небольшим квадратным окошком и˚дверью, обставленной жердями и˚покрытой осокой (болотной травой), чтобы зимой не~заносило снегом. Избушка до˚войны использовалась для приготовления (запарки) корма для свиней.

Очень хотелось поесть. Утром нашли лебеду и˚сварили на˚воде. Пища оказалась какая\=/то горькая, совершенно невкусная. Поэтому бабушка решила идти с˚Мишей в˚соседнюю деревню Богутичи просить милостыню. Один мужчина дал для Миши рваные сапоги, которые оказались на˚2~размера больше его ноги, разжились парой буханок хлеба и˚небольшим кусочком сала. По˚дороге обратно Миша сказал: «Бабушка, я˚буду голодать, умирать, но˚больше не~пойду просить милостыню». Для˚поддержания своего четырёхлетнего ребёнка женщина, проживающая  в˚шалаше, кормила его грудью молоком, а˚сама оставалась голодной.

Однажды ранним утром бабушка пошла в˚лес за˚дровами, чтобы протопить печку. Он находился недалеко от˚избушки. Немцы задержали её в˚лесу, приняли за˚партизанку. Привели под дулом автомата к˚избушке. В~избушке выстроили всех в˚ряд (маму с˚грудным ребёнком на˚руках, бабушку, сестрёнку и˚Мишу), а˚за окошком установили пулемёт, ствол которого через окошко направили в˚грудь бабушки. Таким образом они˚простояли сутки без питания и˚воды, а˚за окном сменялись лишь часовые у˚пулемёта. Бабушка молилась: «Отче наш…» "--- и˚все повторяли за˚ней, произнося слова этой молитвы. Через˚сутки, неожиданно для всех, немцы сняли пулемёт, заколотили дверь гвоздями и˚подожгли навес над дверью из˚жердей и˚осоки, а˚сами уехали. За˚действиями немцев у˚избушки наблюдала из˚окна своего дома соседка. После˚отъезда немцев она прибежала, крикнула: «Горите!» "--- и˚начала вытаскивать нас через окошко. Детишек ей˚удалось вытащить быстро, а с˚женщинами пришлось повозиться. В~это время уже охватил огонь и˚избушку.

Дальнейшая жизнь семьи проходила в˚шалаше из˚стоящих жердей и˚покрытого осокой, среди леса на˚болоте, удалённом от˚деревни на˚7\,км. Для˚обогрева зимой внутри шалаша круглосуточно горел костёр. Для˚пропитания использовали жёлуди и˚гнилую картошку, в˚роли соли "--- удобрения, которые добывали из-под снега. Жёлуди размалывали на˚самодельных жерновах и˚пекли лепёшки, хлеб. Это очень тяжёлая пища, от˚которой по˚весне сводило сухожилия, было очень больно. Спали на˚земле. В~роли подстилки служил лапник из˚сосны, солома. Укрывались различными тряпками.


\begin{wrapfigure}{O}{.4\textwidth}
\centering
\includegraphics[width=.35\textwidth]{burningVillage}
\caption[Горящая деревня. 1941\--1944~гг. БГАКФФД]{Горящая деревня. 1941\--1944~гг. БГАКФФД\footnotemark}
\label{fig:burningVillage}
\end{wrapfigure}
\footnotetext{Источник заимствования "--- Белорусский государственный архив кинофотофонодокументов (БГАКФФД). URL: \url{https://archives.gov.by/.} }

Немцы из˚деревни Вишеньки поехали в˚соседнюю деревню Рудня, согнали всех жителей в˚сарай, сбросили с˚чердака солому, заколотили двери, расставили пулемёты вокруг и˚подожгли сарай.При˚горении сарая люди взломали дверь и˚начали убегать. По˚ним открыли ураганный огонь из˚пулемётов. Перебили всех. В~живых осталась одна женщина по˚имени Разаля. Ей удалось бежать в˚сторону дыма, по˚ржи. На˚руках был маленький ребёнок, которого настигла пуля, а˚она получила несколько ран. Так˚она бежала с˚мёртвым ребёнком на˚руках, имея ранение, не~обращая внимание на˚людей в˚шалашах (куренях). Им удалось её˚схватить и˚постепенно выходить. 

Все дома и˚постройки этой деревни немцы сожгли. Уже будучи взрослым, Миша во˚время очередного своего отпуска побывал на˚месте этой деревни. Рядом˚стоит прекрасная берёзовая роща, в˚мае раздавалось пение соловья и˚щебетание птиц, а˚посреди бывшей деревни стоял большой деревянный крест. Люди так и˚не~поселились на˚этом месте. Угнетающее настроение овладело Мишей: прекрасной природе вокруг сопутствовала мёртвая земля с˚жертвами живших людей. 

\begin{wrapfigure}{O}{.4\textwidth}
\centering
\includegraphics[width=.35\textwidth]{tankNemec}
\caption{Немецкие экранированные танки Pz.Kpfw. III в˚советском селе перед началом операции «Цитадель»}
\label{fig:tankNemec}
\end{wrapfigure}

Однажды зимой, когда взрослые ушли за˚добыванием какой\-/нибудь пищи, а в˚шалаше остались два малыша Миша и˚Есип у˚костра, ствол пушки пробил шалаш. Оказывается, пожаловал по˚замёрзлому болоту немецкий танк. Миша выбежал из˚шалаша и˚босиком по˚снегу, в˚одной рубашке, побежал. Немец не~стрелял, а˚забрался на˚танк, фотографировал и˚заливался громким смехом. Со˚слов людей, схватили бегущего Мишу за˚2\,км от˚своего шалаша. Они растерли Мишу, погрузили ноги в˚холодную воду, а˚затем уложили у˚костра в˚шалаше, накрыли разными тряпками. Миша уснул. Проспал сутки, а˚затем очнулся. Благодаря заботам и˚стараниям неизвестных людей Миша остался жив, даже не~отморозил ноги. За˚это он очень благодарен им.

Следует заметить, что отношения людей тогда были исключительными. Неизвестные делились друг с˚другом всем, отдавая последнее, оказывали всевозможную помощь, что не~всегда увидишь сейчас. В~характере белорусских людей есть много чудесных и˚привлекательных черт. Сейчас многих одолело стяжательство, стремление к˚наживе, деньгам.

Как-то под утро раздалась сплошная стрельба. Немцы стреляли по˚партизанам, а˚те пытались выбить из˚деревни немцев. Миша схватил какое\=/то одеяло подмышку и˚с˚мамой побежал к˚лесу. Сосновый лес был недалеко, стоял понурой стеной. Вдруг Мише резко ожгло левый бок, при этом проронил: «Мама, меня убили». Прибежав в˚лес, мама осмотрела его и˚сказала, что˚пуля пробила одеяло между рукой и˚левой стороной тела, ожгла левую сторону. След от˚этого ожога оставался на˚теле очень продолжительное время. Подобная стрельба между партизанами и˚немцами повторялась очень часто.

В˚зимний период жизнь людей, проживающих в˚шалашах среди болота и˚леса, была невыносимым, тяжёлым испытанием. С˚наступлением весны она становилась немного легче: теплело, с˚питанием проще "--- переходили на˚щавель, лебеду. 

Стрессовое состояние и голод ослабили здоровье Миши. Он заболел, и˚родители решили отправить его в˚соседнюю деревню Богутичи, к˚дальним родственникам. Дом их˚стоял на˚краю деревни, он состоял из˚двух комнат, кладовки. В~600\,м от дома проходила железная дорога. В~углу кладовой лежали какие\=/то снаряды, заходили люди в˚белых халатах. Услышал их˚разговор: «Взяли языка». В~день Мишиного прихода снаряды забрали. 

Резко поднялась температура, уложили в˚постель. Миша в˚это время поднял что-то сверкающее, в˚виде «мундштука». Во˚время сильной температуры он˚перегрыз «мундштук». Свободную часть выбросил, а˚оставшуюся бросил в˚огонь (посмотреть её˚накаливание). Для˚обогрева горел камин, недалеко от˚него сидел ребёнок. Раздался сильный взрыв, ребёнка отбросило на˚кровать, разлетелись горящие угли, Мише осмолило брови. Прибежали взрослые и˚начали тушить загоревшееся. Миша также бегал и˚собирал угли. Из˚чана с˚водой смачивал лицо, оно очень «горело». Оказывается, перегрыз он˚взрыватель к˚какой\=/то мине. Наконец всё улеглось и˚сели завтракать. Миша думал, что его станут очень ругать, ожидал этого, но˚никто не~проронил ни˚одного слова. Почему\=/то ему стало очень обидно. На~следующий день Миша потихоньку сбежал из˚дома и˚через лес добрался до˚своих шалашей. Болезнь постепенно отступила. 

\begin{wrapfigure}{O}{.4\textwidth}
\centering
\includegraphics[width=.35\textwidth]{sculpturBelarusPartizan}
\caption{Скульптура «Белорусские партизаны». Экспонируется на˚станции метро «Белорусская», Москва}
\label{fig:sculpturBelarusPartizan}
\end{wrapfigure}

Молодёжь немцы сгоняли и˚увозили в˚Германию, многих убивали, людей сжигали целыми деревнями. Немцы повсеместно занимались грабежом, забирали и˚увозили всё.

В~начале войны, чтобы немцы не~угнали скот, детишкам с˚коровами пришлось ночевать в˚лесу. Вели они себя совершенно по\=/взрослому. Кормили скот, не~боясь, ночевали в˚лесу. Как-то решили снять с˚привязи коров и˚попасти. Но˚коровы все сорвались и˚бегом, подняв хвосты кверху, убежали из˚леса в˚деревню. Видимо, соскучились по˚хозяевам, тёплым сараям и˚нормальному корму. Детишки не~смогли перехватить их. Но˚всё это стадо захватили так называемые местные партизаны и˚угнали скот. Под˚видом местных партизан, видимо, орудовали мародёры. По˚лесам бродили также бандеровцы, убивали всех подряд: взрослых и˚детей. Следовательно, и в˚лесу находиться было небезопасно. Немцы также часто прочёсывали леса в˚поисках партизан. Поэтому жители деревни были напуганы действиями фашистов и˚подобных «партизан» (мародёров). 

Один из˚таких «партизан» приказал бабушке Миши, чтобы она сняла сапоги. При отказе это проделать, он˚бросил её˚на землю, огрел плетью и˚снял сапоги. Сапоги оказались старыми, рваными и˚он бросил их˚в˚бабушку. Таким образом орудовали так называемые «партизаны».

Настоящие партизаны подрывали железнодорожные пути, совершали налёты на˚немцев, помогали Красной армии в˚тылу фашистов. Поэтому жители деревни сочувствовали и˚всяким образом помогали настоящим партизанам, информировали о˚наличии немцев. Однако были и˚предатели, записывались в˚полицаи, сочувствовали немцам. Таких было единицы. 

\begin{wrapfigure}{O}{.4\textwidth}
\centering
\includegraphics[width=.35\textwidth]{vstrechaOsvoboditelei}
\caption{Жители Гомельщины приветствуют советских воинов\-/освободителей. Ноябрь 1943~года. БГАКФФД}
\label{fig:vstrechaOsvoboditelei}
\end{wrapfigure}

Мише запомнилось наступление частей Красной Армии. Бойцы и˚техника двигались по˚лесной дороге, а˚Миша с˚детишками полдня стояли и˚махали им˚ручками, в˚мыслях шагали с˚ними, чтобы быстрей освободить Родину от˚зверств фашистов. Один солдат дал Мише полбуханки хлеба. Впервые за˚время войны испробовали настоящего хлеба. Это действительно был настоящий праздник для˚Миши и˚всей его семьи. Зимой употребляли в˚основном жёлуди, весной и˚летом спасали щавель и˚лебеда. Вся гигиена заключалась в˚умывании холодной водой без мыла и˚прокаливании одежды (спасение от˚вшей) над костром. Кстати, за˚всё своё детство Миша ни˚разу не~испытал вкус сладкого (конфет).

После освобождения от˚немцев жили в˚землянке, если это возможно назвать жильём. Земля была усеяна различными минами, снарядами. Некоторые люди подрывались на˚минах, находили трупы убитых людей. Охватили болезни: свирепствовали тиф, туберкулёз, малярия, от˚которых многие умирали. Медикаментов никаких не~было. 

Впоследствии военные вместе с˚пленными соорудили Мише с˚семьёй небольшой домишко. В~деревне многие дома остались целыми, довоенной постройки. 

Отдельные партизаны указывали, что разбомбили Мишин дом немцы по˚специальной наводке, из-за того, что будто там˚находились партизаны. Даже называлось конкретное лицо. 

Победу 1945~года встретили жители деревни очень радостно, преобразились даже дети, надеясь на˚счастливую жизнь и˚радостное детство.