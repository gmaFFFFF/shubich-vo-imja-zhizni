\section{Биография М.П. Шубича}

%-- Далее идет составной рисунок
\begin{figure}[H]
\includegraphics[width=\textwidth]{infographic/export/TimeLineSplit_1}
\caption{Инфографика «Биография М.\,П.~Шубича». Начало}
\label{fig:TimeLineSplit}
\end{figure}

\begin{figure}[t]
\ContinuedFloat % продолжение рисунка
\includegraphics[width=\textwidth]{infographic/export/TimeLineSplit_2}
\caption{Инфографика «Биография М.\,П.~Шубича». Продолжение}
% \label здесь не нужна!!! 
\end{figure}

\begin{figure}[t]
\ContinuedFloat % продолжение рисунка
\includegraphics[width=\textwidth]{infographic/export/TimeLineSplit_3}
\caption{Инфографика «Биография М.\,П.~Шубича». Продолжение}
% \label здесь не нужна!!! 
\end{figure}

\begin{figure}[t]
\ContinuedFloat % продолжение рисунка
\includegraphics[width=\textwidth]{infographic/export/TimeLineSplit_4}
\caption{Инфографика «Биография М.\,П.~Шубича». Завершение}
% \label здесь не нужна!!! 
\end{figure}
%-- Составной рисунок

% Вывести все плавающие объекты без начала новой страницы (в этом отличии от \clearpage), если это возможно.
\FloatBarrier

\begin{figure}[h]
\includegraphics[height=\textwidth, width=\textheight, keepaspectratio, angle=90, center]{mapbiograf1}
\caption{Биография М.\,П.~Шубича «на карте»}
\label{fig:mapbiograf1}
\end{figure}

\begin{figure}[h]
\includegraphics[width=\textwidth]{mapbiograf2}
\caption{Биография М.\,П.~Шубича «на карте». Детство и юность}
\label{fig:mapbiograf2}
\end{figure}

%\begin{flushleft}
\begin{table}[t]
{\small
	\begin{tabularx}{\textwidth}{cX}
		01.10.1936	&	Родился в˚деревне Вишеньки Ельского района Полесской (Гомельской) области			\\
		1945\==1949		&	Учился в˚начальной школе д.~Вишеньки Ельского района								\\
		1949\==1951		&	Учился в˚семилетней школе д.~Богутичи												\\
		1951\==1954		&	Учился в˚средней школе г/п Ельск													\\
		1954\==1955		&	Диспетчер\-/кладовщик геофизических партии 1/54, 3/55 Главнефтегеофизика				\\
		1955\==1958		&	Служба в˚рядах Советской Армии (старший вычислитель)								\\
		1958\==1959		&	Рабочий бригадир строительно\-/монтажного управления №~14 Мособлспецстрой №~1		\\
		1959\==1964		&	Студент Московского института инженеров землеустройства							\\
		1964\==1966		&	Инженер\-/землеустроитель, начальник отряда Рязанской землеустроительной экспедиции	\\
		1966\==1969		&	Аспирант кафедры землеустроительного проектирования МИИЗ							\\
		1970\==1972		&	Ассистент кафедры землеустроительного проектирования МИИЗ							\\
		1972\==1992		&	Доцент кафедры землеустроительного проектирования МИИЗ								\\
		1975\==1977		&	Учёба на˚экономическом факультете Марксизма Ленинизма при Московском горкоме КПСС	\\
		1975\==1980		&	Председатель объединённого профсоюзного комитета МИИЗ								\\
		1978\==1980		&	Член горкома профсоюза работников сельского хозяйства г.~Москва						\\
		1980\==1985		&	Заместитель заведующего кафедрой													\\
		1982\==2000		&	Декан заочного факультета МИИЗ (ГУЗ)												\\
		1986\==1989		&	Заместитель заведующего кафедра по˚научной работе									\\
		1992\==2017		&	Профессор кафедры землеустройства ГУЗ												\\
		1995				&	Присвоено учёное звание профессора по˚кафедре землеустройства ГУЗ					\\
		23.08.1999	&	Присвоено почётное звание «Заслуженный землеустроитель Российской Федерации»		\\
		2000\==2003		&	Ответственный секретарь приёмной комиссии ГУЗ										\\
		2004\==2006		&	0,5~ставки профессора кафедры мелиорации и˚геодезии МТСХА							\\
		15.10.2004	&	Присвоено почётное звание «Академик Русской академии»									\\
		23.10.2017	&	Вышел в отставку с должности профессора кафедры земпроектирования ГУЗ									\\
		2018				&	Написание автобиографической повести									\\
	\end{tabularx}
}
\caption{Основные даты жизни и деятельности М.\,П.~Шубича}
\label{tab:biograf}  
\end{table}
%\end{flushleft}

% Вывести все плавающие объекты без начала новой страницы (в этом отличии от \clearpage), если это возможно.
\FloatBarrier

Вся сознательная активная жизнь Михаила Павловича связана с˚МИИЗом (ГУЗом): 1959\==1964~годы "--- студент; 1966\--1969~годы "--- аспирант; 1970\--1972~годы "--- ассистент кафедры земпроектирования; 1972\--1992~годы "--- доцент; 1992\--23~октября 2017~года "--- профессор кафедры землеустройства ГУЗ.

В~1975~году Михаил Павлович поступает на˚экономический факультет Марксизма\-/Ленинизма при Московском горкоме КПСС, который окончил с˚отличием в˚1977~году и˚получил высшее политическое образование. Следует заметить, что после окончания экономического факультета Марксизма\-/Ленинизма, ему было предложено работать у˚них, но он˚остался верен МИИЗу.

В~1975~году Михаил Павлович решением Высшей аттестационной комиссии был утверждён в˚учёном звании доцента кафедры землеустроительного проектирования МИИЗ. Отмечая профессиональную квалификацию той же˚комиссией в˚1995~году присвоено учёное звание профессора. В~2004~году присвоено почётное звание академика Русской Академии.

Михаил Павлович принимал участие в˚собрании избирателей Бауманского избирательного округа г.~Москвы, посвящённое встрече с˚кандидатом в˚депутаты Верховного Совета СССР "--- Генеральным секретарём ЦК КПСС, Председателем Президиума Верховного Совета СССР Леонидом Ильичом Брежневым 2~мая 1979~года в˚Кремлёвском дворце съездов. Сидел в˚третьем ряду, напротив Л.\,И.~Брежнева. После˚собрания состоялся большой концерт. 

С~1980 по˚1985~год являлся заместителем заведующего кафедрой землеустроительного проектирования, а с˚1986 по˚1989~годы заместителем заведующего кафедрой по˚научной работе. Во˚время работы заместителем заведующего кафедрой В.\,П.~Троицкий "--- заведующий кафедрой "--- фактически переложил всю работу на˚заместителя.

По˚линии Министерства образования в˚1984~году прошёл переподготовку «по˚проблемам высшего образования», а в˚1990~году "--- «технология аттестации и˚аккредитации вузов» и˚получил соответствующее удостоверение; с˚27~октября по˚4~ноября 1990~года участвовал в˚учебном семинаре\-/встрече руководителей российских предприятий, учёных с˚представителями делового мира Запада (Италии) и˚получил диплом Туринского Центра малого и˚среднего бизнеса, а˚также проходил стажировку в˚проектных организациях Гипрозема.

Указом Президента Российской Федерации Михаилу Павловичу было присвоено в˚1999~году почётное звание «Заслуженный землеустроитель Российской Федерации». 

С~2004 по˚2007~годы Михаил Павлович работал дополнительно на˚0,5~ставки профессора кафедры мелиорации и˚геодезии Российского государственного аграрного университета "--- Московской сельскохозяйственной академии имени К.\,А.~Тимирязева.

С~5~ноября по˚14~ноября 2014~года прошёл обучение по˚теме: «Ведение государственного кадастра недвижимости в˚электронном виде»; с˚23~марта по˚1~апреля 2015~года обучался по˚дополнительной профессиональной программе «Информационные технологии в˚области землеустройства и˚кадастров»; с˚25 по˚27~января 2017~года "--- программе: «Актуализации федеральных государственных образовательных стандартов высшего образования и˚совершенствование образовательной деятельности по˚направлению подготовки „землеустройство и˚кадастры“». По˚всем этим направлениям получены удостоверения о˚повышении квалификации. 

Кроме общественной работы всё время приходилось вести большую организационно\-/методического работу: был членом учёного совета, членом учебно\-/методической комиссии вуза, председателем комиссии по˚контролю за˚деятельностью администрации, председателем комиссии по˚борьбе с˚пьянством и˚алкоголизмом в˚институте и˚членом этой комиссии Бауманского района г.~Москвы, членом комитета комсомола, эвакуационной комиссии и˚другое. На˚комиссии по˚борьбе с˚пьянством и˚алкоголизмом приходилось разбирать отдельных преподавателей и˚студентов. Член этой комиссии профессор А.\,В.~Маслов всегда требовал сурового наказания.
