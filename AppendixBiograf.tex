\section{Биография М.П. Шубича}

%-- Далее идет составной рисунок
\begin{figure}[H]
\includegraphics[width=\textwidth]{infographic/export/TimeLineSplit_1}
\caption{Инфографика «Биография М.\,П.~Шубича». Начало}
\label{fig:TimeLineSplit}
\end{figure}

\begin{figure}[t]
\ContinuedFloat % продолжение рисунка
\includegraphics[width=\textwidth]{infographic/export/TimeLineSplit_2}
\caption{Инфографика «Биография М.\,П.~Шубича». Продолжение}
% \label здесь не нужна!!! 
\end{figure}

\begin{figure}[t]
\ContinuedFloat % продолжение рисунка
\includegraphics[width=\textwidth]{infographic/export/TimeLineSplit_3}
\caption{Инфографика «Биография М.\,П.~Шубича». Продолжение}
% \label здесь не нужна!!! 
\end{figure}

\begin{figure}[t]
\ContinuedFloat % продолжение рисунка
\includegraphics[width=\textwidth]{infographic/export/TimeLineSplit_4}
\caption{Инфографика «Биография М.\,П.~Шубича». Завершение}
% \label здесь не нужна!!! 
\end{figure}
%-- Составной рисунок

% Вывести все плавающие объекты без начала новой страницы (в этом отличии от \clearpage), если это возможно.
\FloatBarrier

\begin{figure}[H]
\includegraphics[width=\textwidth]{infographic/export/TimeLineScale}
\caption{Инфографика «Биография М.\,П.~Шубича»}
\label{fig:TimeLineScale}
\end{figure}


\todo[Выбор]{Выбрать вариант инфографики 1 л или 4 л}

\begin{figure}[h]
\includegraphics[height=\textwidth, width=\textheight, keepaspectratio, angle=90, center]{mapbiograf1}
\caption{Биография М.\,П.~Шубича «на карте»}
\label{fig:mapbiograf1}
\end{figure}

\begin{figure}[h]
\includegraphics[width=\textwidth]{mapbiograf2}
\caption{Биография М.\,П.~Шубича «на карте». Детство и юность}
\label{fig:mapbiograf2}
\end{figure}

%\begin{flushleft}
\begin{table}[t]
{\small
	\begin{tabularx}{\textwidth}{cX}
		01.10.1936	&	Родился в˚деревне Вишеньки Ельского района Полесской (Гомельской) области			\\
		1945\==1949		&	Учился в˚начальной школе д.~Вишеньки Ельского района								\\
		1949\==1951		&	Учился в˚семилетней школе д.~Богутичи												\\
		1951\==1954		&	Учился в˚средней школе г/п Ельск													\\
		1954\==1955		&	Диспетчер\-/кладовщик геофизических партии 1/54, 3/55 Главнефтегеофизика				\\
		1955\==1958		&	Служба в˚рядах Советской Армии (старший вычислитель)								\\
		1958\==1959		&	Рабочий бригадир строительно\-/монтажного управления №~14 Мособлспецстрой №~1		\\
		1959\==1964		&	Студент Московского института инженеров землеустройства							\\
		1964\==1966		&	Инженер\-/землеустроитель, начальник отряда Рязанской землеустроительной экспедиции	\\
		1966\==1969		&	Аспирант кафедры землеустроительного проектирования МИИЗ							\\
		1970\==1972		&	Ассистент кафедры землеустроительного проектирования МИИЗ							\\
		1972\==1992		&	Доцент кафедры землеустроительного проектирования МИИЗ								\\
		1975\==1977		&	Учёба на˚экономическом факультете Марксизма Ленинизма при Московском горкоме КПСС	\\
		1975\==1980		&	Председатель объединённого профсоюзного комитета МИИЗ								\\
		1978\==1980		&	Член горкома профсоюза работников сельского хозяйства г.~Москва						\\
		1980\==1985		&	Заместитель заведующего кафедрой													\\
		1982\==2000		&	Декан заочного факультета МИИЗ (ГУЗ)												\\
		1986\==1989		&	Заместитель заведующего кафедра по˚научной работе									\\
		1992\==2017		&	Профессор кафедры землеустройства ГУЗ												\\
		1995				&	Присвоено учёное звание профессора по˚кафедре землеустройства ГУЗ					\\
		23.08.1999	&	Присвоено почётное звание «Заслуженный землеустроитель Российской Федерации»		\\
		2000\==2003		&	Ответственный секретарь приёмной комиссии ГУЗ										\\
		2004\==2006		&	0,5~ставки профессора кафедры мелиорации и˚геодезии МТСХА							\\
		15.10.2004	&	Присвоено почётное звание «Академик Русской академии»									\\
		23.10.2017	&	Вышел в отставку с должности профессора кафедры земпроектирования ГУЗ									\\
		2018				&	Написание автобиографической повести									\\
	\end{tabularx}
}
\caption{Основные даты жизни и деятельности М.\,П.~Шубича}
\label{tab:biograf}  
\end{table}
%\end{flushleft}

% Вывести все плавающие объекты без начала новой страницы (в этом отличии от \clearpage), если это возможно.
\FloatBarrier