Ответы М. П.Шубича на вопросы выпускников ГУЗа.

I.
1.1 Данную книгу написал, чтобы молодым людям, проживающим в большинстве своём в «тепличных» условиях показать, что на жизненном пути могут быть трудности, неприятности, преодолеть которые возможно лишь выработав силу воли, имея стремления и поставленные цели.

1.2 Самые большие достижения в том, что окружающие меня люди понимали меня, а я - их. В большинстве своём поддерживали меня.

1.3 Не совсем верующий человек, но не отрицаю религию, которая может сплатить массы людей.

1.4 Родина - место рождения, взросления и становления человека; обретения необходимых человеческих черт, любви к людям и окружающему миру.

1.5 Мои любимые книги, фильмы, песни о честных тружениках (людях), делающих добро и приносящих радость другим людям.

1.6 Считать в уме хорошо обучала советская школа. Достигается это прежде всего тренировками.

1.7 С самого раннего детства я трудился, рассчитывал только на себя. В этом весь смысл. Не имел вредных привычек.

1.8 Я родился на земле, вырос на ней. Люблю землю. Поскольку землеустройство связано с землей, тоя полюбил его.

II.
2.1 В русских людях имеется национальная идея - это прежде всего мир, благополучие каждого, процветание страны. Но ... правительство её не выработало.

2.2 Думается, случись сейчас война, олигархи, зажиточные люди и их сынки сбежали бы за границу. Простые труженики рьяно защищали бы своё отечество.

2.3 Крым был наш и будет в будущем, чтобы не хотелось людям, не любящим Россию.

2.4 Распад СССР заключается прежде всего в предательстве Горбачёва, Ельцина, Кравчука и Шушкевича и с ними подобных и действиях не дружеских нам государств. Восстановление СССР в прежнем виде невозможно. Чтобы подобного не произошло, вся политика нашего государства должна быть направлена на защиту и заботу о простых людях (народа).

2.5 Коммунистическую партию, так как её программа направлена на улучшение жизни простых людей (народа).

2.6 В Государственной Думе много карьеристов, людей, стремящихся к богатству, богачей, которые далеки от народа. Поэтому её работа, принимаемые законы, не всегда направлены на защиту интересов народа. Мне представляется, что спортсменам, артистам и подобным им там делать нечего.

2.7 Избирательная система на наш взгляд еще далека от объективности. По прежнему действует административный ресурс, ресурс партии «Единая Россия». Чтобы как-то исправить положение дел, необходимо прежде всего каким-то образом избавиться от этих ресурсов, усилить контроль со стороны ЦИК, подбирать в избирательные комиссии объективных людей.

2.8 Перед справедливыми законами все граждане должны быть законопослушными, иначе в государстве будет хаос.

2.9.1 Коррупция не изжита. Она существует в вузах и во всех структурах( отраслях). При желании государства ее возможно победить, но пока нет такого желания. В целом она наносит большой вред, развращает граждан, унижает обездоленных, способствует оценке людей не по их достоинствам, знаниям. Более значимыми проблемами являются воспитание молодежи, борьба с наркоманией, проституцией, алкоголизмом, коррупцией.

2.9.2 Для борьбы с коррупцией необходимо улучшить воспитательную работу, усилить контроль и повысить ответственность, более правильно подбирать кадры.
 
2.9.3 К трудовым династиям отношусь положительно, но здесь не должны ущемляться цели и стремления молодых людей.

2.9.4 Взяточничество и откаты - явления вредные, их необходимо всячески устранять.

III.
3.1 Для России капитализм неприемлем, тем более капиталистический рынок. У России в настоящее время не рынок, а базар. Считаю, что наиболее приемлема социалистическая система, создающая условия для раскрытия  способностей каждого человека.

3.2 Для поднятия экономики, во-первых, необходимо убрать либералов с различных руководящих постов в экономике, которые стараются смотреть на Запад и жить под их диктовку. Во-вторых, не дорабатывает в этом плане и Правительство (кстати, там засело много либералов).

IV.
4.1 Содержанием педагогической деятельности является обучение, воспитатние, образование, развитие обучающихся. Поэтому высококвалифицированный педагог  должен владеть и вести на высоком уровне методическую, научно-исследовательскую и воспитательную работу. Причём он должен уметь адаптироваться к постоянно меняющимся условиям, иметь глубокое знание предмета обучения. Поэтому считайте, что я одновременно педагог, землеустроитель и учёный.

4.2 Главный совет - это постигать азы науки, специальности. При этом владеть (выработать) силу воли, быть целеустремлённым, не тратить время впустую.

4.3 Считаю ЕГЭ в учебном процессе не приемлемым. При ответах на поставленные вопросы экзаменующий как бы  занимается гаданием, а не раскрывает свои знания. Считаю, вместо ЕГЭ следует вводить экзамены, чтобы человек мог изучать более обстоятельно предмет, а не заниматься натаскиванием.

4.4 Чтобы построить карьеру, быть хорошим руководителем, считаю необходимым быть объективным, справедливым, постоянно работать над собой (повышать свои знания).

4.5 Улучшение учебного процесса, улучшение условий проживания студентов и работы преподавателей вуза, борьба с негативными процессами, повышение эффективности научных исследований.

4.6 Преподаватель вуза должен быть объективным при оценке знаний студента и не унижать достоинства человека. В необходимых случаях оказывать студенту помощь.

4.7 Нет.

4.8.1 Уровень образования в Советском Союзе был более высокий, чем сейчас в России. Это отмечали даже приезжавшие делегации из США. Много ненужного нам навязали решением Болонской конференции, которое подписала и Россия. С начала 90-х годов (годов хаоса) культивировалась продажа дипломов. Даже отдельные врачи были с купленным дипломом. Поэтому такой специалист с подобным дипломом для общества ничего не представляет. Чтобы диплом был девальвирован, необходимо улучшить уровень преподавания. Считаю, что для этого необходимо улучшить подготовку в школах, более тщательный проводить отбор преподавателей, отменить ЕГЭ, улучшить отбор студентов, отказаться от бакалавриата.

4.8.2 Нужно. Ведь человек добросовестно обучавшийся в вузе более приспособлен к жизни, может быстрее адаптироваться в тех или других условиях, к новым введениям.

4.8.3 При поступлении в вуз(вступительных экзаменах) ориентироваться следует не на выполнение плана любыми способами, доведенного сверху, а отбирать более успешно сдавших вступительные экзамены абитуриентов. Финансирование вуза не должно быть от количества набранных студентов.

4.8.4 Об этом я уже высказался. Считаю, что для России более эффективным было заменить бакалавриат подготовкой специалистов в течение пяти лет.

4.8.5 Должности должны раздаваться не по кумовству(знакомству), а по знаниям. Также при выборности в любые властные органы.

4.9.1 В то время в деревне получить справку, паспорт молодому человеку было невозможно, а тем более как-то ориентироваться, осуществить задуманное. Даже не было материалов, по которым можно было бы ориентироваться. Поэтому всё приходилось делать при представлении подобной возможности.

4.9.2 Студента должна интересовать определённая специальность. Возможностей ознакомиться с ней теперь имеется предостаточно. Поскольку в университете имеется несколько специальностей, то абитуриент не может поступать просто в университет, так как он не сможет овладеть несколькими специальностями за время обучения.

4.9.3 Студент после окончания первого курса может перейти в другой вуз, на любую специальность, если в данном вузе оказалось, что «попал не туда». Для этого ему необходимо досдать предметы, которые предусмотрены по программе в новом вузе на данной специальности.

V.
5.1 В МИИГАиКе ввели ряд специальностей подобных ГУЗу, хотя соответствующих преподавателей не хватало. Они всегда ратовали за объединение с ГУЗом, чтобы овладеть территорией. Считаю, что любое объединение не улучшает учебный процесс, причём есть отличия в учебных программах. Следует заметить также, что ГУЗготовит специалистов прежде всего для сельского хозяйства.

5.2 Вы правы, ГУЗ должен прежде всего производить  набор абитуриентов из сельской местности. Но любой человек (в том числе из города) имеет право на образование. Поэтому ущемлять его права никому не позволено. Хрущёв собирался переселить МИИЗ в сельскую местность. Для этого отсутствует соответствующая база (её следует создавать заново), да с преподавателями будет проблема.

5.3 Выпускник должен не только владеть информацией, соответствующей техникой, но уметь анализировать, обобщать, делать выводы с полученной информации. Думается, что в настоящее время выпускник ГУЗа в основном владеет вычислительной техникой.

5.4 Во-первых, лесоустройство имеет свои особенности, во-вторых, не следует отнимать «хлеб» у других.

5.5 При советской системе для преподавания в вузе очень тщательно производился отбор преподавателей. Не секрет, что в ГУЗе и в частности на кафедре землеустройства работает ряд преподавателей, которых бы раньше «на пушечный выстрел» не допустили. но что поделаешь, «на безрыбье и рак рыба». Считаю, подобное явление недопустимым. Об этом я говорил и ректору ГУЗа проф. С.Н.Волкову.

5.6 Нормальная производственная практика для студента необходима, а тем более для преподавателя необходима. Многие наши преподаватели ГУЗа выросли, воспитались, получили образование в «тепличных» условиях. Это является ненормальным. Считаю, что преподаватель вуза должен не только получить соответствующее образование, но пойти определённую «жизненную» школу, производственную практику по специальности. Такая производственная практика для обучающихся студентов крайне необходима. Она помогает лучше усвоить материал, вести более эффективно научные исследования. В настоящее время она фактически отсутствует.

5.7 Самый «сильный» землеустроитель на кафедре тот, кто в совершенстве владеет своей специальностью, и кто знает современное производство, ведёт постоянно научно-исследовательскую работу.

5.8 Вы совершенно правы, что многие авторы в своих статьях, книгах по землеустройству вставляют автором или под редакцией ректора ГУЗ С.Н. Волкова, хотя он в этом не участвовал. Считаю, это неправильно. Я подобного не делал. Совмещать по положению ректора и заведование кафедрой не положено. Но это у нас в вузе на лицо.

5.9 Считаю, что развивать в нашем вузе бесплатную систему онлайн-курсов не следует.

5.10 Человек думает, что подобное изучение той или иной дисциплины, явления не пригодится. Это ошибочное суждение. Жизнь многогранна, поэтому встречается иногда то, что мы не ожидали, не изучали. Поэтому многосторонние знания всегда пригодятся на жизненном пути. Они помогают человеку адаптироваться в новых, неожиданных условиях.

5.11 ГУЗ в основном соответствует статусу университета, поэтому ему присвоен этот статус. Да, специальность землеустройство - широкая специальность, требующая знаний экологии, земельного кадастра, земельного законодательства и т.д. Землеустройство необходимо почти во всех отраслях, даже для Министерства Обороны. Как можно строить земельные отношения без землеустройства? Беда в том, что государство этого не поняло, не оценило и фактически уничтожило землеустройство.

Не все граждане России могут получить образование очно. Поэтому для получения образования без отрыва от производства созданы заочные факультеты. Практика показала, когда на заочный факультет производился набор только по специальности или родственной специальности, заочники показывали более высокий уровень знаний по сравнению с очниками.

Ошибочное существует суждение, что если в школе ученик обучался на тройки, то в вузе он не может обучаться на отлично. Ведь он повзрослел, появилась мотивация получить образование, стал относиться к своим делам более серьезно.

Красный диплом выдают студентам, получившим за время обучения в вузе, 75\% отличных оценок, не имеющих ни одной тройки. Поэтому не все его могут получить.

5.12.1 От уровня зарплаты зависит стремление работать, повышать свои знания.

5.12.2 Государство еще не обеспечивает достойную заработную плату преподавателям. Заработная плата преподавателя должна быть по положению средней зарплате в экономике России. Подобное явление касается в основном преподавателей всех вузов страны. В отдельных вузах заработная плата несколько выше, чем преподавателей в ГУЗе.

5.12.3 Уровень преподавания, конечно, зависит от заработной платы преподавателя. Более высокую зарплату получают более высококвалифицированные преподаватели. Если была бы высокая заработная плата, то в вузе не было бы «бездарностей».

5.12.4 На этот вопрос ответил в пункте 5.12.2.

5.12.5 Если молодому выпускнику ГУЗа нравится преподавательская деятельность и у него имеются склонности и возможности подготовить диссертацию, то он должен поступить на должность преподавателя. От этого зависит и карьерный рост.

5.13 В настоящее время одним из показателей уровня работы преподавателя является количество публикаций на иностранных языках и ссылок на публикации. Восприятие работы (статьи) во многом зависит от широты и известности тематики, и от того, на каком языке она опубликована. Более узкая тематика воспринимается хуже и может не быть на неё ссылок. Основные иностранные публикации осуществляются на английском языке. Если вы изучали немецкий язык и им владеете, не беда. Главное заключается в том, чтобы хорошо владеть хотя бы одним иностранным языком. Конечно неплохо еще знать и английский язык.

5.14 Молодым специалистам следует поступать в аспирантуру для престижа, достижения профессионального уровня, получения опыта педагогической деятельности в вузе и, естественно, карьеры. Аспирантура при вузах является составной частью единой системы непрерывного образования и степенью послевузовского образования.

5.15 Внедрение результатов исследований является наиболее сложным процессом. В России с внедрением результатов исследований обстоит дело недостаточно хорошо. При допуске к защите требуется представление справки о внедрении, то есть что-то из полученных результатов исследований должно быть внедрено. Поэтому толковать, что диссертационная работа нужна лишь для удовлетворения амбиций автора неправомерно. В повышении квалификации того или иного гражданина заинтересованы не только сам гражданин, но и предприятие, вуз, и в целом общество, так как от этого человека будет больше отдачи. Без исследований и внедрения результатов не может быть прогресса в стране (обществе). При защите диссертации оценивается научная и практическая ценность диссертационной работы.

5.16 Тематика диссертационной работы (её выбор) зависит не только от наличия необходимого материала, но и от актуальности темы и состояния изученности поставленных вопросов объекта исследования и т.д.

5.17 Суждение о том, что кандидатская диссертация должна перевернуть, изменить что-то в обществе неверное. Финансировать исследования аспирантов должно государство, если вуз государственный, так как оно заинтересовано в подготовке кадров, необходимых для конкретных предприятий, вузов.

5.18 Земельная реформа не выполнила поставленных целей. Многое было проведено ошибочно, привело фактически к деградации сельского хозяйства. Непонятно, зачем было преобразовывать прибыльые колхозы и совхозы, создавать отдельные неэффективные крестьянские (фермерские) хозяйства?

Земельная реформа привела к резкому снижению доходности и уровня жизни народа, потере  жизненных ориентиров. Она привела к дестабилизирующим факторам экологическую обстановку. Примеров негативных последствий земельной реформы можно привести множество. Причём о них известно из имеющихся публикаций,поэтому на лекциях не излагалось об этом.

5.19 Пошёл бы учиться и работать в ГУЗ. Мне предлагали работать в Политехническом Музее (после окончания экономического факльтета по линии Горкома КПСС), приглашали в Горком профсоюза, предлагали в два раза большую зарплату, когда был деканом заочного факультета. Но остался верен ГУЗу.

VI.
6.1 Поскольку исследования не приведены и не получены результаты трудно судить о научном и практическом значении диссертационной работы. Думается, по описательной части отдельные разработки могут заинтересовать руководителей сельскохозяйственных предприятий. Плата за информационный продукт будет зависеть от ценности этих разработок и принесут ли они прибыль хозяйству при их внедрении.

6.2 Наиболее актуальными направлениями исследования в землеустройстве являются те, которые будут способствовать развитию землеустроительной науки, увеличению производства и улучшению качества продукции, увеличению прибыльности сельскохозяйственных предприятий; защите земли от негативных последствий и увеличению её производительных свойств.

6.3 Считал и считаю, что колхозы, совхозы имеют ряд преимуществ перед единоличными хозяйствами. Во-первых, в них имеется возможность более широко применить механизацию производственных процессов; во-вторых, внедрять передовые технологии; в-третьих, русский человек более склонен к коллективному.

6.4 Думается, что у Вас несколько предвзятое отношение к журналу «Землеустройство кадастр и мониторинг земель». На мой взгляд, статьи в журнале в основном являются актуальными, интересными по содержанию. Журнал признан даже ВАКом. 

6.5 Дела с землеустройством обстоят плохо. Во-первых, государство уничтожило (ликвидировало)все землеустроительные органы. При Советском Союзе землеустройство являлось государственным мероприятием и финансировалось за счёт государства. В настоящее время такое финансирование отсутствует( лишь финансируются отдельные работы, причём в небольших размерах). Фермерское хозяйство не в состоянии его финансировать. Не в состоянии этого сделать в большинстве своём публичные акционерные общества и другие сельскохозяйственные предприятия.

6.6 Считаю, что внутрихозяйственное землеустройство необходимо и оно будет развиваться. Примером его необходимости является восстановление деградированных земель в сельскохозяйственных предприятиях, крестьянских(фермерских)хозяйствах (процессы деградации земель наблюдаются во всех землепользованиях сельскохозяйственных организаций), введение и освоение научно-обоснованных севооборотов, без которых существовать хозяйство не может, разработка мероприятий по повышению производительных свойств земли и т.д. Внутрихозяйственное землеустройство необходимо проводить на основе разработки бизнес-планов для конкретных хозяйств.

Спасибо за поставленные вопросы, на которые я попытался кратко ответить. В заключении пожелаю читателю успехов во всех делах, творческих дерзаний!