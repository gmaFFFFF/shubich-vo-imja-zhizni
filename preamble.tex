\documentclass[book
,a5paper
,10pt
,russian																%	Заказывает русский стиль оформления заголовков, загружает пакет многоязыковой поддержки Babel с опцией russian, включает \frenchspacing (без дополнительных отбивок после конца предложения) и переопределяет команды \alph, \Alph на алфавитную нумерацию кириллицей (нумерация латиницей сохраняется в командах \alphlatin, \Alphlatin)
,openright															%	Новая глава начинается с нечетной страницы
,cyremdash															% Длинное тире будет несколько короче (что соответствует российским полиграфическим традициям)
%,draft																	%	Режим черновика, отображает ошибки
]{ncc}


%--- Руссификация
\input glyphtounicode.tex 							% Копирование символов в pdf (должен стоять до fontenc)
\pdfgentounicode=1											% ----//----
%Не работает в моём случае
%\usepackage{cmap}											% Копирование символов в pdf (должен стоять до fontenc)

\usepackage[T2A]{fontenc}								% Внутренняя кодировка
\usepackage[utf8]{inputenc}							% Кодировка исходников книги

%При использовании класса ncc использовать babel не нужно
%\usepackage[english,russian]{babel}		% Локализация и переносы

%--- Настройки оформления
\usepackage {indentfirst}								% Отступ и красная строка у первого параграфа
\usepackage{paratype}										% Шрифты paratype ПТ Санс и ПТ Сериф

\usepackage[headings]{ncchdr}						% Горизонтальная линия в колонтитле


\ChapterPrefixStyle{header,toc}					% Добавление префикса “Глава”: header - в колонтитуле; toc - в оглавлении

%--- Типографика
%\usepackage{microtype}									% Висячая пунктуация

% Эмулировать 'желательно' неразрывные пробелы. Подробности тут https://habrahabr.ru/post/339110/. Суть идеи: не следует все предлоги переносить на другую строчку. Некоторые можно оставить.
\usepackage{newunicodechar}
\newcount\afterconjunctionpenalty
\afterconjunctionpenalty=1000						% Величина для контроля желательности неразрывного пробела
\newunicodechar{˚}{\penalty\afterconjunctionpenalty\ }

% Задать коду символа U+2009 (Thin Space, тонкий пробел) значение неразбиваемый нерастяжимый узкий пробел (\,) или неразбиваемый растяжимый узкий пробел (\=,)
\newunicodechar{ }{\=,}



\clubpenalty = 150											% Определяет нежелательность разрыва страницы после первой строки абзаца. Запрет = 10000, по умолчанию 150
\widowpenalty = 1000										% Определяет нежелательность разрыва страницы перед последней строкой абзаца. Запрет = 10000, по умолчанию 150

%\emergencystretch=25pt									% Когда без переполнений сверстать абзац не удаётся, TeX попробует сделать все строки абзаца более разреженными (тем более разреженными, чем больше величина этого параметра). Оптимальное значение параметра равно примерно 20-30 пунктам и подбирается экспериментально


%--- Дополнительные пакеты

% Вставка изображений
\usepackage{graphicx}
\graphicspath{{pictures/}}							% Путь по умолчанию для изображений

% Символ градуса
\usepackage{textcomp}

% Использование minipage
\usepackage{float}			

% Стихи
\usepackage{verse}			

% Обтекание текстом
\usepackage{wrapfig}	



% Гиперсылки в тексте (Лучше добавлять самым последним)
\usepackage[unicode
,pdfdisplaydoctitle={true}							% Отображать в заголовке окна AdobeReader название документа, а не имя файла
,pdflang={ru-RU}												% Язык документа
,hidelinks															% Скрывает выделение ссылок
,hypertexnames={true}										% Если false, то hyperref будет вместо счётчиков LaTeX использовать свои и ссылки в предметном указателе будут работать неврно.
,plainpages={false}											% Если true, то счётчик страниц будет в формате arabic и не будет учитывать римскую нумерацию страниц, но hypertexnames должно быть равно true
,bookmarksnumbered={true}								% Добавить префикс <<Глава №>>
,pdfpagelabels
]{hyperref}



%--- Настраиваемые переносы
\hyphenation{in-ter-na-tion-al
рас-про-стра-ня-ет-ся}

% Борьба с ошибками appendix, заложенными в пакет ncc
\makeatletter
\let \@Asbuk\russian@Alph
\let \@asbuk\russian@alph
\makeatother