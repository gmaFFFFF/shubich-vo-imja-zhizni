% Лицензия
\chapter{Условия использования}

\begin{wrapfigure}{l}{.2\textwidth}
\centering
\includegraphics[width=.19\textwidth]{CC-BY-SA}
\label{fig:CC-BY-SA}
\end{wrapfigure}

\noindent М.\,П.~Шубичу (автору, лицензиару) принадлежат авторские (интеллектуальные) права на˚текст данного литературного произведения (работу). 

\noindent Права на˚иллюстрации к˚работе принадлежат их˚авторам. Если˚не˚указано иное, иллюстративный материал заимствован из˚общедоступных ресурсов Интернета, не˚содержащих указаний на˚авторов этих материалов и˚каких\=/либо ограничений для˚их˚заимствования. Считается, что˚такие иллюстрации перешли в˚общественное достояние.

Лицензиар предоставляет неограниченному кругу лиц (лицензиатам) простую (неисключительную) лицензию на˚условиях открытой лицензии\footnote
{Статья 1286.1 ГК РФ «Открытая лицензия на˚использование произведения науки, литературы или˚искусства».}
\foreignlanguage{english}{Creative Commons Attribution --- ShareAlike~4.0 International} (С~указанием авторства "--- С~сохранением условий 4.0 Всемирная, \textbf{CC~BY\=/SA~4.0}). Лицензиаты могут без˚ограничений распространять данную работу, изменять и˚использовать её в˚любых (в~том˚числе коммерческих) целях при˚условии указания \textbf{авторства} и˚сохранения данной лицензии \textbf{в˚производных работах.}
Исполнение лицензиатом условий данной лицензии считается акцептом её˚условий.

Лицензия \textbf{CC~BY\=/SA~4.0} доступна в˚сети Интернет по˚адресу: \url{http://creativecommons.org/licenses/by-sa/4.0/}. Экземпляр лицензии Вы также можете запросить письмом по˚адресу: \foreignlanguage{english}{Creative Commons: PO~Box~1866, Mountain View, CA~94042, USA.}

Срок действия открытой лицензии устанавливается на˚срок действия исключительного права на˚работу.

Книга подготовлена в˚формате \LaTeXe{}. Исходный код для сборки книги размещён в˚сети Интернет по˚адресу: \url{https://github.com/gmaFFFFF/shubich-vo-imja-zhizni}.{\sloppy

}% sloppy разрешает разреженные, но˚не~выходящие за˚край строки. Режим верстки абзаца определяется в˚тот момент, когда TeX читает пустую строку, завершающую абзац. Поэтому в˚группе \sloppy закрывающая фигурная скобка должна идти после пустой строки.

\todo[Оформление]{Может быть на оборотной стороне планируемой обложки разместить Ваш портрет, который нарисовали архитекторы?}