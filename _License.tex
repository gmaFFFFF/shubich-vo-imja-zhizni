% Лицензия
\chapter{Условия использования}

\begin{wrapfigure}{l}{.2\textwidth}
\centering
\includegraphics[width=.19\textwidth]{CC-BY-SA}
\label{fig:CC-BY-SA}
\end{wrapfigure}

\noindent М.П. Шубичу (автору, лицензиару) принадлежат авторские (интеллектуальные) права на текст данного литературного произведения (работу). 

\noindent Права на иллюстрации к работе принадлежат их авторам. Если не указано иное, иллюстративный материал заимствован из общедоступных ресурсов Интернета, не содержащих указаний на авторов этих материалов и каких-либо ограничений для их˚заимствования. Считается, что такие иллюстрации перешли в общественное достояние.

Лицензиар предоставляет неограниченному кругу лиц (лицензиатам) простую (неисключительную) лицензию на условиях открытой лицензии\footnote
{Статья 1286.1 ГК РФ <<Открытая лицензия на использование произведения науки, литературы или искусства>>.}
Creative Commons Attribution-ShareAlike 4.0 International (С указанием авторства-С сохранением условий 4.0 Всемирная \textbf{CC BY-SA 4.0}). Лицензиаты могут без ограничений распространять данную работу, изменять и использовать её в любых (в˚том числе коммерческих) целях при условии указания \textbf{авторства} и сохранения данной лицензии \textbf{в производных работах.}
Исполнение лицензиатом условий данной лицензии считается акцептом её условий.

Чтобы ознакомиться с экземпляром этой лицензии, посетите \url{http://creativecommons.org/licenses/by-sa/4.0/} или отправьте письмо на адрес Creative Commons: PO Box 1866, Mountain View, CA 94042, USA.{\sloppy

}% sloppy разрешает разреженные, но не выходящие за край строки. Режим верстки абзаца определяется в тот момент, когда TeX читает пустую строку, завершающую абзац. Поэтому в группе \sloppy закрывающая фигурная скобка должна идти после пустой строки.

Срок действия открытой лицензии устанавливается на срок действия исключительного права на работу.