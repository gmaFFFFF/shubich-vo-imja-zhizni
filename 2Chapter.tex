\chapter{Пора учёбы}
В~1945~году Миша пошёл в первый класс. Для˚начальной школы использовался жилой дом, в˚одной половине которого проживала семья. Сколоченные доски заменяли парты. Сидели на˚таких же˚досках. Первой учительницей была Мария Фёдоровна, недостаточно грамотная деревенская женщина, но к˚своей работе относилась ответственно. Писали на˚бересте\footnote{Береста "--- верхний слой (белая наружная часть) коры берёзы.}, а˚чернила делали сами из˚сажи. На˚две деревни Вишеньки и˚Богутичи был один старый букварь. Чтобы˚подготовиться к˚урокам, приходилось за˚4\,км идти за˚книгой. Устанавливалась очерёдность пользования букварём.

Мария Фёдоровна незамужняя, средних лет женщина, с˚сохранившейся девичьей красотой. Впоследствии вышла замуж за˚деревенского парня, который принимал от˚жителей деревни молоко. Даже˚среди жителей была такая частушка:

\settowidth{\versewidth}{Легир молоко принимает,}		% Для определения примерной ширины строки
\begin{verse}[\versewidth]
	Мама, я легира\footnote{Легир "--- деревенское прозвище.} \todo[Текст]{Легир- где ударение?} люблю, \\* 
	Мама, я за легира пойду, \\*	
	Легир молоко принимает, \\*
	И с каждой литры попивает, \\*
	Вот за что я легира люблю. \\!
\end{verse}

Как-то Миша нашёл бутылочку, налил воды, засыпал сажи и˚из покрышки автомобиля вырезал пробку и˚плотно закрыл бутылочку. В~школу уже опаздывал, поэтому пришлось бежать. Был урок арифметики. Впереди˚сидела девушка в˚белой кофточке "--- родственница Марии Фёдоровны. Повернулась лицом к˚Мише, что-то спрашивала по˚арифметике. В~это время Миша вытащил пробку, из˚бутылочки ударил фонтан сажи, который полностью обдал девушку, а˚на потолке образовалось чёрное пятно. Оказалось, когда бежал в˚школу, в˚бутылочке всё разболталось, и˚образовались газы и˚при открытии пробки вырвались наружу. Мишу за˚такой поступок отчислили из˚школы, но˚через 2~дня восстановили.

В~школе было введено чистописание. Тетради ученики должны были приносить на˚проверку учительнице домой. В~один из˚воскресных дней Миша принёс свою тетрадь. Дверь в˚дому была приоткрыта, поэтому спокойно вошёл в˚дом. На˚кровати лежали Мария Фёдоровна и˚мужчина (военный). Тихонько Миша задом вышел из˚дома. Во˚время урока следующего дня учительница отругала Мишу за˚то, что не~принёс на˚проверку тетрадь. Было очень обидно и˚злостно, но˚он только произнёс: «Не~успел сделать!»

В~школе писали диктанты, оценки в˚основном стояли хорошо и˚отлично. Приехавшая из˚района комиссия, нашла в˚этих диктантах много ошибок, и˚выставила двойки. Начали беседовать с˚учениками. Ответ был единый "--- так нас научили. Тогда учительница в˚порыве гнева сказала: «И~ты˚Миша против меня!»

Война, разруха, голод, нарушение обычного жизненного уклада сказалась на˚психологии людей, породили зверя в˚человеке, особенно в˚молодёжи. Они стали несносны друг другу, что проявлялось в˚повседневных делах, даже в˚играх. Подтверждением подобных явлений является случай, произошедший в˚деревне. Более взрослые ребята разложили костёр и˚подговорили малыша прыгать через костёр с˚палкой. Естественно, Андрей прыгнул, но˚попал в˚костёр и˚весь обгорел, побежал по˚улице со˚страшным криком. Второй случай, когда ребята 8-летнего парня разложили в˚лесу, приложили руки, ноги брёвнами и˚оставили в˚лесу. Случайно наткнулась на˚него женщина и˚освободила.

\begin{wrapfigure}{O}{.25\textwidth}
\centering
\includegraphics[width=.2\textwidth]{Kerosene-lamp}
\caption[Лампа керосиновая стенная («стенник») с˚подвесом и˚рефлектором (отражателем)]{Лампа керосиновая стенная («стенник») с˚подвесом и˚рефлектором (отражателем)\footnotemark}
\label{fig:Kerosene-lamp}
\end{wrapfigure}
\footnotetext{Автор: В.~Журов (VladimirZhV), 02.01.2009.}

Дальнейшая учёба (5\==7~класс) проходила в˚деревне Богутичи. Приходилось за˚4\,км ежедневно ходить пешком туда и˚обратно. Учителя там были подготовленные, сочувствующие, старались дать знания ученикам. 

Хотелось добрым словом вспомнить учителя по˚математике Н.\,К.~Бенза, бывшего моряка. Он даже позволял Мише истолковывать своё видение решения той или иной задачи. При˚посещении деревни Вишеньки он˚отцу сказал: «Из˚парня будет толк». Отца это воодушевило.

Дальнейшая учёба (8\==10~класс) проходила в˚городском посёлке Ельск. Ежедневно приходилось ходить туда и˚обратно за˚11\,км. 7\,км дороги проходило через дубовый лес, 3\,км "--- по˚полю и˚1\,км "--- по˚улицам городского посёлка. Зимы тогда были суровые: морозы доходили до˚20 и˚более градусов, снегом иногда заметало под самые крыши домов. Приходилось Мише из˚дома выходить очень рано. 

Однажды замело снегом основательно. По˚дороге не~было ни˚одного человеческого следа. По˚лесу шёл кое-как с˚трудом, а˚по полю так замело, что пришлось ползти. До˚школы добрался лишь тогда, когда занятия уже кончились. Пришлось по˚своим следам возвращаться обратно. Дважды волки загоняли на˚дерево. Приходилось сидеть и˚выжидать, когда они убегут. Это «удовольствие» было небезопасным и˚страшным. Волки долго сидели под деревом и˚выжидали.

Уроки готовили при керосиновой лампе. Уставший иногда засыпал и˚лампой осмаливал брови. Весной передвигаться было несколько легче. Ни˚одного дня не~был на˚квартире, так как нечем было оплачивать. Всего пройдено в˚школу в˚деревню Богутичи 4\,500\,км и˚городской посёлок Ельск "--- более 12\,500\,км. 

Учителя были высококвалифицированными и˚старались больше знаний дать ученикам.

Хорошие слова хотелось сказать в˚адрес директора школы, учителя математики Ф.~Сорокина, завуча "--- учителя по˚русскому языку и˚литературе "--- Френкеля. 

Закончил среднюю школу на˚хорошо и˚отлично (не~было ни˚одной тройки). 25~июня 1954~г. получил аттестат зрелости.

Во˚время каникул Миша всегда работал в˚колхозе\footnote
{Колхоз (сокр. от˚коллективное хозяйство) "--- это способ коллективного ведения сельскохозяйственного производства на˚территории СССР. С~1930\=/х годов основной формой колхозов стала сельскохозяйственная артель "--- производственный кооператив, в˚котором обобществлялись труд и˚основные средства производства (техника, инвентарь, семена, постройки, сельскохозяйственные животные и т.~д.). В~личной собственности крестьян оставались жилой дом и˚подсобное хозяйство (в~т.~ч. продуктивный скот), размеры которого ограничивались уставом артели. Колхозы создавались на˚основе безвозмездной передачи единоличными крестьянскими хозяйствами в˚совместную собственность колхоза средств производства. Земля выбывала из˚единоличного пользования и˚передавалась колхозам в˚бессрочное безвозмездное пользование, оставаясь в˚государственной собственности. Доходы распределялись по˚количеству и˚качеству труда (по~трудодням) продукцией сельского хозяйства вплоть до˚середины 1960\=/х годов, когда трудодни были заменены заработной платой.}%
, чтобы как-то облегчить участь родителей. Был заместителем бригадира. Приходилось бегать и˚обмерять двухметровкой выполненную работу каждой женщины (все работы выполнялись вручную) и˚начислять трудодни. Работал на˚соломокопнителе\footnote
{Соломокопнитель "--- приспособление к˚комбайну, состоящее из˚брезентовой камеры с˚деревянным днищем, передняя сторона которой открыта и˚обращена к˚соломотрясу комбайна. Выбрасываемые соломотрясом солома и˚мякина (отходы при обмолоте и˚очистке зерна хлебных злаков и˚некоторых других культур; полова) постепенно заполняют соломокопнитель, из˚которого они выгружаются опрокидыванием днища.}
на˚комбайне, на˚току\footnote
{Зерновой ток "--- площадка с˚комплексом машин, оборудования и˚сооружений для механизированной послеуборочной обработки зерна в˚колхозах и˚совхозах.}
"--- на˚веялке\footnote
{Веялка "--- сельскохозяйственная машина, предназначенная для отделения зерна от˚мякины.} очищали зерно.

\begin{wrapfigure}{O}{.25\textwidth}
\centering
\includegraphics[width=.2\textwidth]{2metrovka}
\caption[Двухметровка]{Двухметровка\footnotemark}
\label{fig:2metrovka}
\end{wrapfigure}
\footnotetext{Источник заимствования "--- \url{https://fermer.ru/}.}

Трудодни\footnote{Трудодень "--- мера оценки и˚форма учёта количества и˚качества труда в˚колхозах в˚период с˚1930 по˚1966~год. Заработная плата членам колхозов не~начислялась. Каждый колхозник получал за˚свою работу долю колхозного дохода соответственно выработанным им˚трудодням.} начисляли, но˚за них, фактически, ничего не~получали. Один председатель колхоза (местный житель) выдал на˚трудодень по˚300\,грамм зерна (при норме 3\==4 кг ржи на работающую женщину) и˚его осудили на˚3~года.

Немного следует изложить о˚родителях Миши. Отец возвратился с˚войны раненный в˚ногу в˚1947~году. Часть попала в˚окружение, раненый оказался в˚плену у˚немцев. Дважды бежал, но˚ловили и˚возвращали обратно. Содержали русских пленных в˚ужасных условиях, по˚рассказу отца, не~кормили. Если˚они бросались на˚сырую кормовую свёклу, то˚по ним открывали огонь. Насаживали отдельных на˚колья задницей. Особенно усердствовали полицаи из˚русских. 

Однажды с˚отцом пошли на рынок г.п.~Ельск. Во˚время хождения по˚базару отец вдруг резко остановился, весь побледнел, затрясся. «Что˚с˚тобой, папа?» Оказывается он˚увидел человека, который был полицаем и˚в˚лагерях беспощадно издевался над˚русскими военнопленными. На~вопрос: «Не~ошибся ли˚ты?» "--- последовал ответ: «Нет!» Пошли его˚искать, но˚среди толпы людей не~обнаружили. Возможно он˚заметил состояние отца и˚скрылся. Заявили об˚этом в˚милицию.

Всех пленных после освобождения для проверки отправляли на˚войну с˚Японией. Подобная участь постигла и˚отца Миши. Мишу и˚отца удивило, как могло быть, что часть деревенских мужчин оказались не в˚армии во˚время войны. Отдельных из˚них, правда, постигла страшная участь. «Партизаны» согнали их в˚дом, допрашивали, а˚затем выводили и˚расстреливали в˚затылок. Таких оказалось несколько человек. Но˚нигде в˚архивных данных не~нашёл, за˚что их˚расстреляли. Как-то мальчик подошёл к˚ Мише и˚сказал: «Миша, пойдём я˚передам отцу хлеб». Подошли к˚дому, в˚это время вывели его отца. Мальчик подошёл ближе, чтобы передать хлеб, но˚его оттолкнули. В~это время последовал выстрел и˚его отец на˚глазах ребёнка пал мёртвым. Побежали обратно по˚улице, и˚мальчик сквозь слёзы проронил: «Миша, не~говори моей маме». 

После˚возвращения отец Миши работал в˚колхозе, который затем преобразовали в˚совхоз\footnote{Совхоз (сокр. от˚советское хозяйство) "--- это государственное сельскохозяйственное предприятие в˚СССР. В~отличие от колхозов, являвшихся кооперативными объединениями крестьян, созданными на˚средства самих крестьян, совхоз был государственным предприятием. Работающие в˚совхозах были наёмными работниками, получавшими фиксированную заработную плату в˚денежной форме, в˚то˚время как в˚колхозах до˚середины 1960-х использовались трудодни.}

Война, плен, ранение, голод сказались на˚здоровье отца. Заболел раком желудка. Сделали в˚Минске операцию, удалили 2/3~желудка. После˚операции прожил 3~года. Напряжённо работал: косил, носил бревна, складывал в˚стог сено, не~жалел себя. 25~марта 1982~года умер.

\begin{wrapfigure}{O}{.4\textwidth}
\centering
\includegraphics[width=.35\textwidth]{kok-saghyz}
\caption[Кок\=/сагыз. Посадки в˚Америке, 1947~год]{Кок\=/сагыз. Посадки в˚Америке, 1947~год\footnotemark}
\label{fig:kok-saghyz}
\end{wrapfigure}
\footnotetext{\foreignlanguage{english}{W.\,Gorden Whaley and John S.\,Bowen Image courtesy of Ford Motor Co. --- Russian Dandelion (kok-saghyz) An Emergency Source of Natural Rubber.}}

Мама Миши также работала в˚колхозе. В~одно время доводили план выращивания на˚болоте за˚7\,км от деревни кок\=/сагыза\footnote{Кок\=/сагыз "--- один из˚лучших естественных каучуконосов флоры СССР. С~1954~года, в˚связи с˚развитием производства синтетического каучука, плантации кок\=/сагыза более не~культивировались.}. Затем на˚плечах она приносила его домой и˚вместе с˚детьми очищали корни и˚сдавали. Использовали его для изготовления резины.

Сразу после войны, из-за отсутствия техники, лошадей и˚отца, маме вместе с˚детьми приходилось запрягаться в˚плуг, борону и˚на себе пахать, сеять, чтобы как-то пропитаться. В~колхозе и˚на приусадебных участках пахали, сеяли, убирали всё вручную.

После˚освобождения от˚фашистов Миша заболел малярией. Это страшная болезнь. На˚улице жара, а˚тебя трясёт. Не˚помогает никакая одежда, силы оставляют. Даже˚лежащую на˚земле палочку не~можешь переступить. 

Бабушка заварила табак, выращенный на˚огороде, и˚заставила выпить литровую кружку отвара. После˚чего Миша потерял сознание. В~бреду находился трое суток, не~приходя в˚сознание. Но˚когда очнулся, малярии уже не~было. Бабушка пошла на˚большой риск. Решила выживет, но˚не~будет малярии или наступит смерть. Фактически отравила, но˚молодой организм победил. Лекарств, в˚том числе хинина не~было в˚наличии. Впоследствии это коварная болезнь в˚стране была полностью побеждена.

\begin{figure}[h]
\includegraphics[width=.9\textwidth,center]{private/semyaShubicha_clip}
\caption{Семья Миши. В~первом ряду слева направо: отец "--- Павел Ильич, брат "--- Володя, мать "--- Александра Макаровна. Второй ряд справа налево: сестра "--- Галина, Миша. Источник заимствования "--- личный фотоархив М.\,П.~Шубича}
\label{fig:semyaShubicha}
\end{figure}
\todo[Кто стоит слева от Миши?]{Кто стоит слева от Миши?}

Мысли Миши были направлены на˚дальнейшую учёбу, поступление в˚вуз. Подобное желание не~очень одобряла мама. Ей хотелось, чтобы Миша оставался помощником родителей и˚работал в˚колхозе. Следует отдать должное\todo[Ред.]{Может лучше так: Следует поблагодарить отца, который всегда поддерживал Мишу} отцу, который всегда поддерживал Мишу. 

В~то˚время деревенской молодёжи не~давали справки на˚выезд из˚деревни. С~получением паспорта была большая проблема. Эти обстоятельства вынудили Мишу поработать и˚отказаться на˚некоторое время от˚поступления в˚вуз. Тем более подвернулся подходящий случай, пригласили его поработать в˚сейсмическую партию 1/54 Белорусской Главнефтегеофизики, которая расположилась в˚соседней деревне Богутичи.

