\chapter{Жизнь с МИИЗом (ГУЗом)}

\section*{Аспирантура. Преподавательская деятельность}
\label{sec:aspirantura}
\addcontentsline{toc}{section}{\nameref{sec:aspirantura}}

Учёба в˚аспирантуре была напряжённой, интересной. Аспиранты не~только готовились к˚сдаче кандидатского минимума, собирали необходимые материалы и˚работали над диссертацией, но и˚развивались в˚культурном отношении: коллективно посещали музеи, выставки, театры, кинотеатры, бани. Большую роль в˚формировании личности аспиранта сыграл неуставной совет "--- «домогающиеся умышленники МИИЗа», который был организован аспирантами. 

На˚своём заседании «совет» обсуждал, как вести себя во˚время защиты диссертации тому или иному аспиранту, его работу над диссертацией, готовность к˚сдаче кандидатского экзамена, оказывал помощь нуждающимся, стимулировал˚успешную сдачу экзамена, защиту диссертации, рождение ребёнка, собирали взносы. «Совет» сыграл определённую положительную роль в деятельности аспирантов. 	

Сергей Александрович Удачин как-то пригласил Мишу "--- аспиранта к˚себе домой, посадил в˚своём кабинете за˚стол и˚предложил прочесть доклад, с˚которым он˚должен выступать на˚международном симпозиуме, и˚высказать свои замечания.

Миша внимательно прочёл доклад, и˚тут же˚последовал вопрос: «Понравился?» По˚своей наивности Миша ответил, что в˚одном месте не~понравилось, указав, что именно не~понравилось и˚почему. Он ожидал, что впоследствии академик ВАСХНИЛ отругает, устроит взбучку, но˚Сергей Александрович очень спокойно сказал: «Мне тоже здесь не~нравится, переделаю».

С~того же˚дня Сергей Александрович ежедневно вечером (в~одно и˚тоже время) назначил прогулки. Ходили по˚ближайшим от˚дома улицам. Эти прогулки были очень интересными, познавательными и˚поучительными. Он рассказывал, сколько сил и˚энергии пришлось ему потратить на˚сохранение и˚развитие землеустройства, создание научной школы, борьбу с˚различными вредными теориями. В~свою очередь он˚интересовался бытовыми условиями Миши, семейными делами, работой над˚диссертацией.

По˚мере готовности диссертации позвонил С.\,А.~Удачин и˚сказал, чтобы принёс для читки. Прочитавши диссертацию, С.\,А.~Удачин однажды вечером позвонил и˚сказал: «Завтра будем обсуждать, подготовь автореферат». Ночью пришлось потрудиться. Диссертацию на кафедре одобрили с˚первого раза обсуждения. Предстояла рассылка автореферата. Однако, учёный секретарь О.~Козырева сначала запротивилась из-за наличия в˚автореферате как бы секретных данных расчётного количества зерна, которое может произвести область. С.\,А.~Удачин своевременно её˚поправил. Защита состоялась 25~декабря 1969~г. на˚объединённом учёном совете разных специальностей. Защита прошла успешно, и˚Мише была присвоена учёная степень кандидат экономических наук по˚специальности 08594 "--- «экономика, организация и˚планирование народным хозяйством». 

Место будущей работы определили ректор института канд. техн. наук, проф. Н.\,Д.~Ильинский и˚заведующий кафедрой землеустроительного проектирования д-р экон. наук, проф., академик ВАСХНИЛ Сергей Александрович Удачин. После˚защиты диссертации ректор Н.\,Д.~Ильинский вызвал и˚сказал, что есть решение зачислить Мишу на˚кафедру районной планировки, где будет обеспечен дальнейший рост "--- заведование кафедрой. В~этот же˚день встретил С.\,А.~Удачин, который сказал: «Этому не~быть, зачислен будешь на˚кафедру землеустроительного проектирования» и˚на˚следующий день принесли из˚Министерства сельского хозяйства СССР направление по˚распределению на˚кафедру землеустроительного проектирования. Таким образом, была определена дальнейшая судьба Миши, которого зачислили на должность ассистента. Субординация позволила в˚дальнейшем обращаться к˚Мише по˚имени и отчеству (Михаил Павлович). Дальнейшее его формирование в˚качестве педагога проходило под˚влиянием учёных-землеустроителей.

В˚январе 1970 года на˚кафедре земпроектирования в˚это время работали видные советские учёные: академик ВАСХНИЛ, профессор С.\,А.~Удачин, профессора: Н.\,И.~Прокуронов, Н.\,Н.~Бурихин, Г.\,В.~Чешихин, Я.\,М.~Цфасман, доценты: А.\,И.~Гавриленко, В.\,Д.~Кирюхин, К.\,М.~Кирюхина, Е.\,Н.~Первова, А.\,В.~Юнюков, Б.\,Н.~Ширин, Н.\,И.~Ярмак, В.\,П.~Троицкий, Г.\,В.~Симонов и˚другие. 

На˚первых порах среди таких учёных светил Михаил Павлович чувствовал себя как-то неуютно, но˚несмотря на˚это ему˚поручили чтение лекций, руководство курсовым и˚дипломным проектированием, приём экзаменов, ведение практических занятий. Всё˚это Михаила Павловича как-то одухотворило и˚заставило работать над собой, тщательно готовиться к˚любым видам учебных занятий.

Педагогическая деятельность увлекала Михаила Павловича, но˚были серьёзные бытовые трудности "--- неустроенность с˚жильём (семья находилась в˚г. Рязани, проживали в˚маленькой комнатушке без˚удобств). Перспектив с˚жильём в˚институте не~предвиделось, что˚сказалось на˚его настроении и˚в˚какой\=/то степени на˚работе.
Педагогическая деятельность наряду со˚всеми видами учебной работы (чтение лекций, проведение курсового проектирования, семинаров, зачётов, экзаменов, дипломного проектирования, работы с˚аспирантами и~т.~д.) предусматривала проведение методической, научно\-/исследовательской, воспитательной и˚общественной работы.

Одной из˚первых дипломниц была Алевтина Алексеевна Мохнаткина, девушка с˚выраженным характером, трудно поддающаяся убеждениям. Считала, что если принято какое\=/то решение, то˚иного более целесообразного и˚эффективного не~может быть. Михаил Павлович предложил ей˚новое решение по˚организации использования земли в˚колхозе «Россия» Косимовского района Рязанской области (объект её˚дипломного проекта) в˚отличие от˚принятого. В~ответ заявила: «Над˚этим решением думали десять голов, неужели Ваша голова умнее их». Долго пришлось её˚убеждать более объективно смотреть на˚вещи и˚правильно их˚оценивать. Защитилась она на «отлично». После˚защиты преподнесла букет цветов, внутри которых записка на˚вырезанном куске ватмана следующего содержания: «Из~Вас получится очень хороший преподаватель, воспитатель. Вы сумели совершенно изменить меня».

Любой человек, особенно молодой, всегда должен заранее обдумывать свои действия, поступки и˚делать выводы, к˚чему это приведёт впоследствии. Делать выводы после свершившегося факта бывает бесполезно. Девушка Надя была на˚преддипломной практике в˚колхозе «Чёрная речка» Сапожковского района Рязанской области. Михаил Павлович приехал в˚хозяйство (преподаватели разъезжали по˚командировкам, чтобы оказать помощь студентам в˚разработке проекта и˚сборе материала для дипломного проектирования). На˚квартире с˚ней проживали ещё две девушки "--- молодые специалисты. Зашёл в˚квартиру. Посреди˚комнаты стоит чемодан. Надя собралась уезжать. На˚вопрос: «В~чем дело?» "--- ответа не~последовало. Проживающие с˚ней девушки ничего не~сказали, лишь загадочно переглянулись. 

Впоследствии выяснилось, что поздно вечером заехал на˚машине председатель колхоза и˚предложил Наде взять материалы для утверждения в˚районе разработанного проекта. Девушка быстро собралась и не~подумала, что рабочий день уже закончился. Председатель полночи возил её по˚лесу. В~результате она решила убежать с˚практики. 

Михаилу Павловичу хотелось побеседовать с˚председателем колхоза, но он˚всяким образом избегал подобной встречи. Поскольку˚встреча не~состоялась, то˚решено было изложить в˚газете «Приокская правда» о˚пренебрежительных действиях председателя к˚проекту внутрихозяйственного землеустройства и˚нарушениях границ его перенесения. По˚результатам статьи председатель колхоза был снят с˚должности.Вот˚всё, что удалось сделать за˚безнравственный проступок председателя колхоза. На˚сердце у˚Михаила Павловича долго оставался горький осадок, к˚чему могут привести безрассудные, необдуманные действия.

Как-то в начале своей педагогической деятельности Михаил Павлович пришёл в˚группу заочников для˚проведения практических занятий. В˚группе были взрослые люди, опытные производственники, руководители отрядов, отделов, начальники управлений, в˚основном из˚Центрально\-/Чернозёмной зоны (ранее на˚заочный факультет принимали лишь лиц, работающих по˚специальности или˚родственным специальностям). Они как-то настороженно встретили  молодого преподавателя и˚решили устроить проверку, стали задавать различные вопросы. Естественно, в˚этой ситуации выручил его жизненный опыт (работа в˚геофизических партиях «Главнефтегеофизика», служба в˚рядах Советской Армии, работа по˚специальности в˚должностях инженера\-/землеустроителя и˚начальника отряда в˚Рязанской землеустроительной экспедиции). Михаил Павлович спокойно ответил на˚все их˚вопросы, и˚более того задал им ряд вопросов˚землеустроительного характера, но˚ответа не~получил. Всё˚это очень понравилось студентам-заочникам и˚как-то сблизило с ним. Впоследствии и˚после окончания МИИЗа, будучи в˚г.~Москве, они постоянно заходили к Михаилу Павловичу с˚беседами, различными советами. 

Пришлось Михаилу Павловичу однажды давать печальный совет. Как-то к˚нему в˚общежитие зашёл директор Тамбовского филиала ЦЧОГипрозем Владимир Никитич Покидышев. У~них в˚семье была неприятность. У~ребёнка был порок сердца, предстояла в˚Москве операция. Они должны были дать разрешение на˚операцию. Они колебались, как поступить. С~этим вопросом он˚обратился к˚Михаилу Павловичу. Конечно, было трудно что-то посоветовать. Подумавши, Михаил Павлович сказал, что они должны согласиться на˚операцию. Они ещё молодые и, при˚неприятном исходе, в˚состоянии родить ребёнка. Иначе˚придётся мучиться очень долго. Сделали операцию, исход оказался печальным. Через˚год у˚них родился ребёнок. Пришёл к˚Михаилу Павловичу опять В.\,Н.~Покидышев и˚поблагодарил за˚совет. Конечно, Михаил Павлович понимал, что для них это большое горе, но˚пришлось дать непростой совет, заглядывая в˚будущее.

Бытовые неудобства (Михаил Павлович на˚первых порах жил в˚общежитии, а˚семья в˚Рязани), трудности с˚пропиской и˚жильём вносили какую-то неуверенность, тревогу и˚он задавался вопросом: «Что˚делать дальше?» Следует отдать должное С.\,А.~Удачину, несмотря на˚свою большую занятость, он по\=/человечески относился к˚людям, проникался их˚страданиями. Подходил к˚Михаилу Павловичу и˚говорил: «На˚первых порах все трудности необходимо преодолеть, в˚дальнейшем всё будет хорошо». Это˚высказывалось как-то «по\=/отцовски» и˚вселяло в˚него надежды на˚будущее. В˚дальнейшем эти проблемы пришлось решать Михаилу Павловичу самостоятельно: «пробивать» деньги в˚Министерстве сельского хозяйства СССР, заключать долевое участие в˚строительстве.

Поскольку˚в то˚время было много опытных, высококвалифицированных педагогов, то˚негласно ходило мнение, что с˚написанием докторских диссертаций должна наступить очередь. Думается, это было ошибочное мнение, которое тормозило и не~способствовало повышению квалификации молодых.

Доцент А.\,И.~Гавриленко был требовательным преподавателем, не~всегда считался с˚мнением других членов кафедры, представлял собой трудно пробиваемую натуру. Будучи проректором по˚учебной работе, при подаче Михаилом Павловичем документов на˚должность доцента и˚положительном заключении кафедры сказал: «Мы ходили в˚ассистентах по семь и˚больше лет, а ты˚хочешь быть доцентом уже на˚третий год». Фактически продержал документы ещё год, не~давая им ходу. Пришлось Михаилу Павловичу ехать в˚ВАК, где разъяснили, что никаких подобных сроков не~существует. Если˚человек заслуживает, то˚можно выдвигать в˚любое время.

При˚рассмотрении вопроса о˚присвоении звания профессора М.\,П.~Шубичу в˚ВАК позвонил проректор по˚учебной работе ГУЗ профессор С.\,Н.~Волков и˚заявил: «М.\,П.~Шубичу не~присуждайте профессора, так как отсутствуют сейчас у˚него аспиранты». Наличие аспиранта не~имело никакого значения. Причём˚это было в˚то время, когда один аспирант защитился, а˚другого ещё не~закрепили. По˚приезду в˚университет Михаил Павлович зашёл в˚кабинет к С.\,Н.~Волкову и˚сказал: «Независимо˚от твоего звонка, я˚буду профессором». Поддержку в˚присвоении Михаилу Павловичу Шубичу учёного звания профессора оказал ректор института природообустройства. За˚что он выражает ему благодарность. Через˚пару недель вызвали в˚ВАК и˚выдали аттестат профессора. Но˚этот звонок стал для Михаила Павловича сигналом не~работать дальше над докторской диссертацией, т.~к.~С.\,Н.~Волков был председателем Совета по˚присуждению докторских степеней.


\section*{Общественная работа}
\label{sec:socialWork}
\addcontentsline{toc}{section}{\nameref{sec:socialWork}}

Наряду˚с основной работой на˚кафедре много сил и˚времени у Михаила Павловича отнимала общественная работа. Пять лет "---  председатель объединённого профсоюзного комитета МИИЗа, избирался год"--- членом горкома профсоюза работников сельского хозяйства, членом партийного бюро МИИЗа, пять лет заместителем заведующего кафедрой землеустроительного проектирования, около 18 лет "--- деканом заочного факультета, длительное время был членом учебно-методического центра заочной формы обучения по сельскохозяйственным специальностям департамента кадровой политики и образования МСХ, 3 года ответственным секретарём приёмной комиссии и т.д. 
Выполняя эти функции,  нагрузку по˚кафедре он вёл в˚полном объёме, не~снижая её. К~общественной работе относился также ответственно. Старался быть объективным и˚справедливым, хотя иногда это принималось негативно. По˚поводу принятия одного решения на˚заседании профсоюзного комитета вызвал Михаила Павловича ректор, профессор И.\,В.~Дегтярёв в˚свой кабинет и˚требовал его отмены. Но˚когда Михаил Павлович ответил, что решение принималось коллективно, единогласно, и он не~может его отменить, он˚схватил его за˚грудь. Михаилу Павловичу в˚данной ситуации пришлось как-то отвечать взаимностью, и˚только тогда слова Михаила Павловича были восприняты с˚пониманием.

На˚следующий день ректор пришёл в˚56~аудиторию (кабинет профкома), вместо извинения сказал Михаилу Павловичу: «Мы˚оба погорячились». Михаил Павлович посчитал это высказывание своеобразным извинением.

Не˚поддержал В.\,И.~Дегтярёв Михаила Павловича в˚получении неиспользуемых площадей для института в˚посёлке Салтыковка после разморозки труб (такие площади Михаилом Павловичем были подобраны совместно с˚начальником отдела по˚распределению жилья города Балашиха), что не~способствовало нормальному ведению учебного процесса. После˚разморозки труб лекции и˚практические занятия пришлось проводить в˚пальто, зимней шапке, а˚отдельные занятия отменились.

Заместителем председателя профсоюзного комитета длительное время была Пальмира Фёдоровна Белянцева. Председатель профсоюзного комитета работал на˚общественных началах, а˚заместитель получал зарплату. Муж П.\,Ф.~Белянцевой работал в ЦK КПСС\footnote{ЦК КПСС "--- Центральный комитет Коммунистической партии Советского Союза "--- высший партийный орган в˚промежутках между съездами партии.}. Поэтому она чувствовала себя независимой, всемогущей. По˚натуре это была своенравная женщина, старалась полностью подчинить председателя комитета. Во˚время работы председателем Михаил Павлович выявил, что оказанную профсоюзным комитетом помощь студенты не~всегда получали. Поэтому Михаил Павлович решил её˚убрать, хотя ни˚одному председателю не~удавалось это сделать. Во˚время очередной выборной кампании списки членов нового профкома не~распечатал и не~огласил, а˚сделал это во˚время перерыва. После˚перерыва огласил. П.\,Ф.~Белянцева восседала в˚Президиуме, но˚в˚списках она не~значилась. Делегаты проголосовали единогласно, то˚есть без П.\,Ф.~Белянцевой. 

Назавтра она в˚кабинете профкома 56~аудитории устроила страшный скандал, вызвала мужа. Муж оказался очень тактичным человеком. По˚приезду в˚институт, не~заходя в˚кабинет, вызвал Михаила Павловича в˚вестибюль. В~беседе Михаил Павлович нашёл с˚ним взаимопонимание, и˚он˚уехал. Но~П.\,Ф.~Белянцева на˚этом не~остановилась, обратилась в˚следственные органы. Следователю при вызове пришлось подробно объяснять создавшуюся ситуацию. На˚этом всё закончилось. Так˚пришлось избавиться от˚коварной женщины. 

Будучи председателем профкома Михаилу Павловичу приходилось заниматься многими вопросами в˚интересах студентов и˚сотрудников: 
«выколачивать» путёвки в˚дома отдыха и˚санатории, автомашины и˚садовые участки для сотрудников, оформлять демонстрационные колонны;
«выбивать» участок и˚деньги на˚строительство дома для сотрудников;  
следить за˚состоянием дел в˚общежитии, работой кружков, столовой, в˚том числе контролировать качество пищи;
обеспечивать соблюдение техники безопасности в˚МИИЗе;
проводить подготовку ежегодной учёбы партийно\-/профсоюзного актива, культурно\-/массовую работу; 
заниматься профсоюзными взносами и˚подготовкой к˚празднованию 200\=/летия вуза и˚так далее. Многие виды этих работ были неблагодарны, требовали много сил и˚нервов. 

Михаилу Павловичу удалось выбить 10\,гектар земли в˚Жуковском районе Калужской области и˚один гектар в˚Павлово\-/Посадском районе Московской области под дачные участки для сотрудников института, а˚также деньги в˚Министерстве Сельского хозяйства СССР под строительство дома для сотрудников института и˚деньги для долевого строительства. За˚счёт денег на˚долевое строительство удалось поселить четыре семьи сотрудников в˚г.~Балашихе.

Для˚получения земельного участка и˚разрешения на˚строительство дома Михаилу Павловичу пришлось затратить несколько месяцев ежедневных хождений в˚Моссовет\footnote{Московский городской совет (Моссовет) "--- высший орган государственной власти в˚Москве с˚1917 по˚1993~год, предшественник нынешней Мосгордумы.}. Это был год олимпиады (1980~год), что усугубляло положение дел и˚требовало много сил, энергии и˚большой настойчивости. При˚больших усилиях такой участок был выделен в˚Свиблово и˚началось строительство дома. Жилой дом должен был принадлежать институту и˚лишь 30~\% площади, согласно договору, должны были выделить Министерству сельского хозяйства СССР. Из-за недостаточной и неактивной работы в˚этом отношении последователя Михаила Павловича М.\,К.~Недайведова всё было впоследствии переиграно. ГУЗу досталось лишь 30~\% жилой площади. Михаилу Павловичу было очень обидно и˚злостно за˚бездействие при таких громадных затраченных усилиях. Следует отметить, что Михаил Павлович для себя лично никакой выгоды не~извлекал: не~имел садового участка, не~приобрёл машины (хотя сотрудникам распределял), не~претендовал на˚жильё (хотя своего не~имел) и˚должность. Это дало возможность впоследствии ему оправдаться перед Московским городским партийным контролем по˚поводу того, что в˚садоводческий кооператив ГУЗа попали посторонние люди, отдельные сотрудники возвели неустановленных размеров постройки и~т.~д. 

Один случай оставил негативный след до˚настоящего времени. На˚садовом дачном участке, выделенном для сотрудников МИИЗа в˚Балабаново Жуковского района Калужской области, оказались по˚решению Калужского облисполкома посторонние люди. Занялся этим вопросом городской партийный контроль. Каждый вечер вызывали Михаила Павловича на˚протяжении длительного времени, и˚он˚должен был докладывать (хотя Михаил Павлович˚уже и˚не~был председателем профкома), что в˚этом его вины никакой нет.

Даже˚ставился вопрос на˚партийном собрании о˚внесении взыскания Михаилу Павловичу. Фактически это была перестраховка, боязнь за˚себя со˚стороны ректора Ю.\,К.~Неумывакина и˚секретаря парткома А.\,А.~Варламова. Члены партии это не~поддерживали, взыскание не~было вынесено, то˚есть подошли объективно, за˚что Михаил Павлович им˚был очень благодарен. Вопрос о˚посторонних людях в˚кооперативе Михаил Павлович неоднократно ставил на˚заседании профсоюзного комитета и˚перед председателем кооператива, что было зафиксировано протокольно.

Иногда Михаилу Павловичу приходилось отвечать за˚действия других, к˚которым не~имел отношения. За˚минут, примерно, 40 до˚торжественного собрания, посвящённого 200\=/летию со˚дня основания института (21~мая 1979~года) приехал в˚институт начальник Главного управления высшего и˚среднего сельскохозяйственного образования МСХ СССР Иван Павлович Макаров. Ректора И.\,В.~Дегтярёва на˚месте не~было. Секретарь ректора вызвала Михаила Павловича в˚ректорат. Вскоре появился И.\,В.~Дегтярёв. На˚вопрос начальника Главка: «Всё ли˚готово к˚торжественному собранию?» "--- последовал ответ ректора, что Михаил Павлович срывает его. Впервые Михаил Павлович узнал, что должен подготовить это торжественное собрание и вести его. Начальник Главного управления заметил, что у˚тебя всегда «стрелочник» виноват. Почему\=/то этот вопрос был решён неожиданно, хотя программа празднования 200\=/летия МИИЗа была в˚какой\=/то степени известна. Предстояло хотя бы˚разложить по˚рангу поступившие в˚адрес института поздравления, приветственные адреса и˚написать какой\=/то сценарий. Большую помощь Михаилу Павловичу оказал в˚этом проректор по˚учебной работе В.\,Х.~Улюкаев.

\begin{figure}[h]
\includegraphics[width=\textwidth]{private/KrasnoeZnamya_200LetMiiz}
\caption[Вручение Красного трудового знамени в˚честь 200\=/летия МИИЗа. М.\,П.~Шубич крайний справа (изображение обрезано). 1979~год]{Вручение Красного трудового знамени в˚честь 200\=/летия МИИЗа. М.\,П.~Шубич крайний справа (изображение обрезано). 1979~год\footnotemark}
\label{fig:KrasnoeZnamya_200LetMiiz}
\end{figure}
\footnotetext{Источник заимствования "--- личный фотоархив М.\,П.~Шубича.}

За˚изготовление нагрудного знака «200~лет МИИЗ» и˚настольной памятной медали отвечал профком, который с˚поставленной задачей успешно справился.

С~начальником Главного управления высшего и˚среднего сельскохозяйственного образования МСХ СССР И.\,П.~Макаровым Михаил Павлович встретился повторно в˚декабре 1979~года. Во˚время учебы профсоюзного актива впоследствии ещё несколько раз встречались. Его интересовало наше мнение в˚отношении кандидатуры будущего ректора МИИЗ. Михаил Павлович порекомендовал Ю.\,К.~Неумывакина, который защитил докторскую диссертацию. Это решение было зафиксировано протокольно на˚заседании профкома. И.\,В.~Дегтярёв не~справлялся со˚своими обязанностями. Вдобавок ещё пьянствовал с˚заведующим кафедрой В.~Десятовым.

Профком при˚председателе М.П.Шубиче занимался очень разнообразной деятельностью. Запомнились выездные учёбы, которые проводили совместно с˚партийной организацией. Приходилось Михаилу Павловичу заниматься и˚неблагоприятными работами: осуществлять контроль за˚деятельностью администрации(он был председателем комиссии), быть председателем комиссии по˚борьбе с˚пьянством и˚алкоголизмом в˚институте и членом этой комиссии Бауманского района.

Участие в˚работе городского комитета профсоюза работников сельского хозяйства (1978\--1980~годы) в˚каком\=/то отношении позволяло Михаилу Павловичу достигать в˚профсоюзной работе более ощутимых результатов, знакомиться с˚работой других организаций.






\section*{Заочный факультет}
\label{sec:correspondenceFaculty}
\addcontentsline{toc}{section}{\nameref{sec:correspondenceFaculty}}

Работа декана заочного факультета (1982\--2000~годы) при полной нагрузке по˚кафедре отнимала много времени и˚сил. Большое внимание Михаил Павлович уделял контролю за˚проведением и˚качеством занятий, успеваемости студентов\-/заочников.

По˚инициативе Михаила Павловича на˚заочном факультете удалось внедрить подготовку вместо одной специальности инженера\-/землеустроителя ещё четыре: юриспруденция "--- 1993~год, земельный кадастр и˚городской кадастр "--- 1994~год, экономика и˚управление на˚предприятии (операции с˚недвижимым имуществом) "--- 1994~год. Заочный факультет под руководством Михаила Павловича выпустил более 200~специалистов высшей квалификации без отрыва от˚производства. Им лично подготовлено 8~учебных планов для обучения без отрыва от˚производства по˚4~специальностям, которые широко использовались в˚сельскохозяйственных вузах России. 

\begin{figure}[h]
\includegraphics[width=\textwidth]{private/dekanShubich_2000}
\caption[Декан заочного факультета М.\,П.~Шубич (в~кабинете), 2000~год]{Декан заочного факультета М.\,П.~Шубич (в~кабинете), 2000~год\footnotemark}
\label{fig:dekanShubich_2000}
\end{figure}
\footnotetext{Источник заимствования "--- личный фотоархив М.\,П.~Шубича.}

Министерство сельского хозяйства дало согласие на˚организацию в˚рамках университета заочного института, но˚ректор С.\,Н.~Волков высказался против, видимо боялся, что Михаил Павлович возглавит его.

С~неуспевающими студентами М.\,П.~Шубичем проводились собеседования, устанавливал сроки ликвидации задолженностей, которые заносились в˚учётные карточки студентов. При˚неликвидации их˚беспричинно в˚установленные сроки студент предупреждался, а˚затем отчислялся. При˚этом в˚обязательном порядке оформлялась служебная записка с˚пояснениями его отчисления.

При˚проверке работы заочного факультета народным контролем не~было выявлено никаких серьёзных нарушений. Председатель Иванов спрашивает у Михаила Павловича: «Что? У вас нет никаких недостатков в˚работе? Назови хотя бы˚один, я не~буду это записывать протокольно и˚оглашать». Ответно было сказано, что недостатки имеются, мы˚знаем о˚них и˚стараемся исправлять, например, много  студентов отчисляется за˚неуспеваемость. На˚совещании, проводимом на˚городском уровне, прозвучало, что декан всех отчислил. В~тот же˚день явилась к Михаилу Павловичу корреспондентка газеты Московский комсомолец с˚вопросом: «Правда, что ты˚отчислил почти всех студентов факультета?» При˚показе фактического состояния она с˚Михаилом Павловичем согласилась и˚сказала, что нет никаких нарушений. Оказывается, честное и˚справедливое признание может иногда негативно ударить по˚тебе, хотя даются какие\=/то заверения. 
Заочный факультет МИИЗа под руководством Михаил Павловича являлся центром по˚подготовке учебно\-/методической литературы по˚специальности «землеустройство» для всех сельскохозяйственных вузов страны. Учебно\-/методическая литература, прошедшая через заочный факультет, использовалась также на˚очных факультетах вузов.
На˚протяжении всех лет работы деканом факультета Михаил Павлович являлся членом Учебно\-/методического центра заочной формы обучения по˚сельскохозяйственной специальности Главного управления высших учебных заведений Министерства сельского хозяйства СССР. На˚семинарах, которые проводились ежегодно в˚различных городах страны, вырабатывались рекомендации. Они являлись основой для ведения учебного процесса на˚заочных факультетах сельскохозяйственных вузов.

Курировала эту работу от˚Главка Раиса Мироновна Цыбулевская. Участие в˚работе этого центра и˚проверки состояния учебного процесса в˚других вузах по˚линии Центра позволяли знакомиться с˚работой других факультетов вузов, перенимать что-то хорошее, прогрессивное.

\section*{Проверки и аттестации вузов}
\label{sec:deskwork}
\addcontentsline{toc}{section}{\nameref{sec:deskwork}}

По˚линии Министерства образования и˚Учебно\-/методического Центра МСХ СССР Михаилу Павловичу приходилось  участвовать в˚составе комиссии по˚проверке учебного процесса сельскохозяйственных вузов России, аттестации и˚аккредитации вузов. В~составе комиссии он участвовал в˚проверке состояния учебного процесса в˚Омском, Красноярском сельскохозяйственных вузах, Абаканском консультативном центре и~т.~д.

При˚посещении г.~Абакана изменилось представление Михаила Павловича о˚Сибири. Раньше он думал, что Сибирь очень суровый край, и˚мало какие растения там произрастают. Оказалось, что здесь на˚корню созревают помидоры, кукуруза на˚зерно, чего трудно добиться в˚Подмосковье. Абакан "--- красивый город. Жители хвалились, что отдельно от˚жилой вынесена производственная зона, но˚оказывается, не~учли направление вредоносных ветров, разместив жилую зону с˚наветренной стороны. Проехали на˚машине вдоль реки Енисей: туда с˚одной стороны, обратно "--- с˚другой. Побывали в˚Шушенском "--- место ссылки В.\,И.~Ленина. Экскурсовод подробно изложила о˚прелестях этого края, доме А.\,Д.~Зырянова, где жил в˚ссылке и˚как жил В.\,И.~Ленин. Действительно, природа оказалась превосходной. Михаил Павлович в˚порыве восторга даже заявил экскурсоводу: «Не˚могли бы˚меня на˚недельку оставить здесь в˚заточении?» Последовал ответ: «Многого хочешь». Посетили Саяно\-/Шушенскую ГЭС. Полдня Михаил Павлович бродил по˚плотине. При˚этом возмущался, что никто не~спросил зачем он˚здесь, что делает? Ведь˚это охраняемый, серьёзный объект. Видимо, наша безалаберность может привести, а˚иногда и˚приводит к˚серьёзным последствиям. Хотя˚здание штаба недалеко, а на˚холме установлена ракетная установка. По˚дороге обратно заехали в˚карьер добычи мрамора. Добычу его производят двумя способами: взрывом и˚распиливанием на˚плиты. При˚первом способе образуется много мраморной крошки. Невдалеке от˚карьера из неё выложено 500\,м˚дороги. При˚посещении дороги японец воскликнул: «Надо˚же, дороги строят из˚золота!» Мрамор залегает разноцветный. Действительно это богатство, которым следует дорожить. Возмущало Михаила Павловича варварское обращение с˚кедром (вырубается, кучи гниют среди тайги). Ведь˚кедровая сосна является ценным материалом для хозяйственных целей, дающая в˚шишках съедобные семена\-/орехи, из˚которых изготавливают кедровое масло. В~лесу оказалось много различных грибов, но˚жители собирают лишь белые.

Неоднократно Михаил Павлович назначался председателем государственных аттестационных комиссий (ГАК). Был председателем ГАК на˚землеустроительных факультетах Грузинского сельскохозяйственного института и˚Белорусской сельскохозяйственной академии. 

Сейчас даже с˚некоторой иронией Михаил Павлович вспоминает, когда доклады и˚ответы на˚вопросы студентов в˚Тбилисском сельскохозяйственном институте ему˚приходилось выслушивать на˚грузинском языке. Отчёт председателя ГАК заслушивался на˚Совете института. На˚реплику одного профессора грузинского института: «Хорошо бы˚вам изучить грузинский язык», "--- последовал ответ: «Вам в˚соответствии с˚постановлением ЦК следует изучить русский язык». Больше никаких вопросов не~последовало. После˚защиты дипломных проектов и заслушивания отчёта повезли через перевал в˚западную Грузию и˚пришлось участвовать в˚грузинской свадьбе.

В~перестроечное время многие вузы оказались в˚очень затруднительном положении. Изменилось отношение студентов к˚занятиям. Отдельные из˚них стали недостаточно изучать дисциплины учебного плана, а˚начали больше думать о˚работе с˚целью выживания. Зарплата профессорско\-/преподавательского состава была низкой, ниже, чем у˚неквалифицированного работника, что не~способствовало привлечению к˚учебному процессу талантливой молодёжи, не~стимулировало работать с˚полной отдачей сил и˚энергии, а˚вынуждало думать о˚дополнительных подработках. Снизилась воспитательная работа среди молодёжи. Зарплата должна стимулировать профессорско\-/преподавательский состав добросовестно выполнять свои обязанности, повышать профессиональный уровень. Из-за отсутствия средств преподаватели вуза лишились командировок по˚научно\-/исследовательской работе, что отрицательно сказалось на˚эффективности исследований и˚повышении их˚квалификации. 

Перестройка привела к˚деградации сельскохозяйственного производства, фактически к˚ликвидации землеустроительной службы и˚приостановлению производственных практик. 

Неблагодарной и˚тягостной оказалась для Михаила Павловича работа ответственным секретарём приёмной комиссии ГУЗа (2000\--2003~годы). Однажды он шифровал письменные контрольные работы по˚математике, вдруг начали стучать в˚дверь (дверь в˚этих случаях он всегда закрывал на˚замок и в˚кабинете оставался один). На˚вопрос: «Кто стучит?» "--- ответа не~последовало. Он˚подумал, что заведующий кафедрой математики пришёл за˚очередной «порцией» зашифрованных работ, открыл дверь. В~это время в˚кабинет рвалась женщина. Михаил Павлович встал у˚порога и её не~пускал, так как на˚столах были разложены контрольные работы. Из-за˚спины женщины последовал резкий удар кулаком Михаилу Павловичу по˚лицу, после чего он еле устоял на˚ногах. Оказалось, нанёс удар сын доцента Смоленского сельскохозяйственного института, который после удара сбежал. Он решил отомстить за то, что по˚письменному экзамену по˚математике получил неудовлетворительную оценку. Зло своё почему\=/то вылил на˚Михаила Павловича. Длительное время Михаилу Павловичу пришлось ходить с˚чёрными кругами под глазами и˚фактически закрытым левым глазом. Он обратился в˚поликлинику, где сделали заключение, и˚передал дело в˚суд. Но,~к˚сожалению, абитуриент не~понёс никакого наказания, откупился, что часто бывает, особенно в˚перестроечное время. А~мотивировка была та, что в˚это время не~было свидетелей, хотя их˚было достаточно. Не˚помогло Михаилу Павловичу в˚этом вопросе и˚ФСБ. Вот Вам объективность и˚справедливость на˚государственном уровне.

В˚соответствии с˚распоряжением Рособрнадзора Михаил Павлович Шубич являлся экспертом по˚анализу содержания и˚качества подготовки по˚образовательным программам и˚аккредитовал в˚2012~году две специальности в˚ФГБОУ ВПО «Ивановская государственная сельскохозяйственная академия имени академика Д.\,К.~Беляева» и˚четыре специальности в˚ФГБОУ ВПО «Пензенский государственный университет архитектуры и˚строительства». 

Из˚публикаций о˚Михаиле Павловиче Шубиче следует, что его жизнь "--- это пример беззаветного служения Родине, землеустроительной науке, высшему землеустроительному образованию. Михаил Павлович Шубич проявил себя как высококвалифицированный педагог, отличный методист, крупный учёный, активный общественный деятель и˚организатор. На˚всех участках работы он˚относился ответственно, проявлял инициативу, самостоятельность, успешно справлялся с˚любыми делами (проф., д.\,э.\,н., заслуженный работник культуры РФ, В.\,В.~Косинский, журнал «Землеустройство, кадастр и˚мониторинг земель» №~1, 2018~г., "--- с.~73). 

\begin{wrapfigure}{O}{.4\textwidth}
\centering
\includegraphics[width=.35\textwidth]{volkovSN}
\caption[Поздравление ректора ГУЗа С.\,Н.~Волкова выпускников 2007 года на церемонии вручения дипломов]{Поздравление ректора ГУЗа С.\,Н.~Волкова выпускников 2007 года на церемонии вручения дипломов\footnotemark}
\label{fig:volkovSN}
\end{wrapfigure}
\footnotetext{Автор: Ю.\,Н.~Гаврюшкина, 29.06.2007.} 

Эффективность работы преподавателя кафедры во˚многом зависит от˚организации и˚руководства кафедрой заведующего и˚его заместителей. Заведующим кафедры землеустроительного проектирования (землеустройство) после С.\,А.~Удачина были В.\,Д.~Кирюхин, В.\,П.~Троицкий, С.\,Н.~Волков. С.\,Н.~Волков одновременно занимает должность ректора университета и˚заведующего кафедрой. Фактически С.\,Н.~Волков кафедрой не~занимается, а˚отдал на˚откуп заместителям заведующего. Заместителями заведующего кафедрой были "---  Е.\,Н.~Первова, М.\,П.~Шубич, В.\,Н.~Сёмочкин, В.\,В.~Пименов, а в˚последнее время избрана Л.\,Е.~Петрова.

13~октября 2017~г. Михаил Павлович Шубич отправлен на˚заслуженный отдых. Он вполне с˚этим согласен, но˚его крайне возмущает, как проведено это действие. Заведующий кафедрой С.\,Н.~Волков не~изволил вызвать на˚кафедру, сказать несколько тёплых слов, а˚подготовил втихаря уведомление, передал его в˚отдел кадров и˚распорядился вручить М.\,П.~Шубичу под˚расписку. Руководитель с˚нормальными человеческими чертами со˚своими сотрудниками, а˚тем более профессором, заслуженным землеустроителем РФ, человеком, отдавшим всю свою жизнь на˚развитие вуза(проработал в˚МИИЗе, ГУЗе 47~лет), так не должен поступать. Это характеризует С.\,Н.~Волкова как руководителя и˚человека, от˚него иного не~следовало ожидать. По˚этому поводу звонил Михаилу Павловичу профессор, д.\,э.\,н., академик РАН, бывший министр сельского хозяйства В.\,Н.~Хлыстун и˚высказал своё неодобрение такими действиями С.\,Н.~Волкова.

\todo[Ред]{Абзац не к месту. Давай в интервью перенесем}
Вузы в˚настоящее время готовят дипломированных бакалавров и˚магистров. По мнению М.\,П.~Шубича, для страны наиболее целесообразно вести подготовку высококвалифицированных специалистов разных специальностей вместо бакалавров. Бакалавр фактически является неполноценным специалистом, который не~может эффективно решать поставленные задачи. Также целесообразно отказаться от решения Болонской конвенции, которуею подписала Россия, и˚готовить полноценных специалистов в˚течение пяти лет. Наверняка, данную позицию полностью поддержат работодатели. На˚рынке труда более полезным и˚востребованным окажется высококвалифицированный специалист. Для˚подготовки таких специалистов высшему образованию на˚современном этапе необходимы грамотные преподаватели, владеющие современными технологиями педагогической деятельности, постоянно ведущие научно\-/исследовательскую работу.

Жизненный путь Михаила Павловича был тернист, хотя в˚большинстве своём его окружали  хорошие люди, с˚которыми было полное взаимопонимание. К~людям он всегда относится с˚добротой, делится и˚помогает всем, хотя отдельные из˚них не~всегда впоследствии были благодарны за˚эту помощь. К~студентам, особенно трудолюбивым, всегда относился с˚любовью, объективно оказывал помощь в˚учёбе и˚освоении учебного материала.
