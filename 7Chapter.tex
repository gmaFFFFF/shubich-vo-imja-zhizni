\chapter{Жизнь с МИИЗом (ГУЗом)}

Учёба в˚аспирантуре была напряжённой, интересной. Аспиранты не~только готовились к˚сдаче кандидатского минимума, собирали необходимые материалы и˚работали над диссертацией, но и˚развивались в˚культурном отношении: коллективно посещали музеи, выставки, театры, кинотеатры, бани. Большую роль в˚формировании личности аспиранта сыграл неуставной совет "--- «домогающиеся умышленники МИИЗа», который был организован аспирантами. 

На˚своём заседании «совет» обсуждал как вести себя во˚время защиты диссертации тому или иному аспиранту, его работу над диссертацией, готовность к˚сдаче кандидатского экзамена, оказывал помощь нуждающимся, стимулировал˚успешную сдачу экзамена, защиту диссертации, рождение ребёнка, собирали взносы. Считаю, что он˚сыграл определённую положительную роль.

Вспоминается такой случай: Сергей Александрович как-то пригласил Мишу "--- аспиранта к˚себе домой, посадил в˚своём кабинете за˚стол и˚предложил прочесть доклад, с˚которым он˚должен выступать на˚международном симпозиуме, и˚высказать свои замечания.

Внимательно прочёл доклад и˚тут же˚последовал вопрос: «Понравился?» По˚своей наивности, Миша ответил, что в˚одном месте не~понравилось, указав что именно не~понравилось и˚почему. Ожидал, что впоследствии академик ВАСХНИЛ отругает, устроит взбучку, но˚Сергей Александрович очень спокойно сказал: «Мне тоже здесь не~нравится, переделаю».

С~того же˚дня он˚ежедневно вечером (в~одно и˚тоже время) назначил прогулки. Ходили по˚ближайшим от˚дома улицам. Эти прогулки были очень интересными, познавательными и˚поучительными. Он рассказывал сколько сил и˚энергии пришлось ему потратить за˚сохранение и˚развитие землеустройства, создание научной школы, борьбе с˚различными вредными теориями. В~свою очередь он˚интересовался бытовыми условиями Миши, семейными делами, работой над диссертацией.

По˚мере готовности диссертации позвонил С.А. Удачин и˚сказал, чтобы принёс для читки. Вечером однажды позвонил и˚сказал: «Завтра будем обсуждать, подготовь автореферат». Ночью пришлось потрудиться. Диссертацию одобрили с˚первого раза обсуждения. Происходила рассылка автореферата. Учёный секретарь О.~Козырева сначала запротивилась из-за допуска как бы в˚автореферате секретные данные расчётного количества зерна, которые может произвести область. С.\,А.~Удачин своевременное её˚поправил. Защита состоялась 25~декабря 1969~года на˚объединённом учёном совете разных специальностей. Защита прошла успешно и˚была присвоена учёная степень кандидат экономических наук по˚специальности 08594 "--- «экономика, организация и˚планирование народным хозяйством». 

Место будущей работы определили ректор института канд. техн., наук проф. Н.\,Д.~Ильинский и˚заведующий кафедрой землеустроительного проектирования д-р экон. наук, проф., академик ВАСХНИЛ Сергей Александрович Удачин. После˚защиты диссертации ректор Н.\,Д.~Ильинский вызвал и˚сказал, что есть решение зачислить меня на˚кафедру районной планировки, где будет обеспечен дальнейший рост "--- заведование кафедрой. В~этот же˚день встретил С.\,А.~Удачин, который сказал: «Этому не~быть, зачислен будешь на˚кафедру землеустроительного проектирования» и на˚следующий день принесли из˚Министерства сельского хозяйства СССР направление по˚распределению на˚кафедру землеустроительного проектирования. Таким образом, была определена дальнейшая судьба Михаила Павловича, которого зачислили на должность ассистента. 

С~Н.\,Д.~Ильинским приходилось ещё много раз встречаться, будучи председателем комиссии по˚контролю за  деятельностью администрации. Это был обстоятельный, очень уравновешенный, спокойный человек, ректор «законник», строго соблюдал все требования делопроизводства и˚очень серьёзно относился к˚должности ректора. Имел влечение к˚игре в˚шахматы, будучи кандидатом в˚мастера по˚100\=/клеточным шашкам. Недовольство выражал при проигрывании партии в˚шахматы.

На˚кафедре земпроектирования в˚это время работали видные советские учёные: академик ВАСХНИЛ, профессор С.\,А.~Удачин, профессора: Н.\,И.~Прокуронов, Н.\,Н.~Бурихин, Г.\,В.~Чешихин, Я.\,М.~Цфасман, доценты: А.\,И.~Гавриленко, В.\,Д.~Кирюхин, К.\,М.~Кирюхина, Е.\,Н.~Первова, А.\,В.~Юнюков, Б.\,Н.~Ширин, Н.\,И.~Ярмак, В.\,П.~Троицкий, Г.\,В.~Симонов и˚другие. На˚первых порах среди таких светил учёных чувствовал себя Михаил Павлович как-то неуютно, но˚несмотря на˚это поручили чтение лекций, руководство курсовым и˚дипломным проектированием, приём экзаменов, ведение практических занятий. Всё˚это меня как-то охладило и˚заставила работать над собой, тщательно готовить к˚любым видам учебных занятий.\todo[Текст]{Непонятно в какой момент Миша превратился в Михаила Павловича}

Немного следует изложить о˚первых дипломниках. 1970~год начало работы на˚кафедре в˚должности ассистента. Появились первые дипломники. Одной из˚таких дипломниц была Алевтина Алексеевна Мохнаткина, девушка с˚выраженным характером, трудно поддающаяся убеждениям. Считала, что если принято какое\=/то решение, то˚иного более целесообразного и˚эффективного не~может быть. Предлагаю ей˚новое решение по˚организации использования земли в˚колхозе «Россия» Косимовского района Рязанской области (объект её˚дипломного проекта) в˚отличие от˚принятого. В~ответ заявляет: «Над˚этим решением думали десять голов, неужели Ваша голова умнее их». Долго пришлось её˚убеждать более объективно смотреть на˚вещи и˚правильно их˚оценивать. Защитилась она на˚отлично. После˚защиты преподнесла букет цветов, внутри которых записка на˚вырезанном куске ватмана следующего содержания: «С~Вас получится очень хороший преподаватель, воспитатель. Вы сумели совершенно изменить меня».

Любой человек, особенно молодой, всегда должен заранее обдумывать свои действия, поступки и˚делать выводы, к˚чему это приведёт впоследствии. Делать выводы после свершившегося факта бывает бесполезным. Девушка Надя была на˚преддипломной практике в˚колхозе «Чёрная речка» Сапожковского района Рязанской области. Приезжаю в˚хозяйство (преподаватели разъезжали по˚командировкам, чтобы оказать помощь студентам в˚разработке проекта и˚сборе материала для дипломного проектирования). На˚квартире с˚ней проживали ещё две девушки "--- молодые специалисты. Захожу в˚квартиру. Посреди˚стоит чемодан. Надя собралась уезжать. На˚вопрос: «В~чем дело?» "--- ответа не~последовало. Проживающие с˚ней девушки ничего не~сказали, лишь загадочно переглянулись. Впоследствии выясняю, что поздно вечером заехал на˚машине председатель колхоза и˚предложил Наде взять материалы для утверждения в˚районе разработанного проекта. Девушка быстро собралась и не~подумала, что рабочий день уже закончился. Председатель полночи возил её по˚лесу. В~результате она решила убежать с˚практики. Хотелось побеседовать с˚председателем колхоза, но он˚всяким образом избегал подобной встречи. Поскольку˚встреча не~состоялась, то˚решено было изложить в˚газете «Приокская правда» о˚пренебрежительных действиях председателя к˚проекту внутрихозяйственного землеустройства и˚нарушениях границ его перенесения. По˚результатам статьи председатель колхоза был снят с˚должности. Всё, что удалось сделать за˚безнравственный проступок председателя колхоза. На˚сердце долго у˚Миши оставался горький осадок, к˚чему могут привести безрассудные, необдуманные наши действия.

Вспоминается такой забавный случай как первое занятие на˚заочном факультете. Прихожу в˚группу, а˚там все взрослые люди, опытные производственники, руководители отрядов, отделов, начальники управлений, в˚основном из˚Центрально\-/Чернозёмной зоны (ранее на˚заочный факультет принимали лишь лиц, работающих по˚специальности и˚родственным специальностям). Они как-то встретили настороженно молодого преподавателя и˚решили устроить проверку, стали задавать различные вопросы. Естественно, в˚этой ситуации выручил меня жизненный опыт (работа в˚геофизических партиях «Главнефтегеофизика», служба в˚рядах Советской Армии, работа по˚специальности в˚должностях инженера\-/землеустроителя и˚начальника отряда в˚Рязанской землеустроительной экспедиции). Я~спокойно ответил на˚все их˚вопросы и˚более того задал ряд вопросов им˚землеустроительного характера, но˚ответа не~получил. Всё˚это очень понравилось им и˚как-то сблизило. Впоследствии и˚после окончания МИИЗа, будучи в˚г.~Москве, они постоянно заходили ко˚мне с˚беседами, различными советами. 

Пришлось однажды давать печальный совет. Как-то ко˚мне в˚общежитие зашёл директор Тамбовского филиала ЦЧОГипрозем Владимир Никитич Покидышев. У~них в˚семье была неприятность. У~ребёнка был порок сердца, предстояла в˚Москве операция. Они должны были дать расписку на˚операцию. Они колебались, как поступить. С~этим вопросом он˚обратился ко˚мне. Конечно, было трудно что-то посоветовать. Подумавши, я˚сказал, что они должны согласиться на˚операцию. При˚неприятном исходе, они ещё молодые и в˚состоянии родить ребёнка. Иначе˚придётся мучиться очень долго. Сделали операцию, исход оказался печальным. Через˚год у˚них родился ребёнок. Пришёл ко˚мне опять В\,Н.~Покидышев и˚поблагодарил за˚совет. Конечно, понимал, что для них это большое горе, но˚пришлось подойти, заглядывая в˚будущее.

Бытовые неудобства (я~на˚первых порах жил в˚общежитии, а˚семья в˚Рязани), трудности с˚пропиской и˚жильём вносили какую-то неуверенность, тревогу и˚задавался вопросом: «Что˚делать дальше?» Следует отдать должное С.\,А.~Удачину, несмотря на˚свою большую занятость, он по\=/человечески относился к˚людям, проникался их˚страданиями. Подходил ко˚мне и˚говорил: «На˚первых порах все трудности необходимо преодолеть, в˚дальнейшем всё будет хорошо». Это высказывалось как-то «по\=/отцовски» и˚вселяло в˚меня надежды на˚будущее.

Следует отметить, что С.\,А.~Удачин, естественно, был крупным учёным, теоретиком социалистического землеустройства, основателем советской землеустроительной научной школы землеустройства. Будучи единственным академиком\-/землеустроителем ВАСХНИЛ сумел отбить все нападки на˚землеустройство, добился сделать его государственным мероприятием, постоянного его развития. Он непосредственно принимал участие в˚подготовке всех нормативно\-/правовых актов по˚землеустройству и˚землепользованию. 

С.\,А.~Удачин был очень спокойным, внимательным, объективным и˚уравновешенным человеком. При˚принятии любого решения первоначально выслушивал мнения профессорско\-/преподавательского состава кафедры. 

Мне всегда было очень приятно и˚познавательно слушать его при ежедневных прогулках по˚улицам, недалеко от˚дома, когда проживал в˚общежитии.

Но˚считаю себя виноватым перед ним из-за того, что не~прибыл к˚нему, когда мне позвонила жена с˚просьбой явиться к˚нему (должен был быть на˚каком\=/то совещание по˚общественной работе). Отложил свой приход на˚следующий день, а˚завтра узнаю, что его не~стало.

Я.\,М.~Цфасман является крупным учёным советского землеустройства. По~натуре был добрым человеком, всегда трепетно относился при защите курсовых проектов на˚комиссии в˚его подгруппе. В~этих случаях говорил: «Ты видишь ошибки на˚ходу, а я их не~замечаю». В~день похода на˚дачу (Чкаловское) "--- последний день его жизни зашёл ко˚мне в˚21~аудиторию (кабинет декана заочного факультета) и˚очень долго беседовал, даже предлагал деньги на˚приобретение автомашины. Я~отказался от˚подобной услуги из-за того, что не~сумею быстро вернуть. По˚возвращению пообещал угостить меня яблоками из˚своего сада. Фактически получилось, что он˚как бы˚предвидел что-то нехорошее. По˚дороге на˚станцию он˚скончался.

\begin{figure}[h]
\includegraphics[width=\textwidth]{private/dekanShubich_2000}
\caption{Декан заочного факультета М.\,П.~Шубич (в~кабинете), 2000~год}
\label{fig:dekanShubich_2000}
\end{figure}

Симпатизировал мне доцент Б.\,Н.~Ширин своей неподкупностью, объективной требовательностью к˚студентам, справедливостью и˚честностью. Это был высококвалифицированный педагог, хороший воспитатель.

Доцент Е.\,Н.~Первова была очень выдержанным, тактичным и˚требовательным преподавателем, хорошим методистом, воспитателем, умелым организатором производственных практик, всегда доходчиво излагала материал на˚практических занятиях. Длительное время она была заместителем заведующего кафедрой земпроектирования.

Доцент А.\,И.~Гавриленко был требовательным преподавателем, не~всегда считался с˚мнением других членов кафедры, трудно пробиваемая натура. Будучи проректором по˚учебной работе, при подаче мной документов на˚должность доцента и˚положительном заключении кафедры сказал: «Мы ходили в˚ассистентах по семь и˚больше лет, а ты˚хочешь быть доцентом уже на˚третий год». Фактически продержал мои документы ещё год, не~давая им ходу. Пришлось ехать в˚ВАК, где разъяснили, что никаких подобных сроков не~существует. Если˚человек заслуживает, то˚можно выдвигать в˚любое время.

Поскольку˚в то˚время было много опытных, высококвалифицированных педагогов, то˚негласно ходило мнение, что с˚написанием докторских диссертаций должна наступить очередь. Думается, это было ошибочное мнение, которое тормозило и не~способствовало повышению квалификации молодым.

Наряду˚с основной работой на˚кафедре много сил и˚времени занимала общественная работа. С~1975 по˚1980~годы был председателем объединённого профсоюзного комитета МИИЗа, в˚1978\==1980~годы "--- членом горкома профсоюза работников сельского хозяйства, членом партийного бюро МИИЗа. Будучи заместителем заведующего кафедрой, председателем профсоюзного комитета, деканом заочного факультета и˚ответственным секретарём приёмной комиссии, учебную и˚другие виды нагрузок по˚кафедре, вёл в˚полном объёме, не~снижая нагрузки. К~общественной работе относились также ответственно. Старался быть объективным и˚справедливым, хотя иногда это принималось негативно. По˚поводу принятия одного решения на˚заседании профсоюзного комитета вызвал ректор, профессор И.\,В.~Дегтярёв в˚свой кабинет и˚требовал его отмены. Но˚когда я˚ответил, что решение принималось коллективно, единогласно и не~могу его отменить, он˚схватил меня за˚грудь. Мне в˚данной ситуации пришлось как-то отвечать взаимностью и˚только тогда это было воспринято с˚пониманием.

На˚следующий день ректор пришёл в˚56~аудиторию (кабинет профкома), вместо извинения сказал: «Мы˚оба погорячились». Я~посчитал это высказывание своеобразным извинением.

Не˚поддержал В.\,И.~Дегтярёв меня в˚получении неиспользуемых площадей для института в˚посёлке Салтыковка после разморозки труб (такие площади мной были подобраны совместно с˚начальником отдела по˚распределению жилья города Балашиха), что не~способствовало нормальному ведению учебного процесса. После˚разморозки труб лекции и˚практические занятия пришлось проводить в˚пальто, зимней шапке, а˚отдельные занятия отменились.

Заместителем председателя профсоюзного комитета длительное время была Пальмира Фёдоровна Белянцева. Председатель профсоюзного комитета работал на˚общественных началах, а˚заместитель получал зарплату. Муж П.\,Ф.~Белянцевой работал в ЦK КПСС\footnote{ЦК КПСС "--- Центральный комитет Коммунистической партии Советского Союза "--- высший партийный орган в˚промежутках между съездами партии.}. Поэтому она чувствовала себя независимой, всемогущей. По˚натуре это было своенравная женщина, старалась полностью подчинить председателя комитета. Во˚время работы председателем М.\,П.~Шубич выявил, что оказанную профсоюзным комитетом помощь, студенты не~всегда получали. Поэтому решил её˚убрать, хотя ни˚одному председателю не~удавалось это сделать. Во˚время очередной выборной кампании, списки членов нового профкома не~распечатал и не~огласил, а˚сделал это во˚время перерыва. После˚перерыва огласил. П.\,Ф.~Белянцева восседала в˚Президиуме. Но˚в списках она не~значилась. Делегаты проголосовали единогласно, то˚есть без П.\,Ф.~Белянцевой 

Назавтра она в˚кабинете профкома 56~аудитории устроила страшный скандал, вызвала мужа. Муж оказался очень тактичным человеком. По˚приезду в˚институт, не~заходя в˚кабинет, вызвал меня в˚вестибюль. В~беседе мы˚нашли взаимопонимание, и он˚уехал. Но~П.\,Ф.~Белянцева на˚этом не~остановилась, обратилась в˚следственные органы. Следователю при вызове пришлось подробно объяснять создавшуюся ситуацию. На˚этом всё закончилось. Так˚пришлось избавиться от˚коварной женщины. 

Будучи председателем профкома приходилось заниматься многими вопросами в˚интересах студентов и˚сотрудников «выколачивать» путёвки в˚дома отдыха и˚санатории, автомашины и˚садовые участки для сотрудников, оформлением демонстрационных колонн, работой кружков, столовой, контролем качества пищи, выбиванием участка и˚капитальных вложений для строительства дома для сотрудников, состоянием дел в˚общежитии, вопросами техники безопасности в˚МИИЗе, подготовкой ежегодной учёбы партийно\-/профсоюзного актива, профсоюзными взносами, проводить культурно\-/массовую работу, подготовкой к˚празднованию 200\=/летия вуза и˚так далее. Многие виды этих работ были неблагодарны, требовали много сил и˚нервов. 

Мише удалось выбить 10\,гектар земли в˚Жуковском районе Калужской области и˚один гектар в˚Павлово\-/Посадском районе Московской области под дачные участки для сотрудников института, а˚также деньги в˚Министерстве Сельского хозяйства СССР под строительство дома для сотрудников института и˚деньги для долевого строительства. За˚счёт денег на˚долевое строительство удалось поселить четыре семьи сотрудников в˚г.~Балашихе.

Для˚получения земельного участка и˚разрешение на˚строительство дома пришлось затратить несколько месяцев ежедневных хождений в˚Моссовет\footnote{Московский городской совет (Моссовет) "--- высший орган государственной власти в˚Москве с˚1917 по˚1993~год, предшественник нынешней Мосгордумы.}. Это был год олимпиады (1980~год), что усугубляло положение дел и˚требовала много сил, энергии и˚большой настойчивости. При˚больших усилиях такой участок был выделен в˚Свиблово и˚началось строительство дома. Жилой дом должен был принадлежать институту и˚лишь 30~\% площади, согласно договору, должны выделить Министерству сельского хозяйства СССР. Из-за недостаточной и не~активной работы в˚этом отношении моего последователя М.\,К.~Недайведова всё было впоследствии переиграно. Нам досталась лишь 30~\% жилой площади. Мише было очень обидно и˚злостно за˚бездействие при таких громадных затраченных усилиях. Хотя˚Миша для себя лично никакой выгоды не~извлекал: не~имел садового участка, не~приобрёл машины (хотя сотрудникам распределял), не~претендовал на˚жильё (хотя его не~имел) и˚должность. Это дало возможность впоследствии ему оправдаться перед Московским городским партийным контролем по˚поводу того, что в˚наш садоводческий кооператив попали посторонние люди, отдельные сотрудники возвели неустановленных размеров постройки и~т.~д. Вызывали много раз, хотя уже не~был председателем профкома.

Иногда Мише приходилось отвечать за˚действия других, к˚которым не~имел отношения. Вспоминается такой случай: за˚минут, примерно, 40 до˚торжественного собрания, посвящённого 200\=/летию со˚дня основания института (21~мая 1979~года) приехал в˚институт начальник Главного управления высшего и˚среднего сельскохозяйственного образования МСХ СССР Иван Павлович Макаров. Ректора И.\,В.~Дегтярёва на˚месте не~было. Секретарь ректора вызвала М.\,П.~Шубича в˚ректорат. Вскоре появился И.\,В.~Дегтярёв. На˚вопрос начальника Главка: «Всё ли˚готово к˚торжественному собранию?» "--- последовал ответ ректора, что М.\,П.~Шубич срывает его. Впервые узнаю, что я˚должен вести это торжественное собрание. Начальник Главного управления заметил, что у˚тебя всегда «стрелочник» виноват. Почему\=/то этот вопрос был решён неожиданно, хотя программа празднования 200\=/летия МИИЗа была в˚какой\=/то степени известна. Предстояло хотя бы˚разложить по˚рангу, поступившее в˚адрес института поздравления, приветственные адреса и˚написать какой\=/то сценарий. Большую помощь оказал проректор по˚учебной работе В.\,Х.~Улюкаев, за˚что ему благодарен.

\begin{figure}[h]
\includegraphics[width=\textwidth]{private/KrasnoeZnamya_200LetMiiz}
\caption{Вручение Красного трудового знамени в˚честь 200\=/летия МИИЗа. М.\,П.~Шубич крайний справа (изображение обрезано). 1979~год}
\label{fig:KrasnoeZnamya_200LetMiiz}
\end{figure}

За˚изготовление нагрудного знака «200~лет МИИЗ» и˚настольной памятной медали отвечал профком, который с˚поставленной задачей и˚успешно справился.

С~И.\,П.~Макаровым встретился в˚декабре 1979~года. Впоследствии ещё несколько раз встречались. Его интересовало наше мнение в˚отношении кандидатуры будущего ректора МИИЗ. Миша порекомендовал Ю.\,К.~Неумывакина, который защитил докторскую диссертацию. Это решение мы˚зафиксировали протокольно на˚заседании профкома. И.\,В.~Дегтярёв не~справилялся со˚своими обязанностями. Вдобавок ещё пьянствовал с˚заведующим кафедрой В.~Десятовым

Вспоминается такой случай, который оставил негативный след до˚настоящего времени. На˚садовом дачном участке, выделенном для сотрудников МИИЗа в˚Балабанове, Жуковском районе Калужской области, оказались по˚решению Калужского облисполкома посторонние люди. Занялся этим вопросом городской партийный контроль. Каждый вечер вызывали меня на˚протяжении длительного времени, и я˚должен был докладывать (хотя я˚уже не~был председателем профкома), что в˚этом моей вины никакой нет.

Даже˚ставился вопрос на˚партийном собрании о˚внесении взыскания М.\,П.~Шубичу. Фактически это было перестраховка, боязнь за˚себя со˚стороны ректора Ю.\,К.~Неумывакина и˚секретаря парткома А.\,А.~Варламова. Члены партии это не~поддерживали, взыскание не~было вынесено, то˚есть подошли объективно, за˚что я им˚был очень благодарен.

Участие в˚работе городского комитета профсоюза работников сельского хозяйства (1978\--1980~годы) в˚каком\=/то отношении мне позволяло достигать в˚профсоюзной работе более ощутимых результатов, знакомиться с˚работой других организаций.

Работа декана заочного факультета (1982\--2000~годы) при полной нагрузке по˚кафедре отнимала много времени и˚сил. Большое внимание уделял контролю за˚проведением и˚качеством занятий, успеваемости студентов\-/заочников.

\begin{wrapfigure}{O}{.4\textwidth}
\centering
\includegraphics[width=.35\textwidth]{volkovSN}
\caption[Поздравление ректора ГУЗа С.\,Н.~Волкова выпускников 2007 года на церемонии вручения дипломов]{Поздравление ректора ГУЗа С.\,Н.~Волкова выпускников 2007 года на церемонии вручения дипломов\footnotemark}
\label{fig:volkovSN}
\end{wrapfigure}
\footnotetext{Автор: Ю.\,Н.~Гаврюшкина, 29.06.2007.}

По˚инициативе М.\,П.~Шубича на˚заочном факультете удалось внедрить подготовку вместо одной специальности инженера\-/землеустроителя дополнительные ещё 4~специальностей: юриспруденция "--- 1993~год, земельный кадастр и˚городской кадастр "--- 1994~год, экономика и˚управление на˚предприятии (операции с˚недвижимым имуществом) "--- 1994~год. Заочный факультет под руководством М.\,П.~Шубича выпустил более 200~специалистов высшей квалификации без отрыва от˚производства. Им лично подготовлена 8~учебных планов для обучения без отрыва от˚производства по˚4~специальностям, которые широко использовались в˚сельскохозяйственных вузах России. 

Министерство сельского хозяйства дало согласие на˚организацию в˚рамках университета заочного института, но˚реактор С.\,Н.~Волков высказался против, видимо боялся, что М.\,П.~Шубич возглавит его.

С~неуспевающими студентами проводились собеседования, устанавливал сроки ликвидации задолженностей, которые заносили в˚учётные карточки студентов. При˚неликвидации их˚беспричинно в˚установленные сроки студент предупреждался, а˚затем отчислялся. При˚этом в˚обязательном порядке оформлялась служебная записка с˚пояснениями его отчисления.

При˚проверке работы заочного факультета народным контролем не~было выявлено никаких серьёзных нарушений. Председатель Иванов меня спрашивает: «Что˚у вас нет никаких недостатков в˚работе? Назови хотя бы˚один, я не~буду это записывать протокольно и˚оглашать». Ответно мной было сказано, что недостатки имеются, мы˚знаем их и˚стараемся исправлять, например, много отчисляю студентов за˚неуспеваемость. На˚совещании, проводимом на˚городском уровне, прозвучало, что декан всех отчислил. В~тот же˚день явилась корреспондентка газеты Московский комсомолец с˚вопросом: «Правда, что ты˚отчислил почти всех студентов факультета?» При˚показе фактического состояния она со˚мной согласилась и˚сказала, что нет никаких нарушений. Оказывается, честное и˚справедливое признание может иногда негативно ударить по˚тебе, хотя даются какие\=/то заверения. На˚протяжении всех лет работы деканом факультета являлся членом Учебно\-/методического центра заочной формы обучения по˚сельскохозяйственной специальности Главного управления высших учебных заведений Министерства сельского хозяйства СССР. На˚семинарах, которые проводились ежегодно в˚различных городах страны, вырабатывались рекомендации. Они являлись основой для ведения учебного процесса на˚заочных факультетах сельскохозяйственных вузов.

Курировала эту работу от˚Главка Раиса Мироновна Цыбулевская. Участие в˚работе этого центра и˚проверки состояния учебного процесса в˚других вузах по˚линии Центра позволяли знакомиться с˚работой других факультетов вузов, перенимать что-то хорошее, прогрессивное.

Много времени мной уделялось анализу и˚обсуждению учебно\-/методической литературы по˚специальности «землеустройство» для сельскохозяйственных вузов страны. Заочный факультет МИИЗа являлся центром по˚подготовке учебно\-/методической литературы по˚специальности «землеустройство» для всех сельскохозяйственных вузов страны. Учебно\-/методическая литература, прошедшая через заочный факультет, использовалась также на˚очных факультетах вузов.

М.\,П.~Шубичу приходилось по˚линии Министерства образования и˚Учебно\-/методического Центра МСХ СССР участвовать в˚составе комиссии по˚проверке учебного процесса сельскохозяйственных вузов России, аттестации и˚аккредитации вузов. В~составе комиссии участвовал в˚проверке состояния учебного процесса в˚Омском, Красноярском сельскохозяйственных вузах, Абаканском консультативном центре и~т.~д.

При˚посещении г.~Абакана изменилось представление о˚Сибири. Раньше думал, что Сибирь очень суровый край и˚мало какие растения произрастают. Оказалось, что здесь на˚корню созревают помидоры, кукуруза на˚зерно, чего трудно добиться в˚Подмосковье. Абакан "--- красивый город. Жители восхвалялись, что отдельно от˚жилой вынесена производственная зона, но˚оказывается не~учли направление вредоносных ветров, разместив жилую зону с˚наветренной стороны. Проехали на˚машине вдоль реки Енисей: туда с˚одной стороны, обратно "--- с˚другой. Побывали в˚Шушенском "--- место ссылки В.\,И.~Ленина. Экскурсовод подробно изложила о˚прелестях этого края, доме А.\,Д.~Зырянова, где жил в˚ссылке и˚как жил В.\,И.~Ленин. Действительно, природа оказалось превосходной. Даже˚Миша в˚порыве восторга заявил экскурсоводу: «Не˚могли бы˚меня на˚недельку оставить здесь в˚заточении?» Последовал ответ: «Многого хочешь». Посетили Саяно\-/Шушенскую ГЭС. Полдня Миша бродил по˚плотине. При˚этом возмущался, что никто не~спросил зачем он˚здесь, что делает? Ведь˚это охраняемый, серьёзный объект. Видимо, наша безалаберность может привести, а˚иногда и˚приводит к˚серьёзным последствиям. Хотя˚здание штаба недалеко, а на˚холме установлена ракетная установка. По˚дороге обратно заехали в˚карьер добычи мрамора. Добычу его производят двумя способами: взрывом и˚распиливанием на˚плиты. При˚первом способе образуется много окрошки. Невдалеке от˚карьера выложено 500\,м˚дороги из˚мраморной окрошки. При˚посещении дороги японец воскликнул: «Надо˚же, дороги строят из˚золота!» Мрамор залегает разноцветный. Действительно это богатство, которым следует дорожить. Возмущало Мишу варварское обращение с˚кедром (вырубается, кучи гниют среди тайги). Ведь˚кедровая сосна является ценным материалом для хозяйственных целей, дающая в˚шишках съедобные семена\-/орехи, из˚которых изготовляют кедровое масло. В~лесу оказалось много различных грибов, но˚жители собирают лишь белые.

Неблагодарной и˚тягостной оказалась работа ответственным секретарём приёмной комиссии ГУЗа (2000\--2003~годы). Не˚забуду такой непонятный случай: однажды шифрую письменные контрольные работы по˚математике, вдруг начали стучать в˚дверь (дверь в˚этих случаях всегда закрывал на˚замок и в˚кабинете оставался один). При˚вопросе кто стучит, ответа не~последовало. Подумал, что заведующий кафедрой математики пришёл за˚очередной «порцией» зашифрованных работ, открыл дверь. В~это время в˚кабинет рвалась женщина. Встал у˚порога и её не~пускал, так как на˚столах были разложены контрольные работы. Из-за спины женщины последовал резкий удар кулаком мне в˚лицо, после чего еле устоял на˚ногах. Оказалось, нанёс удар сын доцента Смоленского сельскохозяйственного института, который после удара сбежал. Он решил отомстить за то, что по˚математике письменно получил неудовлетворительную оценку. Зло своё почему\=/то вылил на˚меня. Длительное время пришлось ходить с˚чёрным кругом под глазами и˚фактически закрытым левым глазом. Обратился в˚поликлинику, где сделали заключение, и˚передал дело в˚суд. Но,~к˚сожалению, абитуриент не~понёс никакого наказания, откупился, что часто бывает особенно в˚перестроечное время. А~мотивировка была та, что в˚это время не~было свидетелей, хотя их˚было достаточно. Не˚помогло мне в˚этом вопросе и˚ФСБ. Вот Вам объективность и˚справедливость на˚государственном уровне.

В~перестроечное время многие вузы оказались в˚очень затруднительном положении. Изменилось отношение студентов к˚занятиям. Отдельные из˚них стали недостаточно изучать дисциплины учебного плана, больше думать о˚работе с˚целью выживания. Зарплата профессорско\-/преподавательского состава низкая, ниже неквалифицированного работника, что не~способствовало привлечению к˚учебному процессу талантливой молодёжи, не~стимулировало работать с˚полной отдачей сил и˚энергии, а˚думать о˚дополнительных подработках. Снизилась воспитательная работа среди молодёжи. Зарплата должна стимулировать профессорско\-/преподавательский состав, добросовестно выполнять свои обязанности, повышению профессионального уровня. Из-за отсутствия средств преподаватели вуза лишились командировок по˚научно\-/исследовательской работе, что сказалось на˚эффективности исследований и˚повышению их˚квалификации. 

Перестройка привела к˚деградации сельскохозяйственного производства, фактически к˚ликвидации землеустроительной службы и˚приостановлению производственных практик. 

В~1998 и˚2003~годах М.\,П.~Шубич участвовал в˚составе комиссии Минобразования по˚аттестации Белорусской сельскохозяйственной академии, в˚частности аттестовал землеустроительный факультет академии. 

16\==20~января 2012~г. в˚соответствии с˚распоряжением Рособрнадзора участвовал в˚проведении аккредитационной экспертизы ФГБОУ ВПО «Ивановская государственная сельскохозяйственная академия имени Академика Д.\,К.~Беляева» в˚качестве эксперта по˚анализу содержания и˚качества подготовки по˚образовательным программам, реализуемым в˚рамках укрупнённой группы направлений подготовки и˚специальностей: 120000 "--- «геодезия и˚землеустройство»; 120301.65 "--- «землеустройство».

19\==23~ноября 2013~года участвовал в˚проведении аккредитационной экспертизы ФГБОУ ВПО «Пензенский государственный университет архитектуры и˚строительства» в˚качестве эксперта содержания и˚качества подготовки обучающихся и˚выпускников образовательного учреждения специальностей:

\begin{enumerate}
	\item 120301.65(гос\textsubscript{2}) "--- «землеустройство»; 
	\item 120302.65(гос\textsubscript{2}) "--- «земельный кадастр»; 
	\item 120303.65(гос\textsubscript{2}) "--- «городской кадастр»; 
 	\item 120700.68(гос\textsubscript{2}) "--- «землеустройство и˚кадастры», магистры.
\end{enumerate}

Неоднократно М.\,П.~Шубич назначался председателем государственных аттестационных комиссий (ГАК). Был председателем ГАК на˚землеустроительных факультетах Грузинского сельскохозяйственного института и˚Белорусской сельскохозяйственной академии. 

С~некоторой иронией сейчас вспоминается, когда доклады и˚ответы на˚вопросы студентов в˚Тбилисском сельскохозяйственном институте приходилось выслушивать на˚грузинском языке. Отчёт председателя ГАК заслушивался на˚Совете института. На˚реплику одного профессора грузинского института «Хорошо бы˚вам изучить грузинский язык». Последовал ответ: «Вам в˚соответствии с˚постановлением ЦК следует изучить русский язык». Больше никаких вопросов не~последовало. После˚этого повезли через перевал в˚западную Грузию и˚пришлось участвовать в˚грузинской свадьбе.

М.\,П.~Шубич принимал участие в˚собрании избирателей Бауманского избирательного округа г.~Москвы, посвящённое встрече с˚кандидатом в˚депутаты Верховного Совета СССР "--- Генеральным секретарём ЦК КПСС, Председателем Президиума Верховного Совета СССР Леонидом Ильичом Брежневым 2~мая 1979~года в˚Кремлёвском дворце съездов. Сидел в˚третьем ряду, напротив Л.\,И.~Брежнева. После˚собрания состоялся большой концерт. 

С~2004 по˚2007~годы М.\,П.~Шубич работал дополнительно на˚0,5~ставки профессора кафедры мелиорации и˚геодезии Российского государственного аграрного университета "--- Московской сельскохозяйственной академии имени К.\,А.~Тимирязева.

В~1975~году М.\,П.~Шубич поступает на˚экономический факультет Марксизма\-/Ленинизма при Московском горкоме КПСС, который окончил с˚отличием в˚1977~году и˚получил высшее политическое образование.

По˚линии Министерства образования в˚1984~году прошёл переподготовку «по˚проблемам высшего образования», а в˚1990~году "--- «технология аттестации и˚аккредитации вузов» и˚получил соответствующее удостоверение; с˚27~октября по˚4~ноября 1990~года участвовал в˚учебном семинаре\-/встрече руководителей российских предприятий, учёных с˚представителями делового мира Запада (Италии) и˚получил диплом Туринского Центра малого и˚среднего бизнеса, а˚также проходил стажировку в˚проектных организациях Гипрозема.

С~5~ноября по˚14~ноября 2014~года прошёл обучение по˚теме: «Ведение государственного кадастра недвижимости в˚электронном виде»; с˚23~марта по˚1~апреля 2015~года обучался по˚дополнительной профессиональной программе «Информационные технологии в˚области землеустройства и˚кадастров»; с˚25 по˚27~января 2017~года "--- программе: «Актуализации федеральных государственных образовательных стандартов высшего образования и˚совершенствование образовательной деятельности по˚направлению подготовки „землеустройство и˚кадастры“». По˚всем этим направлениям получены удостоверения о˚повышении квалификации. 

Следует заметить, что после окончания экономического факультета Марксизма\-/Ленинизма, ему было предложено работать у˚них, но он˚остался верен МИИЗу.

Действительно, вся сознательная активная жизнь связана с˚МИИЗом (ГУЗом): 1959\==1964~годы "--- студент; 1966\--1969~годы "--- аспирант; 1970\--1972~годы "--- ассистент кафедры земпроектирования; 1972\--1992~годы "--- доцент; 1992\--23~октября 2017~года "--- профессор кафедры землеустройства ГУЗ.

В~1975~году М.\,П.~Шубич решением Высшей аттестационной комиссии был утверждён в˚учёном звании доцента кафедры землеустроительного проектирования МИИЗ. Отмечая профессиональную квалификацию той же˚комиссией в˚1995~году присвоено учёное звание профессора. В~2004~году присвоено почётное звание академика Русской Академии.

При˚решении вопроса профессора в˚ВАК позвонил проректор по˚учебной работе ГУЗ профессор С.\,Н.~Волков и˚заявил: «М.\,П.~Шубичу не~присуждайте профессора, так как отсутствуют сейчас у˚него аспиранты». Наличие аспиранта не~имело никакого значения. Причём˚это было, когда один защитился, а˚другого ещё не~закрепили. По˚приезду в˚университет, я˚зашёл в˚кабинет к С.\,Н.~Волкову и˚сказал: «Независимо˚от твоего звонка, я˚буду профессором». Через˚пару недель вызвали в˚ВАК и˚выдали аттестат профессора. Но˚этот звонок стал для меня сигналом не~работать дальше над докторской диссертацией. С.\,Н.~Волков был председателем Совета по˚присуждению докторских степеней.

С~1980 по˚1985~год являлся заместителем заведующего кафедрой землеустроительного проектирования, а с˚1986 по˚1989~годы заместителем заведующего кафедрой по˚научной работе.

Во˚время работы заместителя заведующего кафедрой В.\,П.~Троицкий "--- заведующий кафедрой "--- фактически переложил всю работу на˚заместителя.

Указом Президента Российской Федерации М.П. Шубичу было присвоена в˚1999~году почётное звание «Заслуженный землеустроитель Российской Федерации». 

Кроме указанных видов общественной работы всё время приходилось вести большую организационно\-/методического работу: был членом учёного совета, членом учебно\-/методической комиссии вуза, председателем комиссии по˚контролю за˚деятельностью администрации, председателем комиссии по˚борьбе с˚пьянством и˚алкоголизмом в˚институте и˚членом этой комиссии Бауманского района г.~Москвы, членом комитета комсомола, эвакуационной комиссии и˚другое. На˚комиссии по˚борьбе с˚пьянством и˚алкоголизмом приходилось разбирать отдельных преподавателей и˚студентов. Член этой комиссии профессор А.\,В.~Маслов всегда требовал сурового наказания.

Все передвижения, успехи в˚жизни давались всегда с˚трудом, с˚какими\=/то препятствиями, требующих преодоления, усилий. Жизненный путь Миши был тернист, хотя в˚большинстве своём окружали  хорошие люди, с˚которыми было полное взаимопонимание. К~людям всегда относился с˚добротой, делился и˚помогал всем, хотя отдельные из˚них не~всегда впоследствии были благодарны за˚эту помощь. К~студентам, особенно трудолюбивым, всегда относился с˚любовью, объективно оказывал помощь в˚учёбе и˚освоении учебного материала.

Вузы в˚настоящее время готовят дипломированных бакалавров и˚магистров. Считаю для страны наиболее целесообразно вести подготовку высококвалифицированных специалистов разных специальностей вместо бакалавров. Бакалавр фактически является неполноценным специалистом, который не~может эффективно решать стоящие задачи. Считал бы˚целесообразным отказаться от Болонской конвеции, которую подписала Россия, и˚готовить полноценных специалистов в˚течение пяти лет. Думаю, полностью поддержат меня в˚этом и˚работодатели. На˚рынке труда более полезным и˚востребованным окажется высококвалифицированный специалист. Для˚подготовки таких специалистов высшему образованию на˚современном этапе необходимы грамотные преподаватели, владеющие современными технологиями педагогической деятельности, постоянно ведущие научно\-/техническую работу.

\todo[Текст]{Сухие цифры про студентов и аспирантов вынес в приложение. В тексте хотелось бы видеть что-то вднохновляющее и жизненное. Вы не могли бы у Косинского в электронном виде взять материалы про студентов из биобиблиографии деятелей землеустроительной науки + у кого-нибудь список студентов после 2006 года?}

\Todo[Прим]{Книги и награды вынес в приложение}

Заведующим кафедры землеустроительного проектирования (землеустройство) после С.\,А.~Удачина были В.\,Д.~Кирюхин, В.\,П.~Троицкий, С.\,Н.~Волков. С.\,Н.~Волков одновременно занимал должность ректора университета и˚заведующего кафедрой. Фактически С.\,Н.~Волков кафедрой не~занимался, а˚отдал на˚откуп заместителям заведующего. Заместителем заведующего кафедрой длительное время был В.\,В.~Пименов, а в˚последнее время стала Л.\,Е.~Петрова. 

С.\,Н.~Волков является хорошим организатором, крупным учёным, умело работает на˚себя. Любит доносчиков, подхалимов, является злопамятным человеком. Если˚человек не~из˚его окружения, или не~из˚ранее сказанного разряда, то˚относится пренебрежительно. 

Человек, занимающий любой пост, в горе, роскоши, должен всегда оставаться с человеческими чертами.