%--- Настройка scrbook

%-- Сноски
\deffootnote[														
1.5em]																	% Отсутп самой сноски
{0em}																		% Отступ вторых строк абзаца
{1.5em}																	% Отступ следующего абзаца
{\textsuperscript{\thefootnotemark}}


%-- Подписи к рисункам, таблицам...
\setkomafont{captionlabel}{\small\bfseries}		% Шрифт для слов Рис., Табл. и т.д.
\renewcommand*{\captionformat}{~}				% Разделитель между подписью типа объекта и его названием (например, здесь разделителем является «-» : Рис. 1.- Пример рисунка)
\setcapindent{0em}											% Отступы второй и следующей строки в подписях
					
%-- Настроить колонтитлы
\usepackage[
headsepline															% Добавляет разделительную линию между верхним колонтитлом и текстом
%,footsepline														% Добавляет разделительную линию между нижним колонтитлом и текстом
]{scrlayer-scrpage}
\clearpairofpagestyles									% Очистить колонтитлы
\ohead{\pagemark}												% Добавить с края страницы её номер
\chead{\headmark}												% Добавить название главы и название секции в верхний колонтитул

%-- Стиль заголовков
%--------------------------------------------------------------------------
% Originalcode von: <http://www.komascript.de/fncychap-Sonny>
% Copyright (c) Markus Kohm
% Version: 2017-05-24
% Changes:
% - 2016-09-02 erste Version
% - 2017-05-24 Anpassung von beforeskip an aktuelles KOMA-Script
% Weitergabe und Verwendung gestattet, solange dieser Hinsweiskommentar
% einschließlich Link und Copyrightinformation erhalten bleibt.

% 1. Emulation von fncychap mit KOMA-Script-Mitteln:
\newlength{\ChapterRuleWidth}\setlength{\ChapterRuleWidth}{.5pt}
\newcommand*{\ChRuleWidth}[1]{\setlength{\ChapterRuleWidth}{\dimexpr #1}}%
\newcommand*{\ChNameVar}{\setkomafont{chapterprefix}}%
\newcommand*{\ChTitleVar}{\setkomafont{chapter}}%
\newcommand*{\ChNumVar}{\setkomafont{chapternumber}}%
\newcommand*{\ChapterNameCase}[1]{#1}
\newcommand*{\ChNameUpperCase}{\let\ChapterNameCase\MakeUppercase}
\newcommand*{\ChNameIs}{\renewcommand*\ChapterNameCase[1]{##1}}
\newcommand*{\ChNameLowerCase}{\let\ChapterNameCase\MakeLowercase}
\newcommand*{\ChapterTitleCase}[1]{#1}
\newcommand*{\ChTitleUpperCase}{\let\ChapterTitleCase\MakeUppercase}
\newcommand*{\ChTitleIs}{\renewcommand*\ChapterTitleCase[1]{##1}}
\newcommand*{\ChTitleLowerCase}{\let\ChapterTitleCase\MakeLowercase}

% 2. Einstellungen für den Stil Sonny:
\KOMAoptions{chapterprefix}% Es ist ein Präfix-Stil
%\ChNameUpperCase										% Привести слово "Глава" в верхний регистр
\newkomafont{chapternumber}{\huge}	% Размер шрифта для номера главы
\let\raggedchapter\raggedleft% Überschriften rechtsbündig
\RedeclareSectionCommand[%
  beforeskip=0\baselineskip,% Отступ от слов "Глава" или рамки
  innerskip=25pt,% Отступ от префикса
  afterskip=20pt,% Отступ от текста главы
  font=\normalfont\sffamily\huge,% Шрифт названия главы
  prefixfont=\normalfont\Large,% Шрифт строки префикса
]{chapter}
\renewcommand*{\chapterformat}{%
  \mbox{\ChapterNameCase{\chapappifchapterprefix{\nobreakspace}}%
    {\usekomafont{chapternumber}{%
        \rule{0pt}{.8\baselineskip}\thechapter\IfUsePrefixLine{}{\enskip}}}%
  }%
}
\renewcommand*{\chapterlineswithprefixformat}[3]{% Ebene, Nummer, Text
  \IfArgIsEmpty{#2}{}{%
    % Die Prefix-Zeile aus Argument 2 wird nur gesetzt, wenn sie vorhanden
    % ist.
    #2%
  }%
  \rule[5pt]{\linewidth}{\ChapterRuleWidth}\\*
  \ChapterTitleCase{#3}\nobreak
  \rule[-5pt]{\linewidth}{\ChapterRuleWidth}
}
% --------------------------------------------------------------------------							% Загловки глав fncychap-Sonny средствами Komma Script.

%-- Стиль секций https://tex.stackexchange.com/questions/355955/change-a-section-or-paragraph-style-in-scrbook
%\makeatletter
%\renewcommand\sectionlinesformat[4]{%
%  \ifstr{#1}{section}
%    {\centering #3\\*#4}
%    {\@hangfrom{\hskip #2#3}{#4}}%
%}
%\makeatother
%\renewcommand\sectionformat{--~\thesection~--}

%-- Макет страницы
\KOMAoptions{DIV=calc 									% Расчёт полезной площади с учетом применяемых шрифтов и др.
,headinclude=true 											% Исключить верхний колонтитул из полей
%,footinclude=true											% Исключить нижний колонтитул из полей
}
\recalctypearea													% Запуск пересчета

%-- Оглавление
\KOMAoptions{toc=chapterentrydotfill}		% Добавить заполнитель точками между главой страницей

%- Добавить слово Глава в оглавление итог
\let\originaladdchaptertocentry\addchaptertocentry
\renewcommand*{\addchaptertocentry}[2]{%
  \IfArgIsEmpty{#1}{% Глава без номера
    \originaladdchaptertocentry{#1}{#2}%
  }{% Глава с номером
    \originaladdchaptertocentry{}{\chapapp~#1. #2}%
  }%
}

%- Добавить слово Глава в оглавление 1 способ
% Решение отклонено, т.к. метод не учитывает главы без нумерации
%\makeatletter
%\let\oldaddchaptertocentry\addchaptertocentry
%\renewcommand{\addchaptertocentry}[2]{%
%\oldaddchaptertocentry{}{\@chapapp{} #1. #2}}
%\makeatother

%- Добавить слово Глава в оглавление 2 способ
% Решение отклонено, т.к. подпись главы идет на таком же отступе как и у приложения
%\RedeclareSectionCommands[tocdynnumwidth]{chapter}	% Автоматически вычислять ширину столбца под номер главы (с учетом слов Глава и Приложение)
%%\RedeclareSectionCommand[tocnumwidth=7em]{chapter}	% Размер ширины столбца под номер главы

%\let\originaladdchaptertocentry\addchaptertocentry
%\renewcommand*{\addchaptertocentry}[2]{%
%  \IfArgIsEmpty{#1}{% Глава без номера
%    \originaladdchaptertocentry{#1}{#2}%
%  }{% Глава с номером
%    \originaladdchaptertocentry{\chapapp~#1}{#2}%
%  }%
%}


