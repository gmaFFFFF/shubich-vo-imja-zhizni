%--------------------------------------------------------------------------
% Originalcode von: <http://www.komascript.de/fncychap-Sonny>
% Copyright (c) Markus Kohm
% Version: 2017-05-24
% Changes:
% - 2016-09-02 erste Version
% - 2017-05-24 Anpassung von beforeskip an aktuelles KOMA-Script
% Weitergabe und Verwendung gestattet, solange dieser Hinsweiskommentar
% einschließlich Link und Copyrightinformation erhalten bleibt.

% 1. Emulation von fncychap mit KOMA-Script-Mitteln:
\newlength{\ChapterRuleWidth}\setlength{\ChapterRuleWidth}{.5pt}
\newcommand*{\ChRuleWidth}[1]{\setlength{\ChapterRuleWidth}{\dimexpr #1}}%
\newcommand*{\ChNameVar}{\setkomafont{chapterprefix}}%
\newcommand*{\ChTitleVar}{\setkomafont{chapter}}%
\newcommand*{\ChNumVar}{\setkomafont{chapternumber}}%
\newcommand*{\ChapterNameCase}[1]{#1}
\newcommand*{\ChNameUpperCase}{\let\ChapterNameCase\MakeUppercase}
\newcommand*{\ChNameIs}{\renewcommand*\ChapterNameCase[1]{##1}}
\newcommand*{\ChNameLowerCase}{\let\ChapterNameCase\MakeLowercase}
\newcommand*{\ChapterTitleCase}[1]{#1}
\newcommand*{\ChTitleUpperCase}{\let\ChapterTitleCase\MakeUppercase}
\newcommand*{\ChTitleIs}{\renewcommand*\ChapterTitleCase[1]{##1}}
\newcommand*{\ChTitleLowerCase}{\let\ChapterTitleCase\MakeLowercase}

% 2. Einstellungen für den Stil Sonny:
\KOMAoptions{chapterprefix}% Es ist ein Präfix-Stil
%\ChNameUpperCase										% Привести слово "Глава" в верхний регистр
\newkomafont{chapternumber}{\huge}	% Размер шрифта для номера главы
\let\raggedchapter\raggedleft% Überschriften rechtsbündig
\RedeclareSectionCommand[%
  beforeskip=0\baselineskip,% Отступ от слов "Глава" или рамки
  innerskip=25pt,% Отступ от префикса
  afterskip=20pt,% Отступ от текста главы
  font=\normalfont\sffamily\huge,% Шрифт названия главы
  prefixfont=\normalfont\Large,% Шрифт строки префикса
]{chapter}
\renewcommand*{\chapterformat}{%
  \mbox{\ChapterNameCase{\chapappifchapterprefix{\nobreakspace}}%
    {\usekomafont{chapternumber}{%
        \rule{0pt}{.8\baselineskip}\thechapter\IfUsePrefixLine{}{\enskip}}}%
  }%
}
\renewcommand*{\chapterlineswithprefixformat}[3]{% Ebene, Nummer, Text
  \IfArgIsEmpty{#2}{}{%
    % Die Prefix-Zeile aus Argument 2 wird nur gesetzt, wenn sie vorhanden
    % ist.
    #2%
  }%
  \rule[5pt]{\linewidth}{\ChapterRuleWidth}\\*
  \ChapterTitleCase{#3}\nobreak
  \rule[-5pt]{\linewidth}{\ChapterRuleWidth}
}
% --------------------------------------------------------------------------