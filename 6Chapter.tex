\chapter{Мечта сбылась}

С~1~сентября 1959~г. началась студенческая жизнь. Нас поселили в˚Черёмушки, в <<Меньшиковы конюшни>>, других \--- в˚Удельное. Общежития на˚Малом Демидовском переулке ещё достраивалось. Мишу избрали председателем студенческого совета. На˚первом курсе пришлось пройти очень тщательную военную комиссию. Из˚института я˚был в˚единственном лице. Прошёл комиссию всех специалистов, даже вертели на˚центрифуге. В~целом признали годным. На˚вопрос по˚заключению комиссии спросили: <<Если˚придётся бросить институт, согласишься?>> Ответ последовал: <<Ни˚за что, ведь я˚уже отслужил>>. Для˚каких целей была организована эта комиссия не~сказали. Среди˚проходящих комиссию ходило мнение, что готовят в˚школу шпионажа. Впоследствии узнали, что отбирали в˚отряд космонавтов.

Годы учёбы были интересными, увлекательными. На˚второй год поселили в˚общежитие на˚Малом Демидовском переулке. Ректор института Л.\,С.~Костарев\todo[Материалы]{Хорошо бы˚найти фотки упомянутых в текст} разрешил устроить новоселье. Поскольку˚ребята прибыли из˚разных уголков, то˚после небольшой пьянки началось выяснение отношений, крупная драка. Досталось и˚Мише как председателю студсовета\footnote{Студенческий совет (студсовет) \--- форма самостоятельной, ответственной общественной деятельности студентов, направленной на˚решение важных вопросов жизнедеятельности студенческой молодёжи, развитие её˚социальной активности.}. Пришлось даже закрывать общежитие, вызывать наряды милиции\footnote{Милиция \--- название органов охраны правопорядка и˚законности (эквивалент современной полиции) в˚СССР и˚России.} и˚скорую помощь.

Миша усердно постигал азы учебных дисциплин, занимался научной и˚большой общественной работой. Принимал активное участие в˚культурно\-/массовых мероприятиях, сам их˚организовывал. На˚протяжении всех лет учёбы приходилось активно заниматься общественной работой: председатель студсовета, член комитета комсомола, член партбюро института. Участвовал в˚работе студенческого научного общества, выступал с˚докладами. На˚Всесоюзной Студенческой конференции в г.~Елгаве (Латвия) получил диплом второй степени. Для˚улучшения своего бюджета приходилось по˚вечерам работать. От˚родителей не~брал ни˚копейки. Наоборот старался им˚помогать. Ведь, работая в˚колхозе, они фактически ничего не~получали. Занятия старался не~пропускать, отставаний в˚учёбе не~было. Удивляют отдельные нынешние студенты, которые не~занимаются, в˚большинстве своём, никакой общественной работой и не~справляются с˚учебными занятиями, ведут пассивный образ жизни. Причём, многие процессы выполняются на˚компьютере. Раньше все чертежи, тексты, вычисления выполняли вручную.

Запомнилось в˚студенческие годы поездка с˚24 января по˚5 февраля 1962~года по˚линии Московского городского комитета партии в˚Венгрию для обмена опытом работы молодёжных организаций. Миша был назначен заместителем руководителя делегации. Ежедневно проходило до˚пяти различных встреч. С~руководителем приходилось выступать на˚них поочерёдно, а˚вечером докладывать в˚наше посольство. До˚нашего сообщения в˚посольстве о˚наших действиях уже было всё известно. Молодёжь теле\-/радиозавода встретила нас холодновато. Пришлось Мише выступить несколько резковато. В~конце выступления приподнял им˚статуэтку бегущего оленя и˚сказал: <<Чтобы ваша страна развивалась такими темпами, как бег этого оленя>>. Ожидал за˚выступление <<взбучки>> в˚посольстве. Но˚сказали, что в˚данной ситуации поступил правильно. Очень гостеприимный приём был на˚паровозоремонтном заводе. Прощальный тёплый вечер был в˚Интуристе. Вечером (после 22.00) ребята делегации по˚улице Будапешта громко запели: <<Широка страна моя родная…>>, \--- за˚что Миша чуть не~схлопотал выговор в˚горкоме партии. Запомнилась встреча в˚Венгерском Парламенте, где представитель подробно изложил о˚работе парламента, путче\footnote{Венгерское восстание 1956~года (23~октября — 9~ноября 1956~г.) (в~посткоммунистический период Венгрии известно как Венгерская революция 1956~года, в˚советских источниках как Венгерский контрреволюционный мятеж 1956~года) — вооружённое восстание против просоветского режима народной республики в˚Венгрии в˚октябре — ноябре 1956~года, подавленное советскими войсками.}, ответил на˚все наши вопросы.

Приятно сейчас вспомнить о˚прослушанных лекциях, сдаче экзаменов, а в˚последствии коллегах по˚работе, таких видных учёных профессоров как С.\,А.~Удачин, Г.\,В.~Чешкин\Todo[Ред.]{Неразборчиво}, Н.\,Н.~Бурихин, Н.\,И.~Прокуронов, Л.\,М.~Ифасман, Н.\,В.~Бочков, И.\,В.~Голубев, В.\,Ф.~Дейнеко, Н.\,Д.~Ильинский,Ю.\,В.~Кемнич, В.\,С.~Косинский, Г.\,А.~Кузнецов, Л.\,С.~Костырев, Е.\,Г.~Ларченко, А.\,В.~Маслов, М.\,А.~Снегиров, А.\,К.~Успенский и˚другие.

С~большой благодарностью всегда относился и˚отношусь к˚руководителю дипломного проекта Е.\,Н.~Первовой. Это высокообразованный, спокойный, тактичный, требовательный преподаватель, умеющий доходчиво излагать материал студентам, хороший методист, воспитатель, организатор производственных практик.

Производственные практики позволяли закрепить теоретический материал, улучшить материальное состояние студентов (определяли на˚должность и˚выплачивали зарплату). Они оставляли приятные воспоминания, позволяли познакомиться с˚производством, использовать материалы производства в˚научных исследованиях, а˚производственникам определиться в˚отношении будущих кадров.

Учебный процесс в˚настоящее время очень страдает из-за отсутствие полноценных производственных практик. Производственные практики сейчас проходит формально, отсутствуют работы, соответствующие учебному процессу. 

Приятные воспоминания оставили производственной практике в˚Горках\-/Ленинских, Пошехоне\-/Володарском\footnote{В~1992~году городу возвращено дореволюционное название Пошехонье.} и˚преддипломная в˚Рязанской области.

В~Горках\-/Ленинских производили съёмку территории усадьбы\-/музея Ленина. Бригада состояла из˚6~человек. В~Пошехоне\-/Володарском \--- полигонометрию, испытывали венгерскую систему\Todo[Ред.]{Что˚это?}. Бригада состояла из˚11~студентов: 3~девушки и˚8~ребят. Миша был в˚обоих бригадах руководителем, зачислен на˚должность инженера. Материалы сдавали непосредственно в˚ГУК\todo[Ред.]{Это Главное управление геодезии и˚картографии при Совете министров СССР (ГУГК)?}. Условия практики в˚Пошехоне\-/Володарском были тяжёлые. Жили в˚палатках. Передвигались вдоль рек Сога и˚Согожа. Питались в˚основном макаронами, готовили девушки. Приходилось рубить просеки\footnote{Пр\'{о}сека \--- узкая полоса, прорубаемая в˚лесу.}. Часто на˚нас нападали местные бандиты (ребята). Стасика Монахова так избили, что у˚него было кровоизлияние головного и˚спинного мозга. Кое-как выжил. Виновниками были два брата, которых впоследствии осудили. Мише после практики пришлось выезжать на˚суд. 

Чтобы˚не~подожгли наши палатки, пришлось поочерёдно выставлять ребят на˚ночь охранниками (часовых). Приборы позволяли работать рано утром и˚вечером. Днём из-за солнца не~могли работать. Однажды ночью Миша возвращался из˚Пошехонья к˚своим палаткам. На˚мосту пересечения рек Сога и˚Согожа встретили три парня. Длина моста составляла около 700\,м. Заявили, что мы˚сейчас тебя утопим. Намерения были вполне серьёзными. Миша старался как-то выиграть время, чтобы обдумать свои действия. Сначала попросил их˚перед смертью закурить, хотя никогда не~курил. Делал вид что курит, тянул время. Вдруг у˚них получилась какая-то несогласованность. Миша заявил, что моим ребятам известно, где я˚нахожусь. При˚исчезновении меня, вас обязательно осудят, а˚руководитель\todo[Ред.]{может вожак или главарь? Непонятно почему он не~сядет?} будет на˚свободе, спокойно жить и˚улыбаться. Ребята приняли его высказывания и˚начали ругаться между собой. Это позволило Мише бегством спастись от˚явной смерти. Хотя˚все были однокашники\footnote{Однокашник \--- товарищ по˚воспитанию, учению, выросший вместе (собственно \--- товарищ по˚столу, по˚питанию).}, студенты, Мише приходилось давать неблагоприятные приказы, которые выполнялись.

Как-то девушкам надоело рано вставать и˚готовить пищу. Они устроили забастовку, отказались готовить. Миша прекрасно понимал их˚положение, но˚выхода никакого не~было. Тогда им˚дали топоры и˚послал рубить просеку. Определили норму работу. По˚возвращению вечером девушки заявили, что лучше будут готовить. Таким образом инцидент был исчерпан. 

Во˚время преддипломной практики самостоятельно составил полностью проект внутрихозяйственного землеустройства по˚двум объектам и˚перенёс их в˚натуру. По˚окончании практики в˚колхозе <<Заветы Ильича>> председатель колхоза А.\,И.~Канаев в˚знак благодарности за˚хорошую работу приподнёс бочонок мёда.

С~учёбой в˚ВУЗе попал под нововведения: сроки учёбы были продлены до˚5,5 лет, а˚диплом получали после года работы на˚производстве при представлении положительной характеристики. В~июне 1964~года защитили дипломные проекты. Выпускники фотографировались, праздновали окончание ВУЗа. Миша засел за˚подготовку к˚экзаменам для поступления в˚аспирантуру. Место дальнейшей работы фактически определила преддипломная производственная практика. Распределили в˚Рязанскую землеустроительную экспедицию. Распределяли на˚работу согласно заявкам организаций и по˚баллам.

Для˚поступления в˚аспирантуру необходимо преодолеть конкурс, который составил два~человека на˚место. Конкурентом Миши был Альберт Крюков. Экзамены в˚аспирантуру успешно сдали оба и˚набрали одинаковые балы. С.\,А.~Удачин как-то заметил Крюкова пьяным, поэтому ему было отказано для поступления. Зачислили Мишу. Руководителем диссертационной работы Миши стал академик ВАСХНИЛ, доктор экономических наук, профессор С.\,А.~Удачин. Он же˚был заведующим кафедрой землеустроительного проектирования.

Для˚учёбы в˚аспирантуре необходимо 2~года отработать на˚производстве.

Во˚время учёбы в˚институте и˚общественной работы председателем студсовета приходилось заниматься неблагодарными делами: следить за˚порядком в˚общежитии, чистотой в˚студенческих комнатах (введено самообслуживание), разбирать на˚заседаниях поведение недисциплинированных студентов, ставить вопросы общежития перед комендантом и˚администрацией института, отстаивать (защищать) отдельных студентов. Следует отметить, что с˚комендантом общежития Л.~Комаровой и проректором по˚административно\-/хозяйственной работе П.\,С.~Лошковым складывались хорошие отношения. Они всегда поддерживали начинания Миши. Хорошо относились к˚нему и˚вахтёры общежития.

Вспоминается случай со˚строительством общежития у˚театра Гоголя. По˚сметам на˚строительство выделено 1~млн. руб. За˚7\,месяцев был освоен лишь 400~тыс. руб., то˚есть было серьёзное отставание в˚строительстве. Тогда ректорат института решил подготовить письмо от˚имени студентов Л.\,И.~Брежневу\footnote{Л.\,И.~Брежнев с˚1964 по˚1982~год занимал должность Генерального секретаря ЦК КПСС (ранее Первый секретарь ЦК КПСС) \--- высшую должность в˚Коммунистической партии Советского Союза. В~1923—1927~годах учился в˚Курском землемерно-мелиоративном техникуме. Получив квалификацию землемера 3\=/го разряда, несколько лет работал землемером-землеустроителем.}. С~письмом направили Мишу. Принял его помощник Брежнева. Но˚письмо всё-таки оказалась у˚Брежнева. В˚результате резко активизировалось строительство. За˚5\,месяцев было освоено 600~тыс. руб. и˚своевременно введено общежитие в˚эксплуатацию.

Часто посещали общежитие с˚проверкой представители райкома партии. Однажды с˚проверкой пришёл секретарь Бауманского райкома партии. При˚обходе этажей общежития выявилась пьянка в˚одной из˚комнат. В~конце коридора этажа секретарь заметил: <<Ведь в˚комнате пьянка>>. Мише ничего не~оставалось, как слукавить: <<Знаю, мы им˚разрешили, у˚одного парня день рождения>>. Зашли в˚эту комнату. Секретарь спросил: <<Кто справляет день рождения?>> Ребята оказались сообразительными и˚один сказал, что у˚него. Хотя˚ничего подобного не~было. Секретарь похвалил, что владеем\todo[Ред.]{Предложение кажется не согласовано: …владеем и˚контролируем ситуацию…} и˚контролируем ситуацию.

Распределились в˚Рязанскую экспедицию\todo[Структ.]{Может имеет смысл разбить главы на параграфы? Особенно этого требует 7. Например, здесь можно добавить параграф Рязанская экспидиция.} Миша и˚Володя Яблоков. Устроились на˚квартиру к˚женщине, проживающей со˚взрослой дочерью. Они стали жить в˚одной комнате, а˚нам предоставили вторую. Предоставили нам постельные принадлежности. Дочь часто заходила в˚нашу комнату.

Однажды, придя с˚работы, мы˚сидели за˚столиком и˚разговаривали. Зашла дочь хозяйки. Мы отнеслись без внимания. Неожиданно она ударила ножом прямо в˚сердце Мишу. Спасло то, что на˚нём был пиджак, рубашка и˚майка. Поэтому только задела немного кожу. Оказывается, она психически ненормальная, и˚когда Миша ухаживал за˚будущей своей женой, она следила за˚каждым шагом. Приревновала и˚решила отомстить. Мы сделали крючок на˚дверях. При˚возвращении домой на˚второй день оказался крючок вырван, а в˚углу стоял приготовленный топор. Пришлось о˚всём рассказать хозяйке. Она спрятала топор и˚все ножи. Нам приходилось ложиться с˚вечера головой в˚одну сторону, а˚затем переворачиваться в˚противоположную.

По˚приезду в˚Рязань в˚обкоме партии Миша встал на˚партийный учёт.

Сначала занимался в˚экспедиции схемой районной планировки, проектами внутрихозяйственного землеустройства. Через˚короткое время назначили начальником отряда. Пришлось осуществлять контроль за˚подготовкой схем и˚проектов землеустройства, принимать перенос проектов в˚натуру. Разработанные проекты в то˚время утверждали на˚горисполкоме\footnote{Горисполком \--- городской исполнительный комитет Совета народных депутатов. Исполнительно-распорядительный местный орган государственной власти в˚СССР, аналог современных администраций.}. Начальнику отряда приходилось докладывать на˚горисполкоме каждый проект. Было дано указание, при составлении проекта внутрихозяйственного землеустройства в˚севооборотах предусматривать 70~\% и˚более зерновых. В~какой-то степени нарушался принцип, что каждая предыдущая культура в˚схеме севооборота должна быть хорошим предшественником для последующей. Пришлось на˚горисполкоме доказывать, что при требовании иметь в˚севообороте ˚70\==80~\% зерновых выполнить этого невозможно. Сначала с˚мнением не~согласились, а˚затем разрешили ему отойти от˚этой установки.

В~отряде работала старшим техником женщина средних лет Роза Стерлигова, имеющая почти взрослых двух детей. Переносила проект внутрихозяйственного землеустройства в˚одном хозяйстве Скопинского района. Необходимо было принять у˚неё работу. Приехавши в˚хозяйство, узнал у˚мальчика, где проживает землеустроитель. Он мне любезно показал дом. С~улицы дверь была закрыта. Поэтому решил пройти с˚чёрного хода. При˚проходе по˚коридору заметил, что в˚чулане лежит Роза с˚каким-то мужчиной. Сделал вид, что не~заметил их и˚вошёл в˚дом. Вскоре появилась и˚Роза, причём одна. <<Миша, мы˚будем ужинать в˚ресторане>> проронила она. Предложила посмотреть кое-какие дела, а˚сама исчезла. Затем появилась хозяйка, небольшого роста, с˚горбинкой женщина. Постелила мне постель и˚так же˚ушла.Чутье подсказывало что-то неладное, не~должно ложиться здесь спать. Поэтому пошёл вечером в˚соседний дом, который примыкал (единой стенкой) к˚данному дому. Там оказалась проживала старушка, а у˚неё квартировалась\footnote{Квартироваться (разг.) \--- жить, проживать в˚съёмной квартире.} зоотехник колхоза. Объяснился и˚попросил зоотехника переночевать в˚соседнем доме, на˚приготовленной мне кровати. Она согласилась, но˚предупредила что у˚неё простыни несвежие. Залёг сразу спать, а˚старушка пристроилась на˚печке. В~2:00 ночи раздался резкий крик и в˚дом вбежала зоотехник в˚одной ночной рубашке. Оказывается, она спала и˚вдруг резко сорвана простынь. Перед˚ней стоит мужчина с˚поднятой рукой, а в˚руке нож. Увидев женщину, он˚растерялся. В~это время выскочила она из˚кровати и˚убежала. Если˚бы Миша там оставался, то˚остался ли в˚живых? Перед˚зоотехником пришлось извиняться. Посмотрели в˚окно, по˚улице шли два мужчины. Утром Миша попросил председатель колхоза вызвать милицию, а˚сам уехал. Работу у˚Стерлиговой Р. не~принимал. Оказывается, это был заключённый, сбежавший из˚тюрьмы. Со˚слов Розы выяснилось, что он˚подумал, что Миша приехал специально к˚ней в˚качестве очередного <<любовника>> и˚решил убить.

Е.\,Н.~Первова по-прежнему привозила в˚Рязань студентов на˚производственную практику. Миша старался помочь, уделял много им˚внимания.

Вспоминается такой случай: во˚время моей работы начальником отряда в˚Рязанской землеустроительной экспедиции приехала Е.\,Н.~Первова, и мы˚поехали к˚студентам по˚хозяйствам (на˚машине), которые были на˚преддипломной практике в˚Милославском районе Рязанской области. Ездили допоздна, фактически наступила ночь, а˚нам необходимо было возвращаться в˚Рязань. Поэтому, чтобы сократить время, поехали по˚полевым дорогам, выбирая покороче путь. Дорога привела нас на˚какой-то холм, взобрались на˚него на˚машине, и˚машина заглохла. Была опасность сорваться с˚холма. Кое-как переехали его, а˚дальше дорога привела к˚реке, и мы˚решили вброд перебраться на˚машине на˚другую сторону реки. Мише пришлось нащупывать пешком более мелкие твёрдые места в˚реке, машина двигалась за˚ним. Естественно было рискованное мероприятие, но˚таким образом удалось перебраться на˚другую сторону реки и˚сократить путь. 

Елена Николаевна сидела на˚переднем сидении машины, рядом с˚шофёром и не~проронила в˚это время ни˚одного слова. Лишь˚по приезду в˚Рязань к˚гостинице она сказала: <<Михаил Павлович, у˚меня сердце было в˚пятках, думала погибнем>>. Она оказалась мужественной и терпеливой женщиной в˚экстремальных условиях.

После двух лет работы пришлось возвращаться в˚аспирантуру, но˚здесь возникла проблема. Рязанский обком партии\footnote{Обком партии (разг.) \--- областной комитет КПСС \--- высший орган областной организации КПСС между областными конференциями КПСС.} отказался дать характеристику в˚связи с˚тем, что числился в˚резерве начальника экспедиции.\todo[Текст]{Современному человеку непонятны причины данного поступка} Попросил дать отрицательную характеристику. Последовал ответ, что тем более этого нельзя. Пришлось агитировать А.\,С.~Козлова поехать в˚Рязань. Он работал главным инженером в Чечено\=/Ингушетии. В~обкоме партии устроили собеседование и˚отказали. Через˚неделю Миша снова поехал с˚А.\,С.~Козловым в˚Рязань. В~обкоме пришлось ему доказывать, что Козлов хороший специалист и˚организатор производства. На˚этот раз удалось убедить. Так˚А.\,С.~Козлов стал начальником Рязанский экспедиции, а˚Мише выдали отличную характеристику. С~Рязанью, в˚частности с˚А.\,С.~Козловым поддерживал постоянную связь. Ездил в˚командировку и на˚выходные, так как в˚Рязани оставалась жена. \todo[Материалы]{Вы ещё интересно рассказывали как какой-то старик заставил Вас сено косить и˚как Вы с˚ним боролись}
\todo[Материалы]{Я какой-то слух ещё слышал про медведя или лося…}