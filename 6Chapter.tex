\chapter{Мечта сбылась}

\begin{wrapfigure}{O}{.4\textwidth}
\centering
\includegraphics[width=.35\textwidth]{private/rectorKostirev}
\caption[Л.\,С.~Костарев. Бывший ректор МИИЗа, заведующий кафедрой экономики]{Л.\,С.~Костарев. Бывший ректор МИИЗа, заведующий кафедрой экономики\footnotemark}
\label{fig:rectorKostirev}
\end{wrapfigure}
\footnotetext{Источник заимствования "--- личный фотоархив М.\,П.~Шубича.}

С~1~сентября 1959~г. началась студенческая жизнь. Мишу поселили в˚Черёмушки, в «Меньшиковы конюшни», других "--- в˚Удельное. Общежитие на˚Малом Демидовском переулке ещё достраивалось. Мишу избрали председателем студенческого совета. На˚первом курсе института пришлось пройти ему очень тщательную военную комиссию. Из˚института Миша был в˚единственном лице. Прошёл комиссию всех специалистов, даже «вертели» на˚центрифуге. В~целом признали годным. Представитель комиссии спросил: «Если˚придётся бросить институт, согласишься?» Последовал ответ : «Ни˚за что, ведь я˚уже отслужил». Для˚каких целей была организована эта комиссия не~сказали. Среди˚проходящих комиссию ходило мнение, что готовят в˚школу шпионажа. Впоследствии узнали, что отбирали в˚отряд космонавтов.

Годы учёбы были интересными, увлекательными. На~второй год поселили в˚общежитие на˚Малом Демидовском переулке. Ректор института Леонид Семёнович Костарев разрешил устроить новоселье. Поскольку˚ребята прибыли из˚разных уголков большой страны, то˚после небольшой пьянки началось выяснение отношений, крупная драка. Досталось и˚Мише как председателю студсовета\footnote{Студенческий совет (студсовет) "--- форма самостоятельной, ответственной общественной деятельности студентов, направленной на˚решение важных вопросов жизнедеятельности студенческой молодёжи, развитие её˚социальной активности.}. Пришлось даже закрывать общежитие, вызывать наряды милиции\footnote{Милиция "--- название органов охраны правопорядка и˚законности (эквивалент современной полиции) в˚СССР и˚России.} и˚скорую помощь.

Миша усердно постигал азы учебных дисциплин, занимался научной и˚большой общественной работой. Принимал активное участие в˚культурно\-/массовых мероприятиях, сам их˚организовывал. На˚протяжении всех лет учёбы приходилось активно заниматься общественной работой: был председателем студсовета, членом комитета комсомола, членом партбюро института. Участвовал в˚работе студенческого научного общества, выступал с˚докладами. На˚Всесоюзной Студенческой конференции в г.~Елгаве (Латвия) получил диплом второй степени. Для˚улучшения своего бюджета приходилось по˚вечерам работать. От˚родителей не~брал ни˚копейки, наоборот, старался им˚помогать. Ведь, работая в˚колхозе, они фактически ничего не~получали. 

Занятия старался не~пропускать, отставаний в˚учёбе не~было. Впоследствии Мишу удивляли отдельные нынешние студенты, которые не~занимаются, в˚большинстве своём никакой общественной работы не~ведут и не~справляются с˚учебными занятиями, ведут пассивный образ жизни. Причём многие процессы в˚настоящее время  выполняются на˚компьютере. Раньше все чертежи, тексты, вычисления выполняли вручную.

\begin{figure}[h]
\includegraphics[width=\textwidth,center]{private/studentBunt1961}
\caption[Студенты проявляют недовольство из-за˚прекращения танцев в˚общежитии (красный уголок). Недовольство вылилось в˚адрес зам. директора Н.\,Н.~Буркина, а˚недалеко, с˚таким же˚успехом, осаждали студсовет, М.\,П.~Шубича. А~они правы… 22:30 05.03.1961]{Студенты проявляют недовольство из-за˚прекращения танцев в˚общежитии (красный уголок). Недовольство вылилось в˚адрес зам. директора Н.\,Н.~Буркина, а˚недалеко, с˚таким же˚успехом, осаждали студсовет, М.\,П.~Шубича. А~они правы… 22:30 05.03.1961\footnotemark}
\label{fig:studentBunt1961}
\end{figure}
\footnotetext{Источник заимствования "--- личный фотоархив М.\,П.~Шубича.}

На всю жизнь запомнилась Мише поездка в˚студенческие годы с˚24 января по˚5 февраля 1962~года по˚линии Московского городского комитета партии в˚Венгрию для обмена опытом работы молодёжных организаций. Миша был назначен заместителем руководителя делегации. Ежедневно проходило до˚пяти различных встреч. С~руководителем делегации приходилось выступать на˚них поочерёдно, а˚вечером докладывать в˚советское посольство. До сообщения в˚посольстве об˚их действиях уже было всё известно. Молодёжь теле\-/радиозавода встретила делегацию прохладно. Пришлось Мише выступить несколько резковато. В~конце выступления он преподнёс статуэтку бегущего оленя и˚сказал им: «Чтобы ваша страна развивалась такими темпами, как бег этого оленя». Ожидал за˚выступление «взбучки» в˚посольстве. Но˚сказали, что в˚данной ситуации поступил правильно. 

Очень гостеприимным приём был на˚паровозоремонтном заводе. Прощальный тёплый вечер был в˚Интуристе. Вечером (после 22:00),прогуливаясь по˚улицам Будапешта, ребята делегации громко запели: «Широка страна моя родная…» "--- за˚что Миша чуть не~схлопотал выговор в˚горкоме партии. Мише запомнилась встреча в˚Венгерском Парламенте, где представитель подробно изложил о˚работе парламента, путче\footnote{Венгерское восстание 1956~года (23~октября "--- 9~ноября 1956~г.) (в~посткоммунистический период Венгрии известно как Венгерская революция 1956~года, в˚советских источниках как Венгерский контрреволюционный мятеж 1956~года) "--- вооружённое восстание против просоветского режима народной республики в˚Венгрии в˚октябре "--- ноябре 1956~года, подавленное советскими войсками.}, ответил на˚все вопросы. На~вопрос: «Почему не~арестовали руководителя путча Мицента и˚не~осудили?» "--- последовал чисто дипломатический ответ: «Зачем содержать, кормить, еще тратить на него деньги?»

С~руководителем делегации решили посетить в˚Будапеште ночной клуб. При˚входе с˚каждого взяли по˚35~форинтов. Столик разместили у˚сцены. К~ним подсадили двух местных девушек. Оказывается, некоторые молодые люди Венгрии целыми сутками пребывали в˚ночном клубе, не~являясь домой. Объяснились с˚девушками при помощи венгерско\-/русского разговорника. Гид находился в˚конце зала, вдалеке от˚нас. Заказали для девушек бутылку вина, закуску, а˚затем они запросили ещё бутылку. Поскольку˚расплачиваться пришлось Мише, то˚его очень беспокоило, хватит ли˚форинтов (выдали им небольшую сумму), так как цену вина и˚угощения они не~знали. Спасло то, что входные форинты вошли в˚оплату. Поведение молодёжи, в˚том числе «наших» девушек было развязным. Они заявили: «Сейчас мы не~любим венгерских парней, а˚любим русских». На˚сцене выступала обнажённая девушка (артистка), часто выпивала вино, опускалась и˚пыталась обнять нас. Миша с руководителем делегации вели себя в˚строгом соответствии с˚инструктажем, полученным в˚Горкоме партии в˚Москве, иначе была бы˚взбучка. Девушки проводили их до˚Интуриста. 

В~то˚время отношение отдельных венгров к˚русским было очень плохим, в˚чем Миша убедился на˚себе.
Однажды днём он˚решил ознакомиться с˚Будапештом. В~одиночку пошёл бродить по˚городу. За˚ним увязалась какая\=/то женщина. Она преследовала его по˚пятам, выражая негодование. Лицо её˚было злым, она что\=/то кричала. Для того чтобы избавиться от˚неё, Миша начал заходить в˚разные переулки. Таким путём удалось оторваться от˚неё, но при˚этом Миша окончательно заблудился. Встретил молодую девушку и˚попытался по\=/немецки спросить, как пройти к˚Интуристу. Девушка заулыбалась и˚сказала: «Лучше спроси по\=/русски». Оказывается, это была русская девушка, которая была замужем за˚венгром и˚проживала уже несколько лет в˚Будапеште. Она и˚довела Мишу до˚Интуриста, за˚что он сердечно её˚поблагодарил. 

Приятно сейчас бывает Мише вспоминать о˚прослушанных лекциях, сдаче экзаменов у таких видных учёных профессоров, а в˚последствии и˚коллегах по˚работе, как С.\,А.~Удачин, Г.\,В.~Чешихин\, Н.\,Н.~Бурихин, Н.\,И.~Прокуронов, Я.\,М.~Ифасман, Н.\,В.~Бочков, И.\,В.~Голубев, В.\,Ф.~Дейнеко, Н.\,Д.~Ильинский,Ю.\,В.~Кемнич, В.\,С.~Косинский, Г.\,А.~Кузнецов, Л.\,С.~Костырев, Е.\,Г.~Ларченко, А.\,В.~Маслов, М.\,А.~Снегиров, А.\,К.~Успенский и˚других.




\section*{Производственная практика}
\label{sec:productionPractices}
\addcontentsline{toc}{section}{\nameref{sec:productionPractices}}
Производственные практики позволяли закрепить теоретический материал, улучшить материальное состояние студентов . Студентов определяли на˚должность и˚выплачивали им зарплату). Они оставляли приятные воспоминания, позволяли познакомиться с˚производством, использовать материалы производства в˚научных исследованиях, а˚производственникам определиться в˚отношении будущих кадров.

Учебный процесс в˚настоящее время очень страдает из-за отсутствия полноценных производственных практик. Производственные практики сейчас проходят формально, отсутствуют работы, соответствующие учебному процессу. 

Приятные воспоминания оставили производственные практики в˚Горках\-/Ленинских, Пошехоне\-/Володарском\footnote{В~1992~году городу возвращено дореволюционное название Пошехонье.} и˚преддипломная в˚Рязанской области.

В~Горках\-/Ленинских производили съёмку территории усадьбы\-/музея Ленина. Бригада состояла из˚6~человек. В~Пошехоне\-/Володарском "--- полигонометрию, испытывали венгерскую систему приборов. Бригада состояла из˚11~студентов: 3~девушки и˚8~ребят. Миша был в˚обоих бригадах руководителем, зачислен на˚должность инженера. Материалы сдавали непосредственно в˚ГУК.\footnote{ГУК "--- Главное управление геодезии и˚картографии при˚Совете Министров СССР.} Условия практики в˚Пошехоне\-/Володарском были тяжёлые. Жили в˚палатках. Передвигались вдоль рек Сога и˚Согожа. Питались в˚основном макаронами, готовили девушки. Приходилось рубить просеки\footnote{Пр\'{о}сека "--- узкая полоса, прорубаемая в˚лесу.}. Часто на˚них нападали местные бандиты (ребята). Стасика Монахова так избили, что у˚него было кровоизлияние головного и˚спинного мозга. Кое-как выжил. Виновниками были два брата, которых впоследствии осудили. Мише после практики пришлось выезжать на˚суд. 

Чтобы˚не~подожгли палатки бригады, приходилось поочерёдно выставлять ребят на˚ночь охранниками (часовых). Приборы позволяли работать рано утром и˚вечером. Днём из-за солнца не~могли работать. 

Однажды ночью Миша возвращался из˚Пошехонья к˚своим палаткам. На˚мосту пересечения рек Сога и˚Согожа его встретили три парня. Длина моста составляла около 700\,м. Заявили Мише: «Мы˚сейчас тебя утопим». Намерения были вполне серьёзными. Миша старался как-то выиграть время, чтобы обдумать свои действия. Сначала попросил их˚перед смертью закурить, хотя никогда не~курил. Делал вид что курит, тянул время. Вдруг у˚них получилась какая-то несогласованность. Миша заявил им, что его ребятам известно, где он находится и при˚его исчезновении вас обязательно найдут и осудят, а˚ваш главарь будет на˚свободе, спокойно жить и˚улыбаться. Ребята клюнули на˚его высказывания и˚начали ругаться между собой. Это позволило Мише бегством по мосту спастись от˚явной смерти. В конце моста на˚него набросилась стая собак и пришлось отбиваться включенным китайским фонариком. Однако они успели порвать Мише брюки и нанести небольшую рану на ноге.

Хотя˚все были однокашники\footnote{Однокашник "--- товарищ по˚воспитанию, учению, выросший вместе (собственно "--- товарищ по˚столу, по˚питанию).}, студенты, Мише приходилось давать неблагоприятные приказы, которые выполнялись. Как-то девушкам надоело рано вставать и˚готовить пищу. Они устроили забастовку, отказались готовить. Миша прекрасно понимал их˚положение, но˚выхода никакого не~было. Тогда им˚дали топоры и˚послали рубить просеку. Определили норму работ. По˚возвращению вечером девушки заявили, что лучше будут готовить. Таким образом инцидент был исчерпан. 

Во˚время преддипломной практики Миша самостоятельно составил полностью проект внутрихозяйственного землеустройства по˚двум объектам и˚перенёс их в˚натуру\footnote{Вынос проекта в натуру "--- это геодезические работы, при˚которых происходит перенос и˚закрепление планового и(или)˚высотного расположения точек какого-либо объекта с˚проекта на˚местность.}. По˚окончании практики в˚колхозе «Заветы Ильича» председатель колхоза А.\,И.~Канаев в˚знак благодарности за˚хорошую работу преподнёс Мише бочонок мёда.

Михайловское производственное колхозно-совхозное управление наградило Мишу почетной грамотой за успешное выполнение комплекса землеустроительных работ в колхозах и совхозах управления.



\section*{Студенческий совет}
\label{sec:studentCouncil}
\addcontentsline{toc}{section}{\nameref{sec:studentCouncil}}

Во˚время учёбы в˚институте и˚общественной работы председателем студсовета приходилось заниматься неблагодарными делами: следить за˚порядком в˚общежитии, чистотой в˚студенческих комнатах (введено самообслуживание), разбирать на˚заседаниях поведение недисциплинированных студентов, ставить вопросы общежития перед комендантом и˚администрацией института, отстаивать (защищать) отдельных студентов. Следует отметить, что с˚комендантом общежития Л.~Комаровой и˚проректором по˚административно\-/хозяйственной работе П.\,С.~Лошковым складывались хорошие отношения. Они всегда поддерживали начинания Миши. Хорошо относились к˚нему и˚вахтёры общежития.

В~работе студсовета случился как-то неприятный случай. В~подвале общежития на˚Малодемидовском переулке находилась прачечная, где студенты стирали и˚сушили своё белье. Из~прачечной стало пропадать нижнее белье девушек. От~них поступали частые жалобы. Поэтому студсовет вынужден был пойти с˚обыском по˚комнатам студентов (милиция заявила, что общественность это может проделать). В~чемодане Дины "--- девушки из˚Молдавии "--- нашли много украденного белья, а˚на˚ней˚несколько украденных трусиков. Чтобы не~поднимать большого скандала, Миша предложил Дине самостоятельно оставить институт. Что~она и˚сделала. С~тех˚пор бельё перестало пропадать.

Однажды произошло серьёзное отставание в˚строительстве общежития у˚театра Гоголя. По˚сметам на˚строительство было выделено 1~млн. руб. За˚7\,месяцев было освоено лишь 400~тыс. руб. Тогда ректорат института решил подготовить письмо от˚имени студентов Л.\,И.~Брежневу\footnote{Л.\,И.~Брежнев с˚1964 по˚1982~год занимал должность Генерального секретаря ЦК КПСС (ранее Первый секретарь ЦК КПСС) "--- высшую должность в˚Коммунистической партии Советского Союза. В~1923"---1927~годах учился в˚Курском землемерно-мелиоративном техникуме. Получив квалификацию землемера 3\=/го разряда, несколько лет работал землемером-землеустроителем.}. С~письмом направили Мишу. Принял его помощник Брежнева. Но˚письмо всё-таки оказалось у˚Брежнева. В˚результате резко активизировалось строительство. За˚5\,месяцев было освоено 600~тыс. руб. и˚общежитие своевременно было введено в˚эксплуатацию.

Часто посещали общежитие с˚проверкой представители райкома партии. Однажды с˚проверкой пришёл секретарь Бауманского райкома партии. При˚обходе этажей общежития выявилась пьянка в˚одной из˚комнат. В~конце коридора этажа секретарь заметил: «Ведь в˚комнате пьянка». Мише ничего не~оставалось, как слукавить: «Знаю, мы им˚разрешили, у˚одного парня день рождения». Зашли в˚эту комнату. Секретарь спросил: «Кто справляет день рождения?» Ребята оказались сообразительными и˚один сказал, что у˚него. Хотя˚ничего подобного не~было. Секретарь похвалил, что владеем ситуацией и контролируем ее.



\section*{Путь в аспиранты}
\label{sec:PathGraduateStudents}
\addcontentsline{toc}{section}{\nameref{sec:PathGraduateStudents}}

С~учёбой в˚вузе Миша попал под нововведения: сроки учёбы были продлены до˚5,5 лет, а˚диплом получали после года работы на˚производстве при представлении положительной характеристики. В~июне 1964~года защитили дипломные проекты. Выпускники фотографировались, праздновали окончание вуза. Миша засел за˚подготовку к˚экзаменам для поступления в˚аспирантуру. Место дальнейшей работы фактически определила преддипломная производственная практика. Распределили в˚Рязанскую землеустроительную экспедицию. Распределяли на˚работу согласно заявкам организаций и по˚баллам.

Для˚поступления в˚аспирантуру необходимо преодолеть конкурс, который составил два~человека на˚место. Конкурентом Миши был Альберт Крюков. Экзамены в˚аспирантуру успешно сдали оба и˚набрали одинаковые балы. С.\,А.~Удачин как-то заметил Крюкова пьяным, поэтому ему было отказано в поступлении, а зачислили Мишу. Руководителем диссертационной работы Миши стал академик ВАСХНИЛ, доктор экономических наук, профессор С.\,А.~Удачин. Он~же˚был заведующим кафедрой землеустроительного проектирования.

Для˚учёбы в˚аспирантуре необходимо 2~года отработать на˚производстве.

Распределились в˚Рязанскую землеустроительную экспедицию Миша и˚Володя Яблоков. Устроились на˚квартиру к˚женщине, проживавшей со˚взрослой дочерью. Они стали жить в˚одной комнате, а˚им предоставили вторую, дали постельные принадлежности. Дочь часто заходила в˚комнату к˚ребятам.

Однажды, придя с˚работы, ребята˚сидели за˚столиком и˚разговаривали. Зашла дочь хозяйки. Они не~обратили на˚неё внимание. Неожиданно она ударила ножом прямо в˚сердце Мишу. Спасло то, что на˚нём был пиджак, рубашка и˚майка. Поэтому только задела немного кожу. Оказывается, она была психически ненормальной, и˚когда Миша ухаживал за˚будущей своей женой, она следила за˚каждым шагом. Приревновала и˚решила отомстить. Ребятам пришлось сделать крючок на˚дверях. При˚возвращении домой на˚второй день крючок оказался  вырван, а в˚углу стоял приготовленный топор. Пришлось обо˚всём рассказать хозяйке. Она спрятала топор и˚все ножи, чтобы обезопасить проживающих. Им~приходилось ложиться с˚вечера головой в˚одну сторону, а˚затем переворачиваться в˚противоположную.

По˚приезду на˚работу в˚Рязань в˚обкоме партии Миша встал на˚партийный учёт.

Во˚время работы инженером\-/землеустроителем при обследовании землепользования одного колхоза Миша встретил старика, который косил овёс. «Что˚ты ходишь с˚бумажкой? Лучше бы˚покосил!» "--- проронил старик. «Давайте попробую, никогда не~косил, а Вы займитесь моей работой», "--- и˚передал старику план землепользования. Конечно, Миша здесь слукавил, что никогда не~брал в˚руки косу. Когда˚Миша прокосил один ряд, старик заметил: «У~тебя получается, а я в˚твоих бумагах ничего не~понимаю». Всё\=/таки старику хотелось как\=/то ущипнуть, задеть Мишу. Тогда он˚предложил побороться. Естественно, Мише не~хотелось уронить своё достоинство и˚он приложил усилия, чтобы побороть сильного старика. Старику, видимо, понравился Миша своими действиями. Вечером ежедневно он˚приходил к˚дому проживания Миши для˚бесед. В~знак признательности он˚подарил Библию с˚просьбой не~критиковать её.

В~экспедиции Миша в˚начале занимался схемой районной планировки, проектами внутрихозяйственного землеустройства. Через˚короткое время был назначен начальником отряда. Пришлось осуществлять контроль за˚подготовкой схем и˚проектов землеустройства, принимать перенос проектов в˚натуру. Разработанные проекты в то˚время утверждали на˚горисполкоме\footnote{Горисполком "--- городской исполнительный комитет Совета народных депутатов. Исполнительно-распорядительный местный орган государственной власти в˚СССР, аналог современных администраций.}. Начальнику отряда приходилось докладывать на˚горисполкоме о˚каждом проекте. Было дано указание, что при составлении проекта внутрихозяйственного землеустройства в˚севооборотах предусматривать 70~\% и˚более зерновых. В~какой-то степени нарушался принцип, что каждая предыдущая культура в˚схеме севооборота должна быть хорошим предшественником для последующей. Пришлось на˚горисполкоме доказывать, что при требовании иметь в˚севообороте ˚70\==80~\% зерновых выполнить этого невозможно. Сначала с˚мнением не~согласились, а˚затем разрешили ему отойти от˚этой установки.

В~отряде работали как инженеры\-/землеустроители, так и˚техники. Инженеры разрабатывали проекты землеустройства, а˚техники проводили обследование землепользований, съёмку местности, корректуру (сличение с˚натурой) планово\-/картографического материала, перенесение проекта на˚местность. Начальник отряда осуществлял приёмку работ, определял качество их˚выполнения с˚выездом в˚колхозы, совхозы. Техник Саша Варсеева, молодая девушка, среднего роста, полной комплекции, с˚большой грудью, крупным симпатичным лицом, переносила проект землеустройства в˚натуру. Приём работы планировался в˚восемь часов утра. Саша заявила, что она не~может проснуться в˚это время. «Меня однажды вынесли с˚кроватью в˚овраг и я не~проснулась». По˚приезду в˚хозяйство Миша поселился у˚хозяйки проживания Саши, а её по˚решению директора совхоза на˚ночь разместили в˚его кабинете. Утром прибежали от˚директора с˚заявлением: «не~можем разбудить» (директору необходимо освободить кабинет). Действительно, кричали, толкали, трясли… Саша не~просыпалась. Только˚когда начали лить на˚неё холодную воду, удалось разбудить.

По˚приезду во˚второе хозяйство на˚приёмку работ оказалось, что почвовед Грызлов, Варсеева и˚другие специалисты проживали во˚вновь выстроенном здании все вместе (койки стояли в˚ряд). Предстояло очень рано выходить в˚поле. Почвовед Грызлов поднимался рано. Саше было шутя предложено привязать её˚ногу за˚сапог почвоведа. Обует сапоги, он˚дёрнет, и˚Саша проснётся. Саша действительно натихоря это проделала. Ночью Грызлов проснулся, быстро вдел ноги в˚сапоги и˚начал бежать по˚своим надобностям во˚двор. Верёвки он не~видел и˚грохнулся со˚всего маха на˚пол. Естественно, всех разбудил, разразился сплошной смех. Впоследствии он˚обижался на˚шутку, проделанную над ним. Ведь˚это была не~шутка, а˚необходимость разбудить Сашу.

Техник Евсеев со˚стажем работы переносил проект внутрихозяйственного землеустройства в˚натуру в˚колхозе Милославского района. Территория землепользования была изрезана сплошными оврагами (сильное проявление линейной эрозии). Отдельные овраги были глубокими и˚длиной 300\==400\,м. При˚перенесении проекта границы полей и˚поворотные точки обозначались установленными знаками (закапывался деревянный столбик и˚окапывался курганом). Вместо˚положенного одного оврага техник закопал столбик на˚границе другого подобного оврага, через 500\,м. Оказывается, он не~ориентировался на˚местности и не~смог показать своё местонахождение на˚плане. Поэтому Миша предложил ему изменить место работы. Договорился с˚архитектурой области, и˚они приняли его. При˚очередной встрече Евсеев поблагодарил Мишу и˚заявил: «Здесь я˚оказался нужным и˚полезным работником, подобных ошибок не~совершаю».

Старший техник "--- женщина средних лет Роза Стерлигова, имевшая двух почти взрослых детей, переносила в˚натуру проект внутрихозяйственного землеустройства в˚одном хозяйстве Скопинского района. Мише необходимо было принять у˚неё работу. Приехавши в˚хозяйство, узнал у˚мальчика, где проживает землеустроитель. Он~ему любезно показал дом. С~улицы дверь была закрыта. Поэтому решил пройти с˚чёрного хода. При˚проходе по˚коридору Миша заметил, что в˚чулане лежит Роза с˚каким-то мужчиной. Сделал вид, что не~заметил их и˚вошёл в˚дом. Вскоре появилась и˚Роза, причём одна. «Миша, мы˚будем ужинать в˚ресторане»,  "--- проронила она. Предложила посмотреть кое-какие дела, а˚сама исчезла. Затем появилась хозяйка, небольшого роста, с˚горбинкой женщина. Постелила Мише постель и˚так же˚ушла.Чутье подсказывало Мише что-то неладное, не~должен ложиться здесь спать. Поэтому он˚пошёл вечером в˚соседний дом, который примыкал (единой стенкой) к˚этому дому. Оказалось, что там проживала старушка, а у˚неё квартировалась\footnote{Квартироваться (разг.) "--- жить, проживать в˚съёмной квартире.} зоотехник колхоза. Миша объяснил  ситуацию и˚попросил зоотехника переночевать в˚соседнем доме, на˚приготовленной ему кровати. Она согласилась, но˚предупредила, что у˚неё простыни несвежие. Миша лёг сразу спать, а˚старушка пристроилась на˚печке. В~2:00 ночи раздался резкий крик и в˚дом вбежала зоотехник в˚одной ночной рубашке. Оказывается, она спала и˚вдруг кто-то резко сорвал с неё одеяло. Перед˚ней стоял мужчина с˚поднятой рукой, а в˚руке у˚него был нож. Увидев женщину, он˚растерялся. В~это время она выскочила из˚кровати и˚убежала. Если˚бы Миша там оставался, то˚остался ли в˚живых? Перед˚зоотехником пришлось извиняться. Посмотрели в˚окно, по˚улице шли двое мужчин. Утром Миша попросил председателя колхоза вызвать милицию, а˚сам уехал. Оказывается, это был заключённый, сбежавший из˚тюрьмы. Со˚слов Розы выяснилось, что он˚подумал, что Миша приехал специально к˚ней в˚качестве очередного «любовника» и˚решил убить. Работу у˚Р.~Стерлиговой Миша не~принимал.

Старший преподаватель ГУЗа "--- Е.\,Н.~Первова по-прежнему привозила в˚Рязань студентов на˚производственную практику. Миша старался помочь, уделял им˚много внимания.

Однажды, во˚время работы Миши начальником отряда в˚Рязанской землеустроительной экспедиции, приехала Е.\,Н.~Первова, и они˚поехали (на˚машине) по˚хозяйствам к˚студентам, которые были на˚преддипломной практике в˚Милославском районе Рязанской области. Ездили допоздна, фактически наступила ночь, а˚им необходимо было возвращаться в˚Рязань. Поэтому, чтобы сократить время, поехали по˚полевым дорогам, выбирая покороче путь. Дорога привела на˚какой-то холм, взобрались на˚него на˚машине, а˚машина заглохла. Была опасность сорваться с˚холма. Кое-как переехали его, а˚дальше дорога привела к˚реке, и они˚решили вброд перебраться на˚машине на˚другую сторону реки. Мише пришлось нащупывать пешком более мелкие и твёрдые места в˚реке, машина двигалась за˚ним. Естественно, это было рискованное мероприятие, но˚таким образом удалось перебраться на˚другую сторону реки и˚сократить путь. 

Елена Николаевна сидела на˚переднем сидении машины, рядом с˚шофёром и не~проронила в˚это время ни˚одного слова. Лишь˚по приезду в˚Рязань к˚гостинице она сказала: «Михаил Павлович, у˚меня сердце было в˚пятках, думала погибнем». Она оказалась мужественной и терпеливой женщиной в˚экстремальных условиях.

После двух лет работы пришлось возвращаться в˚аспирантуру, но˚здесь возникла проблема. Рязанский обком партии\footnote{Обком партии (разг.) "--- областной комитет КПСС "--- высший орган областной организации КПСС между областными конференциями КПСС.} отказался дать характеристику в˚связи с˚тем, что Миша числился в˚резерве начальника экспедиции и˚должен оставаться в˚Рязани, так˚как должен был занять должность начальника экспедиции. Попросил дать отрицательную характеристику. Последовал ответ, что тем более этого делать нельзя. Пришлось агитировать А.\,С.~Козлова поехать в˚Рязань. Он работал главным инженером в Чечено\=/Ингушетии. В~обкоме партии устроили собеседование и˚отказали. Через˚неделю Миша снова поехал с˚А.\,С.~Козловым в˚Рязань. В~обкоме пришлось ему доказывать, что Козлов хороший специалист и˚организатор производства. На˚этот раз членов обкома удалось убедить. Так˚А.\,С.~Козлов стал начальником Рязанский экспедиции, а˚Мише выдали отличную характеристику. С~Рязанью, в˚частности с˚А.\,С.~Козловым, поддерживал постоянную связь. Ездил в˚командировку и на˚выходные к жене, так как она оставалась в Рязани.