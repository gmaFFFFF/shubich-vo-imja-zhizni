\clearpage

\thispagestyle{empty}
\noindent\begin{center}
	% вид издания по целевому назначению;
	{\itshape Литературно-художественное издание\par}
\end{center}	
	% заглавие серии;
	% номер выпуска серии;
\vfill\noindent
\begin{center}
	% полное имя автора (соавторов);
	{\large{\bfseries\MakeUppercase{\AuthorFam}} \AuthorName\\[3ex]}	
	% заглавие издания;			
	{\LARGE\Title}
	% полное имя составителя (составителей).
\end{center}

\vfill\noindent
\begin{center}
	Оформлено в системе MiK\TeX(\LaTeXe), макрос \KOMAScript\\[1ex]
	Исходный код для сборки книги размещён по˚адресу: \footnotesize\url{\Source}
\end{center}
	
\vfill\noindent
\begin{center}
	\small 
	{\itshape Корректор} Д.\,А.~Кулебянов\\[1ex]
	{\itshape Компьютерная вёрстка} М.\,А.~Гришкин\\[1ex]
	{\itshape Набор текста} М.\,Л.~Гришкина\\[1ex]
\end{center}

\vfill\noindent
\rule{\textwidth}{.5pt}
\begin{center}
	\small Подписано в печать  ??.??.2019. Формат 60х84/16. Усл. печ. л. 9,88. Тираж 50 экз. Заказ 100. Гарнитура ПТ~Санс, ПТ~Сериф.		
\end{center}
\rule{\textwidth}{.5pt}

\vfil\noindent		
\begin{center}
		\small
		
		Юридическое имя издателя и его адрес.\\[1ex]
				
		Юридическое имя полиграфического предприятия и его адрес.
\end{center}

\vfil
\noindent
\begin{minipage}[b]{0.2\textwidth}		% Столбец для выравнивания бар-кода по центру страницы
	\LARGE\textbf{\AgeLimit}
\end{minipage}
\hfil
\begin{minipage}[b]{0.5\textwidth}
		\begin{center}
			\begin{pspicture}(-15pt,-5pt)(1.7in,.85in)
				\psbarcode{\ISBN}{includetext width=1.5 height=.7 isbntextfont=PTMono addontextfont=PTMono textfont=PTMono}{isbn}
			\end{pspicture}
		\end{center}
\end{minipage}
\hfil
\begin{minipage}[b]{0.2\textwidth}	
	\begin{flushright}
		\hspace*{\fill}		% \hspace*{\fill}	в отличии от \hfil не съедает пробелы в начале и конце строки
	\end{flushright}
\end{minipage}

	

% Вставить пустую страницу
\clearpage
\thispagestyle{empty}
\label{myLastPage}
\null
\clearpage