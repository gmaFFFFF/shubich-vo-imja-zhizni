\chapter[Воспоминания об учителях и коллегах]{Воспоминания М.\,П.~Шубича об учителях и коллегах}

Сергей Александрович Удачин был крупным учёным, теоретиком социалистического землеустройства, основателем советской землеустроительной научной школы землеустройства. Будучи единственным академиком\-/землеустроителем ВАСХНИЛ, сумел отбить все˚нападки на˚землеустройство, добился сделать его государственным мероприятием, постоянного его развития. Он непосредственно принимал участие в˚подготовке всех нормативно\-/правовых актов по˚землеустройству и˚землепользованию. 

С.\,А.~Удачин был очень спокойным, внимательным, объективным и˚уравновешенным человеком. При˚принятии любого решения первоначально выслушивал мнения профессорско\-/преподавательского состава кафедры. 

Михаил Павловичу всегда было очень приятно и˚познавательно слушать его при ежедневных прогулках по˚улицам, недалеко от˚дома, когда проживал в˚общежитии.

Михаил Павлович считает себя виноватым перед ним из-за того, что не~прибыл к˚нему, когда позвонила жена с˚просьбой явиться к˚Сергею Александровичу. Михаил Павлович должен был быть на˚каком\=/то совещании по˚общественной работе и˚он отложил свой приход на˚следующий день, а˚завтра узнал, что Сергея Александровича не~стало.

С~ректором МИИЗа канд. техн., наук проф. Николаем Дмитриевичем Ильинским Михаилу Павловичу приходилось много раз встречаться как председателю комиссии по˚контролю за  деятельностью администрации. Это был обстоятельный, очень уравновешенный, спокойный человек, ректор «законник», строго соблюдал все требования делопроизводства и˚очень серьёзно относился к˚должности ректора. Имел влечение к˚игре в˚шахматы, будучи кандидатом в˚мастера по˚100\=/клеточным шашкам. Недовольство выражал при проигрывании партии в˚шахматы.

Яков Миронович Цфасман являлся крупным учёным советского землеустройства. По~натуре он˚был добрым человеком, всегда трепетно относился при защите курсовых проектов на˚комиссии в˚его подгруппе. В~этих случаях говорил Михаилу Павловичу: «Ты видишь ошибки на˚ходу, а˚я˚их не~замечаю». В~день похода на˚дачу (Чкаловское) "--- последний день его жизни зашёл к˚Михаилу Павловичу в˚21~аудиторию (кабинет декана заочного факультета) и˚очень долго беседовал, даже предлагал деньги на˚приобретение автомашины. Михаил Павлович отказался от˚подобной услуги из-за того, что не~сумел бы быстро вернуть долг. Я.\,М.~Цфасман пообещал по˚возвращению угостить Михаила Павловича яблоками из˚своего сада. Фактически получилось, что он˚как бы˚предвидел что-то нехорошее. По˚дороге на˚станцию он˚скончался.

Симпатизировал Михаилу Павловичу доцент Борис Николаевич Ширин своей неподкупностью, объективной требовательностью к˚студентам, справедливостью и˚честностью. Это был высококвалифицированный педагог, хороший воспитатель.

С~большой благодарностью Михаил Павлович всегда относился и˚относится к˚руководителю дипломного проекта доценту Елене Николаевне Первовой. Она была высокообразованным, спокойным, выдержанным, тактичным и˚требовательным преподавателем, умеющим доходчиво излагать материал студентам, хорошим методистом, воспитателем, умелым организатором производственных практик. Длительное время она была заместителем заведующего кафедрой земпроектирования.

Профессор Сергей Николаевич Волков является хорошим организатором, крупным учёным, умело работает на˚себя. Любит доносчиков, подхалимов, является злопамятным человеком. Если˚человек не~из˚его окружения, или не~из˚ранее сказанного разряда, то˚относится пренебрежительно. 

Высококвалифицированными педагогами, крупными учёными, хорошими воспитателями на кафедре землеустройства являются: проф., д.\,э.\,н. Владимир Васильевич Косинский, доц. д.\,э.\,н. Тимур Валикович Папаскири, проф. д.\,г.\,н. Александр Владимирович Донцов, проф., д.\,э.\,н.~Елена Вячеславовна Черкашина, проф., к.\,э.\,н. Виталий Николаевич Сёмочкин, проф., к.\,э.\,н. Владимир Владимирович Пименов, проф., к.\,э.\,н. Владимир Васильевич Пронин, доц., к.\,э.\,н. Надежда Михайловна Матасова, доц., к.\,э.\,н. Владимир Павлович Радионов.
