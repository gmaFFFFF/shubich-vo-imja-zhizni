\chapter{Предисловие}
В~книге изложены жизненный путь М.\,П.~Шубича, жизнь взрослых и˚детей Белоруссии под оккупацией немцев, трудности, встречающиеся на˚этом пути и~их˚преодоление, цели и~их˚достижение.

Читатель узнает об отдельных персонажах, работавших с˚автором, его отношении к˚ним: людях, учёных, сформировавших его˚как˚учёного землеустройства, гражданина и˚просто человека. Эта книга "--- учебник жизни человека, его˚переживаний, страданий.

Книга будет интересна прежде всего молодым людям, выпускникам, студентам, школьникам, начинающим свой жизненный путь. Она~расскажет о˚различных трудностях, ухабах, встречающихся на˚жизненном пути, возможностях их˚преодоления при˚выработке силы воли, определённой закалки. По~мнению автора, достижение успехов в˚жизни возможно при˚усидчивости, настойчивости и˚стремлении к˚поставленной цели.

Все факты, изложенные в˚книге, взяты из˚жизни, отсутствует что-либо выдуманное. Эти воспоминания изложены для˚того, чтобы, прочитав˚их, молодые люди поняли, какие беды приносит война, и˚какие трудности она˚принесла взрослым и˚детям.