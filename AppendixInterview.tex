\chapter[Интервью с М.\,П.~Шубичем]{Интервью с˚Михаилом Павловичем Шубичем}

\begin{drama}
	\setlength{\speaksindent}{5em}
	\setlength{\Dlabelsep}{5em}
	\Character{Максим Гришкин:}{max}
	\Character{Михаил Павлович Шубич:}{michael}
	
	\maxspeaks Михаил Павлович, Вы "--- незаурядный человек, у˚Вас большой жизненный и˚профессиональный опыт, по˚принципиальным вопросам Вы имеете твёрдую позицию. Вашим читателям и˚мне в˚особенности будет интересно узнать Ваше мнение по˚наиболее острым проблемам образования, землеустройства, политики и˚экономики. Поэтому разрешите задать Вам несколько порой провокационных вопросов.
	\michaelspeaks Пожалуйста.
	
	\Character{МГ:}{max}
	\Character{МПШ:}{michael}
	\maxspeaks Михаил Павлович, почему Вы написали данную книгу?
	\michaelspeaks Данную книгу написал, чтобы молодым людям, проживающим в˚большинстве своём в «тепличных» условиях, показать, что на˚жизненном пути могут быть трудности, неприятности, преодолеть которые возможно лишь выработав силу воли, имея стремления и˚поставленные цели.
	\maxspeaks Ваше самое большое достижение?
	\michaelspeaks Самое большое достижение в˚том, что окружающие меня люди понимали меня, а я "--- их. В~большинстве своём поддерживали меня.
	\maxspeaks Что˚Вы думаете о˚религии? 
	\michaelspeaks Не~совсем верующий человек, но не~отрицаю религию, которая может сплотить массы людей.
	\maxspeaks Вы родились на˚территории современной Беларуси, служили в˚армии на˚Украине, работали в˚России. Что˚такое для Вас Родина?
	\michaelspeaks Родина "--- место рождения, взросления и˚становления человека, обретения необходимых человеческих черт, любви к˚людям и˚окружающему миру.
	\maxspeaks Про˚что Вы любите читать книги? смотреть фильмы? слушать песни?
	\michaelspeaks Мои любимые книги, фильмы, песни о˚честных тружениках (людях), делающих добро и˚приносящих радость другим людям.
	\maxspeaks Ваши студенты знают, что Вы прекрасно умеете считать в˚уме. Как˚Вы этого достигли?
	\michaelspeaks Считать в˚уме хорошо обучала советская школа. Достигается это прежде всего тренировками.
	\maxspeaks Вы работали до˚80~лет и˚могли бы˚ещё. В~чём секрет Вашего трудового долголетия?
	\michaelspeaks С~самого раннего детства я˚трудился, рассчитывал только на˚себя. Не~имел вредных привычек. В~этом весь секрет. 
	\maxspeaks Вы не~жалеете, что стали землеустроителем?
	\michaelspeaks Я~родился на˚земле, вырос на˚ней. Люблю землю. Поскольку˚землеустройство связано с˚землёй, то я˚полюбил его.

\pagebreak
	
	\StageDir{О~политике}
	\maxspeaks Говорят, что в˚приличном обществе не~принято обсуждать политику. Вообще популярность политических тем означает, что в˚обществе не~всё благополучно. В~нашем обществе, к˚сожалению, есть проблемы и˚противоречия, поэтому я˚вынужден затронуть некоторые политические темы… Есть˚ли у˚русского народа национальная идея?
	\michaelspeaks В~русских людях имеется национальная идея "--- это прежде всего мир, благополучие каждого, процветание страны. Но… правительство её не~выработало.
	\maxspeaks Как˚Вы думаете, если бы˚сейчас повторились события 1941~года, то˚могло бы˚современное поколение выстоять? Или˚мы стали слишком расслабленными и˚больше не~способны на˚такой подвиг?
	\michaelspeaks Думается, случись сейчас война, олигархи, зажиточные люди и их˚сынки сбежали бы за˚границу. Простые труженики рьяно защищали бы˚своё Отечество.
	\maxspeaks События «Крымской весны» разделили наших граждан совершенно разного социального положения и˚политических убеждений на˚два лагеря разделяющих противоположные убеждения: 1) Крым наш 2) Крым незаконно аннексирован у˚Украины. Чей Крым?
	\michaelspeaks Крым был наш и˚будет в˚будущем, чтобы ни~хотели люди нелюбящие Россию.
	\maxspeaks Вы были свидетелем распада СССР. Почему˚это произошло, по˚Вашему мнению? 
	\michaelspeaks Распад СССР произошёл, прежде всего, по˚причине предательства Горбачёва, Ельцина, Кравчука, Шушкевича и с˚им подобным, а˚также действий недружеских нам государств.	
	\maxspeaks Что˚надо сделать России, чтобы подобное не~повторилось? 
	\michaelspeaks Чтобы˚подобного не~произошло, вся политика нашего государства должна быть направлена на˚защиту и˚заботу о˚простых людях (народе).
	\maxspeaks Возможно˚ли восстановление СССР?
	\michaelspeaks Восстановление СССР в˚прежнем виде невозможно. 
	\maxspeaks Какую политическую партию Вы поддерживали на˚последних выборах? 
	\michaelspeaks Коммунистическую партию, так как её˚программа направлена на˚улучшение жизни простых людей (народа).
	\maxspeaks Есть мнение, что Государственная Дума практически не~осуществляет реальную законотворческую деятельность, а послушно «штампует» законы, подготовленные Правительством России. Вас устраивает, как трудятся 450~народных избранников в˚Государственной Думе? 
	\michaelspeaks В~Государственной Думе много карьеристов, людей, стремящихся к˚богатству, богачей, которые далеки от˚народа. Поэтому её˚работа, принимаемые законы, не˚всегда направлены на˚защиту интересов народа. Мне представляется, что спортсменам, артистам и˚подобным им˚там делать нечего.
	\maxspeaks Как˚Вы относитесь к˚выборной избирательной системе? 
	\michaelspeaks Избирательная система, на˚мой взгляд, ещё далека от˚объективности. По\=/прежнему действует административный ресурс "--- ресурс партии «Единая Россия». Чтобы˚как\=/то исправить положение дел, необходимо прежде всего каким\=/то образом избавиться от˚этих ресурсов, усилить контроль со˚стороны ЦИК, подбирать в˚избирательные комиссии объективных людей.	
	
	\pagebreak
	
	\maxspeaks В~соответствии с˚принципом верховенства права никто не~может быть выше закона. Но˚законы пишут не˚боги, а˚несовершенные по˚своей природе люди. Значит˚они в˚соответствии с˚принципом информатики  «мусор на˚входе — мусор на˚выходе» ещё менее совершенны, чем люди. Как˚Вы думаете, мы˚действительно должны принять за˚постулат необходимость слепого подчинения законам, т.~к. это меньшее из˚зол или можно предложить альтернативу?
	\michaelspeaks Перед˚справедливыми законами все граждане должны быть законопослушными, иначе в˚государстве будет хаос.

	\StageDir{Об˚экономике}
	\maxspeaks В.\,И.~Ленин говорил, что политика есть концентрированное выражение экономики, поэтому давайте перейдём к˚этой близкой для Вас, как кандидата экономических наук, теме. Вы работали и˚при социалистической экономике, и˚при капитализме. И~та и˚другая системы в˚чистом виде демонстрируют нам свою несостоятельность. Есть мнение, что будущее за˚неким их˚симбиозом, который, например, складывается в˚Китае. Можно˚ли скрестить ежа с˚ужом (капитализм и˚социализм)?
	\michaelspeaks Для˚России капитализм неприемлем, тем более капиталистический рынок. У~России в˚настоящее время не˚рынок, а˚базар. Считаю, что наиболее приемлема социалистическая система, создающая условия для раскрытия способностей каждого человека.
	\maxspeaks С~90\=/х гг. XX~века Россия вернулась к˚капитализму, практикует неолиберализм, выполняет инструкции МВФ, интегрируется в˚какие\=/то мировые структуры и т.~д. При˚этом на˚мой взгляд экономика России стагнирует. Для˚сравнения за˚такой же по˚продолжительности период с˚1922 по˚1941~гг. практически с˚нуля была построена экономика, способная победить силы объединённой фашистской Европы. Что˚делать? 
	\michaelspeaks Для˚поднятия экономики, во\=/первых, необходимо убрать либералов с˚различных руководящих постов в˚экономике. Либералы стараются «смотреть» на˚Запад и˚жить под их˚диктовку. Во\=/вторых, не~дорабатывает в˚этом плане и˚Правительство (кстати, там засело много либералов).

	\StageDir{О~работе}
	\maxspeaks На˚работе проходит большая часть жизни человека. Хотелось, чтобы Вы поделились с˚читателями своим опытом. Кто Вы в˚большей степени: землеустроитель, педагог, администратор или учёный?
	\michaelspeaks Содержанием педагогической деятельности является обучение, воспитание, образование, развитие обучающихся. Поэтому высококвалифицированный педагог должен владеть и˚вести на˚высоком уровне методическую, научно\-/исследовательскую и˚воспитательную работы. Причём˚он должен уметь адаптироваться к˚постоянно меняющимся условиям, иметь глубокое знание предмета обучения. Поэтому считайте, что я˚одновременно педагог, землеустроитель и˚учёный.
	\maxspeaks Какой˚главный совет Вы могли бы˚дать начинающему специалисту?
	\michaelspeaks Главный совет "--- это постигать азы науки, специальности. При˚этом владеть (выработать) силу воли, быть целеустремлённым, не~тратить время впустую.
	\maxspeaks Длительное время Вы занимали высокую руководящую должность "--- декан факультета. Что˚нужно сделать чтобы построить успешную карьеру? Какие качества нужны хорошему руководителю?
	\michaelspeaks Чтобы˚построить карьеру, быть хорошим руководителем, считаю необходимым быть объективным, справедливым, постоянно работать над собой (повышать свои знания).
	\maxspeaks Студенты знают Вас как строгого и˚принципиального преподавателя. Если˚студент получил  «у˚Шубича» оценку  «Х», то˚это значит, что он˚действительно знает предмет минимум на  «Х». Т.~е. оценка, которую Вы поставили, имеет  «настоящую», хоть и˚нематериальную ценность. Из˚преподавателей ГУЗа, у˚которых мне довелось учиться, похожий  «стиль» был только у˚доцента Кафедры геодезии и˚геоинформатики ГУЗ Елены Георгиевны Парамоновой. Почему˚именно такой  «стиль» работы Вы выбрали? Вы считаете его оптимальным или есть более эффективные подходы к˚работе со˚студентами?
	\michaelspeaks Преподаватель вуза должен быть объективным при оценке знаний студента и не~унижать достоинства человека. В~необходимых случаях оказывать студенту помощь.

	\StageDir{О~высшем образовании}
	\maxspeaks Как˚Вы относитесь к˚ЕГЭ? Оценки абитуриентов стали более объективными?
	\michaelspeaks Считаю ЕГЭ в˚учебном процессе неприемлемым. При˚ответах на˚поставленные вопросы экзаменуемый как бы˚занимается гаданием, а не~раскрывает свои знания. Считаю, вместо ЕГЭ следует вводить экзамены, чтобы человек мог изучать более обстоятельно предмет, а не~заниматься натаскиванием.
	\maxspeaks Можно˚ли сказать, что  «диплом» в˚России по˚сравнению с˚советскими временами «девальвировался» и˚его получают не˚для знаний, а˚для бумажки, хотя прекрасно понимают, что  «зря потратят время»?
	\michaelspeaks Уровень образования в˚Советском Союзе был более высокий, чем сейчас в˚России. Это отмечали даже приезжавшие делегации из˚США. Много ненужного нам навязали решением Болонской конференции, которое подписала и˚Россия. С~начала 90\=/х годов (годов хаоса) культивировалась продажа дипломов. Даже˚отдельные врачи были с˚купленным дипломом. Поэтому такой специалист с˚подобным дипломом для общества ничего не~представляет. Чтобы˚диплом был «ревальвирован», необходимо улучшить уровень преподавания. Считаю, что для этого необходимо улучшить подготовку в˚школах, проводить более тщательный отбор преподавателей, отменить ЕГЭ. При˚поступлении в˚вуз (на˚вступительных экзаменах) ориентироваться следует не на˚выполнение любыми способами плана, доведённого сверху, а˚отбирать более успешно сдавших вступительные экзамены абитуриентов. Финансирование вуза не~должно быть от˚количества набранных студентов. Считаю, что для России более эффективным было заменить бакалавриат подготовкой специалистов в˚течение пяти лет. Бакалавр фактически является неполноценным специалистом, который не~может эффективно решать поставленные задачи. Наверняка, данную позицию полностью поддержат работодатели. На˚рынке труда более полезным и˚востребованным окажется высококвалифицированный специалист. Для˚подготовки таких специалистов высшему образованию на˚современном этапе необходимы грамотные преподаватели, владеющие современными технологиями педагогической деятельности, постоянно ведущие научно\-/исследовательскую работу.	
	\maxspeaks Видите ли Вы разницу между человеком с˚дипломом и˚без˚него?
	
	\pagebreak
	
	\michaelspeaks Да, вижу. Человек, добросовестно обучавшийся в˚вузе, более приспособлен к˚жизни, может быстрее адаптироваться в˚тех или других условиях, к˚новым введениям.		
	\maxspeaks Из˚Ваших воспоминаний следует, что Вы поступили в˚ГУЗ по˚большому счёту случайно, прочитав объявление в˚газете. Вы вполне могли поступить в˚БНТУ или в˚МИИГАиК, находящийся по˚соседству с˚ГУЗом. Почему˚ваш выбор остановился именно на˚ГУЗе?
	\michaelspeaks В~то˚время в˚деревне получить справку, паспорт молодому человеку было невозможно, а˚тем более как\=/то ориентироваться, осуществить задуманное. Даже˚не~было материалов, по˚которым можно было бы˚ориентироваться. Поэтому всё приходилось делать при представлении подобной возможности.
	\maxspeaks Почему˚абитуриент не~может поступить хотя бы˚просто в˚университет, а не на˚факультет. И~дальше выстроить свою образовательную траекторию в˚зависимости от˚своих способностей и˚призвания. Например, одному студенту понравилась геодезия и он˚пошёл дальше её˚изучать "--- для него открывается возможность стать геодезистом. А~другому геодезия даётся с˚трудом, и он˚движется в˚сферу кадастра или землеустройства. 
	\michaelspeaks Студента должна интересовать определённая специальность. Возможностей ознакомиться с˚ней теперь имеется предостаточно. Поскольку˚в университете имеется несколько специальностей, то˚абитуриент не~может поступать просто в˚университет, так как он не~сможет овладеть несколькими специальностями за˚время обучения.
	\maxspeaks Хорошо, почему бы тогда не~оставить возможность студенту, который понял, что он  «попал не˚туда», уйти, не~потеряв потраченное время? Например, набрав определённый набор компетенций в˚течение года, студент должен иметь возможность, грубо говоря, уйти с˚дипломом  «техника землеустроителя», через два года "---  «помощника землеустроителя», а˚через 5~лет уже  «инженера землеустроителя». Это очень важно, потому как свыше 50~\% моих однокурсников уже на˚2~курсе поняли, что они не~хотят заниматься землеустройством. Для˚чего Российская Федерация ещё 3~года тратила на˚них деньги?
	\michaelspeaks Студент после окончания первого курса может перейти в˚другой вуз, на˚любую специальность, если в˚данном вузе оказалось, что «попал не˚туда». Для˚этого ему необходимо досдать предметы, которые предусмотрены по˚программе в˚новом вузе на˚данной специальности.
	\maxspeaks У~Михаила Жванецкого есть замечательное высказывание:  «Когда˚знаешь как, умеешь, но˚уже не~можешь сам "--- ты˚тренер. Когда˚знаешь как, не˚умеешь и˚не~можешь сам "--- ты˚профессор. Когда˚не~знаешь как, не~умеешь и не~можешь сам, но˚можешь наказать, если эти с…ки˚сделают не˚так "--- ты˚президент». Вузовские преподаватели часто обладают только книжными знаниями о˚предмете или имеют минимальный практический опыт. Если˚по теоретическим дисциплинам (математика, философия, история и т.~д.) это нормально, то по˚практическим дисциплинам (земпроектирование, геодезия, межевание) данная ситуация вызывает серьёзные опасения, не˚зря П.\,Л.~Чебышёв говорил:  «теория без практики мертва и˚бесплодна, практика без теории бесполезна и˚пагубна». Вы видите в˚этом проблему?
	\michaelspeaks Нормальная производственная практика для студента, а˚тем более для преподавателя, необходима. Многие наши преподаватели ГУЗа выросли, воспитались, получили образование в «тепличных» условиях. Это является ненормальным. Считаю, что преподаватель вуза должен не˚только получить соответствующее образование, но˚пройти определённую «жизненную» школу, производственную практику по˚специальности. Такая производственная практика для обучающихся студентов крайне необходима. Она помогает лучше усвоить материал, вести более эффективно научные исследования. В~настоящее время она фактически отсутствует.
	
	\StageDir{О~коррупции}
	\maxspeaks Мы часто слышим, что основная проблема России "--- коррупция. Как˚Вы думаете, у˚нашего государства есть более значимые проблемы, чем коррупция?
	\michaelspeaks Более значимыми проблемами являются воспитание молодёжи, борьба с˚наркоманией, проституцией, алкоголизмом.
	\maxspeaks Может быть Вы знаете, как побороть коррупцию? 
	\michaelspeaks Для˚борьбы с˚коррупцией необходимо улучшить воспитательную работу, усилить контроль и˚повысить ответственность, более правильно подбирать кадры.
	\maxspeaks В~не˚очень давние времена дети часто шли по˚стопам своих родителей: у˚пекаря сын становился пекарем, а у˚кузнеца – кузнецом. В~советское время трудовые династии существовали не˚только у˚заводских работников "--- после смерти А.\,Н.~Туполева руководить конструкторским бюро стал его сын А.\,А.~Туполев. В~современной России дети крупных государственных чиновников тоже «идут по˚стопам» своих родителей и˚тоже занимают высокие государственные должности. Как˚Вы относитесь к˚трудовым династиям?
	\michaelspeaks К~трудовым династиям отношусь положительно, но˚здесь не~должны ущемляться цели и˚стремления молодых людей.
	
	\pagebreak
	
	\StageDir{О~Государственном университете по˚землеустройству (ГУЗ)}
	\maxspeaks Следующая группа вопросов будет посвящена работе Вашей альма\-/матер. Образовательные программы ГУЗ и˚МИИГАиК во˚многом пересекаются: и˚там и˚там можно стать геодезистом, заниматься кадастром, развитием территории. Стоит˚ли воссоединить ГУЗ и˚МИИГАиК? 
	\michaelspeaks В~МИИГАиКе ввели ряд специальностей подобных ГУЗу, хотя соответствующих преподавателей не~хватало. Они всегда ратовали за˚объединение с˚ГУЗом, чтобы овладеть территорией. Считаю, что любое объединение не~улучшает учебный процесс, причём есть отличия в˚учебных программах. Следует заметить также, что ГУЗ готовит специалистов прежде всего для сельского хозяйства.		
	\maxspeaks Вам часто встречаются горожане, которые после выпуска планируют связать свою жизнь с˚сельским хозяйством? Много˚ли москвичей переехало в˚глубинку  «поднимать село»? Почему˚в условиях отмены обязательной системы распределения выпускников на˚факультет землеустройства принимают городских жителей и˚москвичей в˚особенности? 
	\michaelspeaks Вы правы, ГУЗ должен прежде всего производить набор абитуриентов из˚сельской местности. Но˚любой человек (в~том числе из˚города) имеет право на˚образование. Поэтому ущемлять его права никому не~позволено. 
	\maxspeaks Уместно ли, что сельскохозяйственный вуз находится в˚городе, в˚котором не~осуществляется сельскохозяйственное производство? Может ГУЗ хотя бы в˚Московскую область перевести?
	\michaelspeaks Хрущёв собирался переселить МИИЗ в˚сельскую местность. Для˚этого отсутствует соответствующая база (её˚следует создавать заново), да с˚преподавателями будет проблема.
	\maxspeaks Мы уже достаточно долго живём в˚так называемую информационную эру, но˚выпускники ГУЗа не~обладают компетенциями в˚сфере проектирования и˚создания компьютерных информационных систем. Это является проблемой?
	\michaelspeaks Выпускник должен не˚только владеть информацией, соответствующей техникой, но˚уметь анализировать, обобщать, делать выводы из˚полученной информации. Думается, что в˚настоящее время выпускник ГУЗа в˚основном владеет вычислительной техникой.
	\maxspeaks Лесоустройство по˚духу достаточно близко к˚землеустройству. Почему˚в ГУЗе нет предмета лесоведение, лесоводство или лесоустройство?
	\michaelspeaks Во\=/первых, лесоустройство имеет свои особенности, во\=/вторых, не~следует отнимать «хлеб» у˚других.
	\maxspeaks У~меня были преподаватели, которые не~знали свой предмет на˚минимально необходимом уровне, или те, которые просто прогуливали занятия. Как˚такое вообще могло быть? 
	\michaelspeaks При˚советской системе для преподавания в˚вузе очень тщательно производился отбор преподавателей. Не˚секрет, что в˚ГУЗе, и в˚частности на˚кафедре землеустройства, работает ряд преподавателей, которых бы˚раньше «на˚пушечный выстрел» не~допустили. Но˚что поделаешь, «на˚безрыбье и˚рак рыба». Считаю, подобное явление недопустимым. Об˚этом я˚говорил и˚ректору ГУЗа проф. С.\,Н.~Волкову.
	\maxspeaks Когда˚я учился в˚ГУЗе, то˚меня удивляло, почему практически все книги по˚землеустройству проходят за˚авторством или под редакцией ректора ГУЗа С.\,Н.~Волкова. Прошло 10~лет, но˚ничего не~изменилось. В~качестве примера можете на˚сайте ГУЗа зайти в˚раздел издательская деятельность (https://www.guz.ru/nauka/izdatelskaya-deyatelnost/knigi/) и˚убедиться, что 19 из˚21~книги сделаны при участии Сергея Николаевича. На˚кафедре землеустройства больше никто книги писать не~умеет? 
	\michaelspeaks Вы совершенно правы, что многие авторы в˚своих статьях, книгах по˚землеустройству вставляют автором или под редакцией ректора ГУЗ С.\,Н.~Волкова, хотя он в˚этом не~участвовал. Считаю, это неправильно. Я~подобного не~делал. 
	\maxspeaks В~последнее десятилетие в˚связи с˚развитием Интернета широкое распространение получает так называемое открытое образование "--- система бесплатных онлайн\-/курсов по˚базовым дисциплинам, изучаемым в˚вузах. Т.\,е.˚любой человек может совершенно бесплатно получать знания, не~выходя из˚дома. Конечно без дипломов и˚сертификатов. Имеет˚ли смысл ГУЗу развивать это направление?
	\michaelspeaks Считаю, что развивать в˚нашем вузе бесплатную систему онлайн\-/курсов не~следует.
	\maxspeaks ГУЗ имеет статус университета. Университет "--- это многопрофильное учебное заведение, являющееся ведущим научно\-/методическим центром в˚соответствующей области деятельности. 
Лично я˚как специалист в˚сфере землеустройства не~вижу принципиальной разницы между следующими факультетами ГУЗа: землеустройство, кадастр недвижимости и с˚некоторой натяжкой городского кадастра. Если˚вывести за˚скобки дитя\footnote{В~90\=/е гг. практически в˚каждом техническом вузе появился юридический или экономический факультет с˚низким уровнем преподавания.}  «лихих девяностых» "--- юридический факультет, то˚останется ещё только один полноценный\footnote{С~точки зрения интервьюера заочный факультет и˚факультет второго высшего образования не~являются  «полноценными» в˚том, смысле, что они готовят тех же˚специалистов, что и  «полноценные» факультеты.} факультет "--- факультет архитектуры. Если˚в учебном заведении 2,5~факультета, то˚его вряд˚ли можно назвать многопрофильным. Вы согласны с˚этим?
	\michaelspeaks Не˚все граждане России могут получить образование очно. Поэтому для получения образования без отрыва от˚производства созданы заочные факультеты. Практика показала, когда на˚заочный факультет производился набор только по˚специальности или родственной специальности, заочники показывали более высокий уровень знаний по˚сравнению с˚очниками.
	\maxspeaks По˚Вашему мнению, ГУЗ заслуживает статус университета?
	\michaelspeaks ГУЗ в˚основном соответствует статусу университета, поэтому ему присвоен этот статус. Специальность землеустройство "--- широкая специальность, требующая знаний экологии, земельного кадастра, земельного законодательства и т.~д. Землеустройство необходимо почти во˚всех отраслях, даже для Министерства Обороны. Как˚можно строить земельные отношения без землеустройства? Беда в˚том, что государство этого не~поняло, не~оценило и˚фактически уничтожило землеустройство.
	\maxspeaks Как\=/то так получилось, что я, практически троечник в˚школе, получил красный диплом в˚вузе. Мне порой даже кажется, что красные дипломы в˚ГУЗе раздавали практически всем, кроме совсем ленивых студентов. По˚моим субъективным ощущениям примерно 30~\% выпускников получили красный диплом. У~меня, как инсайдера, нет оснований считать, что большая доля выпускников с˚красным дипломом связана с˚изначально высокой мотивацией и˚подготовкой студентов, или какими\=/то выдающимися методиками преподавания в˚ГУЗе. Скорее, это следствие небрежной оценки знаний студентов или откровенной безответственности преподавателей. Не˚слишком˚ли много красных дипломов в˚ГУЗе?
	\michaelspeaks Ошибочное существует суждение, что если в˚школе ученик обучался на˚тройки, то в˚вузе он не~может обучаться на˚отлично. Ведь˚он повзрослел, появилась мотивация получить образование, стал относиться к˚своим делам более серьёзно. Красный диплом выдают студентам, получившим за˚время обучения в˚вузе 75~\% отличных оценок, не~имеющих ни˚одной тройки. Поэтому не˚все его могут получить.
	\maxspeaks Многие говорят, что проблемы в˚образовании существуют в˚основном из-за недофинансирования. За˚десять лет своей трудовой деятельности я˚сделал наблюдение, что увеличение заработной платы выше определённого уровня не~оказывает заметного позитивного влияния на˚производительность труда. Наоборот люди начинают создавать видимость напряжённой работы, у˚них появляется страх потерять её, пропадает желание рисковать, необоснованно растёт самооценка. Если˚же зарплата существенно ниже какого\=/то оптимального уровня, то˚часть работников увольняется вне зависимости от˚наличия каких\=/либо нематериальных стимулов, а˚часть всё равно продолжает работать! Вообще зачастую заработная плата бывает слабо связана со˚знаниями человека и˚его трудовой активностью, а˚определяется связями, умением  «продать себя подороже» или банальной удачей. На˚что влияет размер заработной платы?
	\michaelspeaks От˚уровня зарплаты зависит стремление работать, повышать свои знания.
	\maxspeaks Обеспечивает˚ли государство достойную заработную плату преподавателям? 
	\michaelspeaks Государство ещё не~обеспечивает достойную заработную плату преподавателям. 
	\maxspeaks Это касается всех вузов или только ГУЗа?
	\michaelspeaks Подобное явление касается в˚основном преподавателей всех вузов страны. В~отдельных вузах заработная плата несколько выше, чем преподавателей в˚ГУЗе.
	\maxspeaks Влияет˚ли на˚уровень преподавания и˚научной деятельности в˚ГУЗе зарплата? Если˚бы минимальная зарплата в˚ГУЗе стартовала от˚5 средних зарплата в˚регионе, что бы˚это изменило?
	\michaelspeaks Уровень преподавания, конечно, зависит от˚заработной платы преподавателя. Более высокую зарплату получают более высококвалифицированные преподаватели. Если˚была бы˚высокая заработная плата, то в˚вузе не~было бы «бездарностей».
	\maxspeaks Какой˚Вы считаете оптимальной систему оплаты труда преподавателей? Какой˚минимальной допустимый уровень оплаты труда нужен в˚ГУЗе?
	\michaelspeaks Заработная плата преподавателя должна быть сопоставима со˚средней зарплатой в˚экономике России.
	\maxspeaks Разумно˚ли молодому выпускнику планировать карьеру в˚ГУЗе?
	\michaelspeaks Если˚молодому выпускнику ГУЗа нравится преподавательская деятельность и у˚него имеются склонности и˚возможности подготовить диссертацию, то он˚должен поступить на˚должность преподавателя. От˚этого зависит и˚карьерный рост.	
	
	\pagebreak
	
	\maxspeaks Одним из˚показателей оценки эффективности деятельности вузов является количество публикаций в ISI Web of Science и Scopus. Поощряется публикация научных статей именно на˚английском языке. Я~понимаю, что может быть для учёного это хорошо, его заметят и˚возьмут работать где\=/нибудь на˚Западе (конечно, это не~касается землеустроителей). Например, по\=/настоящему выдающийся учёный Г.\,Я.~Перельман опубликовал статьи с˚методом доказательства гипотезы Пуанкаре на˚английском языке. Как˚мне, простому инженеру, получить доступ к˚последним достижениям российской науки, если я, например, в˚школе и˚институте учил немецкий язык. Мне надо ещё один язык выучить для того, чтобы иностранцы особо не~утруждались, перенимая российские достижения? Как˚Вы относитесь к˚такой интернационализации науки?
	\michaelspeaks В~настоящее время одним из˚показателей уровня работы преподавателя является количество публикаций на˚иностранных языках и˚ссылок на˚публикации. Восприятие работы (статьи) во˚многом зависит от˚широты и˚известности тематики, и от˚того, на˚каком языке она опубликована. Более узкая тематика воспринимается хуже и˚может не~быть на˚неё ссылок. Основные иностранные публикации осуществляются на˚английском языке. Если˚вы изучали немецкий язык и им˚владеете, не˚беда. Главное заключается в˚том, чтобы хорошо владеть хотя бы˚одним иностранным языком. Конечно неплохо ещё знать и˚английский язык.
	\maxspeaks В~СССР степень кандидата наук позволяла зарабатывать её˚обладателю больше, а например, в˚академической среде, кандидатская степень являлась обязательным условием для продвижения по˚карьерной лестнице. Сейчас заработные платы кандидатов наук практически не~отличаются от˚зарплат специалистов и˚даже бакалавров. При˚этом когда ты˚учишься в˚якобы бесплатной аспирантуре, ты˚теряешь деньги за˚счёт низких трудовых доходов аспиранта, а˚также несёшь прямые расходы на публикацию статей в˚ВАКовских журналах, особенно в˚каком\=/нибудь иностранном журнале,  «благодарность в˚конверте» членам диссертационного совета за˚труды,  «заносы конвертов» оппонентам и˚рецензентам, ещё нужно не~забыть  «проставиться» после защиты. По˚современным деньгам, наверное, 150\==200~тыс. рублей получается? По˚какой причине следует идти в˚аспирантуру молодым специалистам?
	\michaelspeaks Молодым специалистам следует поступать в˚аспирантуру для престижа, достижения профессионального уровня, получения опыта педагогической деятельности в˚вузе и, естественно, карьеры. Аспирантура при вузах является составной частью единой системы непрерывного образования и˚степенью послевузовского образования.
	\maxspeaks Хорошо. Кандидатская диссертация, как я её˚понимаю, "--- это квалификационная работа, призванная показать квалификацию её˚автора. Конечно, она решает какую\=/то задачу, т.~е. даны начальные условия и˚аспирант должен, используя необходимый аппарат, найти её˚решение, ну,˚или, если необходимый аппарат ещё не~придуман, то˚разработать его. Но˚цель кандидатской диссертации именно показать квалификацию её˚автора, а˚сама задача является вопросом вторичным. У~меня возникает резонный вопрос: если диссертация нужна для того, чтобы удовлетворять амбиции автора и˚способствовать его карьере, то˚почему за˚это платят налогоплательщики? Может быть необходимо отменить диссертации, т.\,е.˚если человек хочет заниматься наукой, пусть он ею и˚дальше занимается, не~тратя своё время и˚деньги на˚решение задач, не~имеющих практической ценности, а˚его квалификацию оценивать какими\=/то иными, менее затратными способами? Зачем˚карьеру преподавателя в˚вузе ставить в˚зависимость от˚наличия или отсутствия у˚него учёной степени? Разве˚учёных званий недостаточно? Пусть˚он лучше 3~года предельно внимательно к˚своим студентам относится, чем за˚счёт налогоплательщиков будет наполнять мусорное ведро тривиальными сведениями, изложенными наукообразным языком и˚невостребованными в˚народном хозяйстве?
	\looseness=-1
	
	\michaelspeaks Толковать, что диссертационная работа нужна лишь для удовлетворения амбиций автора неправомерно. В~повышении квалификации того или иного гражданина заинтересованы не˚только сам гражданин, но и˚предприятие, вуз, и в˚целом общество, так как от˚этого человека будет больше отдачи. Без˚исследований и˚внедрения результатов не~может быть прогресса в˚стране (обществе). При˚защите диссертации оценивается научная и˚практическая ценность диссертационной работы.
	\maxspeaks Если˚посмотреть объективно, сколько кандидатских диссертаций в˚области землеустройства за˚последние 20~лет имели настоящую, а не˚бумажную практическую ценность, что называется  «по˚гамбургскому счёту»? Т.\,е.˚чтобы до˚диссертации все решали эту задачу одним образом, а˚после массово стали делать по\=/другому. Одна? две? или всё\=/таки ни˚одной? 
	\michaelspeaks Суждение о˚том, что кандидатская диссертация должна перевернуть, изменить что\=/то в˚обществе неверное. При˚этом внедрение результатов исследований является наиболее сложным процессом. В~России с˚внедрением результатов исследований обстоит дело недостаточно хорошо. При˚допуске к˚защите требуется представление справки о˚внедрении, то˚есть что\=/то из˚полученных результатов исследований должно быть внедрено.
	\maxspeaks Аспиранты выбирают тему исходя из˚наличия у˚них в˚первую очередь материалов, необходимых для написания диссертации. С~государственной точки зрения это настолько нерациональный подход, что я˚только могу догадываться, почему у˚нас до˚сих пор не~привлекают администрацию вузов к˚ответственности за˚неэффективное расходование денежных средств. Возможно˚ли разработать стратегическую программу исследований, исключительно в˚рамках которой аспиранты могли бы˚выбирать свою тему диссертации? Или˚хотя бы˚разработать индикативный план, который бы в˚рекомендательной форме определял показатели, которым должна отвечать тема исследований?
	\michaelspeaks Тематика диссертационной работы (её˚выбор) зависит не˚только от˚наличия необходимого материала, но и от˚актуальности темы и˚состояния изученности поставленных вопросов объекта исследования и т.~д.
	\maxspeaks Современные исследования, как правило, требуют привлечения нескольких разнопрофильных специалистов и˚значительных материальных затрат. 
	Кто, по Вашему мнению, сейчас должен финансировать исследования аспирантов? Или˚оно вообще не~требуется и˚столкнувшиеся с˚необходимостью финансирования аспиранты просто зашли в˚своих исследованиях в˚тупик?
	\michaelspeaks Финансировать исследования аспирантов должно государство, если вуз государственный, так как оно заинтересовано в˚подготовке кадров, необходимых для конкретных предприятий, вузов.
	\maxspeaks Введение во˚многие предметы в˚ГУЗе начинается с˚рассказов о˚так называемой земельной реформе 1990\=/х гг. При˚этом студентам рассказывают о˚каких\=/то благих целях и˚задачах, которые должны были быть достигнуты в˚ходе её˚проведения. Однако,˚если посмотреть на˚её˚результаты и˚вспомнить под чью диктовку её˚осуществляли, то˚получится, что настоящей целью были грабёж и˚освобождение рынка сельскохозяйственной продукции России для иностранных компаний. Вот эти настоящие цели были блестяще достигнуты. 
Почему˚студентам не~раскрывают истинные причины и˚следствия так называемой земельной реформы? Я~понимаю, что может быть в˚учебниках правду написать сложно, но˚хотя бы на˚лекциях можно об˚этом говорить?
	\michaelspeaks Земельная реформа не~выполнила поставленных целей. Многое было проведено ошибочно, привело фактически к˚деградации сельского хозяйства. Непонятно, зачем было преобразовывать прибыльные колхозы и˚совхозы, создавать отдельные неэффективные крестьянские (фермерские) хозяйства?
	Земельная реформа привела к˚резкому снижению доходности и˚уровня жизни народа, потере жизненных ориентиров. Она привела к˚дестабилизирующим факторам экологическую обстановку. Примеров негативных последствий земельной реформы можно привести множество. Причём˚о них известно из˚имеющихся публикаций, поэтому на˚лекциях не~излагалось об˚этом.
	\maxspeaks Если˚бы Вас избрали ректором ГУЗа, то˚какие стратегические цели Вы бы˚наметили? Есть какие\=/то неотложные меры, которые Вы бы˚осуществили немедленно?
	\michaelspeaks Улучшение учебного процесса, улучшение условий проживания студентов и˚работы преподавателей вуза, борьба с˚негативными процессами, повышение эффективности научных исследований.
	\maxspeaks Заключительный вопрос по˚этой теме: если бы Вы родились в˚2001~году, Вы бы˚пошли учиться в˚ГУЗ?
	
	\pagebreak
	
	\michaelspeaks Пошёл бы˚учиться и˚работать в˚ГУЗ. Мне предлагали работать в˚Политехническом Музее (после окончания экономического факультета по˚линии Горкома КПСС), приглашали в˚Горком профсоюза, предлагали в˚два раза большую зарплату, когда был деканом заочного факультета. Но˚остался верен ГУЗу.

	\StageDir{О~землеустройстве}
	\maxspeaks Спасибо за˚ответы по˚этой непростой теме. Давайте переключимся на˚центральную тему Вашей преподавательской деятельности "--- землеустройство.
Если˚нашу беседу сейчас читает будущий аспирант, то˚ему наверняка будет интересно узнать какие, по˚Вашему мнению, наиболее актуальные научно\-/исследовательские задачи стоят перед учёными "--- землеустроителями.
	\michaelspeaks Наиболее актуальными направлениями исследования в˚землеустройстве являются те, которые будут способствовать развитию землеустроительной науки, увеличению производства и˚улучшению качества продукции, увеличению прибыльности сельскохозяйственных предприятий; защите земли от˚негативных последствий и˚увеличению её˚производительных свойств.
	\maxspeaks Сейчас многие ругают колхозы и˚совхозы за то, что они по˚своей природе не˚эффективны. Как˚Вы к˚ним относитесь? Что˚лучше коллективное хозяйство или единоличный владелец предприятия?
	\michaelspeaks Считал и˚считаю, что колхозы, совхозы имеют ряд преимуществ перед единоличными хозяйствами. Во\=/первых, в˚них имеется возможность более широко применить механизацию производственных процессов; во\=/вторых, внедрять передовые технологии; в\=/третьих, русский человек более склонен к˚коллективному.
	\maxspeaks Я~прочитал подписку журнала «Землеустройство, кадастр и˚мониторинг земель» за˚несколько лет. Среди˚статей, опубликованных в˚этих журналах, было только несколько, написанных интересно, и ни˚одной, в˚которой содержались бы˚хоть сколько\=/нибудь новые идеи или значимые исследования в˚землеустройстве. Возникает ощущение, что в˚журнал пишут исключительно  «для галочки». Он˚отражает реальное состояние землеустроительной науки?
	\michaelspeaks Думается, что у˚Вас несколько предвзятое отношение к˚журналу «Землеустройство, кадастр и˚мониторинг земель». На˚мой взгляд, статьи в˚журнале в˚основном являются актуальными, интересными по˚содержанию. Журнал признан даже ВАКом. 
	\maxspeaks Михаил Павлович, а˚как у˚нас в˚целом обстоят дела с˚землеустройством?
	\michaelspeaks Дела с˚землеустройством обстоят плохо. Во\=/первых, государство уничтожило (ликвидировало)все землеустроительные органы. При˚Советском Союзе землеустройство являлось государственным мероприятием и˚финансировалось за˚счёт государства. В~настоящее время такое финансирование отсутствует (лишь финансируются отдельные работы, причём в˚небольших размерах). Фермерское хозяйство не в˚состоянии его финансировать. Не˚в состоянии этого сделать в˚большинстве своём публичные акционерные общества и˚другие сельскохозяйственные предприятия.
	\maxspeaks Как˚Вы видите будущее внутрихозяйственного землеустройства?
	\michaelspeaks Считаю, что внутрихозяйственное землеустройство необходимо и˚оно будет развиваться. Примером его необходимости является восстановление деградированных земель в˚сельскохозяйственных предприятиях, крестьянских (фермерских)хозяйствах (процессы деградации земель наблюдаются во˚всех землепользованиях сельскохозяйственных организаций), введение и˚освоение научно\-/обоснованных севооборотов, без которых существовать хозяйство не~может, разработка мероприятий по˚повышению производительных свойств земли и˚т.~д. Внутрихозяйственное землеустройство необходимо проводить на˚основе разработки бизнес\-/планов для конкретных хозяйств.
	\maxspeaks Спасибо за˚Ваши ответы!
	\michaelspeaks Спасибо за˚поставленные вопросы, на˚которые я˚попытался кратко ответить. В~заключение пожелаю читателю успехов во˚всех делах, творческих дерзаний!

\end{drama}