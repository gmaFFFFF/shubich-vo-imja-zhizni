\chapter[Интервью с М.\,П.~Шубичем]{Интервью с Михаилом Павловичем Шубичем}

\begin{drama}
	\setlength{\speaksindent}{5em}
	\setlength{\Dlabelsep}{5em}
	\Character{Максим Гришкин}{max}
	\Character{Михаил Павлович Шубич}{michael}
	
	\maxspeaks: Михаил Павлович, Вы незаурядный человек, у Вас большой жизненный и профессиональный опыт, по принципиальным вопросам Вы имеете твёрдую позицию. Вашим читателям и мне в особенности будет интересно узнать Ваше мнение по наиболее острым проблемам образования, землеустройства, политики и экономики. Выражаю надежду, что у нас получится откровенный разговор, т.к. теперь Вы не обязаны, защищая “честь мундира”, говорить что чёрное - это белое. Поэтому разрешите задать Вам несколько порой провокационных вопросов.
	\michaelspeaks: Пожалуйста.
	
	\Character{МГ}{max}
	\Character{МПШ}{michael}
	\maxspeaks: Михаил Павлович, почему Вы написали данную книгу?
	\michaelspeaks: I am Jane Roe.  \direct{\michael{} Задумался}
	\maxspeaks: Ваше самое большое достижение и неудача. Почему?
	\maxspeaks: Что Вы думаете о религии? 
	\maxspeaks: Вы родились на территории современной Беларуси, служили в армии на Украине, работали в России. Что такое для Вас Родина?
	\maxspeaks: Ваша любимая книга? фильм? песня? Если возможно, то поясните чем они Вас “зацепили”.
	\maxspeaks: Ваши студенты знают, что Вы прекрасно умеете считать в уме. В чём секрет?
	\maxspeaks: Вы работали до 80 лет и могли бы ещё. В чём секрет Вашего трудового долголетия?
	\maxspeaks: Вы не жалеете, что стали землеустроителем? Кем бы Вы хотели стать ещё?
	
	\StageDir{О политике}
	\maxspeaks: Говорят, что в приличном обществе не принято обсуждать политику. Вообще популярность политических тем означает, что в обществе не всё благополучно. В нашем обществе к сожалению есть проблемы и противоречия, поэтому я вынужден затронуть некоторые политические темы.
	
	Есть ли у русского народа национальная идея? Какое место в ней занимает идея справедливости? 
	\maxspeaks: Как Вы думаете, если бы сейчас повторились события 1941 года, то могло бы современное поколение  выстоять? Или мы стали слишком расслабленными и больше не способны на такой подвиг?
	\maxspeaks: События Крымской весны разделили наших граждан совершенно разного социального положения и политических убеждений на два лагеря разделяющих противоположные убеждения: 1) Крым наш 2) Крым незаконно аннексирован у Украины. Чей Крым?
	\maxspeaks: Вы были свидетелем распада СССР. Почему это произошло по Вашему мнению? Что надо сделать России чтобы подобное не повторилось? Возможно ли восстановление СССР и в каких формах?
	\maxspeaks: Какую политическую партию Вы поддерживали на последних выборах. Вы изучали её политическую программу или Ваш выбор основан на телевизионных передачах, заметках в газете и уличной агитации? Почему именно она? 
	\maxspeaks: Есть мнение, что Государственная Дума практически не осуществляет реальную законотворческую деятельность, а “послушно штампует” законы, подготовленные Правительством России. Вас устраивает как трудятся 450 народных избранников в Государственной Думе? Как заставить их работать? Не слишком ли их много? Нужны ли законодательные органы власти вообще и субъектовые и муниципальные в частности?
	\maxspeaks: Как Вы относитесь к выборной избирательной системе? Вы верите, что на выборах можно выбрать достойных “слуг народа”. Знаете ли Вы более эффективные альтернативы?
	\maxspeaks: В соответствии с принципом верховенства права никто не может быть выше закона. Но законы пишут не боги, а несовершенные по своей природе люди. Значит они в соответствии с принципом информатики “мусор на входе — мусор на выходе”  ещё менее совершенны чем люди. Как Вы думаете, мы действительно должны принять за постулат необходимость слепого подчинения законам, т.к. это меньшее из зол или можно предложить альтернативу?  
	
	\StageDir{Об экономике}
	\maxspeaks: В.И. Ленин говорил, что политика есть концентрированное выражение экономики, поэтому давайте перейдём к этой близкой для Вас, как кандидата экономических наук, теме. Вы работали и при социалистической экономике и при капитализме. И та и другая системы в чистом виде, демонстрируют нам свою несостоятельность. Есть мнение, что будущее за неким их симбиозом, который, например, складывается в Китае. Можно ли скрестить ежа с ужом (капитализм и социализм)? Какие черты этих систем Вы считаете полезными, а какие вредными для России? 
	\maxspeaks: Следующий извечный вопрос очень сложный и ёмкий. Если Вы сможете ответить на него отсылкой к каким-нибудь близким для Вас по духу литературным источникам - это будет прекрасно. С 90-х гг. XX века Россия вернулась к капитализму, практикует неолиберализм, выполняет инструкции МВФ, интегрируется в какие-то мировые структуры и т.д. При этом очевидно экономика России стагнирует. Для сравнения за такой же по продолжительности период с 1922 по 1941 гг. практически с нуля была построена экономика способная победить силы объединённой фашистской Европы. Почему сейчас у нас ничего не получается с экономикой?  Кто виноват и что делать? 

	\StageDir{О работе}
	\maxspeaks: На работе проходит большая часть жизни человека. Хотелось чтобы Вы поделились с читателями своим опытом.

Кто Вы в большей степени: землеустроитель, педагог, администратор или учёный?
	\maxspeaks: Какой главный совет Вы могли бы дать начинающему [землеустроитель, педагог, администратор или учёный - выбрать из предыдущего ответа]?
	\maxspeaks: Длительное время Вы занимали высокую руководящую должность - декан факультета. Что нужно сделать чтобы построить успешную карьеру? Какие качества нужны хорошему руководителю?
	\maxspeaks: Студенты знают Вас как строгого и принципиального преподавателя. Если студент получил “у Шубича” оценку “Х”, то это значит, что он действительно знает предмет минимум на “Х”. Т.е. оценка, которую Вы поставили имеет “настоящую” хоть и не материальную ценность. Из преподавателей ГУЗа, у которых мне довелось учиться похожий стиль был только у доцента Кафедры геодезии и геоинформатики ГУЗ Елены Георгиевны Парамоновой. Почему именно такой стиль работы Вы выбрали? Вы считаете его оптимальным или есть более эффективные подходы к работе со студентами?

	\StageDir{О высшем образовании}
	\maxspeaks: Как Вы относитесь к ЕГЭ? Оценки абитуриентов стали более объективными?
	\maxspeaks: По различным данным в России примерно 30\% граждан трудоспособного возраста имеют высшее образование. Эта доля увеличивается от поколения к поколению (для сравнения по данным переписи населения 2010 года у поколения, родившегося до 1940 года доля людей с высшим образованием составляет 14\%, а у поколения 1980-1985 гг. - 37\%). При этом, во-первых, высококвалифицированных специалистов почему-то больше не становится, а, во-вторых, для многих видов работ, в квалификационных требованиях к которым заявляется высшее образование, узкоспециализированные знания, полученные в институте не нужны или их можно получить в процессе работы. 
Есть шутка, что когда человек пришёл в институт, то ему говорят: “Забудь всё то чему тебя учили в школе” - а когда он после института приходит на работу, то ему говорят: “Забудь всё то, чему тебя учили в институте”. В этой шутке действительно очень малая доля шутки. Например, про себя я могу сказать, что утратил 80\% знаний, полученных в институте и 70\% - в школе, а 60\% знаний, применяемых в трудовой деятельности, получены в процессе самообразования!
Таким образом, нанимая человека с высшим образованием работодатель, как правило, руководствуется тем, что раз человек имеет высшее образование значит он по сравнению с человеком без диплома более дисциплинирован, ответственен и способен к самоорганизации, а не тем, что он умеет делать какие-то сложные вещи или обладает уникальными знаниями.
Можно ли сказать, что “диплом” в России девальвировался и его получают не для знаний, а для бумажки, хотя прекрасно понимают, что “зря потратят время”? Не с этим ли явлением Вы сталкивались, когда рассказывали, что Вас удивляет почему современный студент не хочет учиться? 
	\maxspeaks: Нужно ли нам такое примитивное ранжирование на людей “с диплом” и “без диплома”, ведь за него не только государство платит деньги, финансируя высшую школу, но и люди растрачивают наиболее продуктивные годы на “бесполезное” занятие? Какие есть альтернативы?
	\maxspeaks: В СССР людей с дипломом было меньше чем в современной России, однако страна развивалась несравненно лучше. Чем больше людей имеет диплом, тем больше людей хочет его получить, даже если нет никакой тяги непосредственно к знаниям. Полагаю, что при прочих равных мы будем наблюдать экспоненциальный рост людей с дипломом, описываемый логистическим\footnote{Логистическое уравнение появилось при рассмотрении модели роста численности населения. Логистическое уравнение создано с учетом следующего допущения: скорость размножения популяции при прочих равных условиях пропорциональна её текущей численности и  количеству доступных ресурсов.
} уравнением (уравнение Ферхюльста). Спрос как известно рождает предложение и мы помимо миллионов людей с “бумажкой” получим тысячи “слабых” вузов. Может чтобы людей с дипломом можно было полноправно относить к интеллектуальной элите надо сократить их долю до 5-10\% от общего числа работоспособных, например, за счёт сложных выпускных экзаменов?
	\maxspeaks: Может быть система бакалавриата, которая в сравнении со специалитетом даёт намного меньше знаний, направлена на решение этой проблемы, что называется “малой кровью”: если людям нужна бумажка с названием “диплом”, то давайте дадим её максимально минимизировав затраты? 
	\maxspeaks: Как решить проблему с людьми у которых есть диплом, но нет знаний? Переаттестовать и лишить дипломов? 
\maxspeaks: Когда я поступал в институт, то не знал какой факультет необходимо выбрать. Мне посоветовали: “Иди на землеустройство” - я и пошёл. Даже на втором курсе я не понимал в чём принципиальная разница между землеустроителем, геодезистом и специалистом кадастра. 
	Из Ваших воспоминаний следует, что Вы поступили в ГУЗ по-большому счёту случайно, прочитав объявление в газете. Вы вполне могли поступить в БНТУ или в МИИГАиК, находящийся по соседству с ГУЗом. Почему система высшего образования построена так, что абитуриент должен принимать решение, определяющее всю его жизнь, не только в условиях отсутствия необходимой информации, но и даже не имея знаний для её корректной интерпретации, т.е. делать случайный выбор?

Почему абитуриент не может поступить хотя бы просто в университет, а не на факультет. И дальше выстроить свою образовательную траекторию в зависимости от своих способностей и призвания. Т.е. вуз мог бы определить по каждой специальности квалификационные требования, например, для землеустроителя требуется базовый уровень геодезии, а для геодезиста - продвинутый.  Тогда студент, получив базовый уровень знаний по геодезии, может выбирать по какой образовательной траектории ему двигаться: одному студенту понравилась геодезия и он пошёл дальше её изучать - для него открывается возможность стать геодезистом. А другому геодезия даётся с трудом и он движется в сферу кадастра или землеустройства. 
	\maxspeaks: Самое главное почему бы не оставить возможность студенту, который понял, что он “попал не туда” уйти не потеряв потраченное время? Например, набрав определённый набор компетенций в течение года студент должен иметь возможность, грубо говоря, уйти с дипломом “техника землеустроителя”, через два года - “помощника землеустроителя”, а через 5 лет уже  “инженера землеустроителя”. Это очень важно потому как свыше 50\% моих однокурсников уже на 2 курсе поняли, что они не хотят заниматься землеустройством. Для чего Российская Федерация ещё 3 года тратила на них деньги налогоплательщиков?
	\maxspeaks: У Михаила Жванецкого есть замечательное высказывание: “Когда знаешь как, умеешь, но уже не можешь сам - ты тренер. Когда знаешь как, не умеешь и не можешь сам - ты профессор. Когда не знаешь как, не умеешь и не можешь сам, но можешь наказать, если эти с...ки сделают не так - ты президент”. Вузовские преподаватели часто обладают только книжными знаниями о предмете или имеют минимальный практический опыт. Если по теоретическим дисциплинам (математика, философия, история и т.д.) это нормально, то по практическим дисциплинам (земпроектирование, геодезия, межевание) данная ситуация вызывает серьёзные опасения, не зря П.Л. Чебышев говорил “теория без практики мертва и бесплодна, практика без теории бесполезна и пагубна”. Вы видите в этом проблему и если да, то как её решать?

	\StageDir{О коррупции}
	\maxspeaks: Коррупция в образовании “избитая” тема. Не знаю как сейчас, но в начале 2000 гг. о каких-либо вымогательствах у подготовленных к экзамену студентов я ничего не слышал, но если у студента не было желания учиться, то он мог, ничем не рискуя, купить зачёт за 50\$, экзамен - за 100\$ (расчёты были в рублях) практически по любому предмету. Это явление было настолько обыденным, что о нём знали все студенты. Открыто обсуждалось кто и сколько “берёт”. Причем если преподаватель “не брал”, то его всегда можно было “обойти” через другого преподавателя или руководство кафедры. Т.е. взятки были полноправной частью системы образования: для успешного окончания института надо было или учиться или платить. Сложилась такая странная ситуация, что общество в целом не одобряло коррупцию, но каждый конкретный индивидуум, когда возникала такая необходимость, охотно шёл и давал деньги, а если у него их не брали, то он очень обижался. 
Я, например, столкнулся с тем, что недавно был вынужден для ребёнка получать базовые медицинские услуги в системе государственной медицины за плату. В причины вдаваться не буду, скажу просто, что это связано с коммерциализацией государственных медицинских услуг. Когда у меня появилась возможность заплатить врачу “налом” я с удовольствием и неоднократно ею воспользовался, т.к. посчитал это справедливым. Думаю, что руководство медицинского учреждения имеет возможность “прикрыть эту лавочку”, но последствием мог стать уход опытных специалистов (в этом медицинском учреждении работают действительно квалифицированные врачи).
	\maxspeaks: Всегда ли коррупция это плохо? Какое из зол меньше: преподаватель остался преподавать, но брал взятки, или пошёл в бизнес как посоветовал Председатель Правительства Д.А. Медведев в ходе общения с участниками форума «Территория смыслов» на Клязьме?
	\maxspeaks: Как Вы думаете какую долю от всех проблем нашего государства занимает коррупция? Какие проблемы более значимы?
	\maxspeaks: Очевидно, что борьба с коррупцией требует значительных материальных затрат. При этом простым повышением зарплат её не победить т.к. определенная категория людей будет брать деньги вне зависимости от зарплаты, меняться будет только размер взятки. Для подтверждения этого вспомните, например, бывшего министра экономического развития А.В. Улюкаева или “Мишу два процента”\footnote{Интервьюер имеет в виду кличку бывшего Председателя Правительства России М.М. Касьянова про которого говорили, что он якобы замешан в каких-то коррупционных вещах. При этом каких-либо доказательств этому не было представлено.}, доходы которых язык не повернётся назвать низкими. Наши чиновники уже много лет пытаются решить эту проблему с тем же успехом с которым пчёлы могут вести борьбу против мёда. Может быть Вы знаете как побороть коррупцию? 
	\maxspeaks: Мы понимаем, что в борьбе с коррупцией одержать окончательную победу невозможно. Какие измеримые показатели могут сказать нам, что мы победили и дальнейшая борьба не имеет смысла? 
	\maxspeaks: В не очень давние времена дети часто шли по стопам своих родителей: у пекаря сын становился пекарем, а у кузнеца – кузнецом. В советское время трудовые династии существовали не только у заводских работников - после смерти А. Н. Туполева руководить конструкторским бюро стал его сын А. А. Туполев. В современной России дети крупных государственных чиновников тоже «идут по стопам» своих родителей и тоже занимают высокие государственные должности. Как Вы относитесь к трудовым династиям в целом и к кумовству в частности?
	\maxspeaks: Как вы думаете следует ли разделять взяточничество и систему откатов\footnote{Откат - это неофициальное название части денежных средств, выделяемых на выполнение работ по государственному заказу, которую исполнитель заказа в “благодарность” за его получение возвращает чиновнику.}? Или это одно и то же явление с разными проявлениями? Как бороться с откатами?

	\StageDir{О Государственном университете по землеустройству (ГУЗ)}
	\maxspeaks: Следующая группа вопросов будет посвящена работе Вашей альма-матер. Среди них есть очень критические, но я прошу отнестись к ним с пониманием, потому что замалчивать реальные проблемы бесперспективно, т.к. рано или поздно они найдут своё решение. Правда в этом случае маловероятно, что оно окажется таким каким мы бы хотели его видеть, как, например, произошло с реформой Российской академии наук: вместо того чтобы академикам самим решить имевшие место проблемы за них это сделали чиновники. В результате по словам президента РАН В.Е. Фортова “...цели и методы реализации реформы оказались далеки от реальных нужд и потребностей и науки, и учёных...”

Образовательные программы ГУЗ и МИИГАиК во многом пересекаются: и там и там можно стать геодезистом, заниматься кадастром, развитием территории. Стоит ли воссоединить ГУЗ и МИИГАиК? Почему? А объединить их с РГАУ-МСХА имени К.А. Тимирязева?
	\maxspeaks: Вам часто встречаются горожане, которые после выпуска планируют связать свою жизнь с сельским хозяйством? Много ли москвичей переехало в глубинку “поднимать село”? Почему в условиях отмены обязательной системы распределения выпускников на факультет землеустройства принимают городских жителей и москвичей в особенности? Вообще уместно ли, что сельскохозяйственный вуз находится в городе, в котором не осуществляется сельскохозяйственное производство? Может ГУЗ хотя бы в Московскую область перевести?
	\maxspeaks: Мы уже достаточно долго живём в так называемую информационную эру, но выпускники ГУЗа не обладают компетенциями в сфере проектирования и создания компьютерных информационных систем? Это является проблемой?
	\maxspeaks: Лесоустройство по духу достаточно близко к землеустройству. Почему в ГУЗе нет предмета лесоведение, лесоводство или лесоустройство?
	\maxspeaks: У меня были преподаватели, которые не знали свой предмет на минимально необходимом уровне или те которые вообще пропускали занятия. Как такое вообще могло быть? Какая в ГУЗе введена система оценки качества преподавания? Требует ли она изменений?
	\maxspeaks: Кто по Вашему мнению самый “сильный” землеустроитель на кафедре землеустройства ГУЗа?
	\maxspeaks: Когда я учился в ГУЗе, то меня и однокурсников удивляло почему практически все книги по землеустройству проходят за авторством или под редакцией ректора ГУЗа С.Н. Волкова. Прошло 10 лет, но ничего не поменялось. В качестве примере можете на сайте ГУЗа зайти в раздел издательская деятельность (https://www.guz.ru/nauka/izdatelskaya-deyatelnost/knigi/) и сами убедитесь, что 19 из 21 книги сделаны при участии Сергея Николаевича. На кафедре землеустройства больше никто книги писать не умеет? 
	\maxspeaks: И ещё, что это за “фетиш” совмещение работы ректора и заведующего кафедрой? это традиция такая? или у заведующего кафедрой работа “не бей лежачего”? 
	\maxspeaks: В последнее десятилетие в связи с развитием Интернета широкое распространение получает так называемое открытое образование - система бесплатных онлайн-курсов по базовым дисциплинам, изучаемым в вузах. Т.е. любой человек может совершенно бесплатно получать знания, не выходя из дома. Конечно без дипломов и сертификатов. Имеет ли смысл ГУЗу развивать это направление?
	\maxspeaks: Уже более 10 лет как я окончил ГУЗ. За этот период меня ни разу не попросили заполнить анкету по тому какие дисциплины мне пригодились, какие нет, каких знаний не хватает и т.д. Возможно я просто не попал в статистическую выборку. ГУЗ получает обратную связь от выпускников с целью корректировки учебных планов или он варится “в собственном соку”?
	\maxspeaks: ГУЗ имеет статус университета. Университет это многопрофильное учебное заведение, являющееся ведущим научно-методическим центром в соответствующей области деятельности. 
Лично я как специалист в сфере землеустройства не вижу принципиальной разницы между следующими факультетами ГУЗа: землеустройство, кадастр недвижимости и с некоторой натяжкой городского кадастра. Если вывести за скобки дитя\footnote{В 90-е гг. практически в каждом техническом вузе появился юридический или экономический факультет с низким уровнем преподавания.} “лихих девяностых” - юридический факультет, то останется еще только один полноценный\footnote{С точки зрения интервьюера заочный факультет и факультет второго высшего образования не являются “полноценными” в том, смысле, что они готовят тех же специалистов, что и “полноценные” факультеты.} факультет - факультет архитектуры. Если в учебном заведении 2,5 факультета, то его вряд ли можно назвать многопрофильным. 

Многие преподаватели ГУЗа жалуются на текущую политику в сфере земельных отношений. В целом среди землеустроителей есть консенсус, что в стране что-то идёт не так. Но! Я не нашел ни на сайте ГУЗа, ни где бы то ни было ещё, ни одной системной программы развития землеустройства и земельных отношений за авторством профессорско-преподавательского состава ГУЗа. Ни одной! Где проекты правильных законов, методички, экономически обоснованные планы развития? Какой же ГУЗ - ведущий научно-методический центр?! Он никого никуда не ведёт, а только следует в фарватере бездарной земельной политики и при этом ноет как всё сейчас плохо и как хорошо было раньше. Для периферийного (второстепенного) института, эта позиция абсолютно понятная и в общем нормальная, но не для университета. 

Хуже того, по моим субъективным ощущениям как выпускника ГУЗа, я получил образование не в периферийном институте, а в хорошем таком, “среднем” ПТУ. Полный список “претензий” к ГУЗу я не буду озвучивать, ограничусь только ключевым фактором - квалификацией выпускников. Оценивая себя и экстраполируя данную оценку на своих сокурсников я могу твёрдо сказать, что выпускник землеустроительного факультета - это специалист “обо всём и ни о чём одновременно”: ему далеко до полноценного геодезиста, агронома, мелиоратора, почвоведа, юриста, экономиста, географа, картографа, оценщика и т.д. Даже примитивным кадастром он сейчас заниматься не может без получения отдельного аттестата. Причём его знания базируются на базе технологического уклада 70-80-х гг. XX века словно информационные технологии ещё не вошли в нашу жизнь.  Кроме того как-то так получилось, что я, практически троечник в школе, получил красный диплом в вузе. Мне порой даже кажется, что красные дипломы в ГУЗе раздавали практически всем, кроме совсем ленивых студентов. По моим субъективным ощущениям примерно 30\% выпускников получила красный диплом. У меня, как инсайдера, нет оснований считать, что большая доля выпускников с красным дипломом связана с изначально высокой мотивацией и подготовкой студентов, или какими-то выдающимися методиками преподавания в ГУЗе. Скорее это следствие небрежной оценки знаний студентов или откровенной безответственности преподавателей. В итоге получается, что изначально поверхностные и устаревшие знания передаются студентам ещё и некачественно! В моём понимании, в институте такого быть не должно. 

По Вашему мнению ГУЗ заслуживает статус университета? Какая реальная польза от этого? Есть ли в ГУЗе люди, которые способны выйти за пределы сложившейся системы и изменить её внутреннее наполнение так чтобы оно соответствовало университетской вывеске? Что в ГУЗе делают или должны делать для того чтобы оправдать данный статус? 
	\maxspeaks: Многие говорят, что проблемы в образовании существуют в основном из-за недофинансирования. За десять лет своей трудовой деятельности я сделал наблюдение, что увеличение заработной платы выше определенного уровня не оказывает заметного позитивного влияния на производительность труда. Наоборот люди начинают создавать видимость напряженной работы, у них появляется страх потерять её, пропадает желание рисковать, необоснованно растет самооценка. Если же зарплата существенно ниже какого-то оптимального уровня, то часть работников увольняется вне зависимости от наличия каких-либо нематериальных стимулов, а часть всё равно продолжает работать!  Вообще зачастую заработная плата бывает слабо связана со знаниями человека и его трудовой активностью, а определяется связями,  умением “продать себя подороже” или банальной удачей.
	\maxspeaks: Сталкивались ли Вы с подобными явлениями или это редкое стечение обстоятельств? Почему так происходит? 
	\maxspeaks: Обеспечивает ли государство достойную заработную плату преподавателям? Это касается всех вузов или только ГУЗа?
	\maxspeaks: Влияет ли на уровень преподавания и научной деятельности в ГУЗе зарплата? Если бы минимальная зарплата в ГУЗе стартовала от 5 средних зарплата в регионе, что бы это изменило?
	\maxspeaks: Какой Вы считаете оптимальной систему оплаты труда преподавателей? Какой минимальной допустимый уровень оплаты труда нужен в ГУЗе
	\maxspeaks: Разумно ли молодому выпускнику планировать карьеру в ГУЗе?
	\maxspeaks: Можно ли ожидать, что при текущем уровне зарплат и общей невостребованности землеустроителей, качество преподавания в ГУЗе не будет ухудшаться. Если нет, то что можно сделать в текущих условиях?
	\maxspeaks: Одним из показателей оценки эффективности деятельности вузов является количество публикаций в ISI Web of Science и Scopus. Поощряется публикация научных статей именно на английском языке. Я понимаю, что может быть для учёного это хорошо, его заметят и возьмут работать где-нибудь на Западе (конечно это не касается землеустроителей). Например, по-настоящему выдающийся учёный Г.Я. Перельман опубликовал статьи с методом доказательства гипотезы Пуанкаре на английском языке. Как мне простому инженеру получить доступ к последним достижениям российской науки если я, например, в школе и институте учил немецкий язык. Мне надо ещё один язык выучить, для того чтобы иностранцы особо не утруждались, перенимая российские достижения? Как Вы относитесь к такой интернационализации науки?
	\maxspeaks: В СССР степень кандидата наук позволяла зарабатывать её обладателю больше а, в некоторых организациях, например, академической среде, кандидатская степень являлась обязательным условием для продвижения по карьерной лестнице. Сейчас заработные платы кандидатов наук практически не отличаются от зарплат специалистов и даже бакалавров. При этом когда ты учишься в якобы бесплатной аспирантуре, ты теряешь деньги за счёт низких трудовых доходов аспиранта, а также несёшь прямые расходы на: публикацию статей в ВАКовских журналах особенно в каком-нибудь иностранном журнале, “благодарность в конверте” членам диссертационного совета за труды, “заносы конвертов” оппонентам и рецензентам, ещё нужно не забыть “проставиться” после защиты. По современным деньгам наверное 150-200 тыс. рублей получается? По какой причине, кроме желания “откосить от армии” или построить карьеру в вузе, следует идти в аспирантуру молодым специалистам?
	\maxspeaks: Кандидатская дисссертация, как я её понимаю, - это квалификационная работа, призванная показать квалификацию её автора. Конечно она решает какую-то задачу, т.е. даны начальные условия и аспирант, должен используя необходимый аппарат найти её решение, ну или если необходимый аппарат ещё не придуман, то разработать его. Но цель кандидатской диссертации именно показать квалификацию её автора, а сама задача является вопросом вторичным. Если посмотреть объективно, сколько кандидатских диссертаций в области землеустройства за последние 20 лет имели настоящую, а не бумажную практическую ценность, что называется “по гамбургскому счёту”? Т.е. чтобы до диссертации все решали эту задачу одним образом, а после массово стали делать по другому. Одна? две? или всё-таки ни одной? 
	\maxspeaks: У меня возникает резонный вопрос: если диссертация нужна только для того чтобы, удовлетворив амбиции автора, доказать наличие у него высокой квалификации, то почему за это платят налогоплательщики? Может быть необходимо отменить диссертации, т.е. если человек хочет заниматься наукой пусть он ей и дальше занимается, не тратя своё время и деньги на решение задач, не имеющих практической ценности, а его квалификацию оценивать какими-то иными менее затратными способами? Зачем карьеру преподавателя в вузе ставить в зависимость от наличия или отсутствия у него учёной степени? Разве учёных званий недостаточно? Пусть он лучше 3 года предельно внимательно к своим студентам относится, чем за счёт налогоплательщиков будет наполнять  мусорное ведро тривиальными сведениями, изложенными наукообразным языком и невостребованными в народном хозяйстве?
	\maxspeaks: Аспиранты выбирают тему исходя из наличия у них в первую очередь материалов, необходимых для написания диссертации. С государственной точки зрения это настолько нерациональный подход, что я только могу догадываться почему у нас до сих пор не привлекают администрацию вузов к ответственности за неэффективное расходование денежных средств. Возможно ли разработать стратегическую программу исследований, исключительно в рамках которой аспиранты могли бы выбирать свою тему диссертации? Или хотя бы разработать индикативный план, который бы в рекомендательной форме определял показатели, которым должна отвечать тема исследований?
	\maxspeaks: Современные исследования как правило требуют привлечения нескольких разнопрофильных специалистов и значительных материальных затрат. Что общественно полезное может “родить” одиночка аспирант? Мне кажется ничего или он должен быть гением. Частота потенциальных гениев, развившихся настолько, чтобы так или иначе обратить на себя внимание в качестве потенциальных талантов согласно В.П. Эфроимсону, вероятно, исчисляется цифрами порядка 1:100 000. Частота же гениев, реализовавшихся до уровня признания их творений и деяний гениальными, вероятно, даже в век почти поголовного среднего и очень часто высшего образования, исчисляется цифрой 1:10 000 000\footnote{Эфроимсон В.П., Гениальность и генетика, М., «Русский мир», 1998 г., с. 20-21.}. Таким образом гениев в землеустройстве ждать не следует. 
	\maxspeaks: В СССР существовала сеть ГИПРОЗЕМов, которые могли финансировать и обеспечивать необходимыми ресурсами исследования. Сейчас их нет. Например, когда я занимался своей диссертацией у меня возник ряд проблем, решение которых требовало от меня привлечения математиков и программистов. На тот момент по моим оценкам требовалось финансирование в размере порядка 100-150\footnote{На момент интервью эта сумма с учётом инфляции составила бы порядка 250-400 тыс. руб.} тыс. руб. Сначала я решил, что заработаю эти деньги и потрачу их на своё исследование, но по мере того как я всё больше и больше работал, желание тратить “свои кровные” на диссертацию становилось всё меньше и меньше пока не исчезло совсем. Мне довелось участвовать и в противоположной ситуации, когда я на возмездной основе для одного аспиранта выполнял отдельный раздел диссертационной работы. По понятным причинам я не буду вдаваться в подробности, но факт заключается в том, что мою работу он оплачивал за свой собственный счёт. Мои услуги стоили ему порядка 20 тыс. руб. 
	\maxspeaks: Кто сейчас должен финансировать исследования аспирантов? Или оно вообще не требуется и столкнувшиеся с необходимостью финансирования аспиранты просто зашли в своих исследованиях в тупик?
	\maxspeaks: Введение во многие предметы в ГУЗе начинается с рассказов о так называемой земельной реформе 1990-х гг. При этом студентам рассказывают о каких-то благих целях и задачах, которые должны были быть достигнуты в ходе её проведения. Однако если посмотреть на её результаты и вспомнить под чью диктовку её осуществляли, то получится, что настоящей целью были грабёж и освобождение рынка сельскохозяйственной продукции России для иностранных компаний. Вот эти настоящие цели были блестяще достигнуты. 
	\maxspeaks: Действительно если есть трактор и он время от времени барахлит, то вряд ли Вы будете разбирать его на части и передавать двигатель трактористу, кабину - механику, плуг - бухгалтеру, а колеса - директору в надежде, что это улучшит его работу. А почему-то с колхозами при поддержке землеустроительного сообщества поступили именно так. Причём некоторые из тех, кто проводил так называемую земельную реформу, сейчас публикуют слёзные статьи о том, что мол были отдельные недоработки, которые и привели к текущей плачевной ситуации. Я думаю, что господам бывшим работникам Госкомзема, соучастникам этой диверсии, наверное следует помнить пословицу: “снявши голову, по волосам не плачут”.
	\maxspeaks: Почему студентам не раскрывают истинные причины и следствия так называемой земельной реформы? Я понимаю, что может быть в учебниках правду написать сложно, но хотя бы на лекциях можно об это говорить?
	\maxspeaks: Если бы Вас избрали ректором ГУЗа, то какие стратегические цели Вы бы наметили? Есть какие-то неотложные меры, которые Вы бы осуществили немедленно?
	\maxspeaks: Заключительный вопрос по этой теме: если бы Вы родились в 2001 году Вы бы пошли учиться в ГУЗ?

	\StageDir{О землеустройстве}
	\maxspeaks: Спасибо за ответы по этой непростой теме. Давайте переключимся на центральную тему Вашей преподавательской деятельности - землеустройство.
Самый первый мой вопрос будет “шкурным”. Я писал диссертацию на тему “…динамическое внутрихозяйственное землеустройство…”. Если рассказать кратко и упрощённо, то суть моего исследования состояла в том, чтобы при внутрихозяйственном землеустройстве выделить редко изменяемые (статические) элементы: дороги, постройки, рабочие участки и т.д., а также подверженные изменению (динамические) элементы:  законтрактованные объёмы производства, ожидаемые рыночные цены и т.д. Затем с помощью оптимизационной модели на основе севооборотов во времени, о которых Вы мне когда-то рассказали, реализовать компьютерный алгоритм способный за мгновение пересчитывать оптимальное проектное решение при изменении динамических элементов. 
Как Вы думаете, если бы мне удалось завершить данную работу, она имела бы настоящую практическую ценность? Т.е. набралось ли хотя бы несколько тысяч организаций, которые готовы были бы платить за информационный продукт на её основе? Или она служила бы прекрасным местозаполнителем на архивных полках и тешила мою самолюбие? Надеюсь на объективный ответ.
Если нашу беседу сейчас читает будущий аспирант, то ему наверняка будет интересно узнать какие по Вашему мнению наиболее актуальные научно-исследовательские задачи стоят перед учёными- землеустроителями.
Сейчас многие ругают колхозы и совхозы за то, что они по своей природе не эффективны. Как Вы к ним относитесь? Что лучше коллективное хозяйство или единоличный владелец предприятия?
Я прочитал подписку журнала Землеустройство за несколько лет. Среди статей, опубликованных в этих журналах, было только несколько, написанных интересно, и ни одной, в которой содержались бы хоть сколько-нибудь новые идеи или значимые исследования в землеустройстве. Возникает стойкое ощущение, что в журнал пишут исключительно “для галочки”. Он отражает реальное состояние землеустроительной науки? Или есть другие журналы в которых есть что-то новое и интересное?
Для того чтобы задать направление нашей заочной беседы по следующему очень сложному вопросу, я сначала его задам, потом попробую ответить на него сам, а затем с Вашей помощью попробуем разобраться где я не прав. 
Если спросить выпускников землеустроительного факультета ГУЗа: “Как у нас обстоят дела с землеустройством?”, то наверно 8 из 10 скажут что-то типа: “Землеустройство умерло осталось одно межевание”. С такой позицией трудно спорить. Действительно легко понять объективную востребованность землеустройства в советский период как обоснованной системы разрешения противоречий между: с одной стороны централизованной системой государственного планирования и государственной собственностью на землю, а с другой стороны интересами формально независимых колхозов. Основываясь на Ваших воспоминаниях о том, как Вы добивались снижения доли зерновых в севооборотах легко увидеть, что сельскохозяйственные предприятия в отношениях с госпланом были слабой стороной и действительно были заинтересованы в землеустройстве. 
Что же изменилось в 1990-е гг.? Прежде всего была прекращена монополия государственной собственности на землю, что сразу “поставило крест” на межхозяйственном землеустройстве\footnote{Межхозяйственное землеустройство - научно-обоснованная система распределения и перераспределения земель, с целью обеспечения их эффективного, рационального использования и охраны прежде всего как основного средства производства в сельском и лесном хозяйстве. Обеспечивает формирование оптимального землепользования сельскохозяйственного предприятия или его упорядочивание (устранение недостатков).}, т.к. у каждого собственника земли свои интересы, зачастую противоречащие интересам других собственников.
Во-вторых, сельское хозяйство фактически довели до банкротства, ему не дают дешёвые кредиты, порой у сельскохозяйственного производителя нет денег на солярку для проведения посевной и уборки урожая. Кто в таких условиях будет инвестировать в развитие в целом и в землеустройство в частности?
В-третьих, государство в 1990-е гг. прекратило плановую закупку продукции сельскохозяйственного производства, отказалось от регулирования рынков сбыта и инвестиций, раздробило сельскохозяйственные землепользования на множество долей, принадлежащих различным людям. Как в таких условиях осуществлять долгосрочное планирование (на 7-10 лет) в сельском хозяйстве, если даже в промышленности перестали строить долгосрочные планы? Проект внутрихозяйственного землеустройства\footnote{Внутрихозяйственное землеустройство - мероприятия по организации рационального использования земельных участков для осуществления сельскохозяйственного производства.} разрабатывается на средне- и долгосрочную перспективу. Поэтому в условиях когда “нам бы только ночь простоять да день продержаться” нет смысла тратить на него деньги.
В-четвертых, если в советское время проект землеустройства был директивным документом, который надо было исполнять, не заботясь о внешнем окружении предприятия, то в условиях капитализма он стал всего лишь одним из разделов комплексного плана развития предприятия (бизнес-плана) - производственным планом. При чём он подчинён рыночной конъюнктуре и поэтому постоянно требует внесения каких-то изменений: заключил в апреле выгодный договор на поставку картофеля - значит в мае ты его уже сажаешь, а не ждёшь пока к тебе приедет землеустроитель и в порядке авторского надзора расскажет о том как надо вести бизнес и почему нельзя нарушать севооборот. 
В-пятых, часто при оценке состояния землеустроенности сельскохозяйственного предприятия землеустроители говорили о том, что севообороты нарушены. При этом естественно подразумевалось, что предприятие плохо соблюдало предыдущий проект землеустройства и оно в этом само виновато. Если бы это были единичные случаи, то наверное с этим можно было бы согласиться. Но на самом деле фактический отказ от мероприятий, заложенных в проекте землеустройства встречался повсеместно и связан он с тем, что сохранившаяся с советских времён методология проведения внутрихозяйственного землеустройства не позволяет вносить оперативные изменения в производственный план: слишком медленно, долго и дорого получается.  В то же время, по-моему мнению, с 1980-х гг. землеустроительная наука фактически не эволюционировала: не осуществлялась системная цифровизация, математическая и программная алгоритмизация и соответственно автоматизация землеустройства. Т.е. современное землеустройство, несмотря на повсеместную компьютеризацию осталось таким же трудоёмким процессом с преобладанием ручного труда как и 30 лет назад. 
В-шестых, схемы землеустройства были вытеснены документами территориального планирования. Участковым землеустройством при отсутствии проекта внутрихозяйственного землеустройства будут заниматься непосредственные исполнители соответствующих работ: мелиораторы, лесомелиораторы, гидромелиораторы, строители и др. Природно-сельскохозяйственное районирование вообще является разовым мероприятием и может выполняться широким кругом специалистов.  Таким образом, землеустройство действительно во многом утратило своё содержание.
Михаил Павлович, имея на руках такую печальную оценку как бы Вы ответили на вопрос: “Как у нас обстоят дела с землеустройством?”
	\maxspeaks: На следующий вопрос, закономерно вытекающий из предыдущего, я также предложу свою версию ответа. Итак, что будет с землеустройством?
На мой взгляд то содержание, которое землеустройство уже утратило ему никак не вернуть, а кадастр так и останется примитивным учётным мероприятием каким мы его знаем сегодня. 
С внутрихозяйственным землеустройством чуть-чуть поинтереснее. Информационные технологии кардинально изменили геодезию и картографию. Теперь любой человек без специального образования может на своём компьютере создавать карты, проводить географический анализ, пользоваться данными дистанционного зондирования земли, с помощью мобильного телефона с приемлемой точностью определять своё местоположение, а с помощью дешёвых беспилотников проводить аэрофотосъёмку. Внутрихозяйственное землеустройство этот путь ещё не прошло, но он неизбежен: рано или поздно вся методика внутрихозяйственного землеустройства будет описана с помощью математических или имитационных моделей и будет запрограммирована в одном вспомогательном и очень маленьком модуле географической информационной системы (ГИС) для сельхозпредприятия. Учитывая то, что “ведущий университет в области землеустройства” - ГУЗ утратил стратегическую инициативу (если он вообще её когда-то имел), этот программный модуль скорее всего придёт к нам “из-за бугра”, точно так же как и основные ГИС. 
ГИС сельхозпредприятия будет скорее всего “облачным сервисом” (основная функциональность будет предоставляться через Интернет) который на начальных этапах будет заниматься в основном вопросами мониторинга и учёта, затем получит функции планирования и оптимизации, потом научится управлять сельскохозяйственными машинами под контролем оператора, а в долгосрочной перспективе станет абсолютно автономной системой, в которой человек будет нужен для  выполнения только отдельных операций, осуществляемых под контролем машины. В этом светлом будущем у землеустроителя останется две роли: 
первая - это “доводка” моделей, заложенных в основу ГИС сельхозпредприятия. Таких высококлассных специалистов готовить в ГУЗе не будут и их нужно очень мало - 10-20 человек; 
вторая - это по-большому счёту оператор ЭВМ, который будет проводить первичную обработку исходных данных для внедрения в ГИС сельхозпредприятия и консультировать работников хозяйства о том как эту ГИС оптимально использовать. Таким специалистам достаточно будет полугодовых курсов, которые в том числе могут проводиться в ГУЗе. Число этих операторов также будет сравнительно небольшим, не думаю, что сильно больше 500-1000 человек.
Как Вы видите будущее внутрихозяйственного землеустройства?

	\StageDir{О планах на будущее}
	\maxspeaks: Вы планируете заниматься репетиторством с отстающими студентами?
	\maxspeaks: Каким проектом Вы планируете заняться после публикации данной книги?
\end{drama}