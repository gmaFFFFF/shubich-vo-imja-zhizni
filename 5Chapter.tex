\chapter{Жизненные испытания}

Приехал в г.~Подольск. Зашёл к˚дальнему родственнику. Дверь открыла высокая женщина, с˚заострённым лицом, большим тонким носом. Лицо отражало какую\=/то злобу. Пришлось объясниться, кто я и˚сразу предупредил, что жить у˚них не~собираюсь. Вечером с˚работы возвратился родственник. Обрадовался моему приезду. Но˚сразу почувствовалось, что в˚квартире, даже им˚управляет жена. Затем зашла сестра хозяйки, женщина средних лет, даже сохранившая девичью красоту. Они возвратились недавно из˚Шпицбергена\footnote{Шпицб\'{е}рген (нем. Spitzbergen) \--- обширный полярный архипелаг, расположенный в˚Северном Ледовитом океане. Самая северная часть Норвегии. 

Значительную, по˚арктическим меркам, хозяйственную деятельность на˚архипелаге помимо Норвегии, согласно особому статусу архипелага, осуществляет только Россия, имеющая на˚острове Западный Шпицберген российский населённый пункт — посёлок Баренцбург, а˚также законсервированные посёлки Пирамида и˚Грумант.}. 
Внешне представилось, что проживают в˚достатке. Начали играть в˚лото на˚деньги. Объяснили смысл игры и˚пригласили меня поиграть. Я~заявил, что нет денег, они мне дали в˚долг. В~этот вечер счастье сопутствовало мне, и я˚выиграл 7~рублей. Лица их˚выражали какое\=/то недовольство. Но˚я подумал, что фортуна улыбается и˚помогает обездоленным. Хозяйка квартиры уложила меня спать у˚порога на˚коврике. Поэтому на˚рассвете сбежал из˚данной квартиры, ни с˚кем не~попрощался. Вскоре Иван развёлся с˚женой. Внутренне я˚приветствовал его поступок.

Из-за отсутствия денег, места проживания решил поработать. Из˚газеты узнал, что Подольстрой набирает рабочих. Прибыл по˚указанному адресу. Зачислили рабочим бригадиром строительно\-/монтажного управления №~14 Мособлспецстрой №~1. Предоставили место проживания в˚бараке. Смущало и˚волновало меня, что в˚паспорте поставили штамп <<зачислен по˚оргнабору>>. 

Барак находился на˚Южном, вблизи магазина, в˚котором продавали спиртное. В~бараке проживали 20~ребят, в˚основном детдомовцы и˚5~семейных (ширм\Todo[Ред.]{Не понятно}). При˚появлении в˚бараке представилась ужасная картина. Ребята лежали в˚кроватях на˚простынях в˚грязных сапогах (в~глине). Оказывается, после получки они пропивали все деньги, не~ходили на˚работу. Даже˚использовался термин <<варим колун\footnote{Кол\'{у}н \--- разновидность топора, предназначенного для колки дров.}>>, то˚есть деньги пропиты, есть нечего (кое-что сшибали\footnote{Сшибать \--- выпрашивать.} у˚семейных). Чем˚вызывали недовольство. При˚выяснении оказалось, что барак являлся местом пьянства городских пьяниц. Ведь˚магазин находился рядом. 

Стояла перед Мишей проблема, как исправить положение дел. Решил основательно познакомиться с˚каждым. Младших по˚возрасту как-то заставить, а˚старших \--- убеждением. На˚моей стороне оказались и˚семейные. Решил ещё действовать через Подольстрой, приобрести что\=/либо для барака. Приобрёл шашки, старый телевизор, какой\=/то стол. Отдельные ребята вижу поверили, надеясь, что он˚может что-то улучшить. Договорились с˚ребятами, что пьянки посторонних лиц в˚бараке прекратить, просто выбрасывать их из˚барака. Ребята поддержали данное предложение. 

Следующего дня Миша отправился принимать бригаду. Девушки работали у˚городской больницы. В~состав бригады входили 19~девушек и˚один мужчина \--- сантехник. Среди˚девушек услышал сплошной мат. Для˚Миши это было неожиданным, впервые. Поэтому для него это было дико, и он˚возвратился обратно в˚барак. По˚дороге сам себе сказал: <<Слюнтяй, возьми себя в˚руки>>. Назавтра опять явился и˚представился. 

Поручено было проложить водопровод от˚артскважины вдоль забора военной части. Представление о˚предстоящей работе не~имелось, поэтому пришлось ночь изучать, как прокладывать трубы.

Работа у˚девушек была тяжёлая. Они верёвками должны перетаскивать трубы диаметром 350\,мм и˚весом 1\,200\,кг˚каждая, укладывать в˚траншею, копать, а˚иногда долбить мёрзлую землю. Трубы укладывали на˚глубину 2\,м, сантехник должен был чеканить, заделывать стыки труб и˚устанавливать смотровые колодцы.

Платили девушкам мало. Поэтому стояла задача создать дружный коллектив, как-то облегчить их˚труд и˚повысить зарплату, их˚заинтересованность. Первоначально решил откровенно побеседовать с˚каждой девушкой персонально. Индивидуальная беседа оказалась откровенной. Каждая из˚них излила своё наболевшее, считая себя неудачницами в˚этой жизни. У~многих была неудачная любовь, даже признались, что пришлось делать аборты.

У~самого денег не~было. Ходил в˚военной форме, сверху на˚работе надевал комбинезон. Поэтому целую неделю вёл голодное существование. Единственным пропитанием была вода. Во˚время обеда девушки расходились по˚столовым. Миша снимал свой комбинезон и˚делал вид, что ушёл на˚обед. Несмотря˚на тяжёлые условия труда, жизни, девушки сохранили человеческие качества, доброту, внимательность, взаимопонимание и˚взаимопомощь. По˚возвращении с <<обеда>> однажды в˚кармане комбинезона оказался кусок хлеба и˚сала. Девушки проследили, что не~обедаю, подложили. Стал спрашивать, кто это сделал. Девушки не~признались. Естественно, этот хлеб и˚сало я˚тут же˚втихаря съел. Так˚они продолжали подкладывать до˚первой моей получки. Сколько˚не~спрашивал, так не~узнал своего спасителя.

После˚получки купил килограмм конфет. Разбил бригаду на˚две части, поручил одинаковые задания. Тем, которые выполняли первыми и˚качественнее, вручил конфеты. Так˚продолжалось некоторое время. Это стимулировало и˚повысило производительность труда.

Однажды возвратился вечером с˚работы в˚барак. Пять человек посторонних распивали в˚бараке водку. Ребята сидели на˚своих кроватях притаившись. Спрашиваю, почему они здесь, ведь мы˚договорились. Начали вместе с˚ребятами выталкивать их из˚барака. Последний оказался сильным. Поэтому пришлось уложить его на˚кровать, а˚второй приложить. При˚занятиях с˚последним один из˚первых где-то схватил топор и˚сзади замахнулся чтобы разбить Мише голову. Благодарен Володе Cтрукову за˚спасение. Он обхватил сзади его руки и˚предотвратил опасность. Пообещал меня встретить. В~г.~Подольске в то˚время процветал бандитизм. Поэтому было принято мной решение действовать на˚опережение. Узнал, где он˚работает и˚когда заканчивает, его путь домой. Встретил его и˚сказал: <<Я~явился, будем выяснять отношения?>> он˚явно не~ожидал подобного и˚ответил: <<Не˚будем>>. Так˚мы разошлись.

Чтобы˚ребята не~пропивали зарплату, пришлось деньги отдельных держать у˚себя. Выдавал им˚лишь на˚одну бутылку. Конечно, это стоило много неприятностей, даже угроз. Приходилось ходить вместе и˚покупать им˚брюки, ботинки. Вспоминается такой случай. Сергей ходил весь оборванный. Не˚хотел идти со˚мной в˚магазин, чтобы приобрести что-то из˚одежды. Тогда я˚предложил ему вечером сходить в˚соседний барак. Там проживали девушки. Никого из˚них я не~знал. Надеялся увидеть там городских, хорошо одетых ребят и˚как-то психологически подействовать на˚Сергея. Надежды мои оправдались. Девушки приняли наш приход приветливо. Действительно там находились городские ребята. Среди˚бесед Сергей мне шепчет: <<Миша, уйдём, мне стыдно>>. В~это время внутренне я˚торжествовал. Это была победа.

По˚положению, я не~имел права рыть траншею, пока не~будет акт от˚представителя Мосэнерго. Был воскресный день. Вызвал представителя. Он сказал, что вдоль забора военной части можете рыть свободно. Подземные провода отсутствуют. На˚предложение составить акт, он˚ответил, что не~следует тратить время в˚воскресный день. Завтра сделаем. На˚своей машине отвёз меня в˚барак. Я~принялся варить картофельный суп на˚воде. Вдруг прибегает второй экскаваторщик и˚кричит: <<Нет экскаватора и˚экскаваторщика, всё сгорело!>>. Побежали через кладбище к˚месту работы. Действительно оказался большой бугор взорванной земли, часть ковша сгорела и в˚кабине сидит экскаваторщик без всяких признаков жизни. Экскаваторщика сбросили на˚землю. Оказывается, наскочили на˚провод очень высокого напряжения, который питал часть заводов г.~Подольска. Вскоре понаехало много черных волг. 

Выяснили, кто ведёт работы и˚сразу составили акт на˚400~тыс. руб.\footnote{Для˚сравнения: средний размер заработной платы в˚СССР в˚1960~году составлял 73,1~руб.} и˚предложили в˚течении получаса устранить аварию. Опасность заключалась в˚том, что мог проходить ток и˚при вскрытии (откапывании) провода могли быть жертвы. Поэтому Миша со˚слезами на˚глазах обратился к˚девушкам, чтобы они одели по˚две пары брезентовых рукавиц и˚поработали так, как в˚жизни ещё не~работали. Девушки полностью осознали создавшуюся ситуацию. К~счастью, при порыве натянулась на˚кабеле обмотка и˚ток не~шёл. Порыв удалось ликвидировать в˚сроки. Тут же˚подъехал чёрный воронок\footnote{Чёрный воронок (фразеологическое сочетание) \--- автомобиль для перевозки арестантов.} и˚Мишу отвезли в˚прокуратуру г.~Подольска. Миша изложил представителю прокуратуры подробно, как обстояло дело. Через˚некоторое время привезли представителя Мосэнерго. На˚вопрос видел ли он˚данного товарища (Мишу), последовал ответ: <<Не˚видел>>. Тогда прокурор спросил Мишу: <<Чем˚он может доказать, что вызывал представителя и˚тот отложил составление акта назавтра?>>

\begin{wrapfigure}{O}{.4\textwidth}
\centering
\includegraphics[width=.35\textwidth]{gazM1}
\caption[Чёрный воронок. Автомобиль ГАЗ-М\=/1 в˚Музее отечественной военной истории.]{Чёрный воронок. Автомобиль ГАЗ-М\=/1 в˚Музее отечественной военной истории\footnotemark.}
\label{fig:gazM1}
\end{wrapfigure}
\footnotetext{Автор: Музей отечественной военной истории. URL: \url{http://www.kskdivniy.ru/}.}

Перед˚глазами Миши, как на˚экране, появилась артскважина, стоящие рядом сторож и˚представитель Мосэнерго. Ответ последовал, что возможно слышал наш разговор стоящий рядом сторож. Поехали за˚сторожем. Миша в˚это время начал седеть, то˚есть на˚голове появилась седина. По˚приезду сторож ответил, что видел этих людей. Слово в˚слово изложил, о˚чём мы˚говорили. Миша при прокуроре вскочил, расцеловал сторожа и˚сунул ему в˚карман деньги, которые у˚него были.

Прокурор заулыбался и˚проронил: <<Ты ещё даёшь взятку>>. Он прекрасно понял то˚напряжение, которое овладело Мишей. Впоследствии, акт на˚400~тыс. руб. был ликвидирован.

\begin{wrapfigure}{O}{.4\textwidth}
\centering
\includegraphics[width=.35\textwidth]{elctrDuga}
\caption{Вольтова (электрическая) дуга.}
\label{fig:elctrDuga}
\end{wrapfigure}

Экскаваторщик ожил. Он оказался в˚вольтовой дуге, в˚шоковом состоянии. В~суматохе его даже забыли. Видимо, лежание в˚сырой земле оказало на˚него положительное влияние. 

Бригада начала перевыполнять план на˚20\==30~\%. Для˚облегчения труда девушек, старались больше использовать бульдозер. Зарплата у˚девушек повысилась. Но˚кое\=/кого это беспокоило. Было представлено начальнику управления анонимное заявление, что Миша приписывает объёмы земли, которые перебросали лопатами девушки. Создали комиссию, которая не~смогла этого доказать.

Однажды два дня подряд не~явился на˚работу сантехник. Пришлось чеканить трубы (заделывать стыки водопровода) и, чтобы привезённый раствор не~пропал, устанавливать смотровые 2~колодца самому, под руководством опытной девушки Любы. Подобной работой ведь раньше не~занимался.

На˚третий день поехал к˚начальнику участка выяснять, почему не~является сантехник. Неожиданно представилась картина, сидят и˚выпивают сантехник и˚начальник участка. Нервы в˚это время были на˚пределе, и˚Миша схватил за˚грудь начальника участка, прижал к˚стенке. Затем позвонил начальнику управления и˚попросил выехать и˚разобраться. При˚этом заявил, что избил начальника участка. К~концу дня явился начальник управления, устроили собрание. На˚собрании выяснилось, что много нарушений со˚стороны начальника участка. Даже˚часть зарплаты отнимал у˚девушек. Были и˚другие крупные нарушения.

Начальника участка снимают с˚должности. Очень взволновал Мишу момент. Стоит рядом с˚начальником отдела кадров Рыконовой бывший начальник участка (они были в˚близких отношениях). Подзывает Мишу и˚говорит Рыконовой: <<С~него будет толк, ты˚побереги его>>. Для˚Миши было это неожиданно, ведь получилось, что сняли его по˚инициативе Миши. 

Вечером к˚бараку подъехала машина. Шофёр сказал, чтобы Миша поехал с˚ним, вызывает Рыконова. Привёз Мишу в˚квартиру Рыконовой. Столик был накрыт. Рыконова предложила Мише поступать в˚строительный институт. Она окажет содействие. Такое предложение почему\=/то разозлило Мишу. Последовал ответ, что этому не~бывать. Это убедило Мишу, что необходимо поступать уже в˚этом, 1959~году.\todo[Текст]{Не совсем ясна мотивация}

Из˚объявления газеты <<Комсомольская правда>> узнал, что принимают на˚первый курс Московского института инженеров землеустройства (МИИЗ).

\begin{wrapfigure}{O}{.4\textwidth}
\centering
\includegraphics[width=.35\textwidth]{guz}
\caption[{Современный вид здания Государственного университета по˚землеустройству (Ранее Московский институт инженеров землеустройства \--- МИИЗ).}]{Современный вид здания Государственного университета по˚землеустройству (Ранее Московский институт инженеров землеустройства \--- МИИЗ)\footnotemark.}
\label{fig:guz}
\end{wrapfigure}
\footnotetext{Автор: А.~Миронова 21.07.2013.}

Жители барака погружались в˚сон, а˚Миша сидел в˚коридорчике и˚готовился к˚экзамену. Необходимые книги предоставили девушки из˚бригады.

Однажды на˚день оставил бригаду и˚поехал в˚Москву узнать все обстоятельства. Зашёл к˚ответственному секретарю в˚приёмной комиссии (к~сожалению фамилию его позабыл). Он оказался сочувствующим человеком. Выдал Мише блокнот, карандаш и˚сразу повёл на˚занятие. Шли последние занятия по˚математике на˚подготовительных курсах. Из˚объяснений преподавателя что-то понимал, что-то не~очень (сказался разрыв между учёбой). Обратился к˚рядом сидящему абитуриенту: <<Ты всё понимаешь?>> Последовал ответ, что не~всё. Этот ответ Мишу <<окрылил>>. Подумалось, что он˚посещал все подготовительные курсы и не~всё понимает, а я˚кое-что понимаю. Следовательно, можно наверстать. Начал активно готовиться к˚экзаменам по˚ночам.

Вступительные экзамены сдал успешно и˚был зачислен в˚число студентов первого курса факультета землеустройства Московского института инженеров землеустройства. Ещё в˚течение месяца не~хотели увольнять с˚работы. Даже˚получил зарплату. Девушки работали. При˚посещении их на˚работе, не~хотели отпускать. Просили их не~оставлять!