\chapter{Жизненные испытания}

Поскольку в˚г.~Москве не~было знакомых, а˚в˚г.~Подольске проживал дальний родственник Иван, то˚Миша решил сначала поехать к˚нему в˚г.~Подольск. Дверь открыла высокая женщина, с˚заострённым лицом, большим тонким носом. Лицо отражало какую\=/то злобу. Пришлось Мише объясниться, кто он и˚сразу предупредить, что жить у˚них не~собирается. Вечером с˚работы возвратился родственник. Обрадовался приезду Миши. Но˚сразу почувствовалось, что в˚квартире даже им˚управляет жена. Затем зашла сестра хозяйки, симпатичная женщина средних лет, ещё˚сохранившая девичью красоту. Они возвратились недавно с˚Шпицбергена\footnote{Шпицб\'{е}рген (нем. Spitzbergen) "--- обширный полярный архипелаг, расположенный в˚Северном Ледовитом океане. Самая северная часть Норвегии. 

Значительную, по˚арктическим меркам, хозяйственную деятельность на˚архипелаге помимо Норвегии, согласно особому статусу архипелага, осуществляет только Россия, имеющая на˚острове Западный Шпицберген российский населённый пункт "--- посёлок Баренцбург, а˚также законсервированные посёлки Пирамида и˚Грумант.}. 
Внешне представилось, что проживают в˚достатке. Начали играть в˚лото на˚деньги. Объяснили смысл игры и˚пригласили Мишу поиграть. Он~заявил, что нет денег, они ему дали в˚долг. В~этот вечер счастье сопутствовало Мише, и он˚выиграл 7~рублей. Лица их˚выражали какое\=/то недовольство. Но˚Миша подумал, что фортуна улыбается и˚помогает обездоленным. Хозяйка квартиры уложила его спать у˚порога на˚коврике. Поэтому на˚рассвете Миша сбежал из˚этой квартиры, ни с˚кем не~попрощавшись. Вскоре Иван развёлся с˚женой. Внутренне Миша приветствовал его поступок.

Из-за отсутствия денег и˚места проживания Миша решил поработать. Из˚газеты узнал, что Подольстрой набирает рабочих. Прибыл по˚указанному адресу. Его˚зачислили рабочим бригадиром строительно\-/монтажного управления №~14 Мособлспецстрой №~1, предоставили место проживания в˚бараке. Смущало и˚волновало Мишу, что в˚паспорте поставили штамп «зачислен по˚оргнабору\footnote{Организационный набор (оргнабор) "--- форма привлечения неквалифицированной рабочей силы на˚промышленные предприятия в˚крупных городах в˚СССР в˚1950\==80-х годах.}». Он думал, что это помешает осуществить мечту"--- поступить в˚институт.

Барак находился на˚Южном. В˚нём проживали 20~ребят, в˚основном детдомовцы, и˚5~семейных пар, размещавшихся за˚ширмами. При˚заселении перед глазами Миши предстала ужасная картина: ребята лежали в˚кроватях на˚простынях в˚грязных сапогах (в~глине). Оказывается, после получки они пропивали все деньги и˚не~ходили на˚работу. Даже˚использовался термин «варим колун\footnote{Кол\'{у}н "--- разновидность топора, предназначенного для колки дров.}», то˚есть деньги пропиты, есть нечего (кое-что сшибали\footnote{Сшибать "--- выпрашивать.} у˚семейных, чем˚вызывали недовольство). При˚выяснении оказалось, что барак являлся сборищем для˚городских пьяниц, т.~к.˚рядом находился магазин, в˚котором продавали спиртное. 

Перед Мишей стояла проблема, как исправить положение дел. Он решил основательно познакомиться с˚каждым из˚проживающих в˚бараке. На младших по˚возрасту как-то подействовать силой, а˚на˚старших воздействовать убеждением. На˚его стороне оказались и˚семейные. Решил ещё действовать через Подольстрой, приобрести что\=/либо для барака. Приобрёл шашки, старый телевизор, какой\=/то стол. Отдельные ребята поверили, надеясь, что он может что-то улучшить. После˚длительных убеждений договорились с˚ребятами, что пьянки посторонних лиц в˚бараке прекратить, просто выбрасывать их из˚барака. Ребята поддержали такое предложение.

Однажды Миша возвратился вечером с˚работы в˚барак. Пять посторонних человек распивали в˚бараке водку. Ребята сидели на˚своих кроватях, притаившись. Миша спросил: «Почему они здесь, ведь мы˚договорились?» Начал вместе с˚ребятами выталкивать их из˚барака. Последний оказался сильным. Поэтому пришлось уложить его на˚кровать, а˚второй кроватью приложить сверху. При˚возне с˚последним один из˚первых где-то схватил топор и˚сзади замахнулся, чтобы разрубить Мише голову. Благодаря Володе Cтрукову Миша был спасён. Володя обхватил сзади его руки и˚предотвратил опасность. Он пообещал Мишу встретить. В~г.~Подольске в то˚время процветал бандитизм, поэтому Миша принял решение действовать на˚опережение. Миша узнал, где этот молодой человек работает, когда заканчивает и˚его путь домой. Потом встретил его и˚сказал: «Я~явился, будем выяснять отношения?» Он˚явно не~ожидал подобного и˚ответил: «Не˚будем». Так˚они разошлись.

Чтобы˚ребята не~пропивали зарплату, пришлось Мише деньги отдельных из них держать у˚себя. Выдавал им˚лишь на˚одну бутылку. Конечно, это стоило многих неприятностей, даже угроз. Приходилось даже ходить вместе с˚ними в˚магазин и˚покупать им˚брюки, ботинки. Один из них "--- Сергей, ходил весь оборванный. Не˚хотел идти с˚Мишей в˚магазин, чтобы приобрести что-то из˚одежды. Тогда Миша˚предложил ему вечером сходить в˚соседний барак. Там проживали незнакомые девушки. Миша надеялся застать у них городских, хорошо одетых ребят и˚как-то психологически подействовать на˚Сергея. Надежды Миши оправдались. Девушки приняли их приветливо. У˚них действительно находились городские ребята. Среди˚одной из˚бесед Сергей прошептал: «Миша, уйдём, мне стыдно». В~это время внутренне Миша˚торжествовал "--- это была победа.

На~следующий день после заселения в˚барак Миша отправился принимать бригаду. Бригада работала у˚городской больницы. В~состав бригады входили 19~девушек и˚один мужчина "--- сантехник. Среди˚девушек услышал сплошной мат. Для˚Миши это было неожиданно и дико. Поэтому он возвратился обратно в˚барак. По˚дороге сам себе сказал: «Слюнтяй, возьми себя в˚руки». Назавтра опять пришел на˚место работы и˚представился девушкам.

Бригаде было поручено проложить водопровод от˚артскважины вдоль забора военной части. Представления о˚предстоящей работе не~имелось, поэтому пришлось ночью изучать, как прокладывать трубы.

Работа у˚девушек была тяжёлая. Они верёвками должны были перетаскивать трубы диаметром 350\,мм и˚весом 1\,200\,кг˚каждая, укладывать в˚траншею, копать, а˚иногда долбить мёрзлую землю. Трубы укладывали на˚глубину 2\,м, сантехник должен был чеканить, заделывать стыки труб и˚складывать из˚кирпича смотровые колодцы.

Платили девушкам мало. Поэтому стояла задача создать дружный коллектив, как-то облегчить их˚труд, повысить зарплату и˚их˚заинтересованность. Первоначально Миша решил откровенно побеседовать с˚каждой девушкой. Каждая из˚них излила своё наболевшее, считая себя неудачницей в˚этой жизни. У~многих была неудачная любовь, а˚некоторые даже признались, что пришлось сделать аборты.

У~самого Миши денег не~было. Поэтому целую неделю он вёл голодное существование. Единственным пропитанием была вода. Во˚время обеда, когда девушки расходились по˚столовым, Миша снимал свой комбинезон, который надевал поверх военной формы, и˚делал вид, что шёл на˚обед. Несмотря˚на тяжёлые условия труда и жизни, девушки сохранили человеческие качества, доброту, внимательность, взаимопонимание и˚взаимопомощь. Однажды, по˚возвращении с «обеда», в˚кармане Мишиного комбинезона оказался кусок хлеба и˚сала. Девушки проследили, что он˚не~обедает, и подложили. Стал спрашивать, кто это сделал. Девушки не~признались. Естественно, этот хлеб и˚сало Миша˚тут же˚втихаря съел. Так˚они продолжали подкладывать до˚первой его получки. Сколько Миша˚не~спрашивал, так и˚не~узнал своего спасителя.

После˚получки Миша купил килограмм конфет, разбил бригаду на˚две части и˚поручил одинаковые задания. Тем, кто выполнял задания первыми и˚качественнее, вручал конфеты и даже раньше времени отпускал домой. Так˚продолжалось некоторое время. Это стимулировало и˚повысило производительность труда. Но~последовало анонимное заявление, что Миша нарушает трудовое законодательство.

По˚существующему положению, Миша не~имел права рыть траншею, пока не~будет составлен акт с Мосэнерго. В~один воскресный день Миша вызвал представителя Мосэнерго, который сказал, что вдоль забора военной части можно рыть свободно, подземные кабели отсутствуют. На~предложение составить акт он˚ответил, что не~следует тратить время в˚воскресный день, завтра сделаем, и˚на˚своей машине отвёз Мишу в˚барак. Миша принялся варить картофельный суп на˚воде. Вдруг прибежал второй экскаваторщик и˚крикнул: «Нет экскаватора и˚экскаваторщика, всё сгорело!» Побежали через кладбище к˚месту работы. Действительно, Миша увидел большой бугор взорванной земли, часть ковша сгорела, а˚в˚кабине сидит экскаваторщик без всяких признаков жизни. Экскаваторщика сбросили на˚землю. Оказывается,при˚рытье траншеи наскочили на˚кабель очень высокого напряжения, который питал часть заводов г.~Подольска. Вскоре понаехало много черных «Волг». 

Выяснили, кто ведёт работы, и˚сразу составили акт на˚400~тыс. руб.\footnote{Для˚сравнения: средний размер заработной платы в˚СССР в˚1960~году составлял 73,1~руб.} и˚предложили в˚течение получаса устранить аварию. Опасность заключалась в˚том, что мог проходить ток и˚при вскрытии (откапывании) провода могли быть жертвы. Поэтому Миша со˚слезами на˚глазах обратился к˚девушкам, чтобы они одели по˚две пары брезентовых рукавиц и˚поработали так, как в˚жизни ещё не~работали. Девушки полностью осознали создавшуюся ситуацию. К~счастью, при порыве на˚кабеле натянулась обмотка и˚ток не~шёл. Порыв удалось ликвидировать в˚сроки. 

\begin{wrapfigure}{O}{.4\textwidth}
\centering
\ifPubTypeEBook
	\includegraphics[width=.35\textwidth]{gazM1_select}
\else
	\includegraphics[width=.35\textwidth]{gazM1_select_bw}
\fi
\caption[Чёрный воронок. Автомобиль ГАЗ-М\=/1 в˚Музее отечественной военной истории]{Чёрный воронок. Автомобиль ГАЗ-М\=/1 в˚Музее отечественной военной истории\footnotemark}
\label{fig:gazM1}
\end{wrapfigure}
\footnotetext{Автор: Музей отечественной военной истории. URL: \scriptsize\url{http://www.kskdivniy.ru/}.}

Тут же˚подъехал чёрный воронок\footnote{Чёрный воронок (фразеологическое сочетание) "--- автомобиль для перевозки арестантов.} и˚Мишу отвезли в˚прокуратуру г.~Подольска. Миша изложил представителю прокуратуры подробно, как обстояло дело. Через˚некоторое время привезли представителя Мосэнерго. На˚вопрос видел ли он˚данного товарища (Мишу), последовал ответ: «Не˚видел». Тогда прокурор спросил Мишу: «Чем можешь доказать, что вызывал представителя Мосэнерго и˚тот отложил составление акта на завтра?» 
Перед˚глазами Миши, как на˚экране, появилась артскважина, стоящие рядом сторож и˚представитель Мосэнерго. Последовал ответ, что, возможно, их разговор слышал стоящий рядом сторож. Поехали за˚сторожем. Миша в˚это время начал седеть, то˚есть на˚голове появилась седина. По˚приезду сторож ответил, что видел этих людей. Слово в˚слово изложил то, что говорил Миша. Миша при прокуроре вскочил, расцеловал сторожа и˚сунул ему в˚карман деньги, которые у˚него были.

Прокурор заулыбался и˚проронил: «Ты~ещё даёшь взятку». Он~прекрасно понял то˚напряжение, которое овладело Мишей. Впоследствии, акт на˚400~тыс. руб. был ликвидирован.

Экскаваторщик после˚потери сознания очнулся. Во˚время обрыва электропровода высокого напряжения он оказался в˚вольтовой дуге, в˚шоковом состоянии. В~суматохе про˚него даже забыли. Видимо, лежание на˚сырой земле оказало на˚него положительное влияние. 

\begin{wrapfigure}[15]{O}{.4\textwidth}
\centering
\ifPubTypeEBook
	\includegraphics[width=.35\textwidth]{elctrDuga}
\else
	\includegraphics[width=.35\textwidth]{elctrDuga_bw}
\fi
\caption{Вольтова (электрическая) дуга}
\label{fig:elctrDuga}
\end{wrapfigure}

Бригада начала перевыполнять план на˚20\==30~\%. Для˚облегчения труда девушек старались больше использовать бульдозер. Зарплата у˚девушек повысилась. Но˚кое\=/кого это беспокоило. Было представлено начальнику управления анонимное заявление, что Миша приписывает объёмы земли, которые перебросали лопатами девушки. Создали комиссию, которая не~смогла этого доказать.

Однажды, два дня подряд не~явился на˚работу сантехник. Пришлось Мише чеканить трубы (заделывать стыки водопровода) и, чтобы привезённый раствор не~пропал, сложить 2~смотровых колодца самому под руководством опытной девушки Любы. Подобной работой Миша раньше не~занимался.

На˚третий день Миша поехал к˚начальнику участка выяснить, почему сантехник не~выходит на˚работу. Неожиданно представилась картина: сидят и˚выпивают сантехник и˚начальник участка. Нервы в˚это время были на˚пределе, и˚Миша схватил за˚грудь начальника участка, прижал к˚стенке. Затем позвонил начальнику управления и˚попросил приехать и˚разобраться. При˚этом заявил, что избил начальника участка. К~концу дня явился начальник управления, устроил собрание. На˚собрании выяснилось, что со˚стороны начальника участка было много нарушений, он даже˚отнимал у˚девушек часть зарплаты. За эти и˚другие крупные нарушения начальника участка сняли с˚должности. 

Очень взволновал Мишу один момент. Подозвал Мишу бывший начальник участка и˚сказал Рыконовой "--- начальнику отдела кадров: «С~него будет толк, ты˚побереги его». Для˚Миши это было неожиданно, ведь начальника участка сняли по˚его инициативе. 

\begin{wrapfigure}{O}{.4\textwidth}
\centering
\ifPubTypeEBook
	\includegraphics[width=.35\textwidth]{guz}
\else
	\includegraphics[width=.35\textwidth]{guz_bw}
\fi
\caption[Современный вид здания Государственного университета по˚землеустройству (Ранее Московский институт инженеров землеустройства "--- МИИЗ)]{Современный вид здания Государственного университета по˚землеустройству (ранее Московский институт инженеров землеустройства "--- МИИЗ)\footnotemark}
\label{fig:guz}
\end{wrapfigure}
\footnotetext{Автор: А.~Миронова, 21.07.2013.}

Вечером к˚бараку подъехала машина. Шофёр сказал, чтобы Миша поехал с˚ним, т.~к. его вызвала Рыконова. Привёз Мишу в˚квартиру Рыконовой. Столик был накрыт. Рыконова предложила Мише поступать в˚строительный институт, она может оказать содействие. Такое предложение почему\=/то разозлило Мишу. Последовал ответ, что этому не~бывать. Подобные обстоятельства убедили Мишу, что необходимо поступать уже в˚этом, 1959~году.

Из˚объявления газеты «Комсомольская правда» Миша узнал, что принимают на˚первый курс Московского института инженеров землеустройства (МИИЗ).

Жители барака погружались в˚сон, а˚Миша сидел в˚коридорчике (чулане) и˚готовился к˚экзамену. Необходимые книги предоставили девушки из˚бригады.

Однажды на˚день оставил бригаду и˚поехал в˚Москву узнать условия поступления в˚институт. Зашёл к˚ответственному секретарю приёмной комиссии (к~сожалению, фамилию его Миша позабыл). Он оказался сочувствующим человеком, выдал Мише блокнот, карандаш и˚сразу повёл на˚занятие. Шли последние занятия по˚математике на˚подготовительных курсах. Из˚объяснений преподавателя Миша что-то понимал, что-то не~очень (сказался разрыв в учёбе). Обратился к˚рядом сидящему абитуриенту: «Ты всё понимаешь?» Последовал ответ, что не~всё. Этот ответ Мишу «окрылил». Подумалось: «Он˚посещал все подготовительные курсы и не~всё понимает, а я˚кое-что понимаю. Следовательно, можно наверстать». И~Миша начал активно готовиться к˚экзаменам по˚ночам.

Вступительные экзамены Миша сдал успешно и˚был зачислен в˚число студентов первого курса факультета землеустройства Московского института инженеров землеустройства. Ещё в˚течение месяца его не~хотели увольнять с˚работы и даже˚платили зарплату. Девушки работали. При˚посещении их˚на˚работе не~хотели отпускать и просили их˚не~оставлять!