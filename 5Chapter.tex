\chapter{Жизненные испытания}

Поскольку в˚г.~Москве не~было знакомых, а˚в˚г.~Подольске проживал дальний родственник \Todo[вар.]{дальний родственник Иван}, то˚Миша решил сначала поехать к˚нему в˚г.~Подольск. Дверь открыла высокая женщина, с˚заострённым лицом, большим тонким носом. Лицо отражало какую\=/то злобу. Пришлось Мише объясниться, кто он и˚сразу предупредить, что жить у˚них не~собирается. Вечером с˚работы возвратился родственник. Обрадовался приезду Миши. Но˚сразу почувствовалось, что в˚квартире даже им˚управляет жена. Затем зашла сестра хозяйки, \Todo[вар.]{симпатичная} женщина средних лет, даже\Todo[вар.]{ещё} сохранившая девичью красоту. Они возвратились недавно из˚Шпицбергена\footnote{Шпицб\'{е}рген (нем. Spitzbergen) "--- обширный полярный архипелаг, расположенный в˚Северном Ледовитом океане. Самая северная часть Норвегии. 

Значительную, по˚арктическим меркам, хозяйственную деятельность на˚архипелаге помимо Норвегии, согласно особому статусу архипелага, осуществляет только Россия, имеющая на˚острове Западный Шпицберген российский населённый пункт "--- посёлок Баренцбург, а˚также законсервированные посёлки Пирамида и˚Грумант.}. 
Внешне представилось, что проживают в˚достатке. Начали играть в˚лото на˚деньги. Объяснили смысл игры и˚пригласили Мишу поиграть. Он~заявил, что нет денег, они ему дали в˚долг. В~этот вечер счастье сопутствовало Мише, и он˚выиграл 7~рублей. Лица их˚выражали какое\=/то недовольство. Но˚Миша подумал, что фортуна улыбается и˚помогает обездоленным. Хозяйка квартиры уложила его спать у˚порога на˚коврике. Поэтому на˚рассвете Миша сбежал из˚этой квартиры, ни с˚кем не~попрощавшись. Вскоре Иван развёлся с˚женой. Внутренне Миша приветствовал его поступок.

Из-за отсутствия денег и˚места проживания Миша решил поработать. Из˚газеты узнал, что Подольстрой набирает рабочих. Прибыл по˚указанному адресу. Зачислили рабочим бригадиром строительно\-/монтажного управления №~14 Мособлспецстрой №~1. Предоставили место проживания в˚бараке. Смущало и˚волновало Мишу, что в˚паспорте поставили штамп «зачислен по˚оргнабору\footnote{Организационный набор (оргнабор) "--- \Todo[Текст]{Определение оргнабору}.}».\Todo[Вопрос]{Почему смутил организационный набор?} 

Барак находился на˚Южном, вблизи магазина, в˚котором продавали спиртное. В~бараке проживали 20~ребят, в˚основном детдомовцы, и˚5~семейных пар, размещавшихся за ширмами. В~бараке перед глазами Миши предстала ужасная картина: ребята лежали в˚кроватях на˚простынях в˚грязных сапогах (в~глине). Оказывается, после получки они пропивали все деньги, не~ходили на˚работу. Даже˚использовался термин «варим колун\footnote{Кол\'{у}н "--- разновидность топора, предназначенного для колки дров.}», то˚есть деньги пропиты, есть нечего (кое-что сшибали\footnote{Сшибать "--- выпрашивать.} у˚семейных, чем˚вызывали недовольство). При˚выяснении оказалось, что барак являлся местом пьянства\todo[вар.]{сборища для˚} городских пьяниц, ведь рядом находился магазин. 

Перед Мишей стояла проблема, как исправить положение дел. Решил основательно познакомиться с˚каждым.\todo[ред.]{След. предложение надо переделать, т.к. в совокупности с предыдущим непонятно звучит - заставить познакомиться что ли?} Младших по˚возрасту как-то заставить, а˚старших "--- убеждением. На˚его стороне оказались и˚семейные. Решил ещё действовать через Подольстрой, приобрести что\=/либо для барака. Приобрёл шашки, старый телевизор, какой\=/то стол. Отдельные ребята поверили, надеясь, что он˚может что-то улучшить. Договорились с˚ребятами, что пьянки посторонних лиц в˚бараке \todo[вар.]{нужно} прекратить, просто выбрасывать их из˚барака. Ребята поддержали данное\Todo[вар.]{такое} предложение. 

На~следующий день Миша отправился принимать бригаду. Девушки работали у˚городской больницы. В~состав бригады входили 19~девушек и˚один мужчина "--- сантехник.\todo[вар.]{Бригада работала у˚городской больницы, и˚в˚её состав входили 19~девушек и˚один мужчина "--- сантехник.} Среди˚девушек услышал сплошной мат. Для˚Миши это было неожиданным, впервые. Поэтому для него это было дико, и˚он˚возвратился обратно в˚барак. По˚дороге сам себе сказал: «Слюнтяй, возьми себя в˚руки». Назавтра опять явился и˚представился.\todo[ред.]{Мб переформулировать без слова явился}

Бригаде было поручено проложить водопровод от˚артскважины вдоль забора военной части. Представления о˚предстоящей работе не~имелось, поэтому пришлось ночью изучать, как прокладывать трубы.

Работа у˚девушек была тяжёлая. Они верёвками должны перетаскивать трубы диаметром 350\,мм и˚весом 1\,200\,кг˚каждая, укладывать в˚траншею, копать, а˚иногда долбить мёрзлую землю. Трубы укладывали на˚глубину 2\,м, сантехник должен был чеканить, заделывать стыки труб и˚устанавливать смотровые колодцы.

Платили девушкам мало. Поэтому стояла задача создать дружный коллектив, как-то облегчить их˚труд, повысить зарплату и˚их˚заинтересованность. Первоначально Миша решил откровенно побеседовать с˚каждой девушкой персонально. Индивидуальная беседа оказалась откровенной. Каждая из˚них излила своё наболевшее, считая себя неудачницей в˚этой жизни. У~многих была неудачная любовь, а˚некоторые даже признались, что пришлось сделать аборт.

У~самого Миши денег не~было. Ходил в˚военной форме,\todo[вар.]{поверх которой на работе} сверху на˚работе надевал комбинезон. Поэтому целую неделю вёл голодное существование. Единственным пропитанием была вода. Во˚время обеда девушки расходились по˚столовым. Миша снимал свой комбинезон и˚делал вид, что ушёл на˚обед. Несмотря˚на тяжёлые условия труда, жизни, девушки сохранили человеческие качества, доброту, внимательность, взаимопонимание и˚взаимопомощь. По˚возвращении с «обеда» однажды в˚кармане Мишиного комбинезона оказался кусок хлеба и˚сала. Девушки проследили, что он˚не~обедает, подложили. Стал спрашивать, кто это сделал. Девушки не~признались. Естественно, этот хлеб и˚сало Миша˚тут же˚втихаря съел. Так˚они продолжали подкладывать до˚первой его получки. Сколько Миша˚не~спрашивал, так и~не~узнал своего спасителя.

После˚получки купил килограмм конфет. Миша разбил бригаду на˚две части и˚поручил одинаковые задания. Тем, кто выполнял задания первыми и˚качественнее, вручал конфеты и даже раньше времени отпускал домой. Так˚продолжалось некоторое время. Это стимулировало и˚повысило производительность труда. Но~\todo[вар.]{от˚анонимного недображелателя поступило} последовало заявление, что Миша нарушает трудовое законодательство.

Однажды Миша возвратился вечером с˚работы в˚барак. Пять посторонних человек распивали в˚бараке водку. Ребята сидели на˚своих кроватях, притаившись. Миша спросил: «Почему они здесь, ведь мы˚договорились?» Начал вместе с˚ребятами выталкивать их из˚барака. Последний оказался сильным. Поэтому пришлось уложить его на˚кровать, а˚второй приложить. При˚занятиях с˚последним один из˚первых где-то схватил топор и˚сзади замахнулся, чтобы разрубить Мише голову. Благодаря Володе Cтрукову Миша был спасён. Он обхватил сзади его руки и˚предотвратил опасность. \todo[ред.]{Немного бы переделать, надо пояснить, кто обещал встретить, так-то понятно, но не литературно}Пообещал Мишу встретить. В~г.~Подольске в то˚время процветал бандитизм. Поэтому Миша принял решение действовать на˚опережение. Узнал, где он˚работает и˚когда заканчивает, его путь домой. Встретил его и˚сказал: «Я~явился, будем выяснять отношения?» Он˚явно не~ожидал подобного и˚ответил: «Не˚будем». Так˚они разошлись.

Чтобы˚ребята не~пропивали зарплату, пришлось Мише деньги отдельных из них держать у˚себя. Выдавал им˚лишь на˚одну бутылку. Конечно, это стоило много неприятностей, даже угроз. Приходилось ходить вместе и˚покупать им˚брюки, ботинки. Один из них "--- Сергей, ходил весь оборванный. Не˚хотел идти с˚Мишей в˚магазин, чтобы приобрести что-то из˚одежды. Тогда Миша˚предложил ему вечером сходить в˚соседний барак. Там проживали девушки. Никого из˚них Миша не~знал. Надеялся увидеть там городских, хорошо одетых ребят и˚как-то психологически подействовать на˚Сергея. Надежды Миши оправдались. Девушки приняли их приход приветливо. Действительно, там находились городские ребята. Среди˚бесед Сергей прошептал: «Миша, уйдём, мне стыдно». В~это время внутренне Миша˚торжествовал. Это была победа.

\begin{wrapfigure}{O}{.4\textwidth}
\centering
\includegraphics[width=.35\textwidth]{gazM1}
\caption[Чёрный воронок. Автомобиль ГАЗ-М\=/1 в˚Музее отечественной военной истории]{Чёрный воронок. Автомобиль ГАЗ-М\=/1 в˚Музее отечественной военной истории\footnotemark}
\label{fig:gazM1}
\end{wrapfigure}
\footnotetext{Автор: Музей отечественной военной истории. URL: \url{http://www.kskdivniy.ru/}.}

По˚положению\todo[вар.]{Согласно положению о производстве работ}, Миша не~имел права рыть траншею, пока не~будет акт от˚представителя Мосэнерго. Был воскресный день. Миша вызвал представителя. Он сказал, что вдоль забора военной части можете рыть свободно, подземные кабели отсутствуют. На˚предложение составить акт он˚ответил, что не~следует тратить время в˚воскресный день, завтра сделаем. На˚своей машине отвёз Мишу в˚барак. Он~принялся варить картофельный суп на˚воде. Вдруг прибежал второй экскаваторщик и˚крикнул: «Нет экскаватора и˚экскаваторщика, всё сгорело!» Побежали через кладбище к˚месту работы. Действительно, Миша увидел большой бугор взорванной земли, часть ковша сгорела, а˚в˚кабине сидит экскаваторщик без всяких признаков жизни. Экскаваторщика сбросили на˚землю. Оказывается,\todo[вар.]{при рытье траншеи} наскочили на˚провод очень высокого напряжения, который питал часть заводов г.~Подольска. Вскоре понаехало много черных «Волг». 

Выяснили, кто ведёт работы, и˚сразу составили акт на˚400~тыс. руб.\footnote{Для˚сравнения: средний размер заработной платы в˚СССР в˚1960~году составлял 73,1~руб.} и˚предложили в˚течение получаса устранить аварию. Опасность заключалась в˚том, что мог проходить ток и˚при вскрытии (откапывании) провода могли быть жертвы. Поэтому Миша со˚слезами на˚глазах обратился к˚девушкам, чтобы они одели по˚две пары брезентовых рукавиц и˚поработали так, как в˚жизни ещё не~работали. Девушки полностью осознали создавшуюся ситуацию. К~счастью, при порыве натянулась на˚кабеле обмотка и˚ток не~шёл. Порыв удалось ликвидировать в˚сроки. Тут же˚подъехал чёрный воронок\footnote{Чёрный воронок (фразеологическое сочетание) "--- автомобиль для перевозки арестантов.} и˚Мишу отвезли в˚прокуратуру г.~Подольска. Миша изложил представителю прокуратуры подробно, как обстояло дело. Через˚некоторое время привезли представителя Мосэнерго. На˚вопрос видел ли он˚данного товарища (Мишу), последовал ответ: «Не˚видел». Тогда прокурор спросил Мишу: «Чем можешь доказать, что вызывал представителя и˚тот отложил составление акта назавтра?» 

\begin{wrapfigure}{O}{.4\textwidth}
\centering
\includegraphics[width=.35\textwidth]{elctrDuga}
\caption{Вольтова (электрическая) дуга}
\label{fig:elctrDuga}
\end{wrapfigure}
Перед˚глазами Миши, как на˚экране, появилась артскважина, стоящие рядом сторож и˚представитель Мосэнерго. Последовал ответ, что, возможно, их разговор слышал стоящий рядом сторож. Поехали за˚сторожем. Миша в˚это время начал седеть, то˚есть на˚голове появилась седина. По˚приезду сторож ответил, что видел этих людей. Слово в˚слово изложил, о˚чём они˚говорили. Миша при прокуроре вскочил, расцеловал сторожа и˚сунул ему в˚карман деньги, которые у˚него были.

Прокурор заулыбался и˚проронил: «Ты~ещё даёшь взятку». Он прекрасно понял то˚напряжение, которое овладело Мишей. Впоследствии, акт на˚400~тыс. руб. был ликвидирован.

Экскаваторщик ожил\todo[вар.]{выжил или был жив}. Он оказался в˚вольтовой дуге, в˚шоковом состоянии. В~суматохе про˚него даже забыли. Видимо, лежание на˚сырой земле оказало на˚него положительное влияние. 

Бригада начала перевыполнять план на˚20\==30~\%. Для˚облегчения труда девушек старались больше использовать бульдозер. Зарплата у˚девушек повысилась. Но˚кое\=/кого это беспокоило. Было представлено начальнику управления анонимное заявление, что Миша приписывает объёмы земли, которые перебросали лопатами девушки. Создали комиссию, которая не~смогла этого доказать.

Однажды два дня подряд не~явился на˚работу сантехник. Пришлось чеканить трубы (заделывать стыки водопровода) и, чтобы привезённый раствор не~пропал, сложить 2~смотровых колодца самому, под руководством опытной девушки Любы.\todo[вар.]{, ведь раньше подобной работой Миша не занимался.} Подобной работой ведь раньше не~занимался.

На˚третий день Миша поехал к˚начальнику участка выяснять, почему не~является\todo[вар.]{не выходит на работу} сантехник. Неожиданно представилась картина: сидят и˚выпивают сантехник и˚начальник участка. Нервы в˚это время были на˚пределе, и˚Миша схватил за˚грудь начальника участка, прижал к˚стенке. Затем позвонил начальнику управления и˚попросил выехать и˚разобраться. При˚этом заявил, что избил начальника участка. К~концу дня явился начальник управления, устроили собрание. На˚собрании выяснилось, что со˚стороны начальника участка было много нарушений. Он даже˚отнимал у˚девушек часть зарплаты. Были и˚другие крупные нарушения.

\begin{wrapfigure}{O}{.4\textwidth}
\centering
\includegraphics[width=.35\textwidth]{guz}
\caption[Современный вид здания Государственного университета по˚землеустройству (Ранее Московский институт инженеров землеустройства "--- МИИЗ)]{Современный вид здания Государственного университета по˚землеустройству (Ранее Московский институт инженеров землеустройства "--- МИИЗ)\footnotemark}
\label{fig:guz}
\end{wrapfigure}
\footnotetext{Автор: А.~Миронова, 21.07.2013.}

Начальника участка сняли с˚должности. Очень взволновал Мишу один момент:\todo[вар.]{он шёл и увидел, как стояли рядом начальник отдела кадров Рыконова и бывший начальник участка (они были в˚близких отношениях). Бывший начальник участка подозвал Мишу и˚говорит Рыконовой:}стоит рядом с˚начальником отдела кадров Рыконовой бывший начальник участка (они были в˚близких отношениях). Подзывает Мишу и˚говорит Рыконовой: «С~него будет толк, ты˚побереги его». Для˚Миши это было неожиданно, ведь бывшего начальника участка сняли по˚его инициативе. 

Вечером к˚бараку подъехала машина. Шофёр сказал, чтобы Миша поехал с˚ним, т.~к. его вызвала Рыконова. Привёз Мишу в˚квартиру Рыконовой. Столик был накрыт. Рыконова предложила Мише поступать в˚строительный институт, а она\todo[вар.]{могла бы оказать} окажет содействие. Такое предложение почему\=/то разозлило Мишу. Последовал ответ, что этому не~бывать. Это\todo[вар.]{событие} убедило Мишу, что необходимо поступать уже в˚этом, 1959~году.\todo[Текст]{Не совсем ясна мотивация}

Из˚объявления газеты «Комсомольская правда» Миша узнал, что принимают на˚первый курс Московского института инженеров землеустройства (МИИЗ).

Жители барака погружались в˚сон, а˚Миша сидел в˚коридорчике (чулане)  и˚готовился к˚экзамену. Необходимые книги предоставили девушки из˚бригады.

Однажды на˚день оставил бригаду и˚поехал в˚Москву узнать \todo[вар.]{условия поступления в˚институт} все обстоятельства. Зашёл к˚ответственному секретарю в˚приёмной комиссии (к~сожалению фамилию его Миша позабыл\todo[вар.]{удалить слова в скобках}). Он оказался сочувствующим человеком, выдал Мише блокнот, карандаш и˚сразу повёл на˚занятие. Шли последние занятия по˚математике на˚подготовительных курсах. Из˚объяснений преподавателя Миша что-то понимал, что-то не~очень (сказался разрыв в учёбе). Обратился к˚рядом сидящему абитуриенту: «Ты всё понимаешь?» Последовал ответ, что не~всё. Этот ответ Мишу «окрылил». Подумалось: «Он˚посещал все подготовительные курсы и не~всё понимает, а я˚кое-что понимаю. Следовательно, можно наверстать». И~Миша начал активно готовиться к˚экзаменам по˚ночам.

Вступительные экзамены Миша сдал успешно и˚был зачислен в˚число студентов первого курса факультета землеустройства Московского института инженеров землеустройства. Ещё в˚течение месяца не~хотели увольнять с˚работы. Даже˚получил зарплату. Девушки работали. При˚посещении их˚на˚работе не~хотели отпускать и просили их˚не~оставлять!