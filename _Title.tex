% Оформление по ГОСТ Р 7.0.4-2006
% Пользовательские переменные
\newcommand{\AuthorFam}{Шубич}
\newcommand{\AuthorName}{Михаил Павлович}
\newcommand{\AuthorIO}{М.\,П.}
\newcommand{\Title}{Во имя жизни}
\newcommand{\Subject}{Автобиографическая повесть}	
\newcommand{\YearPub}{2018}
\newcommand{\CityPub}{Москва}
\newcommand{\Publishers}{Самиздат\todo[Орг]{Надо исправить издательство}}
\newcommand{\UDK}{82\==94}
\newcommand{\BBK}{84}
\newcommand{\AuthorMark}{Ш 95}
\newcommand{\AgeLimit}{16\,+}
\newcommand{\ISBN}{978\==5\==11111\==111\==1}

%!!!! Не забудь временно включить пакет auto-pst-pdf чтобы воссоздать штрих-код
\newcommand{\ISBNbar}{978-5-11111-111}	% ISBN для создания штрих кода. Последнее число можно не вводить. Оно будет вычислено автоматически. Причина вывода в отдельную переменную - команда для построения штрих кода требует использовать в качестве разделителя дефис '-', а не короткое тире '–' или '\=='


\newcommand{\Author}{\AuthorName\ \AuthorFam}
\newcommand{\AuthorAbbrA}{\AuthorIO~\AuthorFam}
\newcommand{\AuthorAbbrB}{\AuthorFam~\AuthorIO}

\newcommand{\BibInfo}{\Title~: \MakeLowercase{\Subject}~/ \AuthorAbbrA. ~— М.~:~\Publishers, \YearPub.~— \pageref{myEndRomanPage}, \pageref{myLastPage}~с.~: ил.}

% Настройки шрифтов
\setkomafont{publishers}{\normalfont}
\setkomafont{publishers}{\normalfont\large}
\setkomafont{date}{\normalfont\large}
\setkomafont{subject}{\normalfont\itshape}

% Пользовательский блок
\newcommand{\bibInd}[3]{%
		\pbox[t]{\linewidth}{УДК \\ББК }
		\pbox[t]{\linewidth}{%
		 #1\\				% УДК
		 #2\\				% ББК
		 #3}}				% авторский знак

%-- Титульный лист средствами Koma Script
% Шмуцтитул - страница, размещённая перед титульным листом для защиты последнего от грязи и порчи
%\extratitle{\vspace*{4\baselineskip}
%\begin{center}
%\textbf{\huge\AuthorAbbr\\[3ex]\Title}
%\end{center}}

% Основной титульный лист
%\titlehead{Колонтитул}									% Колонитутл титульного листа
\subject{\Subject}											% Тема книги
\title{\Title}													% Название книги
%\subtitle{Подзаголовок}								% Подзагаловок
\author{\Author}												% Автор1  \and Автор2  \and Автор 3...
\date{\YearPub}													% Дата издания
\publishers{\CityPub\\\Publishers}			% Издательство
%\thanks																% \thanks - создаст сноску, но должен использоваться в любом из вышеуказанных элементов

% Оборот титульного листа
%\uppertitleback{Аннотация}										%	Верх оборота титульной страницы
%\lowertitleback{Создано в {\KOMAScript} and {\LaTeX}}			% Низ оборота титульной страницы

% Посвящение
%\dedication{Посвящается ....\\с любовью}			% Кому книга посвящается
												
%\maketitle																			% Создать титульный лист

%-- Пользовательский титульный лист

\begin{titlepage}
% Титул
	\begin{center}
		\usekomafont{author}{\Author}
		\null\vfill
		\resizebox{\linewidth}{!}{\usekomafont{title}{\Huge\Title}}	% Масштабировать текст до ширины страницы
		\vskip 1em
		\usekomafont{subject}{\Subject}		
		\vfill
		\usekomafont{publishers}{\CityPub}
		\vskip .5em
		\usekomafont{publishers}{\Publishers}
		\vskip .5em
		\usekomafont{date}{\YearPub}
	\end{center}
	
% Оборот титула
	\clearpage
	\thispagestyle{empty}		
	% Библиографический индекс сверху			
	\noindent
	\bibInd{\UDK}{\BBK}{\AuthorMark}
	
	\vfill
	\noindent
	\hspace{1em}\hphantom{\AuthorMark}	% Вставляет пустой блок, нулевой высоты и толщины, но шириной как у аргумента. Есть ещё \vphantom и \phantom
	\begin{minipage}[t]{\linewidth-1em-\widthof{\AuthorMark}}
		% Год основания серии (подсерии), сведения о лицах, принимающих участие в создании серии (подсерии).
		% Если соавторов четыре и более, их имена помещают на обороте титульного листа. Перед именами соавторов на обороте титульного листа приводят слова «Авторы», «Авторский коллектив» и т.п.
		% Сведения об утверждении издания в качестве учебного пособия, учебника или официального издания, сведения о переводчике, составителе, ответственном (научном) редакторе, иллюстраторе и прочих лицах, принимавших участие в создании издания.
		% Сведения об издании, с которого сделан перевод, приводят путем воспроизведения на контртитуле или на обороте титульного листа сведений, помещенных на титульной странице оригинального издания.
		% Сведения о том, что издание является перепечаткой, и сведения об издании, с которого осуществлена перепечатка.
		
		\footnotesize\hspace{1.5em} В~качестве фона для˚обложки использована картина Ары Нориковича Хачатряна «Ветер» (год создания "--- 2005, размер 30х50 см., картон/масло). Право использовать картину при оформлении книги передано на основании лицензионного договора от 13.12.2018.
	\end{minipage}
	
	\vfill
	% Библиографическое описание
	\noindent
	\hspace{1em}\hphantom{\AuthorMark}
	{\bfseries\AuthorAbbrB}
	
	\noindent
	\AuthorMark\hspace{1em}
	\begin{minipage}[t]{\linewidth-1em-\widthof{\AuthorMark}}
		{\hspace{1.5em}\BibInfo\\[1ex]}
		{\hspace{1.5em}ISBN \ISBN\\}
		{\hspace{1.5em}Знак информационной продукции \large\textbf{\AgeLimit}}
	\end{minipage}
	
	\vfill
	\noindent
	\hspace{1em}\hphantom{\AuthorMark}
	\begin{minipage}[t]{\linewidth-1em-\widthof{\AuthorMark}}
		% Издательская аннотация
		\footnotesize\hspace{1.5em}	Что такое путь человека? Где он начинается, куда ведёт, как проходит, от˚чего зависит? На~эти и˚многие другие вопросы Михаил Павлович Шубич отвечает в своей книге.

\hspace{1.5em}Крупная фигура советской и˚российской науки. Кандидат экономических наук, профессор, академик и˚заслуженный землеустроитель Российской Федерации. Человек несгибаемой воли, принципиальный, требовательный, преданный профессии и˚делу всей своей жизни. Он~стал не˚только тактичным наставником, мудрым учителем, добрым воспитателем для многих молодых людей, но и˚верным другом.

\hspace{1.5em}Послевоенное время, тяжелый физический труд с˚малых лет, голод, лишения, неутомимая жажда знаний, желание «стать человеком» и˚стремление к˚свету сделали Михаила Павловича тем, кем мы знаем его сегодня. Благодаря этой книге у˚читателя появилась возможность познакомиться с этим Человеком!
	\end{minipage}
		
	\noindent	
	\begin{flushright}
		\bibInd{\UDK}{\BBK}{}	
	\end{flushright}
	
	% ISBN и авторские права
	\vfill
	\noindent
	\parbox[b]{.5\linewidth}{\small\bfseries ISBN \ISBN}
	\hfill
	\parbox[b]{\widthof{\small©\,Иллюстрации. Права}}
	{\raggedright\small
	 ©\,\AuthorAbbrA, \YearPub\\
	 ©\,Иллюстрации. Права принадлежат авторам	
	}
	
\end{titlepage}