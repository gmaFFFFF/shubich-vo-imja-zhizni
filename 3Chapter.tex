\chapter{Необходимость поработать}

Инженерно\-/технический состав сейсмической партии включал: топографов, бурильщиков, сейсмографов, взрывников. В~состав партии входили также шофёры, обслуживающий персонал, рабочие. Многие инженерно-технические работники (ИТР) были со˚своими жёнами. Возглавляли сейсмическую партию начальник и˚его заместитель. Негласно заместитель должен был ещё осуществлять контроль за˚деятельностью начальника партии. Задачей сейсмической партии был поиск нефти и˚других полезных ископаемых на˚территории Белоруссии. Топографы прокладывали профили\footnote{Профиль "--- это \Todo[Определение]{Прошу дать определение слову профиль}}  и˚отмечали пикеты (точки) для бурения скважин. Бурильщики бурили скважины, взрывники закладывали в˚шурфы заряды и˚осуществляли взрывы. Сейсмографы по˚колебаниям земли при взрывах определяли залежи полезных ископаемых.

Работа партий была сезонная. Весной, летом и˚осенью они работали в˚окрестностях Белоруссии, в˚начале зимы сдавали отчёт, оборудование на˚базу. Зимой ремонтировали оборудование и˚технику. Затем снова начинали сезон.

Базировалась сейсмическая партия №~1/54 в с.~Богутичи, №~3/55 "--- в с.~Михайловке под г.~Мозырь. Заместителем начальника партии №~1/54 был Машук, №~3/55 "--- полковник особого отдела В.\,И.~Доцевич.

На˚балансе партии была техника, автомашины, трактор С\=/100, запчасти, оборудование, спецодежда, горюче\-/смазочные материалы (ГСМ) и~т.~д. Взрывной склад, охраняемый женщинами с˚ружьём, размещался вне населенного пункта. ГСМ хранились непосредственно при стоянке автомашин. На˚товарном складе хранилось сейсмооборудование, буровое оборудование, запчасти для автомашин, спецодежда, палатки, спальные принадлежности, продукты и˚продовольствие\todo[Ред.]{В чём разница между продукатами и продовольствием?}.

\begin{wrapfigure}{O}{.4\textwidth}
\centering
\includegraphics[width=.35\textwidth]{traktorS100}
\caption[Трактор Сталинец\=/100 (С\=/100)]{Трактор Сталинец\=/100 (С\=/100)\footnotemark}
\label{fig:traktorS100}
\end{wrapfigure}
\footnotetext{Источник заимствования "--- \url{http://techstory.ru/trr_foto/s80_100.htm}.}

Зачислили Мишу в˚1954~году на˚должность диспетчера и˚кладовщика. Причём˚сразу поручили вникнуть и˚составить отчёт. Что˚успешно была выполнено. Руководству это понравилось.

Работа Миши заключалась в˚том, чтобы явиться раньше всех, выписать и˚выдать путевые листы, заправить машины, вести учёт ГСМ, а˚дальше ехать в˚районный или областной центр за˚продуктами, производить продажу\todo[Ред.]{Кому продавать?} продуктов и˚отпускать необходимые материалы, начислять зарплату шофёрам. Шофёры работали сдельно. Зарплата начислялась от˚пробега автомобиля, километража. На˚подотчёте Миши были сотни тысяч рублей в˚переводе на˚деньги. Самым ответственным считалось составление и˚сдача отчёта в˚экспедицию.

Работа в˚сейсмических партиях №№~1/54 и 3/55 позволила глубже познать людей, закалила духовно. Остались очень хорошие воспоминания о˚заместителе начальника партии, полковнике бывшего особого отдела В.\,И.~Доцевиче. Он хорошо разбирался в˚людях, правильно оценивая их, был объективным, честным. Хотя˚он˚был уже опытным, взрослым человеком, познавшим жизнь, а˚Миша несмышлёным парнишкой, они подружились. При˚поездках зачастую сопровождали друг друга.

Случались в˚работе и˚курьезные случаи, даже стоившие жизни.\todo[Ред.]{Вроде никто не умер? М.б. ...случаи, которые могли даже стоить жизни.} 

Иногда бурильщики и˚сейсмологи уезжали от˚базы на˚300\,км и˚оставались там на˚несколько дней. Поэтому решено было отвезти им˚горючее и˚продукты питания. Бочки с˚бензином и˚маслом были привязаны сзади кузова. По˚приезду на˚место Миша, поставив на˚кабину весы, отпускал продукты, а˚шофёры заправляли свои машины бензином прямо «с колёс». Вдруг раздался резкий крик: «Горишь!» На˚Мише загорелась телогрейка, он˚выскочил из˚машины, горящую телогрейку с˚него сорвали и˚быстро потушили огонь. Рядом˚стояла машина со˚взрывчаткой. Шофёр растерялся и˚не~мог завести машину. Сотрудники кое-как вытолкнули эту машину, а˚бурмастер В.~Чернышев каким\=/то образом сбросил горящую бочку с˚бензином с˚кузова автомашины, получив, естественно, ожоги. Горящую машину удалось откатить от˚бочки, а˚огонь затушить. Кузов автомашины обгорел. Кто-то заправил бензином керогаз\footnote{Керогаз "--- это нагревательный прибор, работающий на˚керосине. Предназначен для˚приготовления пищи, кипячения воды и˚др.}и˚загорелась палатка\todo[Ред.]{Предложение не связано с пердыдущим}. Используя огнетушители, огонь удалось локализовать. На˚обгоревшей машине, фактически без кузова, возвратился Миша на˚базу в˚Богутичи. Утром следующего дня телеграммой вызывали в˚прокуратуру Мозыря. Устроили допрос: сознательно ли˚был поджог? кто это сделал? почему? и~т.~д. Так˚осталось неясным, почему случился пожар. Мишу отпустили. Можете представить сами, что он пережил за˚это время.

Однажды утром по˚рации буровой мастер передал, что один бурстанок не~явился на˚профиль и˚попросил, чтобы Миша на˚закреплённой за˚ним машине проехал по˚трассе. На˚мосту через реку вдруг увидели сидящего шофёра. На˚вопрос: «Где бурстанок?» Он˚пальцем показал в воду и˚проронил: «Там». Шофёр был пьян. Оказывается, проезжая по˚одной деревне, он˚купил самогон и˚напился. На˚мосту ехала встречная машина, он˚не~уступил и˚поехал на˚встречу. В~результате бурстанок оказался под водой. Сняли с˚профиля трактор и˚все автомашины. С~их помощью бурстанок удалось вытащить из˚реки. Машина не~была повреждена, даже завелась, но˚часть оборудования была потеряна\Todo[вар.]{утонула}. Стоимость его была возмещена из˚зарплаты шофёра, который вдобавок ещё получил выговор. 

В~сейсмических партиях работали смелые, хорошо знающие своё дело люди. Жены ИТР бездельничали, поэтому иногда требовали, чтобы для них устраивали увеселительные мероприятия.

Решили как-то в˚воскресенье отвезти их на˚реку. Сделали сидение и˚разместили в˚кузове грузовой автомашины. В~углу кузова стоял огнетушитель. Во˚время движения автомашины он˚почему\=/то сработал. Одна женщина схватила его и по˚ошибке обдала струей из˚огнетушителя сидящих женщин. В~результате у˚некоторых на˚одежде оказались дырки, кислота сработала, но обошлось без ожогов. Пришлось срочно возвращаться на˚базу. Часто впоследствии, вспоминая подобные поездки, строго соблюдали требования\Todo[Определение]{Чего? Прошу разъяснить последнее предложение}.

Однажды ночью подняли Мишу по˚тревоге, приехала за˚ним машина. Оказалось, ударил около деревни Добрынь фонтан нефти. Фонтан был сильный. Утихомирили его лишь прилетевшие из г.~Баку специалисты. Но˚больше он˚не~проявлял себя, а˚нефть исчезла. 

Поскольку˚шофёрам начисляли зарплату от˚километража, то˚они подкручивали одометры\footnote
{Одометр "--- это прибор для измерения количества оборотов колеса. При помощи него измеряется пройденный путь.}. В~результате оказалась\todo[вар.]{образовалась, получилась} экономия бензина. Завоз бензина производился строго по˚графику. Цистерны были заполнены, сливать некуда. Пришлось дать немного бензина колхозу, и˚кто-то доложил в˚прокуратуру и˚опять вызвали в г.~Мозырь. Мише пришлось сослаться на˚постановление ЦK партии Белоруссии, согласно которому организации, работавшие на˚территории землепользования колхоза, должны помогать ему техникой и˚людьми. Поскольку˚это мы˚не~можем сделать, то˚оказали помощь бензином. В~прокуратуре заявили, что сделано было правильно.

Как\=/то в˚субботу Миша зашёл на˚квартиру проживания начальника партии в с.~Богутичи. Дом находился рядом с˚базой. В~квартире была страшная пьянка. Можно выделить четыре способа опьянения: первый "--- увеселение, второй "--- возбуждение, третий "--- оглупление и˚четвёртый "--- оскотнение. У˚всех уже была четвертая степень опьянения. Мише начальник партии налил стакан водки и˚предложил выпить. На˚ответ Миши, что вообще не~употребляет спиртного и не~может выпить, начальник партии поставил ружье (он˚имел своё ружьё) и˚приказал: «Пей!» Выбрав момент, когда он˚чуть отвернулся, Миша прыгнул в˚окно (окно было открыто). Вдогонку раздался выстрел. Миша уже был на˚земле, и˚пуля пролетела над головой. Из-за оскотнения человека Миша чуть не~лишился жизни в˚молодом возрасте.

Второй сезон база была сосредоточена в с.~Михалки. Поселился Миша в˚доме одной старухи лет 60. Это была высокая, сухая, очень жадная старушка. Даже её˚родной брат, работавший в˚сейсмической партии механиком, предупредил, что с˚ней не~уживался ещё ни˚один человек, всегда уходили. Этим он в˚чём-то заинтриговал Мишу. За˚квартиру ей˚исправно платил. Но˚она заявила: «Почему не~привёз  дров?» Привёз дрова. Далее она потребовала привезти сена и˚принести сахара. И~это было сделано. «Почему˚для лампы не~принёс бензина?» Миша знал, что бензин следовало заправлять в˚лампу, смешав с˚керосином. Он специально залил бензин в˚лампу, а˚валенки\todo[ред.]{про валенки не вяжется, к чему про них?} в˚бензине положили на˚печку. Лампа вспыхнула огнём, в˚дому стоял страшный запах. После˚всего этого старушка ничего не~попросила. При˚сдаче отчёта приехала в г.~Мозырь и˚сказала: «Миша поживи у˚меня, даже денег за˚квартиру не~возьму». Так была усмирена жадность старухи.

В~то˚время с˚продуктами в˚деревне было плохо,поэтому по˚возможности приходилось помогать жителям. За˚это часто доставалось Мише от˚жён ИТР партии. Как-то на˚склад к˚ Мише пришёл старик в˚совершенно рваной телогрейке. Старик ему понравился, был какой\=/то благородный. Миша ему предложил снять свою телогрейку, взамен отдал совершенно новую. Списать её˚Мише не~составляло труда. Старик ушёл счастливый, весь сиял. Через˚2~дня он˚пришёл опять и˚пригласил в˚гости. Обижать старика не~хотелось. Поэтому Миша сказал Есипу, который работал шофёром, что в˚воскресенье пойдём в˚гости и˚его задачей будет выпить водки. Миша в˚то время вообще не~употреблял ничего спиртного. Есип согласился, даже обрадовался такому предложению. 

Старик нарезал хлеба, сала и˚поставил бутылку самогона. Затем налил в˚гранёные стаканы водки. Есип выпил полстакана и˚сразу завалился. Пришлось его на˚плечах нести к˚фельдшеру по˚огородам. Она тут же˚вызвала скорую, которая увезла его в˚больницу в˚Мозырь. Было сделано заключение, если бы˚Есип выпил чуть больше, то˚смерть наступила тут же. Оказывается, чтобы водка была крепче, старик настоял её˚на курином помёте.

Бывает, что из˚добрых побуждений, не~зная к˚чему это приведёт, можно сделать большое зло, даже отправить человека на˚тот свет. 

Закончился очередной сезон работы партии. 7~октября 1955~года Миша закончил составлять отчёт и в˚этот день получил повестку. Через˚день следовало явиться в˚военкомат (забирали в˚армию). Отчёт представил В.\,И.~Доцевичу, а˚сам уехал домой. Устроили небольшой вечер\todo[ред.]{Предложение ни к месту}. Во˚время службы в г.~Киеве Миша получил от В.\,И.~Доцевича письмо следующего содержания: «Отчёт приняли. Всё˚в порядке. Лишь˚в˚шахматах недостаёт одной фигуры. Мы с˚тобой после армии ещё должны поработать». Про фигуру в шахматах была шутка. Сообщил, что его определили начальником отдела кадров на˚Мозырьский перерабатывающий завод.

В~армии Миша часто вспоминал о˚людях, с˚которыми работал. Удивлялся профессионализму шофёра Миши Кухновцева. Он всегда работал с˚топографами. На˚карте ему покажут место в˚лесу, куда он˚должен их˚доставить. Он долго смотрит на˚карту, затем садится за˚руль и˚поехал. В~итоге привозит по˚лесу на˚указанное место. Как˚ему это удавалось, никто не~мог понять, а˚он никогда не~объяснял.