\section{Стихотворения, посвященные М.\,П.~Шубичу}

\poemtitle{Всего лишь пять…!?}									% Название стихотворения
\settowidth{\versewidth}{Дней всего лишь, но \textit{<неразборчиво>}}		% Для определения примерной ширины строки
\begin{verse}[\versewidth]
Тебе коллега, тёзка           \\
Стих я˚посвящаю,               \\
Пусть˚чупр\'{и}на\footnote{Чупр\'{и}на, чуб "--- прядь волос, спадающая на˚лоб.} твоя в˚блёсках, \\
Но˚душа, душа… без края \\!

\vin Знаю я˚уже давно               \\						% \vin - горизонтальный отступ
\vin Норов твой мужицкий.            \\
\vin Всё, что в˚жизни нам дано,      \\
\vin Как˚ни странно, вам нам близко. \\!

В ГУЗе пашешь ты˚давно,         \\
Четверть века уж˚минуло.        \\
Всяко было, но˚гумн\'{о}\footnote{Гумн\'{о} "--- помещение, сарай для сжатого хлеба.}           \\
Настежь пред тобой раскрылось. \\!

\vin Землемерская стезя             \\
\vin Словно˚омут засосала           \\
\vin Так, что, Миша, ты не˚зря        \\
\vin Бульбу\footnote{Бульба (белорусское) "--- картошка.} ел, а то и˚сало \\!

Нынче ты˚профессор, чай,        \\
Впереди˚ещё «коньки»             \\
И не˚вздумай невзначай         \\
Снять узду, уйдя с˚пути  				\\!

\vin Нынче люд и не˚поймёт          \\
\vin Даже˚если упираться…         \\
\vin Бросишь камень в˚огород        \\
\vin Будешь жизнь всю разбираться… \\!

Двигайся вперед на …ять\footnote{Ять (просторечное, шутливое) "--- как следует, очень хорошо.}      \\
Путь твой жизненный моложе     \\
Старше я˚тебя на˚пять          \\
Дней всего лишь, но˚дороже \\!


\vin Наше общее начало              \\
\vin Кредо "--- суть всей нашей жизни  \\
\vin Нам не~нужно опах\'{а}ло\footnote{Опах\'{а}ло "--- веер, преимущественно больших размеров и˚причудливой формы.},          \\
\vin Св\'{а}ры \footnote{Св\'{а}ра "--- шумная перебранка, ссора.}, св\'{о}ры\footnote{Св\'{о}ра (переносное значение) "--- сопровождающие кого\=/нибудь люди, приспешники, свита}, то˚бишь трижды \\!

От˚души я˚поздравляю           \\
Счастья, здоровья желаю        \\
Будь, Мишань, ты на˚коне       \\
Все воздастся… и… втройне…		\\!
\end{verse}
\attrib{Михаил Теодорович Колчун}


\clearpage							% Напечатать все плавающие объекты и начать новую страницу
\poemtitle{М.\,П.~Шубичу, в день его…}									% Название стихотворения
\settowidth{\versewidth}{И вместе с˚дружным коллективом ГУЗа страстно}		% Для определения примерной ширины строки
\begin{verse}[\versewidth]
Сегодня вызывают беспокойство\\*
В стране хреновые дела с˚землеустройством.\\*
Они на˚всех ложатся тяжким грузом…\\* 
Но˚как же˚вовремя Вы появились в˚ГУЗе!\\!

\vin Вы, видя на˚родной земле огрехи,\\*
\vin Свои на˚каждом пробивали вехи \\*
\vin И вместе с˚дружным коллективом ГУЗа страстно \\*
\vin Повсюду ратовали за˚кадастры.\\!

Известно, как Вы за˚землеустройство бились \\*
И до˚чего в˚итоге докатились:\\*
Везде, где только прикасались Ваши руки,\\*
Есть перлы землеустроительной науки.\\!

\vin Как˚вы старались навести порядки \\*
\vin В полях, где положенье было гадким.\\*
\vin Пусть˚ректор был всегда и˚всюду первым,\\*
\vin Вы щекотали землеустроительные нервы.\\!

Желаем, чтобы имя ШУБИЧ громко \\*
Вселяло уважение потомкам \\*
И чтобы Вашу грудь украсил орден \\*
Или˚значок, как у˚Косинского Володи.\\!

\vin И чтобы жизнь осталась полнокровной,\\*
\vin Успехов Вам и˚крепкого здоровья!\\*
\vin Простите, может быть и не˚красиво,\\*
\vin Вас поздравляет коллектив из˚РИО. \\!

\end{verse}
\attrib{Коллектив Редакционно-издательского отдела ГУЗ (РИО)}
\attrib{18.10.2006}


\clearpage							% Напечатать все плавающие объекты и начать новую страницу
\poemtitle[«Михаил "--- ты не Ботвинник»]{* * *}									% Название стихотворения
\settowidth{\versewidth}{Всё они решают смело}		% Для определения примерной ширины строки
\begin{verse}[\versewidth]
Михаил "--- ты не Ботв\'{и}нник\footnote{Михаил Моисеевич Ботв\'{и}нник "--- советский шахматист, 6-й в истории шахмат и 1-й советский чемпион мира.}	\\
Ты ведь Шубич, дорогой,	\\
Почему же пешки ходят	\\
За широкую твоей спиной.	\\!

\vin Всё они решают смело	\\
\vin Кто кому и как когда	\\
\vin Часть идёт от Вас налево	\\
\vin За зачётом "--- кто куда	\\!

Как суров ты, неподкупен	\\
С III-им курсом земфак\'{а}	\\
Но куда исчезла строгость	\\
Пред заочником тогда?	\\!

\vin Мы тебя благославляем	\\
\vin На дальнейшие года	\\
\vin Быть нежнее Вам желаем	\\
\vin Ваши искренние друзья	\\!
\end{verse}
\attrib{Доцент Валентина Ивановна Король}
\attrib{01.10.1996}