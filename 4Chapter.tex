\chapter{Служба в˚армии}
В~военкомате городского поселения Ельск собрали всех, разделили на˚группы, объяснили о˚поведении и˚службе и˚отправили на˚вокзал. Погрузили в˚товарный вагон и˚поехали. Куда˚едем не~сказали. Ехали двое суток, долго стояли в˚разных тупиках. Здорово изголодались, прихваченные с˚дома запасы кончились. Да˚и купить продуктов не~представлялось возможным. Наконец оказались в˚Киеве. Привели сразу в˚баню, постригли, помыли и˚выдали военную одежду и˚направили в˚школу сержантского состава. Оказалось, впоследствии, что служить придётся в˚дивизии резерва Верховного Главного Командования. В~школе сержантского состава обучались год. Определили Мишу в˚группу обучаться на˚вычислителя.

\begin{wrapfigure}{O}{.4\textwidth}
\centering
\includegraphics[width=.35\textwidth]{bm-24}
\caption[Боевая машина БМ\=/24 в˚боевом положении.]{Боевая машина БМ\=/24 в˚боевом положении\footnotemark.}
\label{fig:bm-24}
\end{wrapfigure}
\footnotetext{Источник заимствования \--- Боевая машина БМ\=/24 (индекс 8У31). Руководство службы. 1958~г.}

Порядки были очень строгие. Подъём и˚отбой должны были осуществить за˚несколько секунд, засекали по˚последнему. Если˚не~укладывались, то˚это повторялось долго, пока не~вкладывались в˚норматив. Много отводилось время строевой и˚физической подготовке, а˚также занятиям в˚классах. Очень часто устраивались кроссы на˚3,0 и˚5,0\,км с˚полной выкладкой. На˚соревнованиях Мише часто приходилось бегать за˚других (подставным). В~увольнение не~отпускали. В~баню водили строем. Командиром взвода был сержант эстонец Кивите, который относился к˚ребятам предвзято, старался выслужиться. На˚кроссах также время определялось по˚последнему прибегшему. Если˚не~укладывались в˚нормативы, то˚это повторялось чуть ли не~ежедневно. 

Через˚год окончили школу, прошли стажировку. Миша стажировку проходил нам БМ\=/24 (боевая машина 24, бывшие Катюши). Затем распределили по˚частям. 

Мишу определили в˚часть, в˚которой командиром был генерал\-/майор Филипп Филиппович Капуста. Во˚время войны он˚командовал партизанским отрядом. В~книге <<Война в˚тылу врага>>, Героя Советского Союза полковника Григория Матвеевича Линькова описывается соединение отряда партизан Капусты с˚отрядом генерал\-/майора Александра Николаевича Сабурова. Зачислили Мишу сначала вычислителем батареи, которой командовал капитан Голованов, затем вычислителем дивизиона и˚наконец вычислителем части. Задачей вычислителя было во˚время боевых стрельб расшифровывать метеоогневой, определить заряд, доворот, дальность полёта снаряда и˚передать это радисту, который затем передаст на˚батарею. Необходимо было вложиться в˚какие\=/то секунды. От˚вычислителя зависит во˚многом будут ли˚поражены цели и˚какую оценку выставят за˚стрельбы. Замены вычислителю не~было, он˚был в˚единственном лице. Будучи вычислителем части, часто приходилось быть дежурным по˚штабу части. 

\begin{figure}[h]
	\centering
	
	\begin{minipage}[h]{0.3\textwidth}			
		\includegraphics[width=\textwidth]{kapusta}
		\caption{Генерал\-/майор Ф.\,Ф.~Капуста (1907--1973).}
		\label{fig:kapusta}		
	\end{minipage}	
	\hfill
	\begin{minipage}[h]{0.3\textwidth}			
		\includegraphics[width=\textwidth]{saburov}
		\caption{Генерал\-/майор А.\,Н.~Сабуров (1908—1974).}
		\label{fig:saburov}
	\end{minipage}
	\hfill
	\begin{minipage}[h]{0.3\textwidth}		
		\includegraphics[width=\textwidth]{linkov}
		\caption{Портрет Бати (Командира партизанского соединения Белоруссии Героя СССР Г.\,М.~Линькова (1899\--1961)). Художник: М.\,И.~Шац. 1947.}
		\label{fig:linkov}
	\end{minipage}
\end{figure}

Поэтому с˚командиром части приходилось сталкиваться часто. Генерал\-/майор Капуста по˚натуре был спокойным уравновешенным человеком, страшным матершинником. При˚самоволке он˚солдат не~отправлял на˚гауптвахту, а˚вызывал в˚кабинет (кабинет был большой вытянутой формы), снимал сапог и˚гонял вокруг столов пока не~стукнет по˚спине, а˚затем отправит со˚словами <<иди служи>>. Солдаты не~обижались на˚него.

Во˚время учений в˚районе сосредоточения в˚лесу показывали кино. Перед˚учением выступал Капуста с˚напутствием: <<Кино крутив, салом кормив, так воюйте>>. Однажды услышал такую реплику. Командир дивизии генерал\-/майор начал ругать Капусту, в˚ответ послышал: <<Ты генерал и я˚генерал, иди ты…>>. Капуста как-то расчувствовался и˚рассказал, что во˚время войны его вызвали в г.~Москву и˚Калинин спросил: <<Что˚тебе дать генерала или героя?>>. В~ответ Капуста ответил: <<Героем я˚ещё успею быть, дайте генерала>>. 

В~функции вычислителя начальник штаба части полковник Корабличенко вменил в˚обязанность составлять расписания занятий, даже разрешил подписывать при его отсутствии. Утверждал расписание заместитель командира части по˚строевой подготовке полковник Грудинин. Однажды в˚воскресенье принёс на˚утверждение расписание. Грудинин не~стал подписывать. Пошёл проверил, всё правильно. Снова не~утвердил. Оказывается, не~поставил <<гвардии полковник>>. Исправил ошибку, Грудинин утвердил. При˚этом спросил, кто его подписал. Он уже успел проверить, что Корабличенко отсутствовал. За˚обман Мише объявил гауптвахту, но не~посадили. В~связи с˚предстоящим учением начальник штаба части отменил наказание.

В~батарее Голованова служил Дженик Оганесович Петросян. Миша его хорошо знал, часто беседовал. Дженик не~хотел служить, выдавал себя за˚психически ненормального. На˚посту, при смене караула, засылал патрон в˚патронник и˚грозился расстрелять разводящего. Попав в˚госпиталь, он˚сбежал (дезертировал). Вызвал командир части Мишу и˚поставил задачу, чтобы он˚его словил. Время не~ограничено и˚всё, что необходимо будет предоставлено. Начинайте действовать с˚сегодняшнего дня. Две недели Миша находился в˚поисках Петросяна. Продукты питания ему привозили на˚указанные точки. Выдали гражданскую одежду. Однажды удалось по˚различным связям выяснить, что Дженик связан с˚Люсей Прозумент, работавшей в˚психдиспансере и˚проживающей на˚его территории. Проживали Люся с˚мамой в˚однокомнатной квартире. Психдиспансер находился на˚окраине г.~Киева. Территория вокруг была изрезана оврагами, заросшими кустарником, деревьями. Удалось хорошо познакомиться с˚соседями Люси. 

Ходить по˚территории диспансера было жутковато, так как иногда встречались проживающие в˚нём. Соседи мне сказали, что Дженик ночевал в˚квартире Люси, а˚когда появился там я, то˚несколько ночей Люся не~ночевала дома. Пригласил Люсю и˚целый день бродили с˚ней по˚территории. Надеясь, что она укажет местонахождение Дженика. Но˚этого не~произошло, тогда отправил её в˚часть. В~части она также ничего не~признала. Оказывается, они несколько ночей ночевали под корнями вывернутого дерева. Однажды в˚овраге, под деревом я˚обнаружил следы солдатских сапог. Измерил, это оказался 41~номер. Предположение было, что это следы Петросяна. На˚суде Петросян сказал, что, когда ты˚измерял след, он˚находился на˚дереве с˚ножом, хотел зарезать, но˚появилась какая\=/то жалость. Сожалею, заявил Петросян, что не~сделал подобное. 

Устраивал при помощи присланных солдат из˚части, дежурство у˚дома. Но˚это также ничего не~дало. Видимо, устал я˚окончательно. Однажды решил съездить в˚часть и˚хоть нормально поесть. На˚трамвайной остановке ко˚мне подходит молодой парень с˚листком бумаги и˚спрашивает знаю ли я˚часть 41401. Попросил у˚него листок, развернул, а˚там написано Дженис Оганесович Петросян тот, кого я˚ищу. Меня охватила внутренняя дрожь, которая, видимо, передалось спрашивающему. Попытался спокойным тоном ему ответить, что часть эту знаю, нахожусь в˚увольнении и˚еду в˚дивизию. Смогу его сопровождать. Задача стояла заманить его в˚часть. Но˚собеседник заявил, что он˚опаздывает на˚работу и˚поспешил уйти. Впоследствии на˚суде выявилось, что этот парень следил за˚мной и˚должен был убрать меня. 

\begin{wrapfigure}{O}{.4\textwidth}
\centering
\includegraphics[width=.35\textwidth]{kievVokzal}
\caption{Паровоз ТЭ\=/7397, ст. Киев-Пасс., Киев. Автор: ЦГКА Украины, 28.05.1955.}
\label{fig:kievVokzal}
\end{wrapfigure}

Однажды утром при беседе с˚соседями Люси выявилось, что видели у˚Люси Петросяна в˚гражданской форме. Сразу проскользнула мысль, что сегодня он˚будет уезжать. Сразу же˚поехал в˚часть, переоделся в˚военную форму и˚отправился на˚Киевский вокзал. Предварительно изучил все входы и˚выходы, расписание поездов на˚Ереван и˚Баку. Предупредил военного патруля, что выполняю задание, и˚чтобы меня не~задержали, если не~отдам честь. 

По˚расписанию должен был отправляться поезд на˚Баку. При˚выходе из˚зала ожидания я˚встал за˚дверью и˚высматривал выходящих. Объявили, что до˚отхода поезда оставалось 5\,минут, провожающим покинуть вагоны. Вдруг вижу из˚зала ожидания идёт Петросян. Поравнявшись со˚мной, схватил его. Он начал вырываться и˚кричать, что военный напал на˚него. Начали подходить гражданские, поэтому пришлось крикнуть военному патрулю, помоги мне. Они схватили его и на˚машине нас доставили к˚коменданту города. 

Был первый этаж. Комендант сидел за˚столом, рядом было открыто окно. Ввели Петросяна, и он˚сразу бросился в˚окно. За˚окном был часовой и˚схватил его. Комендант заявил: <<браток, у˚меня ещё ни˚один не~убегал>>. 

На˚машине меня отвезли в˚часть. Дежурным по˚части был капитан Голованов. Дал команду поварам в˚столовой накормить меня до˚отвала. Пригласил меня назавтра в˚кино. Сидели с˚одной стороны меня его жена, а с˚другой \--- командир батареи. Но˚я, видимо, за˚всё время так устал и˚стрессовое состояние, что уснул не~видел ничего. Очнулся, когда написано конец. Жена спросила, понравилось ли˚кино. Она прекрасно видела моё состояние. 

Люся уволилась с˚работы, выписалась из г.~Киева. Видимо собиралась уезжать, но у˚Петросяна оказался один билет. Он её˚обманул. Состоялся суд. Петросян и˚целая группа людей, связанных с˚ним, осуждена. По˚Люсе вынесено частное определение, не~прописывать в г.~Киеве. Такая участь постигла этих молодых людей.

Командир части Мише объявил отпуск, но он им не~воспользовался. Начинались крупные учения. Учение затянулись.

Однажды во˚время боевых стрельб снаряд разорвался в˚50\,м от˚нашего наблюдательного пункта (НП). Осколками срезало ножку буссоли, обвалило бруствер лежащего моего ПУМО (прибор управления миномётным огнём). К~счастью никого не~задело. Оказывается, батарейный вычислитель Шахов неверно рассчитал данные, за˚что понёс наказание. А~Мише вменили в˚обязанности ещё проверять расчёты батарейных вычислителей. 

Вспоминается ещё такой случай. Шла бесконечная стрельба (учения). Двое суток никто не~сомкнул глаз. На˚третьи сутки образовался какой\=/то перерыв. Оставили дежурных, а˚остальным решено было вздремнуть. Залегли в˚ряд. Автоматы подложили под себя, а˚стволы наружу. Вдруг машина проехала в˚конце наших голов по˚стволам автоматов. Видимо, не~суждено было нам умереть. Ведь˚машина могла проутюжить и˚нас. Шофёр нас в˚траве не~заметил.

\begin{wrapfigure}{O}{.4\textwidth}
\centering
\includegraphics[width=.35\textwidth]{svaku}
\caption[СВАКДКУ 1952~год.]{СВАКДКУ 1952~год\footnotemark.}
\label{fig:svaku}
\end{wrapfigure}
\footnotetext{Источник заимствования \--- \url{http://svaku.ru/forum/gallery/}}

В~начале службы в˚части, Грудинин \--- заместитель командира части по˚строевой подготовке \--- отобрал несколько человек для учёбы в˚Сумское высшее командное артиллерийское училище (СВАКДКУ). В~числе них оказался и˚Миша, который не~желал поступать. Но˚приказ пришлось выполнять. В~училище явились через неделю (задержались у˚друга в г.~Сумих). Миша успешно сдал 2~экзамена и˚видя, что остальные будут сданы и˚зачислен. Написал рапорт об˚отказе для поступления на˚имя начальника училища, генерала\-/лейтенанта. Вызов его не~задержался. Он так отчитывал, что пот из˚Миши стекал ручьём. Наконец рапорт был подписан и˚дано предписание в˚течение суток явиться в˚часть, в г.~Киев. Поезд ходил один раз в˚сутки, причём уже ушёл. Выполнить приказ, естественно, Миша не~смог. За˚что Грудинин пригрозил отправить в˚Полтавское пехотное училище, но к˚счастью это не~было выполнено. 

Часто дивизию и˚входящие в˚неё части, руководство подвергались проверкам вышестоящими организациями боевую готовность знания своих обязанностей. Подобной проверке однажды подверглась и˚часть генерала\-/майора Капусты. Часть ночью была поднята по˚тревоге. Летом находились в˚лагерях. Предстояло командиру части выдержать экзамен. Экзаменовал представитель из˚округа. На˚наблюдательном пункте за˚столиком разместились Миша с˚прибором управление миномётным огнём (ПУМО), с˚одной стороны командир части, с˚другой проверяющий. Командиру части предстояло расшифровать метеоогневой, определить заряд, доворот и˚дальность полёта снаряда и˚передать радисту для отправления на˚огневые позиции. Стреляли боевыми снарядами. Так˚должно продолжаться до˚тех пор, пока не~будет поражена цель. 

\begin{wrapfigure}{O}{.4\textwidth}
\centering
\includegraphics[width=.35\textwidth]{BNTU}
\caption{Белорусский национальный технический университет (БНТУ). Главный корпус. Автор: Gruszecki, 29.05.2010}
\label{fig:BNTU}
\end{wrapfigure}

Миша должен был проконтролировать правильность расчётов. Но˚негласная задача заключалась ещё в˚том, чтобы незаметно от˚проверяющего оказать помощь генералу. Уже первые расчёты командира части оказались не~совсем верными. Незаметно пришлось подсовывать правильное решение. Видимо, напряжение было такое, что Миша перестал что\=/либо видеть. Попросил прекратить стрельбу, вышел из НП, но˚зрение не~восстановилось. Обратно возвратился по˚стенке \textbf{не~пойму слово}. Стрельба больше не~возобновилась. Командиру части поставили положительную оценку. По˚возвращении в˚палатку, генерал приказал дневальному Мишу не~будить. Проспал ночь и˚ещё полдня. После˚сна поднялся подошёл к˚соску умывальника и в˚тумане увидел его. На˚радостях во˚всё горло раздался крик: <<вижу!>>. Пересказать состояние Миши в˚это время очень трудно, просто нужно ощущать на˚себе.

Однажды часть подняли по˚тревоге. Выстроили всех на˚плацу и˚каждого второго, третьего из˚шеренги поступила команда выйти. Такая участь пришлась солдату, стоящему рядом с˚Мишей. Всех их с˚техникой отправили в˚Чечню, многие из˚них обратно не~возвратились. Служили в то˚время три года. Лица, убегающие в˚самоволку, должны были дослужить дополнительно. Максимов, из˚взвода управления, набрал дополнительно полгода. Однажды на˚Новый год приказали его найти и˚привезти в˚часть. Задача стояла неблагодарная. Помогло знание место его нахождения. В~то˚время появился секретный приказ, если пришёл вызов из˚ВУЗа, то˚могли демобилизовать из˚армии досрочно. Такой вызов Миша получил из˚Минского политехнического института. Кончался последний год службы.

\begin{wrapfigure}{O}{.4\textwidth}
\centering
\includegraphics[width=.35\textwidth]{GAZM20Pobeda(Gwafton)}
\caption{ГАЗ-М\=/20 <<Победа>>. Classic Motor Show parking lot in Lahti, Finland. Автор: Gwafton, 08.05.2010}
\label{fig:GAZM20Pobeda(Gwafton)}
\end{wrapfigure}

Обратился к˚командиру части и˚показал вызов. Капуста сказал, чтобы сдал всё, будешь уволен сегодня. Быстро приступив к˚сдаче. Повстречал заместителя командира части по˚политической части. Он не~разрешил Мише сдачу и˚привёл опять командиру части с˚вопросом <<кто будет вычислять?>> Капуста ответил: <<мы с˚тобой>>. Приказал явиться к˚штабу части в˚14:00. При˚появлении в˚указанное время был крайне удивлён. Офицеры штаба были построены и˚перед ними Капуста изложил Мише душевное отцовское напутствие. Сфотографировали у˚знамени части. Капуста приказал сопровождать до˚пристани Кальное взводу управления под командованием командира взвода. Разрешил взять его машину Победу. Естественно, Миша был растроган таким вниманием и˚заботой к˚простому солдату. 

На˚полученные деньги удалось купить чемодан и˚осталось денег немного. Дорога до˚Минска была бесплатной. В~Минске поселили в˚общежитии. Денег на˚пропитание хватило лишь сдать 2~экзамена, экзамены сданы на˚четыре и˚пять. На˚столах хлеб был бесплатным. Заказывал лишь по˚четыре стакана чая. Питался два раза в˚сутки. Дальше денег не~осталось ни˚копейки. Решил уехать домой. По˚проспекту Сталина до˚вокзала добрался пешком. В~это время осуществлялась реконструкция проспекта. За˚мной шла цыганка и˚просила осчастливить её˚ручку. Злость овладела мной, готов растерзать её. От˚голодного солдата требуют денег.

На˚поезде ехал <<зайцем>>. Проводники понимали объяснение солдата, не~имеющего ничего в˚кармане. В~Калинковиче военный патруль снял с˚поезда. Собирались обратно отправить в˚часть. 

Родители очень обрадовались возвращению. Дома пробыл два дня. Решил уехать поступать. Денег хватило лишь на˚билет. От˚родителей Миша старался денег не~брать, а по˚возможности помогать им. Так˚как у˚них денег не~было.

Шёл 1958~год.