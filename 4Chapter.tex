\chapter{Служба в армии}
В~военкомате городского поселения Ельск собрали всех, разделили на˚группы, объяснили о˚поведении и˚службе и˚отправили на˚вокзал. Погрузили в˚товарный вагон и˚поехали. Куда˚повезут не~сказали. Ехали двое суток, долго стояли в˚разных тупиках. Здорово изголодались, прихваченные из˚дома запасы кончились. Да˚и купить продуктов не~представлялось возможным. Наконец оказались в˚Киеве. Привели сразу в˚баню, постригли, помыли, выдали военную одежду и˚направили в˚школу сержантского состава. Оказалось, впоследствии, что служить придётся в˚дивизии резерва Верховного Главного Командования. В~школе сержантского состава обучались год. Определили Мишу в˚группу обучаться на˚вычислителя.

\begin{wrapfigure}{O}{.4\textwidth}
\centering
\includegraphics[width=.35\textwidth]{bm-24}
\caption[Боевая машина БМ\=/24 в˚боевом положении]{Боевая машина БМ\=/24 в˚боевом положении\footnotemark}
\label{fig:bm-24}
\end{wrapfigure}
\footnotetext{Источник заимствования "--- Боевая машина БМ\=/24 (индекс 8У31). Руководство службы. 1958~г.}

Служили в˚то˚время в˚армии три~года. Порядки были очень строгие. Подъём и˚отбой должны были осуществить за˚несколько секунд, засекали по˚последнему. Если˚не~укладывались, то˚это повторялось долго, пока не~укладывались в˚норматив. Много времени отводилось строевой и˚физической подготовке, а˚также занятиям в˚классах. Очень часто устраивались кроссы на˚3,0 и˚5,0\,км с˚полной выкладкой. На˚соревнованиях Мише часто приходилось бегать за˚других (подставным). В~увольнение не~отпускали. В~баню водили строем. Командиром взвода был сержант эстонец Кивитс, который относился к˚ребятам предвзято, старался выслужиться. На˚кроссах также время определялось по˚последнему прибегшему. Если˚не~укладывались в˚нормативы, то˚это повторялось чуть ли не~ежедневно.


\begin{figure}[H]
\includegraphics[width=.85\textwidth,center]{private/serjantShubich}
\caption{Миша в˚армии. Источник заимствования "--- личный фотоархив М.\,П.~Шубича}
\label{fig:serjantShubich}
\end{figure}


Через˚год окончили школу, прошли стажировку. Миша стажировку проходил на БМ\=/24 (Боевая машина реактивной системы залпового огня М-24, современный аналог «Катюши»). Затем распределили по˚частям. 

Мишу определили в˚часть № 41401, в˚которой командиром был генерал\-/майор Филипп Филиппович Капуста. Во˚время войны он˚командовал партизанским отрядом. В~книге «Война в˚тылу врага» Героя Советского Союза полковника Григория Матвеевича Линькова описывается соединение отряда партизан Капусты с˚отрядом генерал\-/майора Александра Николаевича Сабурова. Зачислили Мишу сначала вычислителем батареи, которой командовал капитан Голованов, затем вычислителем дивизиона и˚наконец вычислителем части. Задачей вычислителя было во˚время боевых стрельб расшифровывать метеоогневой\todo[Ред.]{Дать определение слову метеоогневой}, а именно, определить заряд, доворот, дальность полёта снаряда и˚передать это радисту, который затем передаст\todo[вар.]{данные} на˚батарею. Необходимо было уложиться в˚какие\=/то секунды. От˚вычислителя зависит во˚многом будут˚ли поражены цели и˚какую оценку выставят за˚стрельбы\todo[вар.]{От˚вычислителя многое зависит: будут˚ли поражены цели и˚какую в˚результате оценку выставят за˚стрельбы}. Замены вычислителю не~было, он˚был в˚единственном лице. Будучи вычислителем части, Мише часто приходилось быть дежурным по˚штабу части. 

\begin{figure}[h]
	\hfil	
	\begin{minipage}[b]{0.32\linewidth}			
		\center{\includegraphics[width=.9\linewidth]{kapusta}}
	\end{minipage}	
	\hfil	
	\begin{minipage}[b]{0.32\linewidth}	
		\center{\includegraphics[width=.9\linewidth]{saburov}}
	\end{minipage}	
	\hfil	
	\begin{minipage}[b]{0.32\linewidth}		
		\center{\includegraphics[width=.9\linewidth]{linkov}}
	\end{minipage}	
	\hfil
	\\[1ex]	
	\begin{minipage}[b]{.96\linewidth}
		\begin{tabular}{p{.32\linewidth}p{.32\linewidth}p{.32\linewidth}}
			\centering а) & \centering б) & \centering в) 
		\end{tabular}
	\end{minipage}	
	\captionsetup{skip=1ex}							% Указание размера отступа от подписи рисунка
	\caption{Командиры партизанских соединений: 
	а)~Генерал\-/майор Ф.\,Ф.~Капуста (1907–1973), 
	б)~Генерал\-/майор А.\,Н.~Сабуров (1908–1974), 
	в)~Герой СССР полковник Г.\,М.~Линьков (Батя, 1899–1961). Портрет Бати, художник: М.\,И.~Шац. 1947}
\end{figure}

Поэтому с˚командиром части приходилось сталкиваться часто. Генерал\-/майор Капуста по˚натуре был спокойным, уравновешенным человеком, страшным матерщинником. При˚самоволке\todo[вар.]{При˚выявлении случаев самовольного отлучения из части} он˚солдат не~отправлял на˚гауптвахту, а˚вызывал в˚кабинет (кабинет был большой вытянутой формы), снимал сапог и˚гонял вокруг столов, пока не~стукнет по˚спине, а˚затем отправлял со˚словами: «Иди, служи!» Солдаты не~обижались на˚него.

Иногда устраивались соревнования между взводами, отделениями. На˚одном подобном соревновании между отделениями соревнующиеся должны были пробежать \Todo[Текст]{сколько километров?} км, затем преодолеть полосу препятствия, проползти под натянутой проволокой, пробежать по˚бревну, спрыгнуть и˚через 50\,м финишировать. Миша бежал в˚отделении последним. Поэтому перед ним была поставлена задача "--- победить. С~задачей он˚справился, опережал соперника. Но˚когда прыгнул с˚бревна, упал и не~смог встать. Подвернул ногу. Отправили в˚лазарет. Заведующим лазарета был сын командира дивизии. На˚второй день он˚заявил Мише, что неделю будет отсутствовать,\todo[вар.]{т.~к.} покупает мотоцикл. Видимо, сыну командира дивизии все «позволено». Мише поручено «принимать» больных и˚оказывать им «лечение». По˚возвращении он˚заявил, что Миша остаётся его помощником. Предстояло очередное учение части. Начальник штаба части предложил Мише убежать из˚лазарета, иначе они не~могут дать команду на˚возвращение в˚часть. Что˚он и˚сделал. Так˚закончилась его «работа» в˚лазарете.

Во˚время учений в˚районе сосредоточения в˚лесу показывали кино. Перед˚учением выступал Капуста с˚напутствием: «Кино крутив, салом кормив, так воюйте». Однажды услышал такую реплику. Командир дивизии генерал\-/майор начал ругать Капусту, в˚ответ услышал: «Ты генерал и я˚генерал, иди ты…». Капуста как-то расчувствовался и˚рассказал, что во˚время войны его вызвали в г.~Москву и˚Калинин спросил: «Что˚тебе дать генерала или героя?» В~ответ Капуста сказал: «Героем я˚ещё успею быть, дайте генерала». 

В~функции вычислителя начальник штаба части полковник Корабличенко вменил в˚обязанность составлять расписания занятий, даже разрешил подписывать при его отсутствии. Утверждал расписание заместитель командира части по˚строевой подготовке полковник Грудинин. Однажды в˚воскресенье Миша принёс на˚утверждение расписание. Грудинин не~стал его подписывать. Миша пошёл и˚проверил, всё правильно. Снова не~утвердил. Оказывается, не~поставил «гвардии полковник». Исправил ошибку, Грудинин утвердил. При˚этом спросил, кто его подписал. Он уже успел проверить, что Корабличенко отсутствовал. За˚обман Мише объявил гауптвахту, но˚не~посадили. В~связи с˚предстоящим учением начальник штаба части отменил наказание.


В~батарее Голованова служил Джоник Оганесович Петросян. Миша его хорошо знал, они часто беседовали. Джоник не~хотел служить, выдавал себя за˚психически ненормального. На˚посту, при смене караула, засылал патрон в˚патронник и˚грозился расстрелять разводящего. Попав в˚госпиталь, он˚сбежал (дезертировал). Вызвал командир части Мишу и˚поставил задачу, чтобы он˚его поймал. Время не~ограничено и˚всё, что необходимо будет предоставлено. Нужно было начинать действовать с˚сегодняшнего дня. Две недели Миша находился в˚поисках Петросяна. Продукты питания ему привозили на˚указанные точки. Выдали гражданскую одежду. Однажды удалось по˚различным связям выяснить, что Джоник связан с˚Люсей Прозумент, работавшей в˚психдиспансере и˚проживающей на˚его территории. Проживали Люся с˚мамой в˚однокомнатной квартире. Психдиспансер находился на˚окраине г.~Киева. Территория вокруг была изрезана оврагами, заросшими кустарником, деревьями. Удалось хорошо познакомиться с˚соседями Люси. 

Ходить по˚территории диспансера было жутковато, так как иногда встречались проживающие в˚нём. Соседи Мише сказали, что Джоник ночевал в˚квартире Люси, а˚когда появился там Миша, то˚несколько ночей Люся не~ночевала дома. Миша пригласил Люсю и˚целый день бродил с˚ней по˚территории, надеясь, что она укажет местонахождение Джоника. Но˚этого не~произошло, тогда он отправил её в˚часть. В~части она также ничего не~признала. Оказывается, они несколько ночей ночевали под корнями вывернутого дерева. Однажды в˚овраге, под деревом Миша обнаружил следы солдатских сапог. Измерил, это оказался 41~номер. Предположение было\todo[вар.]{Сразу возникло предположение}, что это следы Петросяна. На˚суде Петросян сказал, что, когда Миша измерял след, он˚находился на˚дереве с˚ножом, хотел зарезать, но˚появилась какая\=/то жалость. «Сожалею, "--- заявил Петросян, "--- что не~сделал подобное».  

\begin{wrapfigure}{O}{.4\textwidth}
\centering
\includegraphics[width=.35\textwidth]{kievVokzal}
\caption{Паровоз ТЭ\=/7397, ст.~Киев-Пасс., Киев. Автор: ЦГКА Украины, 28.05.1955}
\label{fig:kievVokzal}
\end{wrapfigure}

Для поимки дизертира Миша устраивал при помощи присланных солдат из˚части дежурство у˚дома. Но˚это также ничего не~дало. Устал Миша окончательно. Однажды решил съездить в˚часть и˚хоть нормально поесть. На˚трамвайной остановке к˚Мише подошёл молодой парень с˚листком бумаги и˚спросил, знает˚ли он˚часть №~41401. Миша попросил у˚него листок, развернул, а˚там написано: «Джоник Оганесович Петросян» "--- тот, кого он˚ищет. Мишу охватила внутренняя дрожь, которая, видимо, передалось спрашивающему. Миша решил заманить его в˚часть и˚попытался спокойным тоном ему ответить, что часть эту знает, сам находится в˚увольнении и˚едет в˚дивизию, поэтому сможет его сопровождать.  Но˚собеседник заявил, что он˚опаздывает на˚работу и˚поспешил уйти. Впоследствии на˚суде выявилось, что этот парень следил за˚Мишей и˚должен был убрать его.

Однажды утром при беседе с˚соседями Люси выявилось, что Петросяна видели у˚Люси в˚гражданской форме. У Миши проскользнула мысль, что сегодня он˚будет уезжать. Сразу же˚поехал в˚часть, переоделся в˚военную форму и˚отправился на˚Киевский вокзал. Предварительно изучил все входы и˚выходы, расписание поездов на˚Ереван и˚Баку. Предупредил военный патруль, что выполняет задание, и˚чтобы его не~задержали, если не~отдаст честь. 

По˚расписанию должен был отправляться поезд на˚Баку. При˚выходе из˚зала ожидания Миша встал за˚дверью и˚высматривал выходящих. Объявили, что до˚отхода поезда оставалось 5\,минут, провожающим покинуть вагоны. Вдруг видит "--- из˚зала ожидания идёт Петросян. Миша дождался пока он˚поравняется с˚ним и˚схватил его. Он~начал вырываться и˚кричать, что военный напал на˚него. Начали подходить гражданские, поэтому Мише пришлось крикнуть военному патрулю, чтобы помогли. Они~схватили его и на˚машине вместе с˚Мишей доставили к˚коменданту города. 

Был первый этаж\todo[вар.]{Кабинет коменданта располагался на˚первом этаже}. Комендант сидел за˚столом, рядом было открыто окно. Ввели Петросяна, и он˚сразу бросился в˚окно. За˚окном был часовой и˚схватил его. Комендант заявил: «Браток, у˚меня ещё ни˚один не~убегал». 

На˚машине Мишу отвезли в˚часть. Дежурным по˚части в˚тот день был капитан Голованов. Дал команду поварам в˚столовой накормить Мишу до˚отвала. Пригласил его назавтра в˚кино. С˚одной стороны Миши сидела его жена, а с˚другой "--- командир батареи. Но, видимо, Миша за˚всё время так устал и˚стрессовое состояние дало о себе знать, что он˚уснул и˚ничего не~видел. Очнулся, когда написано: «Конец». Жена капитана спросила: «Понравилось кино?» Она прекрасно видела Мишино состояние. 

Люся\todo[вар.]{, подруга Петросяна,} уволилась с˚работы, выписалась из г.~Киева. Видимо собиралась уезжать, но у˚Петросяна оказался один билет. Он её˚обманул. Состоялся суд. Петросян и˚целая группа людей, связанных с˚ним, осуждена. По˚Люсе вынесено частное определение "--- не~прописывать в г.~Киеве. Такая участь постигла этих молодых людей.

Командир части Мише объявил отпуск, но он им не~воспользовался. Начинались крупные учения. Учения затянулись.\todo[вар.]{удалить}

Однажды во˚время боевых стрельб снаряд разорвался в˚50\,м от˚наблюдательного пункта, в˚которм находился Миша. Осколками срезало ножку буссоли\footnote{Буссоль "--- это геодезический инструмент для измерения горизонтальных углов при˚съёмках на˚местности, специальный вид компаса. 

Артиллерийская буссоль "--- это вид буссоли, применяемой в˚артиллерии для˚определения магнитных азимутов и˚дирекционных углов, ориентирования орудий и˚приборов в˚заданном направлении, измерении расстояния, засечки целей, а˚также для˚наблюдения и˚разведки.}, обвалило бруствер\footnote{Бруствер "--- это насыпь в˚фортификационном сооружении, предназначенная для˚удобной стрельбы, защиты от˚пуль и˚снарядов, а˚также для укрытия от˚наблюдения противника; вместе с˚тем бруствер служит для˚образования боевой позиции, а˚в˚укреплениях представляет и˚дополнительную преграду на˚случай штурма.}, на котором находился Мишин прибор управления минометным огнём. К~счастью, никого не~задело. Оказывается, батарейный вычислитель Шахов неверно рассчитал данные, за˚что понёс наказание. А~Мише вменили в˚обязанности ещё проверять расчёты батарейных вычислителей. 

Как-то~шла бесконечная стрельба (учения). Двое суток никто не~сомкнул глаз. На˚третьи сутки образовался какой\=/то перерыв. Оставили дежурных, а˚остальным разрешено было вздремнуть. Залегли в˚ряд. Автоматы подложили под себя, а˚стволы наружу. Вдруг машина проехала в˚конце голов по˚стволам автоматов. Видимо, не~суждено было Мише с товарищами умереть. Ведь˚машина могла проутюжить их. Шофёр их в˚траве не~заметил.

\begin{wrapfigure}{O}{.4\textwidth}
\centering
\includegraphics[width=.35\textwidth]{svaku}
\caption[СВАКДКУ 1952~год]{СВАКДКУ 1952~год\footnotemark}
\label{fig:svaku}
\end{wrapfigure}
\footnotetext{Источник заимствования "--- \url{http://svaku.ru/forum/gallery/}.}

В~начале службы в˚части Грудинин "--- заместитель командира части по˚строевой подготовке "--- отобрал несколько человек для учёбы в˚Сумское высшее командное артиллерийское училище (СВАКДКУ). В~числе них оказался и˚Миша, который не~желал поступать. Но˚приказ пришлось выполнять. В~училище явились через неделю (задержались у˚друга в г.~Сумы). Миша успешно сдал 2~экзамена. Видя, что остальные будут сданы, и˚его зачислят, написал рапорт об˚отказе для поступления на˚имя начальника училища "--- генерала\-/лейтенанта. Вызов его не~задержался. Генерал так отчитывал, что пот по˚Мише стекал ручьём. Наконец рапорт был подписан, и˚дано предписание в˚течение суток явиться в˚часть, в г.~Киев. Поезд ходил один раз в˚сутки, причём\todo[вар.]{в˚этот день он} уже ушёл. Выполнить приказ, естественно, Миша не~смог. За˚что Грудинин пригрозил отправить его в˚Полтавское пехотное училище, но, к˚счастью, это не~было выполнено. 

Часто руководство дивизии и˚входящих в˚неё частей подвергалось проверкам вышестоящими организациями на боевую готовность, знание своих обязанностей. Подобной проверке однажды подверглась и˚часть генерала\-/майора Капусты. Часть ночью была поднята по˚тревоге. Летом находились в˚лагерях. Предстояло командиру части выдержать экзамен. Экзаменовал представитель из˚округа. На˚наблюдательном пункте за˚столиком разместились Миша с˚прибором управления миномётным огнём, с˚одной стороны "--- командир части, с˚другой "--- проверяющий. Командиру части предстояло расшифровать метеоогневой, определить заряд, доворот и˚дальность полёта снаряда и˚передать радисту для отправления на˚огневые позиции. Стреляли боевыми снарядами. Так˚должно продолжаться до˚тех пор, пока не~будет поражена цель.

\begin{wrapfigure}{O}{.4\textwidth}
\centering
\includegraphics[width=.35\textwidth]{BNTU}
\caption[Белорусский национальный технический университет (БНТУ). Главный корпус]{Белорусский национальный технический университет (БНТУ). Главный корпус\footnotemark}
\label{fig:BNTU}
\end{wrapfigure}
\footnotetext{Автор: Gruszecki, 29.05.2010.}

Миша должен был проконтролировать правильность расчётов. Но˚негласная задача заключалась ещё в˚том, чтобы незаметно от˚проверяющего оказать помощь генералу. Уже первые расчёты командира части оказались не~совсем верными. Незаметно пришлось подсовывать правильное решение. Видимо, напряжение было такое, что Миша перестал что-либо видеть. Он попросил прекратить стрельбу, вышел из наблюдательного пункта, но˚зрение не~восстановилось. Обратно возвратился по˚стенке блиндажа. Стрельба больше не~возобновилась. Командиру части поставили положительную оценку. По˚возвращении в˚палатку, генерал приказал дневальному Мишу не~будить. Он проспал ночь и˚ещё полдня. После˚сна поднялся подошёл к˚соску умывальника и в˚тумане увидел его. На˚радостях во˚всё горло раздался крик: «Вижу!» Пересказать состояние Миши в˚это время очень трудно, просто нужно ощущать на˚себе.

\begin{wrapfigure}{O}{.4\textwidth}
\centering
\includegraphics[width=.35\textwidth]{GAZM20Pobeda(Gwafton)}
\caption[ГАЗ-М\=/20 «Победа»]{ГАЗ-М\=/20 «Победа»\footnotemark}
\label{fig:GAZM20Pobeda(Gwafton)}
\end{wrapfigure}
\footnotetext{\foreignlanguage{english}{Classic Motor Show parking lot in Lahti, Finland}. Автор: Gwafton, 08.05.2010.}

Однажды часть подняли по˚тревоге. Выстроили всех на˚плацу. Каждому второму\--третьему из˚шеренги поступила команда "--- выйти. Такая участь пришлась солдату, стоящему рядом с˚Мишей. Всех их с˚техникой отправили в˚Чечню, многие из˚них обратно не~возвратились\footnote
{Здесь речь идёт о массовых беспорядках в г.~Грозном, 23\==31~августа 1958~г. Причиной беспорядков стала обострившаяся межнациональная напряжённость, в˚связи с˚реабилитацией репрессированных народов (чеченцев и˚ингушей) и˚восстановлении Чечено-Ингушской АССР, на˚фоне крупных ошибок в˚работе КПСС и Совета министров Республики}. 

\todo[ред.]{Наверху (2 абз.) уже было про три года}Служили в то˚время три года. Лица, убегающие в˚самоволку, должны были дослужить дополнительно. Максимов, из˚взвода управления, набрал дополнительно полгода. Однажды на˚Новый год Мише приказали его найти и˚привезти в˚часть. Задача стояла неблагодарная. Помогло знание места его нахождения. 

В~то˚время появился секретный приказ, если пришёл вызов из˚вуза, то˚могли демобилизовать из˚армии досрочно. Такой вызов Миша получил из˚Минского политехнического института. Кончался последний год службы.

\begin{figure}[h]
\includegraphics[width=\textwidth]{private/nagradaShubichBoevoeZnamya}
\caption{Фотография Миши с˚боевым знаменем части. 1958~год. Источник заимствования "--- личный фотоархив М.\,П.~Шубича}
\label{fig:nagradaShubichBoevoeZnamya}
\end{figure}

Миша обратился к˚командиру части и˚показал вызов. Капуста сказал ему, чтобы он всё сдал и тогда будет уволен сегодня. Быстро приступил к˚сдаче. Повстречал заместителя командира части по˚политической части. Он не~разрешил Мише сдачу и˚привёл опять к командиру части с˚вопросом: «Кто будет вычислять?» Капуста ответил: «Мы с˚тобой». Приказал явиться к˚штабу части в˚14:00. При˚появлении в˚указанное время был крайне удивлён. Офицеры штаба были построены и˚перед ними Капуста изложил Мише душевное отцовское напутствие. Сфотографировали у˚знамени части. Капуста приказал сопровождать до˚пристани Кальное взводу управления под командованием командира взвода. Разрешил взять его машину Победу. Естественно, Миша был растроган таким вниманием и˚заботой к˚простому солдату. 

При демобилизации Миша получил денежное довольствие. На˚полученные деньги удалось купить чемодан и˚осталось немного денег. Дорога до˚Минска была бесплатной. В~Минске Мишу поселили в˚общежитии.\todo[вар.]{Питался он два раза в˚сутки в˚столовой при общежитии. На˚столах лежал бесплатный хлеб и к˚нему Миша заказывал по˚четыре стакана чая. Денег на˚пропитание хватило ненадолго: Миша успел сдать лишь 2 экзамена, причём на на˚«четыре» и˚«пять».} Денег на˚пропитание хватило лишь сдать 2~экзамена, экзамены сданы на˚«четыре» и˚«пять». На˚столах хлеб был бесплатным. Заказывал лишь по˚четыре стакана чая. Питался два раза в˚сутки. Дальше денег не~осталось ни˚копейки, он голодал. Решил уехать домой. По˚проспекту Сталина до˚вокзала добрался пешком. В~это время осуществлялась реконструкция проспекта. За˚Мишей шла цыганка и˚просила осчастливить её˚ручку. Злость овладела им, готов был растерзать её "--- от˚голодного солдата требуют денег.

На˚поезде ехал «зайцем». Проводники понимали объяснение солдата, не~имеющего ничего в˚кармане. В~Калинковиче военный патруль снял Мишу с˚поезда. Собирались обратно отправить в˚часть. 

Родители очень обрадовались возвращению сына. Дома он пробыл три дня и˚решил\todo[вар.]{снова ехать поступать в институт} уехать поступать. Денег хватило лишь на˚билет. От˚родителей Миша старался денег не~брать, а по˚возможности помогать им. Так˚как у˚них денег не~было.

На˚дорогу до˚Москвы родители денег дали, а˚затем появилась проблема, как жить дальше без денег и˚жилья. Шёл 1958~год.

К~людям Миша всегда относился любезно, помогал всем, чем мог, даже делился последним куском хлеба, дорожил дружбой. Другом по˚учёбе в˚школе был Валик Гриневич. В~юности договорились помогать друг другу, даже материально. Когда˚Валик служил в˚армии (его призвали раньше Миши), присылал ему письма и˚деньги. Соблюдал договорённость. Во~время службы Миши от˚Валика получено лишь два письма. В~первом он˚сообщал, что женится на˚дочери учителя Шевцова "--- Маше (у~них\todo[вар.]{у~учителя Шевцова} было шестеро детей). В~ответном письме поздравил его с˚таким событием. Маша была стройная, молодая девушка, с˚голубыми глазами, небольшим носиком, пышными волосами, нежной кожей лица, симпатичной. От˚неё исходила какая\=/то свежесть, энергия. Во˚втором письме Валик сообщил, что все разладилось и он на˚Маше не~женится.

По˚приезду в д.~Вишеньки вечером Миша пошёл в˚клуб. Клуб представлял собой небольшое деревянное здание, скамейки, сколоченные из˚досок. В~клубе Валик показывал кино (работал он˚киномехаником). Миша договорился с˚Валиком и˚Машей, что после кинофильма будет сопровождать их˚домой. Задачей Миши было сблизить их, наладить потерянные взаимоотношения. После˚просмотра кинофильма втроём отправились к˚дому Валика. По˚дороге Миша всяким образом старался восстановить их˚взаимоотношения, доказывал, какой хороший Валик парень. Дойдя до˚дома Валика, хотел оставить их˚одних. Но˚Маша спросила, придёт ли˚Миша завтра в˚клуб и˚предложила её˚проводить. В~этот момент до˚сознания Миши дошло, что в˚любви не~может быть помощников.

В~следующий раз по˚приезду в˚деревню во˚время каникул решил встретиться с˚другом. Миша пригласил Валика к˚себе в˚дом, угостил, но, к˚сожалению, никакой беседы не~получилось. Миша пытался вести беседы на˚разные темы, но˚было полное непонимание друг друга. Валика больше уже интересовала водка, пьянка, какие\=/то незначительные дела. Мише стало ясно, что человек полностью деградировал, живёт без целей, обволокла его повседневность деревенской жизни. Он уже был женат на˚недалёкой такой же˚женщине. С~этого дня закончилась дружба двух молодых людей.
