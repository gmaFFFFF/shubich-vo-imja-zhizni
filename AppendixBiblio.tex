\section{Краткая библиография}

Всего опубликовано 138~научных и˚учебно\-/методических работ, общим объёмом более 180~печатных листов. Большое внимание уделялось подготовке учебных пособий, многие из˚которых были мной подготовлены впервые: наиболее значимые из˚них являются: 

\begin{enumerate}
	\item Внутрихозяйственное землеустройство сельскохозяйственной организации: Учебное пособие / С.\,Н.~Волков, М.\,П.~Шубич, Е.\,В.~Черкашина, В.\,В.~Пименов , Е.\,А.~Дуплицкая, М.\,Р.~Мячина. Под ред. С.\,Н.~Волкова.-М., 2017.- 194 с.
	\item Эколого-хозяйственная оценка территории сельскохозяйственных организаций : метод. указ./ авт. сост. М.\,П.~Шубич, А.\,П.~Исаченко; под ред. М.\,П.~Шубича; Гос. ун-т по землеустройству; Каф. землеустройства. -М., 2009. -58 с.: табл.
	\item Шубич, М.\,П. Землеустроительное проектирование. Размещение и устройство территории ягодников, рабочий проект на их создание и устройство: учеб. пособие.-М.,2011.-82с.
	\item \Todo[Материал]{Не нашёл точное библиографическое описание книг}
	\item Устройства территории садов; 	
	\item Образования садоводческих, виноградарских и˚ягодниководческих агропромышленных организаций и˚устройство их˚территорий;	
\end{enumerate}	
\Todo[Материал]{Как-то эти книги надо назвать}

\begin{enumerate}
	\item Землеустроительное проектирование: учеб. пособие/ сост. С.\,Н.~Волков, В.\,П.~Троицкий, М.\,П.~Шубич и др; ГУЗ .-М., 2002.-120с.
	\item Практикум по внутрихозяйственному землеустройству сельскохозяйственного предприятия : Учеб. пособие для вузов/ С.\,Н.~Волков, М.\,П.~Шубич, А.\,В.~Купчиненко, В.\,В.~Пименов, Е.\,В.~Черкашина; Гос. ун-т по землеустройству. -М.: ГУЗ, 2003. -163 с.
\end{enumerate}
\todo[Материалы]{Вы не могли бы у Косинского в электронном виде взять эти материалы из биобиблиографии деятелей землеустроительной науки?}