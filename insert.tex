\chapter{Вставки}


Вставка 1

В˚зимний период жизнь людей, проживающих в˚шалашах среди болота и˚леса, была невыносимым, тяжёлым испытанием. С˚наступлением весны она становилась немного легче: теплело, с питанием проще "--- переходили на˚щавель, лебеду. Вкус настоящего хлеба испытал Миша во˚время наступления наших войск, когда один солдат дал полбуханки хлеба. Это действительно был настоящий праздник для˚Миши и˚всей его семьи. Кстати, за˚всё своё детство Миша ни˚разу не~испытал вкус сладкого (конфет).

Вставка 2

Молодёжь немцы сгоняли и˚увозили в˚Германию, многих убивали, людей сжигали целыми деревнями. Неизвестная участь постигла и˚дедушку Миши. Посадили на˚мотоцикл и˚увезли. Больше его никто не~видел.
Немцы повсеместно занимались грабежом, забирали и˚увозили всё.

Вставка 3

Ни˚одного дня не~был на˚квартире, так как нечем было оплачивать. Даже не~было часов, чтобы ориентироваться во˚времени. Иногда среди ночи поднимаешься и˚идёшь. При подходе к˚г.п. Ельск начинало светать, а˚когда ещё было темно приходилось ожидать на˚железнодорожной насыпи.

Вставка 4

Несмотря˚на все послевоенные трудности, трату много сил и времени на˚дорогу, старался Миша не~подводить учителей и˚учиться хорошо.
Несколько обидно становится ему за˚отдельных нынешних школьников и˚студентов, которым предоставлены все условия для˚учёбы и˚жизни, но˚они плохо занимаются, не~стараются усвоить материал, иногда просто бездельничают, ведут иждивенческий образ жизни.

Вставка 5

Техники, лошадей в˚послевоенное время не~было. Поэтому все работы выполняли силами работающих: вручную косили траву, жали рожь и˚другие зерновые культуры, сеяли. Лошадей, которые оставались, нечем было кормить. Они не~могли стоять на˚собственных ногах.

Вставка 5

Педагогическая деятельность увлекала Михаила Павловича, но˚были серьёзные бытовые трудности "--- неустроенность с˚жильём (семья находилась в˚г. Рязани, проживали в˚маленькой комнатушке без˚удобств). Перспектив с˚жильём в˚институте не~предвиделось, что˚сказалось на˚его настроении и˚в˚какой-то степени на˚работе.
Педагогическая деятельность наряду со˚всеми видами учебной работы (чтение лекций, проведение курсового проектирования, семинаров, зачётов, экзаменов, дипломного проектирования, работы с˚аспирантами и˚т.д.) предусматривала проведение методической, научно-исследовательской, воспитательной и˚общественной работы.
 
Вставка 6

В˚соответствии с˚распоряжением Рособрнадзора Михаил Павлович Шубич являлся экспертом по˚анализу содержания и˚качества подготовки по˚образовательным программам и˚аккредитовал в˚2012 году две специальности в˚ФГБОУ ВПО «Ивановская государственная сельскохозяйственная академия имени академика Д.К. Беляева» и четыре специальности в˚ФГБОУ ВПО «Пензенский государственный университет архитектуры и˚строительства». 

Из˚публикаций о˚Михаил Павловиче Шубиче следует, что его жизнь - это пример беззаветного служения Родине, землеустроительной науке, высшему землеустроительному образованию.
Михаил Павлович Шубич проявил себя как высококвалифицированный педагог, отличный методист, крупный учёный, активный общественный деятель и организатор. На всех участках работы он относился ответственно, проявлял инициативу, самостоятельность, успешно справлялся с любыми делами. (Профессор, д.э.н., заслуженный работник культуры РФ, В.В. Косинский, журнал «Землеустройство, кадастр и мониторинг земель» №~1, 2018~г., - с.`~73).
Эффективность работы преподавателя кафедры во˚многом зависит от˚организации и˚руководства кафедрой заведующего и˚его заместителей.

Вставка 7

13 октября 2017 года Михаил Павлович Шубич отправлен на˚заслуженный отдых. Он вполне с˚этим согласен, но˚его крайне возмущает, как проведено это действие. Заведующий кафедрой С.Н. Волков не изволил вызвать на˚кафедру, сказать несколько тёплых слов, а подготовил втихаря уведомление, передал его в˚отдел кадров и˚распорядился вручить М.П. Шубичу под˚расписку. Руководитель с˚нормальными человеческими чертами со˚своими сотрудниками, а˚тем более профессором, заслуженным землеустроителем РФ, человеком, отдавшим всю свою жизнь на развитие вуза(проработал в МИИЗе, ГУЗе 47 лет), занимавшему разнообразные должности: 5 лет председателя объединённого профкома, 3 года члена горкома профсоюза работников сельского хозяйства, 5 лет заместителя заведующего кафедрой, 3 года заместителя заведующего кафедрой по˚научной работе, 18 лет декана заочного факультета, 3 года ответственного секретаря приёмной комиссии и т.д. И даже с˚ассистентом так нельзя поступать. Это характеризует С.Н. Волкова как руководителя и˚человека, от˚него иного не следовало ожидать. По˚этому поводу звонил Михаилу Павловичу профессор, д.э.н., академик РАН, бывший министр сельского хозяйства В.Н. Хлыстун и˚высказал своё неодобрение такими действиями С.Н. Волкова.