\chapter{Вставки}
Вставка˚ настр. 18

Как\=/то в˚субботу зашёл на˚квартиру проживания начальника партии в с.~Богутичи (фамилию его не~помню). Дом находился рядом с˚базой. В~квартире была страшная пьянка. Можно выделить четыре способа опьянения: первый "--- увеселение, второй "--- возбуждение, третий "--- оглупление и˚четвёртый "--- оскотнение. Во˚всех уже была четвертая степень опьянения. Мише начальник партии наливает стакан водки и˚предлагает выпить. На˚ответ Миши, что вообще не~употребляет спиртного и не~может выпить, начальник партии поставил ружье (он˚имел своё ружьё) и˚приказал: «Пей!». Выбрал момент, когда он˚чуть отвернулся, Миша прыгнул в˚окно (окно было открыто). Вдогонку раздался выстрел. Миша уже был на˚земле, и˚пуля пролетела над головой. Из-за оскотнения человека Миша чуть не~лишился жизни в˚молодом возрасте.

Вставка на˚стр.21 

Иногда устраивались соревнования между взводами, отделениями. Об˚одном подобном соревновании между отделениями следует изложить.

Соревнующиеся должны пробежать \Todo[Текст]{сколько километров?} км, затем преодолеть полосу препятствия, проползти под натянутой проволокой, пробежать по˚бревну, спрыгнуть и˚через 50\,м˚был финиш. Миша бежал в˚отделении последним. Поэтому перед ним была поставлена задача "--- победить. С~задачей он˚справился, опережал соперника. Но˚когда прыгнул с˚бревна, упал и не~смог встать. Подвернул ногу. Отправили в˚лазарет. Заведующим лазарета был сын командира дивизии. На˚второй день он˚заявил Мише, что неделю будет отсутствовать, покупает мотоцикл. Видимо, сыну командира дивизии все «позволено». Мише поручено «принимать» и˚больных и˚оказывать им «лечение». По˚возвращении он˚заявил, что Миша остаётся его помощником. Предстояло очередное учение части. Начальник штаба части предложил Мише убежать из˚лазарета, иначе они не~могут дать команду на˚возвращение в˚часть. Что˚он и˚сделал. Так˚закончилась его «работа» в˚лазарете.

Вставка на˚стр. 29

На˚дорогу до˚Москвы родители денег дали, а˚затем появилась проблема, как жить дальше без денег и˚жилья. Шёл 1958~год.

К~людям Миша всегда относился любезно, помогал всем, чем мог, даже делился последним куском хлеба, дорожил дружбой. Другом по˚учёбе в˚школе был Валик Гриневич. В~юности договорились помогать друг другу, даже материально. Когда˚Валик служил в˚армии (его призвали раньше Миши), присылал ему письма и˚деньги. Соблюдал договорённость. Во˚время службы Миши от˚Валика получено лишь два письма. В~первом он˚сообщал, что женится на˚дочери учителя Шевцова "--- Маше (у~них было шестеро детей). В~ответном письме поздравил его с˚таким событием. Маша была стройная, молодая девушка, с˚голубыми глазами, небольшим носиком, пышными волосами, нежной кожей лица, симпатичной. От˚неё исходила какая\=/то свежесть, энергия. Во˚втором письме Валик сообщил, что все разладилось и он на˚Маше не~женится.

По˚приезду в д.~Вишеньки вечером Миша пошёл в˚клуб. Клуб представлял собой небольшое деревянное здание, скамейки, сколоченные из˚досок. В~клубе Валик показывал кино (работал он˚киномехаником). Договорился с˚Валиком и˚Машей, что после кинофильма буду сопровождать их˚домой. Задачей Миши было сблизить их, наладить потерянные взаимоотношения. После˚просмотра кинофильма втроём отправились к˚дому Валика. По˚дороге Миша всяким образом старался восстановить их˚взаимоотношения, доказывал, какой хороший Валик парень. Дойдя до˚дома Валика, хотел оставить их˚одних. Но˚Маша спросила придёт ли˚Миша завтра в˚клуб и˚предложила её˚проводить. В~этот момент до˚сознания Миши дошло, что в˚любви не~может быть помощников.

В~следующий раз по˚приезду в˚деревню во˚время каникул решил встретиться с˚другом. Миша пригласил Валика к˚себе в˚дом, угостил, но, к˚сожалению, никакой беседы не~получилось. Миша пытался вести беседы на˚разные темы, но˚было полное непонимание друг друга. Валика больше уже интересовала водка, пьянка, какие\=/то незначительные дела. Мише стало ясно, что человек полностью деградировал, живёт без целей, обволокла его повседневность деревенской жизни. Он уже был женат на˚недалёкой такой же˚женщине. С~этого дня закончилась дружба двух молодых людей.

Вставка на˚стр.40

С~руководителем делегации решили посетить в˚Будапеште ночной клуб. При˚входе с˚каждого взяли по˚35~форинтов. Столик наш разместили у˚сцены. К~нам подсадили двух местных девушек. Оказывается, отдельные молодые люди Венгрии целыми сутками пребывали в˚ночном клубе, не~являясь домой. Объяснились с˚девушками при помощи венгерско\-/русского разговорника. Гид находился в˚конце зала, вдалеке от˚нас. Заказали для девушек бутылку вина, закуску, а˚затем они запросили ещё бутылку. Поскольку˚расплачиваться пришлось Мише, то˚очень беспокоило хватит ли˚форинтов (выдали нам небольшую сумму), так как цену вина и˚угощения мы не~знали. Спасло то, что входные форинты вошли в˚оплату. Поведение молодёжи, в˚том числе наших девушек было развязным. Они заявили: «Сейчас мы не~любим венгерских парней, а˚любим русских». На˚сцене выступала обнажённая девушка (артистка), часто выпивала вино. опускалась и˚пыталась обнять нас. Мы вели себя в˚строгом соответствии с˚инструктажем, полученным в˚Горкоме партии в˚Москве, иначе была бы˚взбучка. Девушки проводили нас до˚Интуриста. 

В~то˚время отношение отдельных венгров к˚русским было очень плохим, в˚чем Миша убедился на˚себе.
Днём однажды он˚решил ознакомиться с˚Будапештом. В~одиночку пошёл бродить по˚городу. За˚ним увязалась какая\=/то женщина. Она преследовала его по˚пятам, выражая негодование. Лицо её˚было злым, она что\=/то кричала. Задачей Миши было избавиться от˚неё, начал заходить в˚разные переулки. Таким путём удалось избавиться от˚неё, но˚окончательно заблудился. Встретил молодую девушку и˚попытался по\=/немецки спросить, как пройти к˚Интуристу. Девушка заулыбалась и˚сказала: «Лучше спроси по\=/русски». Оказывается, это была русская девушка, которая была замужем за˚венгром и˚проживает уже несколько лет в˚Будапеште. Она довела до˚Интуриста, за˚что Миша сердечно её˚поблагодарил. 

Вставка на˚стр. 45

В~отряде работали как инженеры\-/землеустроители, так и˚техники. Инженеры разрабатывали проекты землеустройства, а˚техники проводили обследование землепользований, съёмку местности, корректуру(сличение с˚натурой) планово\-/картографического материала, перенесение проекта на˚местность. Начальник отряда осуществлял приёмку работ, определял качество выполнения с˚выездом в˚колхозы, совхозы. Техник Саша Варсеева, молодая девушка, среднего роста, несколько утолщённой комплекции с˚большой грудью, крупным симпатичным лицом переносила проект землеустройства в˚натуру. Приём работы планировался в˚восемь часов. Саша заявила, что она не~может проснуться в˚это время. «Меня однажды вынесли с˚кроватью в˚овраг и я не~проснулась». По˚приезду в˚хозяйство Миша поселился у˚хозяйки проживания Саши, а её по˚решению директора совхоза на˚ночь разместили в˚его кабинете. Утром прибежали от˚директора с˚заявлением «не~можем разбудить» (директору необходимо освободить кабинет). Действительно, кричали, толкали, трясли… Саша не~просыпалась. Только˚когда начали лить на˚неё холодную воду, удалось разбудить.

По˚приезду во˚второе хозяйство на˚приёмку работ оказалось, что почвовед Грызлов, Варсеева и˚другие специалисты проживали во˚вновь выстроенном здании все вместе (койки стояли в˚ряд). Предстояло очень рано входить в˚поле. Почвовед Грызлов поднимался рано. Саше шутя предложено привязать её˚ногу за˚сапог почвоведа. Обует сапоги, он˚дёрнет, и˚Саша проснётся. Саша действительно натихоря это проделала. Ночью Грызлов проснулся, быстро вдел ноги в˚сапоги и˚начал бежать по˚своим надобностям во˚двор. Верёвки он не~видел и˚грохнулся со˚всего маха на˚пол. Естественно, всех разбудил, воцарился сплошной смех. Впоследствии он˚обижался на˚шутку, проделанную над ним. Ведь˚это была не~шутка, а˚необходимость разбудить Сашу.

Техник Евсеев со˚стажем работы переносил проект внутрихозяйственного землеустройства в˚натуру в˚колхозе Милославского района. Территория землепользования изрезана сплошными оврагами (сильное проявление линейной эрозии). Отдельные овраги были глубокими и˚длиной 300\==400\,м. При˚перенесении проекта границы полей и˚поворотные точки обозначались установленными знаками (закапывался деревянный столбик и˚окапывался курганом). Вместо˚положенного одного оврага техник закопал столбик на˚границе другого подобного оврага, через 500\,м. Оказывается, он не~ориентировался на˚местности и не~смог показать наше местонахождение на˚плане. Поэтому предложено ему изменить место работы. Договорился с˚архитектурой области, и˚они приняли его. При˚очередной встрече Евсеев поблагодарил Мишу и˚заявил: «Здесь я˚оказался нужным и˚полезным работником, подобных ошибок не~совершаю».

Вставка на˚стр. 47

Хотелось ещё поведать читателю об˚одном забавном случае. Во˚время работы инженером\-/землеустроителем Миша проводил в˚одном колхозе Рязани корректуру планово\-/картографического материала. Старик на˚поле косил овёс. «Что˚ты ходишь с˚бумажкой? Лучше бы˚покосил!», "--- проронил старик. «Давайте попробую, никогда не~косил, а Вы займитесь моей работой», "--- и˚передал старику план землепользования. Конечно Миша здесь слукавил, что никогда не~брал в˚руки косу. Когда˚Миша прокосил один ряд старик заметил: «У~тебя получается, а я в˚твоих бумагах ничего не~понимаю». Всё\=/таки старику хотелось как\=/то ущипнуть, задеть Мишу. Тогда он˚предложил побороться. Естественно, Мише не~хотелось уронить своё достоинство и˚приложил усилия, чтобы побороть сильного старика. Старику видимо понравился Миша своими действиями. Вечером ежедневно он˚приходил к˚дому проживания Миши. В~знак признательности он˚подарил Библию с˚просьбой не~критиковать её. 

Вставка на˚стр.52

Немного следует изложить о˚первых дипломниках. 1970~год начало работы на˚кафедре в˚должности ассистента. Появились первые дипломники. Одной из˚таких дипломниц была Алевтина Алексеевна Мохнаткина, девушка с˚выраженным характером, трудно поддающаяся убеждениям. Считала, что если принято какое\=/то решение, то˚иного более целесообразного и˚эффективного не~может быть. Предлагаю ей˚новое решение по˚организации использования земли в˚колхозе «Россия» Косимовского района Рязанской области (объект её˚дипломного проекта) в˚отличие от˚принятого. В~ответ заявляет: «Над˚этим решением думали десять голов, неужели Ваша голова умнее их». Долго пришлось её˚убеждать более объективно смотреть на˚вещи и˚правильно их˚оценивать. Защитилась она на˚отлично. После˚защиты преподнесла букет цветов, внутри которых записка на˚вырезанном куске ватмана следующего содержания: «С~Вас получится очень хороший преподаватель, воспитатель. Вы сумели совершенно изменить меня».

Любой человек, особенно молодой, всегда должен заранее обдумывать свои действия, поступки и˚делать выводы, к˚чему это приведёт впоследствии. Делать выводы после свершившегося факта бывает бесполезным. Девушка Надя была на˚преддипломной практике в˚колхозе «Чёрная речка» Сапожковского района Рязанской области. Приезжаю в˚хозяйство (преподаватели разъезжали по˚командировкам, чтобы оказать помощь студентам в˚разработке проекта и˚сборе материала для дипломного проектирования). На˚квартире с˚ней проживали ещё две девушки "--- молодые специалисты. Захожу в˚квартиру. Посреди˚стоит чемодан. Надя собралась уезжать. На˚вопрос: «В~чем дело?», "--- ответа не~последовало. Проживающие с˚ней девушки ничего не~сказали, лишь загадочно переглянулись. Впоследствии выясняю, что поздно вечером заехал на˚машине председатель колхоза и˚предложил Наде взять материалы для утверждения в˚районе разработанного проекта. Девушка быстро собралась и не~подумала, что рабочий день уже закончился. Председатель полночи возил её по˚лесу. В~результате она решила убежать с˚практики. Хотелось побеседовать с˚председателем колхоза, но он˚всяким образом избегал подобной встречи. Поскольку˚встреча не~состоялась, то˚решено было изложить в˚газете «Приокская правда» о˚пренебрежительных действиях председателя к˚проекту внутрихозяйственного землеустройства и˚нарушениях границ его перенесения. По˚результатам статьи председатель колхоза был снят с˚должности. Все, что удалось сделать за˚безнравственный проступок председателя колхоза. На˚сердце долго у˚Миши оставался горький осадок, к˚чему могут привести безрассудные, необдуманные наши действия.

Вставка на˚стр. 59

При˚посещении г.~Абакана изменилось представление о˚Сибири. Раньше думал, что Сибирь очень суровый край и˚мало какие растения произрастают. Оказалось, что здесь на˚корню созревают помидоры, кукуруза на˚зерно, чего трудно добиться в˚Подмосковье. Абакан "--- красивый город. Жители восхвалялись, что отдельно от˚жилой вынесена производственная зона, но˚оказывается не~учли направление вредоносных ветров, разместив жилую зону с˚наветренной стороны. Проехали на˚машине вдоль реки Енисей: туда с˚одной стороны, обратно "--- с˚другой. Побывали в˚Шушенском "--- место ссылки В.\,И.~Ленина. Экскурсовод подробно изложила о˚прелестях этого края, доме А.\,Д.~Зырянова, где жил в˚ссылке и˚как жил В.\,И.~Ленин. Действительно, природа оказалось превосходной. Даже˚Миша в˚порыве восторга заявил экскурсоводу: «Не˚могли бы˚меня на˚недельку оставить здесь в˚заточении?». Последовал ответ: «Многого хочешь». Посетили Саяно\-/Шушенскую ГЭС. Полдня Миша бродил по˚плотине. При˚этом возмущался, что никто не~спросил зачем он˚здесь, что делает? Ведь˚это охраняемый, серьёзный объект. Видимо, наша безалаберность может привести, а˚иногда и˚приводит к˚серьёзным последствиям. Хотя˚здание штаба недалеко, а на˚холме установлена ракетная установка. По˚дороге обратно заехали в˚карьер добычи мрамора. Добычу его производят двумя способами: взрывом и˚распиливанием на˚плиты. При˚первом способе образуется много окрошки. Невдалеке от˚карьера выложено 500\,м˚дороги из˚мраморной окрошки. При˚посещении дороги японец воскликнул: «Надо˚же, дороги строят из˚золота!» Мрамор залегает разноцветный. Действительно это богатство, которым следует дорожить. Возмущало Мишу варварское обращение с˚кедром (вырубается, кучи гниют среди тайги). Ведь˚кедровая сосна является ценным материалом для хозяйственных целей, дающая в˚шишках съедобные семена\-/орехи, из˚которых изготовляют кедровое масло. В~лесу оказалось много различных грибов, но˚жители собирают лишь белые.

Вставка на˚стр.63

Вузы в˚настоящее время готовят дипломированных бакалавров и˚магистров. Считаю для страны наиболее целесообразно вести подготовку высококвалифицированных специалистов разных специальностей вместо бакалавров. Бакалавр фактически является неполноценным специалистом, который не~может эффективно решать стоящие задачи. Считал бы˚целесообразным отказаться от … конференции, которую подписала Россия, и˚готовить полноценных специалистов в˚течение пяти лет. Думаю, полностью поддержат меня в˚этом и˚работодатели. На˚рынке труда более полезным и˚востребованным окажется высококвалифицированный специалист. Для˚подготовки таких специалистов высшему образованию на˚современном этапе необходимы грамотные преподаватели, владеющие современными технологиями педагогической деятельности, постоянно ведущие научно\-/техническую работу.


